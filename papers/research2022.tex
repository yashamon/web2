\documentclass{amsart}
\usepackage{graphicx} 
\usepackage{color} 
 % \usepackage{biblatex} 
\usepackage{hyperref} 
\usepackage{graphicx}  
\usepackage{verbatim}  
\usepackage[svgnames,pdf]{pstricks}
\usepackage{pst-plot,pst-eucl}
\hypersetup{backref,pdfpagemode=FullScreen,colorlinks=true}
\usepackage[autostyle]{csquotes}
% \usepackage[
%     backend=biber,
%     % defernumbers=true,
%     style=numeric,
%     % autocite=plain, 
%     sorting=none,
%     % sortlocale=us_US,
%     % natbib=true,
%     url=true, 
%     doi=false,
%     eprint=true
% ]{biblatex}
% \addbibresource{~/workspacemodules/link.bib}
\usepackage{appendix, url}
  % \definecolor{gray}{rgb}{.80, .80, .80} 
  % \pagecolor {gray}
%  \color {black}
% \usepackage{url}        
%  \usepackage {hyperref}
\usepackage{amsmath, amssymb, mathrsfs, amsfonts, amsthm} 
\usepackage{tikz-cd}
%  \usepackage[all]{xy}
\newcommand{\ls}{\Omega \textrm {Ham} (M,\omega)} 
\newcommand{\freels}{L \textrm {Ham}} 
\newcommand{\eqfreels}{L ^{S^1} (\textrm {Ham})}   
\newcommand{\cM}{\mathcal M^*_ {0,2}(P_D, {A}, \{J_n\})}
\newcommand{\cMc}{\mathcal M^*_ {0,1}(P ^{\chi} _{f \ast g}, {A}, \{J_n\})}
\newcommand{\ham}{\ \textrm {Ham} (M, \omega)}
\newcommand {\df} {\Omega^t _{r,s}}   
\newcommand{\cMn}{\mathcal M_ {0}(A, \{J_n\})}  
\newcommand{\plus}{(s \times M \times D ^{2})_{+}}
\newcommand{\minus}{(s \times M\times D^2)_\infty}
\newcommand{\delbar}{\bar{\partial}}  
\newcommand {\base} {B}
\newcommand{\ring}{\widehat{QH}( \mathbb{CP} ^{\infty})}
\newcommand{\neigh}{{ \mathcal
{B}} _{U _{\max}}}
\newcommand{\paths}{\mathcal {P} _{g, l,
m} (\phi _{ \{l\}} ^{-}, \phi _{ \{m\}} ^{+})}
% \newcommand {\Q}{ \operatorname{QH}_* (M)}
\newcommand{\Om}{\bar{\Omega}}
% \newcommand{\G}{\bar {\Gamma}}
\newcommand{\fiber}{X}
%\newcommand{\P}{ P ^{\chi} _{f \ast g}}
\newcommand{\ilim}{\mathop{\varprojlim}\limits}
\newcommand{\dlim}{\mathop{\varinjlim} \limts}
\newcommand {\rect} {[0,4] \times [0,1]}
% \DeclareMathOperator{}{H}
% \DeclareMathOperator{\cohH}{H^*}
% \DeclareMathOperator{\C}{\mathbb{R} \times S ^{1}}
\DeclareMathOperator{\card}{Card}
\DeclareMathOperator{\kareas}{K ^s-\text{area}}
\DeclareMathOperator{\rank}{rank}
\DeclareMathOperator{\sys}{sys}
\DeclareMathOperator{\Sing}{Sing}
% \DeclareMathOperator{\2-sys}{2-sys}
% \DeclareMathOperator{\i-sys}{i-sys}
%  \DeclareMathOperator{\1,n-sys}{(1,n)-sys}
% \DeclareMathOperator{\1-sys} {1-sys}
\DeclareMathOperator{\ho}{ho}
\DeclareMathOperator{\karea}{K-\text{area}}
\DeclareMathOperator{\pb}{\mathcal {P} ^{n} _{B}}
\DeclareMathOperator{\area}{area}
\DeclareMathOperator{\ind}{index}
\DeclareMathOperator{\codim}{codim}

\DeclareMathOperator{\su}{\Omega SU (2n)}
\DeclareMathOperator{\im}{im}
\DeclareMathOperator{\coker}{coker}
\DeclareMathOperator{\interior}{int}
\DeclareMathOperator{\I}{\mathcal{I}}
\DeclareMathOperator{\cp}{ \mathbb{CP}^{n-1}}
\DeclareMathOperator {\cA} {\mathcal{A}_{\infty} }
\DeclareMathOperator{\Aut}{Aut}
\DeclareMathOperator{\Vol}{Vol}
\DeclareMathOperator{\Diff}{Diff}    
\DeclareMathOperator{\sun}{\Omega SU(n)}
\DeclareMathOperator{\gr}{Gr _{n} (2n)}
\DeclareMathOperator{\holG}{G _{\field{C}}}
\DeclareMathOperator{\Core} {Core}
\DeclareMathOperator{\energy}{c-energy}
\DeclareMathOperator{\qmaps}{\overline{\Omega}}
\DeclareMathOperator{\hocolim}{colim}
\DeclareMathOperator{\image}{image}
% \DeclareMathOperator{\maps}{\Omega^{2} X}
\DeclareMathOperator{\AG}{AG _{0} (X)}
\DeclareMathOperator{\maps}{{\mathcal {P}}}
\newtheorem {theorem} {Theorem} [section] 
\newtheorem {note} [theorem]{Note} 
\newtheorem {problem} [theorem]{Research Problem} 
\newtheorem{conventions}{Conventions}
\newtheorem {hypothesis} [theorem] {Hypothesis} 
\newtheorem {conjecture} [theorem] {Conjecture}
\newtheorem{lemma}[theorem] {Lemma} 
\newtheorem {claim} [theorem] {Claim}
\newtheorem {question}  [theorem] {Question}
\newtheorem {definition} [theorem] {Definition} 
\newtheorem {proposition}  [theorem]{Proposition} 
\newtheorem {research} {Research Objective}
\newtheorem {corollary}[theorem]  {Corollary} 
\newtheorem {example} [theorem]  {Example}
\newtheorem {axiom}{Axiom}  
\newtheorem {notation}[theorem] {Notation}
\newtheorem {terminology}[theorem] {Terminology}
\newtheorem {remark} [theorem] {Remark}
% \newtheorem {axiom2} {Axiom 2}
% \newtheorem {axiom3} {Axiom 3}
\numberwithin {equation} {section}
% \DeclareMathOperator{\image}{image}
\DeclareMathOperator{\grad}{grad}
\DeclareMathOperator{\obj}{obj}
\DeclareMathOperator{\colim}{colim}
\begin{document}
\author{Yasha Savelyev} 
\address{University of Colima, Bernal Díaz del
Castillo 340,
Col. Villas San Sebastian,
28045, Colima, Colima,
Mexico}
\email{yasha.savelyev@gmail.com}
\title {Research statement}
% Conceptually, the goal here is a construction which functorially translates a Hamiltonian fibre bundle to a certain ``derived vector bundle'' over the same space, with fiber an $A _{\infty} $ category. This ``derived vector bundle'' must remember the continuity of the original bundle. Concretely, 
% We associate a $\infty$-equivalence class of an $A
% _{\infty}$-category $Fuk (P)$ to every Hamiltonian $M$ bundle $M \hookrightarrow P \to X$ over a
% smooth manifold $X$. (For certain $ (M, \omega)$.) This category in a sense
% globalizes the Fukaya category of fibers. In particular this gives a
% new invariant of smooth manifolds by taking  for example $ \mathbb{CP} ^{n-1}
% \hookrightarrow P \to X ^{n}$ to be the projectivization of the complexified tangent bundle.
% The Hochschild
% homology of this $A _{\infty}$-category is expected to be Hutchings' family
% Floer homology of the natural Floer family over the free loop space of $X$,
% determined by $P$. This is verified here for the bundle $P=M
% \times S^1 \to S ^{1}$, which turns out to be a very interesting
% calculation. As a peculiar special case, taking our bundle over $X$ to be
% $ X \times (\mathbb{CP} ^{0} =pt)$, we obtain a linear category $Fuk
% (X)$ whose definition involves no analysis.  Its Hochschild homology is
% expected to be $H (LX, HH (Fuk ( \mathbb{CP} ^{0}))$, i.e. $H (LX, \mathbb{F})$,
% where $ \mathbb{F}$ is the base coefficient field for the Fukaya category. \end {abstract}
\maketitle    
% \noindent Email: yasha.savelyev@gmail.com \\
% phone: +34 657 61 6972
Broadly speaking I work in geometry-topology, 
with much of my work exploring interrelations of 
algebraic topology with differential geometry, and 
in particular symplectic geometry via Floer, 
Gromov-Witten theory, Fukaya categories, Hofer 
geometry and in part dynamical systems.

My recent research is along four distinct lines. 
\begin{enumerate}
   \item In symplectic geometry I have been working on developing the global Fukaya category. Here the basic object is a symplectic 
manifold $(M, \omega) $, $\omega$ a closed 
non-degenerate differential 2-form. And this has 
a fascinating infinite dimensional transformation 
group of Hamiltonian symplectomorphisms $Ham (M,  
\omega) $, with a natural bi-invariant Finsler norm called 
the Hofer metric. One culmination of my program is 
the proof, ~\cite{citeSavelyevGlobalFukayacategoryI},  of one 
sense of a conjecture of Teleman on existence of 
natural ``continuous'' action of the group of Hamiltonian 
symplectomorphisms $Ham (M, \omega)$ on the Fukaya 
category of $(M, \omega ) $. In particular this 
leads to new invariants of smooth manifolds, new 
Chern-Gauss-Weil type curvature bounds for 
singular connections, and other applications, 
~\cite{citeSavelyevGlobalFukayacategoryII}.
\item Also in symplectic geometry I am working on 
development of the theory of locally conformally 
symplectic or $lcs$ manifolds, which suitably 
generalize both symplectic and contact manifolds. 
The idea is to use tools like (pseudo)-holomorphic curves, 
to find ``rigidity phenomena'' in $lcs$ geometry analogous to, for example, the Gromov non-squeezing theorem, and the Weinstein 
``conjecture'' in symplectic/contact geometry. 
Here my main contribution is toward developing the statement and 
partial verification of an analogue of the 
Weinstein conjecture dubbed ``conformal symplectic 
Weinstein conjecture'', ~\cite{citeSavelyevConformalSymplectic}.   This has an intriguing 
connection with the theory of elliptic curves of 
complex algebraic geometry. I have also studied 
the non-squeezing problem in $lcs$ geometry in 
~\cite{citeSavelyevLCSnonsqueezing}. 
\item At the intersection of algebraic topology 
with differential geometry, I have introduced and developed the 
notion of a ``smooth simplicial set'', which in 
particular leads to a solution of a long-standing 
problem of the construction of the universal Chern 
Weil homomorphism for Frechet Lie groups (under 
the condition the group have the homotopy type of 
CW complex), ~\cite{citeSavelyevSmoothSimplicial}. 
\item Finally, I have been fascinated for many 
years by some problems in logic. I have learned the subject 
in large part by email correspondence with the 
philosopher-logician Peter Koellner \footnote{Whose kindness has been astonishing.}.  My contribution here has been the 
first generalization of the (crucially) second 
incompleteness theorem of G\"odel to stably 
computably enumerable formal systems, strictly 
generalizing the classical setting of computably 
enumerable formal systems, ~\cite{citeSavelyevIncompleteness}. 
\end{enumerate}

%
%
% Let $Ham (M,\omega ) $ denote the Frechet Lie group of
% Hamiltonian symplectomorphisms of a monotone
% symplectic manifold $(M, \omega) $.  
% Let $NFuk (M, \omega)$ be the $A
% _{\infty} $-nerve of the Fukaya category $Fuk
% (M, \omega)$, and let $(|\mathbb{S}|, NFuk (M, \omega))$ denote the $NFuk
% (M, \omega)$ component of the ``space  of
% $\infty$-categories'' 
% $|\mathbb{S}| $. Using Floer-Fukaya theory for
% a monotone $(M, \omega)$ we construct a natural up
% to homotopy classifying map 
% \begin{equation*}
%    BHam (M, \omega) \to (|\mathbb{S}|, NFuk (M, \omega)).
% \end{equation*}
% This verifies one sense of a conjecture of
% Teleman on existence of action of $Ham (M ,
% \omega)$ on the Fukaya category of $(M, \omega )
% $. This construction is very closely related to
% the theory of the Seidel homomorphism and the
% quantum characteristic classes of the author, and 
% this map is intended to be the deepest expression of their underlying geometric theory.
% In part II the above map is shown  to be nontrivial by an explicit calculation. In particular, we arrive at a new non-trivial
% ``quantum'' invariant of any smooth manifold,
% which motives the statement of a kind of ``quantum'' Novikov conjecture.
% %  Together with an explicit
% % analytic/algebraic calculation in part II we show that there is an injection of
% % $\mathbb{Z}$ into $\pi _{4} (\mathbb{S} , NFuk
% % (S ^{2} )).$ 
% % Conjecturally the above classifying map(s) are expected to recover 
% %   other ``quantum'' invariants of the Hamiltonian fibration $P$, like the quantum
% %  characteristic classes, or Seidel representation when $X= S ^{2}$, and we
% %  sketch this reconstruction here via  Toen's derived Morita theory.  
% % Given the classifying space $\mathbb{S}$ we may extract from it new invariants of $A _{\infty} $
% % Other methods include the theory of
% % $\infty$-categories of Joyal, particularly after Lurie. In part II we compute the
% % Global Fukaya category for a particular Hamiltonian $S ^{2} $-fibration over $S
% % ^{4} $ and give some applications to Hofer geometry. This part needs a heavier
% % technique in geometric analysis. In a further sequel,  we will use the
% % calculation in part II to deduce non-triviality of $\pi_4$ of a certain ``space"
% % of $A _{\infty} $-categories.
% % One of the author's main personal motivations for developing this construction,
% % is a certain rigidity conjecture in Hofer geometry, which we review here. 
% % .  This is the global asf
% % Fukaya category of $P$.  In particular this gives a local Floer
% % theoretic invariant of smooth manifolds by taking $P$ to be the projectivization
% % of the complexified tangent bundle. 
% % 
% %  We will also describe how to recover from this invariant the known
% % invariants of such a fibration
% % With the latter being more speculative. 
% %  In particular we conclude that these invariants are local in
% % $X$, which is already remarkable for the Seidel representation. 
% % This 
% %  with the $A _{\infty}$-$dg$-nerve of Lurie-Tanaka. Moreover we have
% % symplectic geometry yielding a instant purely algebraic  prediction, as quantum
% % characteristic classes are a ring homomorphism Toen's group isomorphism should extend to a ring isomorphism. 
% % The methods other than the classical methods of Fukaya categories, involve the
% % theory simplicial sets, and more prominently $\infty$-categories of Joyal. To
% % relate global Fukaya category of $P$ to known invariants we need to Toen's
% % derived Morita theory, and the theory of inner fibrations of Joyal and further
% % developed by Lurie. However the majority of the paper is differential geometric
% % in nature. 
% \end {abstract}
%
%  Smooth fibrations over a Lorentz 4-manifold with fiber a Calabi-Yau
% 6-fold are a model for the physical background in string theory. This suggests that there may be some string theory linked mathematical invariants of such a fibration. Indeed, when the structure group of $M \hookrightarrow P \to X$
% can be reduced to the group of Hamiltonian
% symplectomorphisms of $M$, (with its $C ^{\infty}$
% topology)  in which
% case $P$ is called a \emph{Hamiltonian fibration}, 
% there are a couple of basic invariants of such a fibration based on
% Floer-Gromov-Witten theory. One such example is the Seidel representation
% \cite{citeSeidelpi1ofsymplecticautomorphismgroupsandinvertiblesinquantumhomologyrings} and
% the related quantum characteristic classes of the author
% \cite{citeSavelyevQuantumcharacteristicclassesandtheHofermetric}. Related
% invariants are also proposed by Hutchings
% \cite{citeHutchingsFloerhomologyoffamilies.I}.
% Even earlier there is work on
% parametric Gromov-Witten invariants of Hamiltonian fibrations by Le-Ono
% \cite{citeLeOnoParametrizedGromov-Witteninvariantsandtopologyofsymplectomorphismgroups} and Olga Buse \cite{citeOlgaBuse}.
% At the same time, Costello's  theorem
% \cite{citeCostelloTopologicalconformalfieldtheoriesandCalabi-Yaucategories} 
% on reconstruction of topological conformal
% field theories from Calabi-Yau $A _{\infty}$ categories suggests that the
% above invariants must have a similar reconstruction principle. 
% % There is also a much
% % more general approach to this reconstruction via the so called Cobordism
% % Hypothesis of Baez-Dolan, whose proof appears in Lurie's
% % \cite{citeLurieOntheclassificationoftopologicalfieldtheories}.
%
%
% For a given Hamiltonian fibration $P$ as above, the $A _{\infty} $ Fukaya
% categories of the fibers fit into a ``family'', although exactly what this ``family'' should mean is a non-trivial problem by itself, since we must somehow remember the continuity of $P$. Then our basic idea is that associated to a Hamiltonian fibration there
% should be a classifying map from $X$ into an appropriate ``classifying'' space of $A _{\infty} $ categories, from
% which the other invariants can be reconstructed via a version of 
% Toen's derived Morita theory, 
% \cite{citeToenThehomotopytheoryofdg-categoriesandderivedMoritatheory}.  
% This can also be understood to say that $Ham (M ,
% \omega) $ naturally (continuously) acts on $Fuk
% (M)  $, verifying in one sense a conjecture of
% Teleman.  We say more
% on this in Section \ref{sectionHochschild}.  
%
%  This paper will be mostly self-contained, as we will explain many (especially algebraic) concepts used.
% % At
% % present a suitable formulation of such a space does not yet exist.
% % So instead we use the theory of $A _{\infty}
% % $-nerve, to instead construct a classifying map into the space of
% % $\infty$-categories. 
% % ideas of Costello/Lurie on the extended $2d$ topological field
% % theories \cite{}, if
% % one wants to go beyond genus 0.
% % In this paper an $\infty$-category
% % will mean a $\infty$-category, and an $\infty$-groupoid will mean a Kan complex, e.g. the
% % singular set of a topological space. $\infty$-categories, which were initially considered by
% % Boardman and Vogt \cite{citeBoardmanVogtHomotopy}, and intensely studied by Joyal
% %  \cite{citeJoyalNoteson$\infty$-categories}, and Lurie
% %  \cite{citeLurieHighertopostheory},
% % are likely technically the simplest models for
% % $\infty$-categories. 
% % As Kan complexes for us model $\infty$-groupoids, $\infty$-categories maybe
% % considered to be a relaxation of the data of an $\infty$-groupoid, hence Kan
% % complex,
% % whereby some $1$-morphisms may not be
% % invertible. A fair bit of all this theory will be reviewed.
% % A version of this space of $A _{\infty} $ categories appeared in an
% % earlier draft of the paper but as it turns out to be a bit technical
% % we postpone the full
% % construction to a latter note, \cite{}, where it plays a more direct
% % role.  
% % However most of the geometric information in the above mentioned hypothetical
% % functor is contained in the following data. 
% \subsection {A functor from the category of smooth 
% simplices of $X$ to $A _{\infty}$ categories} \label{sec:IntroAfunctor}
% % As mentioned the most important geometric point is the construction of a
% % classifying map $f _{P, \mathcal{D}}: X \to \mathbb{S} $. Here $\mathcal{D}$
% % is certain auxiliary Floer theory perturbation  data, with homotopy class of $f
% % _{P, \mathcal{D}} $ independent of the choice of this data. This is done by
% % first constructing a certain fibration over the singular set of $X$: $X
% % _{\bullet} $. 
% The first basic ingredient for our construction is
% as follows.
% Given $P$ as above, and a choice of
% geometric-analysis theoretic perturbation data
% $\mathcal{D}$, to each smooth simplex
% $$\Sigma: \Delta ^{n} \to X,$$ we associate an $A
% _{\infty} $ category $F (\Sigma)$.  This data
% $\mathcal{D} $ 
% involves certain compatible choices of Hamiltonian
% connections and almost complex structures, similar
% to the kind in the construction of the Seidel
% morphism
% ~\cite{citeSeidelpi1ofsymplecticautomorphismgroupsandinvertiblesinquantumhomologyrings}. 
% This will be discussed in Section \ref{section:dataD}.
%
%
% A key geometric ingredient is the following. Let
% ${\mathcal{R}}_{d}$, $d \geq 2$,  denote
% the  moduli space of Riemann surfaces which are topologically
% disks with $d+1$ punctures on
% the boundary. Let $\overline{\mathcal{R}}_{d}$
% denote the standard compactification. We construct natural, axiomatically determined maps from the universal
% curves \footnote{Technically from certain spaces
% obtained from the universal curves.}  over $\overline{\mathcal{R}}_{d} $, for
% each $d$, into the standard topological
% simplices $\Delta ^{n} $.
%  This topological-combinatorial connection of the universal curves with simplices is new, and is likely of independent interest.
% % and
% % indeed discovery of
% % these maps appears to be new. 
% % In some sense what these maps do is allow us to join
% %  the action of Stasheff's
% % original topological $A _{\infty}$ operad, 
% % and the Fukaya-Stasheff chain $A _{\infty}$ operad. \textcolor{blue}{necessary?} 
% % However analytically there is nothing really new happening, only the underlying
% % geometry is somewhat new. 
%
% Let $X _{\bullet} $ denote the smooth singular set
% of $X$ and let $P$ be as above. Let $$\Delta^{} (X)
% = \Delta /X _{\bullet}$$ denote the category of
% simplices of $X _{\bullet }$, see
% Section \ref{section:simplexCategory} for
% the particulars.
%
% The above data is then extended to a functor
% \begin{equation*} \label{eqF}
%    F _{P}: \Delta (X) \to A _{\infty}-Cat ^{unit},
% \end{equation*}
% with $A _{\infty}-Cat ^{unit}$
% the category of small, unital, $\mathbb{Z} _{2} $-graded $A _{\infty} $ categories over $\mathbb{Q}$, with morphisms strict embeddings, which are moreover
% quasi-equivalences.   
%
% We had mentioned above the ``space of $A
% _{\infty}$ categories''. However, technically it
% will be simpler to work with a related space of $\infty$-categories we denote by
% $|\mathbb{S}| $, discussed
% in Appendix \ref{section:prelimQuasi}.  
% Slightly more
% explicitly, it is the geometric realization
% of a certain Kan complex $\mathbb{S} $  whose
% vertices are $\infty$-categories, and whose edges
% are  equivalences of $\infty$-categories,  called
% categorical equivalences.
%
% The
% connection of the functor $F _{P}$ with $\mathbb{S} $  comes via the
% nerve functor $$N: A _{\infty}-Cat ^{unit} \to
% sSet,$$
% with right-hand side the category of simplicial
% sets.  The functor $N$ is an analogue for
% $A _{\infty} $ categories 
% of the classical nerve construction, which is due
% to Grothendieck.  The $A _{\infty}$ version, first
% suggested in Lurie~\cite{citeLurieHigherAlgebraa}, can
% be considered to be a special case of the more
% general nerve construction for simplicial
% categories,  and was developed by Faonte
% \cite{citeFaonteSimplicialNerve}.
% See also, Tanaka
% \cite{citeLeeAfunctorfromLagrangiancobordismstotheFukayacategory}.
%  What will be crucial for us
% is that
% $N$ takes an $A _{\infty}$ category to a
% $\infty$-category.
%
% One basic reason that working with $\mathbb{S} $
% is useful, is that it 
% will allow us to convert all the algebraic data of the
% functor $F$ above, to the data of a single
% combinatorial-topological 
% object, which we call the global Fukaya
% category $Fuk _{\infty} (P)$, as appearing in the
% title of the paper. More specifically,
% $Fuk _{\infty} (P)$ has the structure of
% $\infty$-fibration \footnote
% {$\infty$-fibration is our name for a
% categorical fibration over a Kan complex.} 
% \begin{equation}
%    \label{eq:Fukinfty}
%    NFuk (M,  \omega) \to  Fuk
% _{\infty} (P) \to X _{\bullet},
% \end{equation}
% described in Section \ref{sec:GlobalFukaya}.
%  This will be
% crucial for computations in Part II
% ~\cite{citeSavelyevGlobalFukayacategoryII}.
%
% Together with suitable invariance, under
% deformation of the perturbation data $\mathcal{D}
% $, the $\infty$-fibration \eqref{eq:Fukinfty}
% leads to the
% following theorem. Denote the
% connected
% component of an element $\mathcal{X} \in
% \mathbb{S} (0) $ (corresponding to an
% $\infty$-category)    by
% $(\mathbb{S}, \mathcal{X}) $, cf. Definition
% \ref{def:connectedComponent}. In what follows $\mathcal{X}
% $  will be $NFuk (M,\omega) $.  
%
% %  , and use the
% % straightening theorem of Lurie, specifically
% % \ref{corollaryStraightening}, to obtain the following:
% \begin{theorem} \label{thmInfinityUniversal}
% For $(M,\omega)$ a monotone symplectic manifold, 
% and $M \hookrightarrow P \to X$  a smooth Hamiltonian
%    fibration over a smooth manifold $X$,
% there is a natural up homotopy map
%   $$cl _{P}: X  \to |(\mathbb{S}, NFuk
%    (M, \omega))|,$$ with $|\cdot|$ still denoting
%    geometric realization. Moreover, this extends to the
%    universal level, so that there is a natural up
%    to homotopy map  $$cl: BHam
% (M,\omega) \to |(\mathbb{S}, NFuk (M, \omega))|, $$   
%   corresponding to the universal Hamiltonian $M$-fibration
% $p: E _{M} \to BHam (M,  \omega) $. This is
%    natural so that $$[cl _{P}]
%    = [cl] \circ [\widetilde{cl} _{P}],$$ for
%    $\widetilde{cl} _{P}: X \to BHam (M,  \omega)
%    $ the classifying map of the Hamiltonian
%    fibration $P$, and $[\cdot] $  denoting the
%    homotopy class.
% \end{theorem}   
% Natural up to homotopy just means that the map is
% natural in the homotopy category of topological
% spaces, with morphisms
% homotopy classes of continuous maps. 
% The name $cl _{P}$  comes from the fact that in a
% certain sense $cl _{P}$ is classifying. In fact it
% classifies the fibration $Fuk _{\infty} (P) \to X 
% _{\bullet}
% $.   The
% proof of this theorem is in Section \ref{sectionFukayaHochschild}.
%
% This theorem can also be interpreted to say that $Ham (M,
% \omega)$ ``continuously acts'' on $NFuk (M) $.
% \begin{remark} For example, working in the
%    category of simplicial sets, suppose we have a
%    simplicial action of $Ham (M, \omega) _{\bullet }
%    $ on $NFuk (M, \omega )$. Then we have an
% induced simplicial map $$BHam (M, \omega) _{\bullet} \to BAut (N Fuk (M,
%    \omega)), $$ with the group of simplicial
%    automorphisms $Aut (N Fuk (M,  \omega) ) $
%    interpreted as a simplicial group,  and where $BG 
%    $  denotes the simplicial  nerve
%    of a simplicial group $G$. And
%    it's easy to see that there is a natural map
%    $$BAut (N Fuk (M,
%    \omega) )  \to (\mathbb{S}, N Fuk (M,
%       \omega)), $$ by
%    the construction of $\mathbb{S} $. 
%    Now starting with  a simplicial map  
%    \begin{equation}
%       \label{eq:BHam}
%      BHam (M, \omega) _{\bullet } \to
%       (\mathbb{S}, N Fuk (M,
%       \omega))
%    \end{equation}
%    we do not generally get an induced homomorphism
%    $$Ham (M,  \omega) _{\bullet} \to Aut (NFuk (M, \omega)
%    ),
%    $$ naturally. But if we  take the based
%    loop space of both sides of \eqref{eq:BHam}
%     we get something   approaching this
%     homomorphism, since $\Omega B Ham (M,
%    \omega) _{\bullet} \simeq Ham (M,  \omega)
%    _{\bullet} $ (simplicial homotopy equivalence).  And since
%     the simplicial $H$-space $$\Omega (\mathbb{S}, NFuk (M,
%     \omega) ) $$
%    in a sense ``extends'' the 
%     group $$Aut (NFuk (M, \omega)),$$   (loops in
%    $\mathbb{S} $ based at $N Fuk (M,  \omega) $
%    are in correspondence with categorical
%    self-equivalences of $NFuk (M, \omega ) $,
%    which generalize simplicial
%    automorphisms, cf. Appendix
%    \ref{appendix.quasi}.)  
%    % For another
%    %  perspective on this see Oh-Tanaka~\cite{cite},
%    %  where instead of using language of the Nerve,
%    %  they use the language of localizations of
%    %  categories. 
%    % It should be noted that a earlier  
%    %  arxiv version (v3) of the present paper indicated a
%    %  direct construction of ``the space'' of $A
%    %  _{\infty}$ categories, without using the
%    %  nerve. This is mostly of conceptual interest
%    %  but is still planned to be done in the
%    %  future.  
% \end{remark}
%  
% The continuous action of $Ham (M,  \omega) $ on
% $NFuk (M, \omega )$  is one interpretation of the existence of a
% ``continuous action'' of $Ham (M, \omega) $ on $Fuk
% (M, \omega) $.  
% And this verifies one sense of a 
% conjecture of Teleman ICM 2014 on the existence of such an action.
% A discrete version of such an action can
% be found in
% Seidel~\cite[Section
% 10c]{citeSeidelFukayacategoriesandPicard-Lefschetztheory}.
% Another interpretation of this ``continous
% action'' appears in the work
% of Oh-Tanaka ~\cite{citeOhTanakaCoherentActions}. There the functor of
% type $F _{P}$ above is converted to a map of spaces by
% means of localizations of categories. This provides an
% alternative algebraic topological
% perspective.
%
% 
% In part II
% \cite{citeSavelyevGlobalFukayacategoryII} of this
% paper, this map is shown to be homotopically non-trivial in a specific example,
% and some possibly unexpected geometric applications of this are
% developed.
% % Then to this $A _{\infty} $ category $F (\Sigma)$ we
% % associate a quasi-category $NF (\Sigma)$, for $N$ the nerve as previously
% % described. Taking the colimit over all simplices of $X _{\bullet} $ we obtain a
% % (co)-Cartesian fibration over $X _{\bullet} $, from which the map to
% % $\mathbb{S}$ is obtained via Lurie's straightening theorem \ref{thm.straightening}.
% % These are analogues, in the world of quasi-categories of Grothendieck
% %  fibrations.
% % We denote by $ \infty-\mathcal {C}at$ the category of
% % quasi-categories, by $A _{\infty}-Cat ^{unit} $ the category of 
% % cohomologically unital $ A
% % _{\infty} $-categories, by $$N: A _{\infty}-Cat ^{unit} \to \infty-\mathcal
% % {C}at$$ the $A _{\infty}$-nerve functor of Lurie and Tanaka, and $\Delta
% % /X _{\bullet} $ the category of simplices of $X _{\bullet} $.
% % \begin{theorem} \label{thm.cocart1} After a choice of auxiliary perturbation
% % data $ \mathcal {D} $, there is a natural $\infty$-category $Fuk _{\infty} (P,
% % \mathcal {D}) = \colim _{\Delta/X _{\bullet} } NF, $
% % and a (co)-Cartesian fibration $$N Fuk (M) \hookrightarrow Fuk _{\infty} (P,
% %  \mathcal {D}) \to X _{\bullet},$$ whose  equivalence class depends only on
% %  the Hamiltonian isomorphism class of $P$. 
% % %  This construction is natural in the
% % %  following sense. There is a universal (co)-Cartesian fibration $\mathcal{P} \to
% % %  \mathcal{A}_{\infty}$ and $Fuk _{\infty} (P, \mathcal{D}) $ is (canonically) isomorphic to the pull-back by
% % %  $f _{P, \mathcal{D}} $ of $\mathcal{\mathcal{P}}$. 
% % \end{theorem}
% % Using this we get using Lurie's straightening theorem for such fibrations:
% % \begin{theorem} \label{thmInfinity}
% % For $(M,\omega)$ a monotone symplectic manifold, a Hamiltonian fibration $M \hookrightarrow P \to X$
% % together with some choice of auxiliary perturbation data $\mathcal{D}$
% % induces 
% % a classifying map $$cl ({P}): X \to (\mathbb{S}, NFuk (M)),$$ % % (More explicitly the
% % % maximal $\infty$-subgroupoid  of the $\infty$-category of $\infty$-categories,
% % % in the component of $NFuk (M)$). 
% % The homotopy class of $cl (P)$ is independent of the
% % choice of $\mathcal{D}$.
% % % where $\Core
% % % (A _{\infty}-Cat _{\bullet} ^{\mathbb{K}}   )$ denotes the maximal Kan subcomplex of the quasi-category
% % % $A _{\infty}-Cat _{\bullet} ^{\mathbb{K}}   $.
% % \end{theorem}
% %
% % However
% % it is a folklore theorem of Lurie (also directly communicated to the author)  that the
% % category of (pre)-triangulated small $A _{\infty} $ categories over $\mathbb{Q}$, with general $A _{\infty}
% % $ functors as morphisms is equivalent to the category of (pre)-triangulated
% % dg-categories over $\mathbb{Q}$, as homotopical categories, that is as
% % categories equipped with a class of morphisms called weak equivalences. In
% % both cases the weak equivalences are just quasi-equivalences. 
% % Moreover they are
% % both equivalent as homotopical categories to 
% %  the
% %  category of stable
% %  $\infty$-categories, with categorical equivalences as weak equivalences, 
% %  with the dg-$A _{\infty}$ nerve functor forming the 
% %  equivalence. See \cite[Chapter 1]{citeLurieHigherAlgebraa} for a discussion of stable
% %  $\infty$-categories. See also 
% % \cite{citeFaonteSimplicialNerve} where it is shown that the nerve functor produces a stable
% % $\infty$-category. Finally an abstract equivalence of the type claimed by Lurie
% % is given in the 
% % differential graded context in
% %  \cite{citeCohnStablekLinear}. 
% % % Dwyer-Kan simplicial localization
% % %   \cite{citeDwyerKanSimpliciallocalizationofCategories} 
% % %   of the category of
% % %   stable
% % %   $\infty$-categories, with categorical equivalences as weak equivalences 
% % Let
% % $\widehat{\mathbb{S}} $ denote the maximal Kan subcomplex of the
% % $\infty$-category of stable  $\infty$-categories.
% % It follows by Lurie's folklore theorem and Toen's theorem above that for a
% % rational (pre)-triangulated $A
% % _{\infty} $ category $C$,
% %   there are  isomorphisms: 
% %   \begin{align} \label {eq.conjecturaltoen1} \Psi ^{T}: \pi
% %      _{i} (\widehat{\mathbb{S}}, NC) \to HH ^{2-i} (C), i >2, \\
% % \label {eq.conjecturaltoen2} \Psi ^{T} _{2} : \pi _{2}  (\widehat{\mathbb{S}},
% % NC) \to HH
% % ^{0} (C) ^{*}.
% % \end{align}
% % % with $(A _{\infty}-Cat _{\bullet}, C  )$ denoting the $C$ component of the
% % % maximal Kan subcomplex of the 
% % % quasi-category $A _{\infty}-Cat _{\bullet},$ described as a simplicial set in Section
% % % \ref{section.algebraic}.
% % \subsubsection {Lifting the construction to (pre)-triangulated $A _{\infty} $ 
% % categories over $\mathbb{Q}$} \label{section:lift}
% %   Note that  the classifying maps $f _{P} $
% %   can lifted to maps $$\widehat {f} _{P}: X _{\bullet}  \to \widehat 
% %   {\mathbb{S}}$$ by working with graded rational (pre)-triangulated $A
% %   _{\infty} $ categories. To do this we must in principle work with Lagrangian submanifolds equipped
% %   with a grading \cite{citeSeidelGradedLag} or so called brane structure, in order to get a $\mathbb{Z}$
% %   grading on the $hom$ complexes. Given this we may work over $\mathbb{Q}$ as
% %   we chose to be in the monotone setting. Finally we must (pre)-triangulate
% %   our $A _{\infty} $ categories $F (\Sigma)$, with the common way to do this is
% %   by taking the associated $A _{\infty} $ category of twisted complexes, see
% %   for instance \cite[Sc. 3l]{citeSeidelFukayacategoriesandPicard-Lefschetztheory}. 
% % \subsubsection {Quantum characteristic classes}
% % There are certain ``quantum characteristic classes" (see
% %   \cite{citeSavelyevQuantumcharacteristicclassesandtheHofermetric}, or
% %   \cite{citeSavelyevBottperiodicityandstablequantumclasses} for a friendlier
% %   discussion (less generality) and explanation of why they are ``characteristic") constructed via
% %   Gromov-Witten theory, and associated to a Hamiltonian fibration $M \hookrightarrow
% %   P \to X$:
% %   \begin{equation*}
% %       \Psi _{P} : \pi _{*} (X, x) \to QH (M) \quad (* \geq 2 \text{ and dropping the
% %       grading on $QH (M)$}).
% %       \end {equation*}
% % Here $QH (M)$ denotes either the quantum cohomology or the quantum homology of
% % $(M, \omega)$.
% % These generalize the Seidel representation
% % \cite{citeSeidel$pi_1$ofsymplecticautomorphismgroupsandinvertiblesinquantumhomologyrings}.
% %       \begin {equation*}
% %      S _{P} : \pi _{2} (X) \to QH (M) ^{*}.   
% %   \end{equation*}
% % \subsubsection {Statement of the conjecture}
% % \begin {conjecture}
% %  Assuming that the Kontsevich conjecture  \cite{citeKontsevichHomologicalalgebraofmirrorsymmetry} 
% %   $$HH ^{*}  (Fuk (M)) \simeq
% %   QH  ^{*} (M), \text{\quad (up to grading conventions)}$$ holds for our symplectic manifold $(M,
% %   \omega)$ we have that: 
% %  \begin{align*}
% %     \Psi _{P} & = \Psi ^{T} \circ (\widehat{f} _{P} )_{*,i} \quad i>2  \\
% %     S _{P} & = \Psi ^{T} _{2} \circ (\widehat{f} _{P} )_{*, 0}, \quad i=2.
% %  \end{align*} 
% %  \end {conjecture}
% \subsection {Towards new invariants and quantum Novikov conjecture} By the above discussion we automatically obtain a
% new  invariant of a Hamiltonian fibration $M \hookrightarrow P \to X$ as the
% homotopy class of the classifying map
% $cl _{P}: X \to |\mathbb{S}|$.  
%
% It may be difficult to get intrinsic motivation for Hamiltonian fibrations for a
% reader outside of symplectic geometry, as a start one may read
% \cite{citeGuilleminLermanEtAlSymplecticfibrationsandmultiplicitydiagrams}.
% However, as
% one particular case we can fiberwise projectivize the complexified tangent bundle:
%  $$P (X) = P (TX \otimes \mathbb{C}),$$ of a
%  smooth manifold $X$. This $P (X)$ in particular has the structure of a
% smooth Hamiltonian fibration with fiber
% $\mathbb{CP} ^{r-1} $ for $r$ the real dimension
% of $X$. In this way we also
%  get a new invariant of a smooth $r$ manifold $X$,
%  given by the homotopy class of the classifying
%  map $$cl _{P (X)}: X \to (\mathbb{S}, NFuk (\mathbb{CP}
%  ^{r-1} )),
%  $$ 
% induced by Theorem \ref{thmInfinityUniversal}.
% %  It should be remarked that
% %  a priori the homotopy class of $cl (P(X)) $ depends on the smooth
% %  structure of the tangent bundle. However it immediately follows from the universal
% %  construction in Section \ref{sectionFukayaHochschild}, (Theorem \ref{theoremTopologicalInvariant}) that only the topological type of the tangent
% %  bundle is detected by the homotopy class $[cl (P(X))] $. 
%
% Recall that Pontryagin classes of a smooth manifold are defined as Chern classes of its complexified tangent bundle. Novikov has shown that rational Pontryagin
% classes are
% topologically invariant. 
% It is then very natural to ask the
%  following, ``quantum'' variant of the Novikov conjecture:
%  \begin{question} Suppose that $f: X \to Y$ is a homeomorphism of smooth
% manifolds. Is $cl _{P(X)} $ homotopic to $cl
%     _{P (Y) } \circ f$? 
% %  What if $f$ is a
% %     homotopy equivalence? This ``quantum
% %     analogue'' of the Novikov conjecture.
%  \end{question}  
% I suspect that the answer is yes,  simply because
% the whole construction involves a kind of integration
% theory, not fantastically far removed from
% Chern-Pontryagin theory, (if we understand
% ``Gromov-Witten counts'' as integration). This would lead to
% further questions however, of how the resulting
% invariants are related to more classical
% topological invariants of smooth manifolds.   And
% a wealth of other questions.
%
% % Why may one expect the answer of yes? One substantial reason is that like the
% % rational Pontryagin
% % classes of $X$ the invariant $[cl (P(X))]$ is
% % based on rational counts, that is the construction 
% % uses $\mathcal{Q} $ coefficients Fukaya category.  
% % \begin{remark}
% %    \label{remark:}
% % We may use more general field but, as far the
% % author understands no extra information is
% % obtained at least in the setting here. \end{remark}
% % 
% % (
% % more general  coefficients, but its very dubious
% % any additional information would be obtained  
% % meaning that we are working with $$ 
% % 
% % (we passed through rational Fukaya categories). But of course the answer
% The answer of ``no'' is possibly even more interesting,
% since it means that our construction gives new
% smooth invariants of manifolds via
% holomorphic curves in symplectic geometry.
% % Indeed the question is so interesting that we feel it is one of the main reasons for existence of this paper.
% \subsection {Hochschild and geometric Hochschild cohomology and homotopy groups
% of $Ham (M, \omega)$} 
% \label{sectionHochschild}
% This section is an excursion, meant to relate our geometric theory with the algebraic derived Morita theory of Toen.
% For an $A _{\infty} $ category $C$ we define $$HH _{geom} ^{2-i} (C) = \pi _{i}
% (\mathbb{S}, NC), \quad i>2.
% $$ The left-hand side is named geometric Hochschild cohomology, the name and notation will be justified shortly. By Theorem \ref{thmInfinityUniversal} above we then get:
%  \begin{theorem} \label{thmGroupHomoIntro} For $ (M,\omega)$ monotone,  there is
%    a natural group homomorphism 
% \begin{equation} \label{eqClassifying1}
%    \pi _{i-1} (Ham (M, \omega), id)  \to
%    HH ^{2-i} _{geom}  (Fuk (M, \omega)), \quad i > 2.     % &  \left( \pi _{i-1} Ham (M, \omega) = \pi _{i} BHam (M, \omega) \right) \to
%    % HH  ^{2-i} (Fuk (M)). \label{eqClassifying2}.
% \end{equation}
% % When $i=2$ the second homomorphism has image in the group of invertibles of $HH
% %    ^{0} (Fuk (M))$.
% \end{theorem}
% $HH ^{*} (Fuk (M, \omega )) $ is known to be
% isomorphic to $QH ^{*} (M) $ in some cases, for
% example in the monotone setting, relevant to us
% here, this is due to
% Sheridan~\cite{citeNickSheridanOntheFukayaCategory}. 
% And so the above morphism, when $i>2$, has the
% same formal form as (a special case of) the author's quantum characteristic classes
% \cite{citeSavelyevQuantumcharacteristicclassesandtheHofermetric},
% taking the form of  homomorphisms:
% $$\Psi: (\pi _{k} (\Omega Ham (M,  \omega), id)
% \simeq \pi _{k+1} (Ham (M,  \omega), id))  \to QH _{2n+k}
% (M,\omega), \quad $$ 
% where $2n=\dim M$. 
% This is provided there is a connection between $HH ^{*} (Fuk (M)) $ and $HH ^{*} _{geom}
% (Fuk (M))$. Such a connection is described
% further below. 
%  This would be the most basic form of the
% ``reconstruction'' that was mentioned in the
% first paragraph of the paper. 
%
%
%
%
% % We of course expect that in those cases one has a coincidence of the above
% % mentioned invariants, with the corresponding invariants in Theorem
% % \ref{thmGroupHomoIntro}, 
% In Part II we calculate  with Hamiltonian $S ^{2} $ fibrations over $S ^{4} $ to
% get:
% \begin{theorem}  \label{thm:pi4} The map
%    $$\left(\pi _{3} (Ham (S ^{2},  \omega), id) =
%    \mathbb{Z} \right) \to \left(HH _{geom} ^{-2} (Fuk (S ^{2} )) = \pi _{4} 
%    (\mathbb{S}, NFuk (S ^{2} )) \right),$$
% determined by \eqref{eqClassifying1} is an injection.
% \end{theorem}
% This has some possibly surprising consequences,
% particularly for the theory of singular
% connections.
% % The fact that there is such (arguably) interesting
% % mathematics associated to smooth Hamiltonian fibrations suggests that perhaps  the
% % space-time model for string theory background should likewise be a Hamiltonian
% % fibration. By this we mean the mathematical models of string theory may make
% % sense in such a context. There is no physical claim here, a priori. 
%  \subsubsection {Geometric Hochschild cohomology and  Toen's derived Morita theory
%  }
%  \label{section:Toen}
% A small disclaimer. $HH _{geom} ^{*} (C)  $ is just a name for an object 
% whose construction is immediate from work of Joyal
% and Lurie, and quite possibly
% appears elsewhere. We claim no originality for this construction.
% What may however be
% interesting is the connection to symplectic geometry that we discover
% in these papers.
% % ,h and perhaps $HH _{geom} ^{*} (C)  $ deserves a more careful
% % study on its own. It appears that some study of
% % related 
%
%
%
% Let us then very briefly indicate the connection of $HH _{geom} ^{*} (C)  $ with Hochschild
% cohomology via Toen's derived Morita theory. Let
% $dg-Cat$  denote the category of differential
% graded categories, a.k.a. dg categories with
% morphisms quasi-equivalences. 
% % Some terminology:  a \emph{component}  of
% % $C \in dg-Cat$  is the set \footnote {It is a set
% % if we are implicitly working with
% % Grothendieck universes, otherwise a
% % class. We ignore the set theoretic foundations here.
% % Toen's own paper makes use of universes,
% % see also ~\cite{citeSavelyevSmoothSimplicial} which treats these issues
% % explicitly in similar context.}
% % of dg categories quasi-equivalent to
% % $C$.  We denote the component of $C$ by
% % $(dg-Cat, C) $. 
% % We describe here a version of a theorem from
% % \cite{citeToenThehomotopytheoryofdg-categoriesandderivedMoritatheory} which will be used in part II, 
% % and some seemingly natural conjectures in connection with this. While we have no very strong
% % evidence for these beyond the result in part II, we feel that they may be central to the subject of global
% % Fukaya category and may help
% % the reader orient algebraically, and so should be stated.
% \begin{theorem} [Corollary 8.4 \cite{citeToenThehomotopytheoryofdg-categoriesandderivedMoritatheory}] \label{thmToen}
%       For a
%     small
%     dg category $C$, (with cohomological grading conventions) there are natural isomorphisms
%   \begin{align} \label {eq.toen1} \pi
%      _{i} (|dg-Cat|, C) \simeq HH ^{2-i} (C), \text{ for }i >2, \\
%      \label {eq.toen2} \pi _{2}  (|dg-Cat|, C)
%      \simeq HH ^{0} (C) ^{\times} ,
% \end{align}
% with $HH ^{0} (C) ^{\times} $ denoting  the multiplicative group of invertible
% elements, and with $|dg-Cat|$ denoting the  geometric
% realization of the nerve of $dg-Cat$, a.k.a. the
% classifying space. Here $C \in |dg-Cat|$ is the element
%    corresponding to $C \in dg-Cat$.
% \end{theorem} 
% % We shall take it for granted that this extends to the case of $A _{\infty} $
% % categories. 
%
% On the other hand the nerve functor $N$ naturally induces a homomorphism, 
% \begin{equation*}
%    N _{*} : \pi
%      _{i} (|dg-Cat|, C)| \to \pi
%      _{i} (|\infty-\mathcal{C}at|, NC) \simeq \pi
%    _{i} (|\mathbb{S}|, NC).
% \end{equation*}
% When $C$ is a $\mathbb{Z}
% $-graded, (pre)-triangulated dg category over
% $\mathbb{Q} $ there are folklore theorems of Lurie
% (personal communication) to the
% effect that this is an isomorphism. 
%
%
% 
%
% Thus, in this case, for $i>2$ $$HH ^{2-i} (C) = \pi _{i}
% (\mathbb{S}, NC) = HH _{geom} ^{2-i} (C),    $$ by our definition.  This
% extends to $\mathbb{Z}$-graded rational (pre)-triangulated $A _{\infty} $
% categories, along
% the lines of Faonte~\cite{citeFaonteFunctors}.
% \begin{remark}
%    \label{remark:triangule}
% As I
% understand, but I am not a Fukaya category expert,  
% these hypotheses apply to at least
% monotone symplectic manifolds, if we
% pre-triangulate the Fukaya categories.  It is important to note however that we do not
% pre-triangulate the Fukaya categories in the
% main construction of the paper, it should be
% possible to do that, following the same
% ideas,   but this possibly loses
% information, and it may make the computation
%    in Part II ~\cite{citeSavelyevGlobalFukayacategoryII} more difficult. Without pre-triangulating the
% connection of $HH ^{*} _{geom}$ and $HH ^{*}$
% appears to be more complicated.  It is also worth
% noting that even if we did identify  $HH ^{*}
% _{geom}$ and $HH ^{*}$ then there is still  a
% hard geometric problem of identifying the
% actual morphisms - the quantum characteristic
%    classes/Seidel morphism and the morphisms from the data of
%    $F$. So all in all the reconstruction 
% is still an open problem.   In
%    Oh-Tanaka~\cite{citeOhTanakaCoherentActions} a
% different approach is taken. Starting with the
% functor   $F$,  or a close  cousin, the authors
% use categorical techniques of localization, which
% allows to avoid introduction of the space
%    $\mathbb{S} $. 
% However, the above problem of identifying the
% morphisms remains.
% \end{remark}
%
% \begin{remark}
% In the case of $i=2$, by Theorem
%    \ref{thmInfinityUniversal}, we have a homomorphism 
% \begin{equation*}
%    \pi _{1} (Ham (M,  \omega), id) \to \pi _{2} (\mathbb{S},
%    N Fuk (M,  \omega) ).    
% \end{equation*}
% So again, if we could again identify $\pi _{2}
%    (\mathbb{S}, N Fuk (M,  \omega) )$ with $(HH
%    ^{0} (Fuk (M,  \omega) )) ^{\times} $ and the latter
%    with $QH _{2n} (M) $, then we would, a priori
%    only in form,
%    recover the Seidel homomorphism \cite{citeSeidelpi1ofsymplecticautomorphismgroupsandinvertiblesinquantumhomologyrings}. 
%    %    %    
% % which ``corresponds to'' the Seidel homomorphism
% % is a bit special since the correspondence in Theorem \ref{thmToen} works
% % differently when $i=2$.
% \end{remark}
% \subsection {Organization}   Section 3 is
% concerned with preliminaries. The crucial
% construction of the system of maps from the
% universal curves to $\Delta^{n} $ is in Section 4.
% Perturbation data $\mathcal{D} $  is constructed
% in Section 5. The main functor $F$ is constructed
% in Section 6. Finally, the global Fukaya category
% in constructed in Sections 7,8. Section 7
% contains the proofs of the main Theorems
% \ref{thmInfinityUniversal},
% \ref{thmGroupHomoIntro}.
% % As will likely be apparent our construction should be highly amenable to combinatorial methods, in
% % particular we plan give in the future a combinatorial construction of the
% % classifying map $cl (X)$ for a combinatorial manifold, which would immediately give
% % combinatorial invariance of $[cl (X)]$. 
% \subsection{Acknowledgements}
% I would like to thank Octav Cornea and Egor Shelukhin for
%  discussions and support,  Kevin Costello and Paul Seidel for interest. Hiro
%  Lee Tanaka for enthusiasm,  generously providing me with
%  an early draft of his thesis and
%  finding a number of misprints in a draft of the paper. Bertrand Toen for
%  explaining to me an outline of the proof of some conjectures and for enthusiastic response. 
% I also thank Jacob Lurie, for feedback on some
% questions. Special thanks to the referees for
% much 
% help in the shaping of this paper.
% % providing the
%   % statement of Claim \ref{lemma.private}.
% The paper was primarily written while I was a CRM-ISM
% postdoctoral fellow, and I am grateful for the wonderful research atmosphere
% provided by CRM-Montreal. It was then substantially revised during my stay at RIMS at Kyoto university, and I also thank the staff and Kaoru Ono for great
% hospitality and discussions. 
% \section {Notations and conventions and large categories}
% We use diagrammatic order for composition of morphisms in the Fukaya
% category, and in $\infty$-categories  so $f \circ g$ means $$\cdot \xrightarrow{f}
% \cdot \xrightarrow{g} \cdot,$$ as reversing order
% for composition in $\infty$-categories is geometrically very confusing, since morphisms are identified with edges of
% simplices. Elsewhere, we use the more common
% Leibnitz functional convention. Although this is somewhat contradictory in practice things should be clear from context. By simplex and notation $\Delta ^{n} $ we will interchangeably mean the topological $n$-simplex and the standard representable $n$-simplex
% as a simplicial set, for the latter we may also write $ \Delta ^{n} _{\bullet}
% $.  
%
% Given a category $C $ the over-category of  an object $c \in C$ is denoted by
% $C/c$. We say that a morphism in $C$ is \emph{over $ c $} exactly if it is a morphism
% in the over-category of $c$.  
%
% Given an $A _{\infty} $ category by the nerve we always mean the $A _{\infty}
% $ nerve $N$, as previously described. 
%
% Some of our $\infty$-categories  are
% ``large'' with proper classes of simplices instead of sets. The standard formal treatment of this
% is to work with Grothendieck universes. This is a widely accepted extension of set theory. 
% % $\mathcal{U}$. In our case this would
% % just mean restricting to $\mathcal{U}$ small $A _{\infty} $ categories over
% % $\mathbb{K}$ with $\mathbb{K} \in \mathcal{U}$.
% We shall not however need to make this
% explicit. 
% % well except for Theorem \ref{thm:pi4} proved in Part II, since we at least need
% % that the set of objects of
% % $Fuk (S ^{2} )$ is contained in $\mathcal{U}$, such a $\mathcal{U}$ exists by
% % Grothendieck axiom of universes.
% For reference one paper that does make this kind of thing explicit is
% \cite{citeToenThehomotopytheoryofdg-categoriesandderivedMoritatheory} also
% previously cited. 
% \section{Preliminaries} \label{section:construction} 
% % We now outline our construction of the space $ \mathcal {V}$ as well as of the
% % map $X \to \mathcal {V}$ induced by a Hamiltonian bundle.  
% % We will have to get a
% % little technical at this point. 
% % \section {Outline of arguments}
% % We are going to  define here a topological space $ \widetilde{\mathcal
% % {V}}$ as a representing space for a very concrete functor $ \mathcal {F}: Top
% % ^{op} \to SemiRing$, that is a contravariant functor from the homotopy category
% % of topological spaces to the category of semi rings. (We want to avoid
% % mentioning group completion for simplicity sake.) The space $ \mathcal {V}$ is
% % itself a semi-ring space in the homotopy category and if  $ \mathcal {V} \simeq
% % \widetilde{ \mathcal {V}}$ then the first pair of research problems would be naturally solved.
% % Recall that a simplicial set $S _{\bullet}$, is a functor $ \mathcal
% % {S} _{\bullet}: \Delta ^{op} \to Set$, where $ \Delta ^{op}$ denotes the
% % opposite category to $\Delta$: the category of combinatorial simplices, 
% % whose objects are non negative integers and morphisms not strictly increasing
% % maps \begin{equation*} \{0 <1 < \ldots < n\} \to \{0 <1 < \ldots < m\}.
% % \end{equation*}
% %  We will denote the objects of $\Delta ^{op}$ by $ [n]$, and $S
% %  _{\bullet} ( [n])$ by $ S _{n}$.
% % And for $X$ a smooth manifold we denote by $X $ the total  smooth
% % singular set of $X$. In other words this is the simplicial set defined by: $X ([n])= X _{n}$
% % is the set of all smooth maps $\Delta ^{n} \to X$ for $\Delta
% % ^{n}$ denoting the standard topological $ n$-simplex. 
% % The map $f _{P}: X \to \mathcal {V}$ associated to a Hamiltonian bundle, will
% % actually be induced by a morphism of simplicial sets $X \to \mathcal
% % {V} _{\bullet}$. 
% \subsection {The simplex category of a smooth
% manifold $X$}  \label{section:simplexCategory}
% Let $\Delta$ denote the category of combinatorial
% simplices, whose objects are totally ordered
% finite sets $[n] = \{0, \ldots,
% n\} $, with $hom _{\Delta } ([n], [m] ) $ the set
% of non-strictly increasing maps   
% \begin{equation*} \{0, \ldots, n\} \to \{0, \ldots,  m\}.
% \end{equation*} 
% A simplicial set $S _{\bullet}$ is a functor $ {S}
% _{\bullet}: \Delta ^{op} \to Set$.  We will
% usually write $S _{\bullet } (n) $   instead of 
%  $S
% _{\bullet } ([n]) $, and this is called the \emph{set of
% $n$-simplices} of $S _{\bullet }$. 
%
% A map of simplicial set $f: A _{\bullet } \to S
% _{\bullet }$ is a natural transformation of the
% corresponding functors. 
%
%  Let $\Delta ^{n} _{\bullet}$  denote the
% simplicial set $\Delta ^{n}
%  _{\bullet} = hom _{\Delta} (\cdot, [n])$. Then we
% have the \emph{category of simplices}  over $S _{\bullet}$, $\Delta/S _{\bullet}$, whose set of objects is the set of natural transformations
% $Nat (\Delta ^{n} _{\bullet}, S _{\bullet})$ and morphisms commutative diagrams
% $$
% \begin{tikzcd}
%  \Delta ^{n} _ {\bullet} \ar [rd] \ar[r] & \Delta ^{m} _{\bullet}
%  \ar[d] \\ & S _{\bullet},  \\
%  \end{tikzcd}
% $$
%  s.t. the natural transformations $\Delta ^{n} _{\bullet} \to \Delta ^{m} _{\bullet}$
%  are induced by maps $[n] \to [m]$. To simplify
%  notation we
%  rename:
% \begin{equation*}
%    \Delta  (S _{\bullet}):= \Delta/S _{\bullet}.
% \end{equation*}
%
% 
% Let $\Delta ^{n}$ denote the standard
% topological $n$-simplex, 
% i.e.  $$\Delta^{n} := \{(x _{1}, \ldots, x _{n})
% \in \mathbb{R} ^{n} \,|\, x _{1} + \ldots + x _{n} \leq 1,
% \text{ and } \forall i:  x _{i} \geq 0  \}. $$ The
% vertices of $\Delta^{n} $ are assumed ordered in
% the standard way $0, \ldots, n$.  
% Let $X$ be a smooth manifold.   We say that
% $\Sigma: \Delta ^{n} \to X$ is a \textbf{\emph{smooth map}}
% if it has an extension  $V \subset \mathbb{R} ^{d}
% \to X$, for $V \supset \Delta ^{n} $ some open set.  
% \begin{definition}\label{def:}
%  We say that a smooth 
% map $\Sigma: \Delta^{n}  \to X$  is
% \textbf{\emph{collared}} if there is a
% neighborhood $U \supset \partial \Delta^{n} $  in
% $\Delta^{n} $, such that $\Sigma| _{U} =
% \Sigma \circ ret$ for $ret: U \to \partial
% \Delta^{n}  $  some smooth retraction.   Here
% smooth means that $ret$ has an extension to a
%    smooth
%  map $V \subset \mathbb{R} ^{d} \to \mathbb{R}
%    ^{d}  $, with $V \supset \partial \Delta^{n} $
%    open in $\mathbb{R} ^{d} $.
%
% \end{definition}
% For $X$ a smooth manifold, define a simplicial set $X _{\bullet }$  by: $$X _{\bullet} (n) := C
% ^{\infty} _{col} (\Delta ^{n}, X),$$  with the right-hand side the set of
% all smooth collared maps
% $\Delta ^{n} \to X$.   It is easy to see that $X _{\bullet} $ is a Kan complex.    The same
% surely holds without the collared condition but
% the proof is more difficult. \footnote {A
% reference is not known to me.}   For simplicity, we
% will work with collared simplices throughout and
% this may no
% longer be mentioned.
%
% In this case the simplex
% category $$\Delta (X) :=\Delta^{}  (X _{\bullet })
% $$   can be 
% elaborated as follows. It is the category with
% objects smooth, collared maps $\Sigma: \Delta ^{n}
% \to X $. A morphism $f$  from $\Sigma _{1}$ to
% $\Sigma _{s}$ is a commutative diagram
% \begin{equation} \label{eq:morphismovercategory} 
% \begin{tikzcd}
% \Delta ^{n} \ar [rd, "\Sigma _{1}" ] \ar[r,
%    "f"]& \Delta ^{m} \ar[d, "\Sigma _{2}"] \\
% & X,
% \end{tikzcd}
% \end{equation}
%  and top horizontal arrow a
% \textbf{\emph{simplicial map}}, also denoted
% $f$, that
%  is an affine map taking vertices to vertices preserving the order.  We say that $\Sigma: \Delta ^{n} \to X$ is \emph{non-degenerate} 
% if it does not fit into a commutative diagram
%  \begin{equation*} 
% \begin{tikzcd}
% \Delta ^{n} \ar [rd, "\Sigma"]  \ar[r] & \Delta ^{m} \ar[d] \\
% & X, 
% \end{tikzcd}
% \end{equation*}
% with $m < n$. 
% % We define recursively $\Sigma$ to be $ndc$ if it is non-degenerate and all of its faces are $ndc$.
%
% We will denote by $Simp (X)$ the full subcategory
% of $\Delta (X)  $, consisting of its non-degenerate objects. The significance   of
% $Simp (X) $   is that the perturbation data in the
% construction of $F$ (as in Section
% \ref{sec:IntroAfunctor}) of the introduction,
% must first be constructed in the context of $Simp
% (X) $, and then formally extended to all
% simplices.  This is necessary to insure
% functoriality of $F$ on $\Delta (X) $.
%
% % and the
% % morphisms to be face maps, that is morphisms
% % induced by monomorphisms in $\Delta$. 
%
% %  which is defined as follows. If $\Delta ^{n}$ 
% % now denotes the standard $n$-simplex: $\Delta ^{n} = hom _{\Delta} (\cdot, [n])$, then $Simp (X)$
% %  is a category with objects natural transformations: $Nat (\Delta ^{n}, X)$ and morphisms commutative diagrams,
% %  \begin{equation*} 
% %  \xymatrix {\Delta ^{n} \ar [rd] \ar[r]& \Delta ^{m} \ar[d] \\
% %  & X},  \\
% %  \end{equation*}
% %  with natural transformation $\Delta ^{n} \to \Delta ^{m}$ induced by maps $[n]
% %  \to [m]$. 
% 
%
%
% % The functor $ \mathcal {F}$ assigns to $X$ the semi-ring of natural
% % isomorphism classes of functors $Simp (X) \to A_{\infty}-Cat$. 
% % The semi-ring structure comes from the semi-ring structure of $
% % A_{\infty}-Cat$ with respect to disjoint union and tensor product. An elementary
% % application of Brown representabilty theorem shows that this functor is representable. 
% % Note that our simplical set $ \mathcal {V} _{\bullet}$ is actually a semi-ring
% % object in the category of simplicial sets. As $ A_{\infty}-Cat$ is itself a semi-ring with respect to
% % disjoint union and tensor product, so there is an induced operations on functors
% % $\Delta ^{n} \to A_{\infty}-Cat$, which is well behaved with respect to natural
% % equivalences. 
% % Consequently the geometric realization $| \mathcal {V}
% % _{\bullet}|$ is a semi-ring space. We have already mentioned that this
% % should be the case on the level of homotopy category, but in our model this is a strict
% % semi-ring structure.
% \subsection{Preliminaries on Riemann surfaces}
% \label{sec:preliminariesRiemannSurfaces}
% Much of this material is adopted from
% the book of
% Seidel~\cite{citeSeidelFukayacategoriesandPicard-Lefschetztheory}.
% Although there are some notation changes, to fit better
% with our goals.  Some other notions like the
% linear ordering, appearing further on, might be
% new, at least in present type of context. 
%
% Let $S'$ be a nodal, connected, simply
% connected, Riemann surface, with each
% smooth component topologically a disk $D ^{2}$  with some
% marked points on the
% boundary, indexed by a finite set $I$.   
%  Removing the marked points we
% obtain a surface $S$ with ends alternatively
% called \emph{punctures}. However, it is
% sometimes simpler to represent $S$ as the original
% compact surface $S'$ with marked points.
% The ends/marked points are labeled by $\{e _{i} \} _{i \in I} $. 
% The nodal points of $S'$ are denoted by $\{n
% _{j} \} _{j \in J} $, again for some index set
% $J$, and these are distinct from the set of marked
% points $\{e _{i} \} _{i \in I} $. 
%
% For each $j \in J$ we have a
% pair  $S _{j, \pm} $ of smooth components of $S$, that are topologically 
% disks with punctures $$\{e _{i} \} _{i \in I _{j, \pm}}  \subset
%  \{e _{i} \} _{i \in I},
%  $$ we explain the signs $\pm$ shortly, for now
%  they just distinguish the pair of components 
%  $S _{j,+}$  and $S _{j,-}$.
% More explicitly, $I _{j,+}$  respectively $I _{j,-}$  are just
% the subsets of $I$ corresponding to the
% punctures on the components $S _{j,+}$
% respectively $S _{j,-}$.
% %   When a pair of smooth components 
% % of $S$ have a node $n _{j} $ in common, we denote the pair 
% %  having the  node in common, are topologically  disks with punctures $$\{e _{i} \} _{i \in I _{j, \pm}}  \subset
% %  \{e _{i} \} _{i \in I} , $$ on the
% %  boundary.
%  If we remove the node $n _{j}$  from $S$ then  $$S _{j, \pm} ^{\circ} := S _{j,
%  \pm} - n _{j}   $$ has an
%  additional puncture $n _{j, \pm} $ called the
%  \textbf{\emph{node end}}.
% %  , with $n _{j} $ called the \emph{node puncture or end}.
%
%
% We distinguish one end of $S$ as the root,
% to be denoted as $e _{0}$. Using the clockwise
% orientation of the boundary of $S$, and if
% $\card (I) = d$  we then have
% an induced ordering, $e _{0}, \ldots, e _{d}$  of
% the punctures.
%
% It is
% sometimes convenient to depict such Riemann surfaces as rooted
% semi-infinite trees, embedded in the plane.
% We do this by assigning a vertex
% to each smooth component as above, a half infinite edge to each marked point, and an edge to each nodal point, as depicted in Figure  \ref{bubblestotrees}. 
% \begin{figure}[h]
%  \includegraphics[width=3in]{Tree.pdf}
% % \scalebox{.9}{\input{Tree.pdf}}
%  \caption {} \label{bubblestotrees}
% \end{figure} 
% We say that $S$ is \textbf{\emph{stable}} if 
% for the associated tree the valency of each vertex is at least 3. 
%
% To make some arguments and notation  cleaner we also introduce a linear ordering on the smooth components of $S$, or
% vertices, by ``order of
% operation'' defined as follows.
% The component with the root semi-infinite edge $e _{0} $ will be called the root vertex denoted by $\omega$. In terms of the associated tree for the surface we have
% a pre-order on vertices given  by the distance to the root vertex, (by giving each edge length
%  1). To get an actual order, first
% isometrically embed the tree in the plane, while
% preserving the clockwise ordering of each
% half-infinite edge, corresponding
% to the ordering of the punctures.
% Then clockwise order vertices equidistant to the root, as in Figure \ref{bubblestotrees}.
% We shall denote by ${\alpha}$ the furthermost
% component from $\omega$, i.e. it is the greatest
% element with respect to our order.  Then ${
%    \beta}$ is the next furthermost component, etc. (Pretending that we can't run out of letters.) 
% Note, that $\alpha$ may
% not be the leftmost component, in fact
% ``leftmost''
% may be ambiguous (dependent on embedding) for vertices not equidistant to
% $\omega$. 
% \begin{remark} \label{remark:}
% This correspondence of letters to the order 
% may seem counterintuitive, but this is motivated
%    by idea that these trees are operadic trees
%    determining composition. More explicitly, later
%    on this is the composition
%    in certain $A _{\infty}$ Fukaya categories. Composition
%    corresponding to furthermost elements from
%    $\omega$   is performed first. Hence $\alpha$
%    corresponds to the first operation we need to perform.
%    Although the operations corresponding
%    to components equidistant
%    from $\omega$ can be performed in any order. 
%     \end{remark}
% %  with  boundary
% % components of $S _{d}$  labeled by $L _{i}$ and  ends at the punctures
% %  identified with strips, see figure \ref{figuresigm}.
% As part of the data,  we may ask for a holomorphic
% diffeomorphism at each $i$'th end, having the name of the end: $$e _{i}:  [0,1] \times (0,
%   \infty) \to S,$$ $i \neq 0$. And at the 0'th puncture we ask for a holomorphic diffeomorphism $$e _{0}: [0,1] \times (-\infty, 0) \to S.
%   $$  
% These charts will be called
%   {\emph{strip end charts}}. 
%
% When $S'$ is not nodal, these strip end charts have the property that $$S - ends:= S - \cup
% _{i \in \{0, \ldots, d\}} \image (e _{i}) $$ is a compact surface with
% corners. 
%
% % Let $ e _{i}^{t}$ denote the  restriction of the maps above to $ [0,1]
% %   \times [t, \infty)$, 
% Let $S _{j, \pm} ^{\circ}, S _{j, \pm}   $ be as above. 
% We further specify the $\pm$
%  distinction so that  $S _{j,-} > S _{j, + }  $ with respect to the linear order above.
%   And we may ask for a similar pair of  strip
% charts \begin{align}
%    &  e ^{}
%   _{j,-}: [0,1] \times (-\infty, 0) \to S ^{\circ}  _{j,-}, \\ 
%    &  e ^{} _{j,+}: [0,1] \times (0,
%   \infty) \to S ^{\circ}  _{j,+}
%  \end{align}  
%  at the $n _{j, {\pm} } $ ends.
%   % Likewise $ e _{j, \pm  }^{t}$ will
%   % denote the restrictions of the above diffeomorphisms to $ [0,1]
%   % \times [t, \infty)$, respectively to $[0,1] \times (-\infty,t]$.
%   The data of all such strip charts for a given
% $S$, will be called a \emph{a strip end
% structure}.  
%
%
%   % (We don't distinguish any ends as positive or negative,
%   % as Seidel \cite{citeSeidelFukayacategoriesandPicard-Lefschetztheory}, as in the present context it would only be cosmetic.)
%
% The moduli space of the Riemann surfaces as above,
% with $\card I=d$, will be denoted by $\overline{\mathcal {R}} _{d}$. 
% (Note that Seidel \cite{citeSeidelFukayacategoriesandPicard-Lefschetztheory} calls our $ \overline{\mathcal
%   {R}} _{d}$ by $ \overline{\mathcal {R}} _{d+1}$.) 
% $\overline{\mathcal {R}} _{d}$ is a real dimension $d-2$  manifold with
% corners. We will also denote by $\mathcal{R} _{d} \subset
% \overline{ \mathcal{R}} $  the
% subspace corresponding to non-nodal surfaces.
%
%
% For $d \geq 2$ let $\rho': {\mathcal {S}} _{d}
%   \to \overline{\mathcal {R}} _{d}$ denote the universal family of the Riemann surfaces
%   $S$, as above. 
% Denote by $$\rho: {\mathcal {S}} ^{\circ}  _{d}
%   \to \overline{\mathcal {R}} _{d},$$ this universal family where the nodal
%   points of the surface fibers have been removed.
%     \begin{notation}
% We denote by $ \mathcal{S} _{d,r} $ and sometimes just by $\mathcal{S}_{r} $ the fiber $\rho ^{-1} (r) $, for $r \in \overline{\mathcal{R}}_{d}$. 
% \end{notation} 
%
%   
% Choose $r$-smooth (varying smoothly with respect
% to  $r$)  families $ \{e _{i,r} \}$, $\{e _{j,\pm,
% r} \}$ of  strip end structures for the entire
%   universal family $ {\mathcal {S}}_d \to \overline{ \mathcal {R}}
%   _{d}$, (note that further on $r$ is suppressed). These choices have to be  consistent with gluing in the natural sense   
% %   A priori ``smooth'' only makes sense over the
% %   interior of $ \mathcal {R} _{d+1}$, but it is possible to make sense of this over the whole of $
% %   \overline{ \mathcal {R}} _{d+1}$, by choosing strip like ends for each family
% %   $ \mathcal {S} _{k} \to \mathcal {R} _{k}$, $2 \leq k \leq d$ in a way that is
% %   consistent with gluing. This
%    as explained in \cite[Section 9g]{citeSeidelFukayacategoriesandPicard-Lefschetztheory}. 
% %    We
% %   will have to reiterate this point later in a few different incarnations. 
%  We will keep track of these systems of choices of strip end structures only implicitly.
%  
%
% %  In the case of $ \overline{\mathcal {R}} _{4}$, we  label the component furthest
% % from $root$ as $\alpha$, the next component  as $\beta$ and
% % the root component as $\gamma$. 
% % % stable
% % % rooted planar tree with 3 interior vertices, encoding the
% % % corresponding composition operation for the $A _{\infty}$ structure on $\Pi
% % % (\Delta ^{n})$. 
% % % See \textcolor{red}{figure here} below. 
% % %  Stable means the valency of each vertex is at least 3. 
% % %  From now on when we say
% % %  \emph{stable tree} we will mean the tree as above. 
% % %  Let us denote by
% % % $\alpha$ the vertex furthest away from the root, $\beta$ the next vertex, and by
% % % $\gamma$ the vertex closest to the root for the usual metric on a graph.
% % Denote by $S _{\alpha}$ the collection of marked points on $\alpha$, likewise
% % with $\beta, \gamma$. This determines
% % a subset $mor (S _{\alpha})$ of the composable sequence $ (m_1, \ldots, m_4)$,
% % and likewise with $\beta, \gamma$, (note that $S _{\gamma}$ could be empty).
% % 
% % % The two sides of the pentagon incident to our corner can
% % %  be distinguished by asking if the corresponding stable Riemann surface with 2
% % %  components has a component with marked points $S  _{\alpha}$. 
% % %  If it does we call the face $\alpha$. 
% \subsubsection{Metric characterization of the moduli
% space} % (fold)
% \label{sec:Metric characterization of the moduli
% space}  It will be helpful to recall the 
% characterization of the moduli space $ {\mathcal
% {R}} _{d}$ and its compactification  $
% \overline{\mathcal {R}} _{d}$,  
%  in terms of hyperbolic metrics   on the punctured
% disks $S _{r}$. 
% The family $ \{{\mathcal {S}} _{d,r} \}
% _{r \in   {\mathcal{R} } _{d}}$ is in a
% bijective correspondence with a  suitably
% universal family $\{\mathcal {M}et _{r}\} = \{\mathcal {M}et _{d,r } \}$ of constant curvature $-1$
% metrics on the disk with $d+1$ punctures on the boundary. Under this correspondence the complex structure on $ \mathcal {S} _{r}$ is just the conformal
% structure induced by $ \mathcal {M}et _{r}$. This is of course classical, to see
% all this use Schwartz reflection to ``double''
% each $\mathcal {S} _{r}$ to a connected genus 0 Riemann surface $\mathcal {D}_{r}$, without boundary. This determines an embedding of $
% {\mathcal {R}} _{d}$ into the moduli space $ {M}
% _{0,d+1}$ of Riemann surfaces that are
% topologically $S ^{2}$  with $d+1$ points removed.  As $d>2$,  by
% the uniformization theorem, each $\mathcal {D} _{r}$ is a quotient of the disk by a subgroup of $PSL (2, \mathbb{R})$, which must also preserve the
% hyperbolic metric. Therefore, $ \mathcal {S} _{r}$ inherits a hyperbolic
% metric, that we call $Met _{r}$. 
%
% The metric point of view gives an illuminating description of the
% compactification  $ \overline{\mathcal{R}}
% _{d}$, for $d \geq 2$. Starting with some $ \mathcal {S} _{r}$ and taking
% $r$ to a boundary stratum, corresponds to 
% some fixed collection of embedded,  disjoint
% geodesics on $ \mathcal {S} _{r}$, with boundary
% in the boundary of $\mathcal{S} _{r} $,   have
% their length shrunk to zero. Each boundary stratum of
% $\overline{\mathcal{R} } _{d}$ is completely determined by such a collection of
% geodesics.
% \subsubsection{Gluing} % (fold)
% \label{sec:Gluing}
% The gluing construction
% (see for example
% \cite{citeSeidelFukayacategoriesandPicard-Lefschetztheory})
% takes a surface $\mathcal{S} _{r} $, $
% r \in \partial \overline {\mathcal
% {R}} _{d}$ and produces a surface with one less node. This gluing  is determined by gluing parameters which we parametrize by $ [0,1)$, assigned to each node. For us $0$ means don't glue, and 1 is meant to correspond to some small value of the gluing parameter used in actual gluing. We will write $d _{\alpha, \beta} $ for the parameter used in the gluing of components $\alpha, \beta $, and likewise with other components.
%
% The gluing construction for parameters in $ [0,1)$ determines an open
% neighborhood called \textbf{\emph{gluing neighborhood}}
%  of the boundary of $ \overline {\mathcal
% {R}} _{d}$.
%  A \textbf{\emph{gluing normal neighborhood}} of the boundary of $\overline {\mathcal
% {R}} _{d}$: will be an open neighborhood (usually
% denoted $N$) of the boundary, deformation retracting to the boundary, contained in the gluing neighborhood.
%
% The gluing construction also induces a kind of thick-thin
% decomposition of the surface, with thin parts conformally identified with $
% [0,1] \times (0, l)$ for $l>0$     determined by the corresponding gluing
% parameter.  This decomposition is not intrinsic,
% as it depends in particular on the choice of the
% family of strip end structures.
% % For $r$ as above  $ \mathcal {S} _{r}$ is obtained
% % by first gluing $\alpha$ component with $\beta$ or $\gamma$ component, with gluing parameter
% %  $d _{\alpha} <1$ and then gluing the $\beta$ component to resulting surface,
% %  with gluing parameter $d _{\beta}<1$. 
%  However, instructively these gluing parameters can be
%  thought of as lengths of geodesic segments, for example $m _{\alpha}$, $m
%  _{\beta}$ in figure \ref{figuregeodsegm}, and the thin parts are closely
%  related to thin parts of thick-thin decomposition in hyperbolic geometry.
% \def\svgwidth{2in}
% \begin{figure} 
% \centering 
% \includegraphics[width=3in]{figuregeodsegm.pdf}
% % \input{figuregeodsegm.pdf}
% \caption {This diagram is only schematic. The embedding into the plane is not meant to be holomorphic or isometric for the natural hyperbolic structure on
% the surface.}
%  \label {figuregeodsegm}
% \end{figure} 
% \subsection{Hamiltonian fibrations} % (fold)
%  \label{sec:HamiltonianBundles}  
% A \textbf{\emph{Hamiltonian fibration}}  is a smooth fiber bundle $$M \hookrightarrow P \to X,$$
% with structure group $\mathcal{G} = \text {Ham}(M, \omega) $
% with its $C ^{\infty} $ Frechet topology. A
% \textbf{\emph{Hamiltonian connection}}, in this
% paper usually denoted by caligraphic letters of
% the kind $\mathcal{A} $,  is just
% an Ehresmann $\mathcal{G} $-invariant connection for a Hamiltonian
% fibration. When $P = M \times X$  is trivial, such
% $\mathcal{A} $ 
% may be specified as a one form valued in $C
% ^{\infty} _{norm} (M) $ - the space of  smooth
% functions on $M$ with mean 0. 
% \section {A system of natural maps from the
% universal curve to $\Delta ^{n} $}
% \label{sec:systemOfmaps}
% We explain here a remarkable connection between the universal curve over
% $\overline{\mathcal{R}} _{d}  $ and the standard
% topological simplices $\Delta
% ^{n} $. This will be used in our construction but may be of independent
% interest.
%
% Let $\Pi (\Delta ^{n}) $ be the small groupoid, whose
% objects set $obj$ is the set of vertices
% of $\Delta ^{n}$. The morphisms set $hom$ is
% the set of affine maps $m: [0,1] \to \Delta ^{n}$, 
% (possibly constant) sending end points to the
% vertices. The source map $$s: hom \to obj$$ is defined by
% $s (m) = m (0)  $ and the target map $$t: hom \to obj$$ is
% defined by $t(m) = m (1)$.  Thus, the set $hom
% (v_1, v _{2}) 
%   $  of
% morphisms between $v_1, v_2 \in \Pi (\Delta ^{n})
% $ consists of a single morphism.  It is the unique
% affine map $m:[0,1] \to \Delta^{n}  $, satisfying
% $m (0)  = v_1$  and $m (1) = v _{2}$. 
% Consequently, the composition maps in $\Pi (\Delta
% ^{n}) $ are forced, by the above uniqueness.   And
% the identity at $v$ is the constant $m: [0,1] \to
% \Delta^{n} $, $m (0) = m (1) =v$.
% \begin{notation}
%    \label{} We denote the composition of
%    $m _{1}, m _{2}  $ by $m _{1} \cdot m _{2}  $.
%    The order is diagrammatic, so that this means
%    the composition
%  \begin{equation*}
%       \cdot \xrightarrow{m _{1}} \cdot
%     \xrightarrow{m _{2}} \cdot.
%    \end{equation*}
%      \end{notation}
% We say that $(m_1, \ldots , m _{d} ) $ is a
% \emph{composable chain} of morphisms $m_i \in
% hom(\Pi (\Delta ^{n} ))$ if $t(m _{i-1}) = s (m_i)
% $, for all $2 \leq i \leq d$ .  For future use let $m _{i-1,i} $ denote the unique morphism from the $i-1$ vertex to the $i$ vertex in $\Delta^{n} $.
%
%
%
%
%
% The goal is to construct a  ``natural'' system of
% maps $${u} (m _{1}, \ldots, m _{d}, n):
% {\mathcal {S}} ^{\circ}  _{d}  \to \Delta ^{n},$$ $d \geq 2$, for each composable chain $(m_1, \ldots ,m _{d} )$. 
% To give a preview, the reason why we work with
% $\mathcal{S} ^{\circ }_{d}$ instead of
% $\mathcal{S} _{d}$, is that we need  
% partial naturality conditions of the Definition
% \ref{def:partialnaturality} below and certain
% gluing relations, which are part of the full
% naturality conditions.  Together these conditions force
% that at the nodes the surface maps to an edge of
% $\Delta ^{n}$. Given continuity, this is of course impossible if the
% edge is non-constant, unless the node itself is removed. 
%
%
%
%
% Let us order the boundary components $0, \ldots, d$ of a surface
% $\mathcal{S} _{r}$, $r \in \mathcal{R} _{d}$, as
% follows.  The ordering is
% clockwise with 0 the component on the
% left of the $e _{0}$ end, given that we have chosen the
% embedding with $e _{0}$ pointing downward.   See
% Figure \ref{fig:sidenumberedsurface}.
% \begin{figure}[h]
%   \includegraphics[width=3in]{labelsidesurface.pdf}
% % \scalebox{.9}{\input{labelsidesurface.pdf}}
%  \caption {} \label{fig:sidenumberedsurface}
% \end{figure} 
%
% % (note we will explain some of the ``standard'' terminology in what follows shortly.)
%
% \begin{definition}\label{def:partialnaturality}
% We say that ${u} (m _{1}, \ldots, m _{d}, n):
% {\mathcal {S}} ^{\circ}  _{d}  \to \Delta ^{n}$ 
%    satisfies \textbf{\emph{partial naturality}}   if
%    the following holds.
% \begin{enumerate} 
%    \item \label{axiom:partial1} The map ${u} (m _{1}, \ldots, m _{d},
%       n)$ is continuous and its restrictions to
%       $$\rho ^{-1} (\mathcal{R} _{d}  \subset
%       \overline {\mathcal{R}}_{d})  $$ is smooth.
% \item \label{axiom:partial2} Let $ 
% {u} (m _{1}, \ldots, m _{d}, n,r)$ denote the restriction of  ${u} (m _{1}, \ldots, m _{d},
% n)$ to $ \mathcal {S} _{r}$, and let $$m_0:=m_1 \cdot
% \ldots \cdot m_d$$ be the composition in $\Pi (\Delta ^{n}) $.
% Then given the strip end charts $e _{k}:
% [0,1] \times (0, \infty) \to \mathcal {S}
% _{r}$ at each $e _{k}$ end,  $1 \leq k \leq d$, 
% $$e _{k} \circ {u} (m _{1}, \ldots, m _{d}, n,r)=
%       m _{k} \circ pr$$ for $pr: [0,1] \times (0, \infty) \to
%       [0,1]  $  the projection.
% \item \label{axiom:partial3} Likewise, in the strip
%    end coordinates $e _{k}:
%    [0,1] \times (- \infty,   0) \to \mathcal {S}
%       _{r}$ at the end $e _{0}$ of $\mathcal{S}
%       _{r}$, 
% ${u} (m _{1}, \ldots, m _{d}, n,r)$ has the form of the projection to $
% [0,1]$ composed with $m _{0}$.  
%
% %       Likewise in the strip coordinates at the node ends 
% %       $e _{j,\pm}
% % ^{1}: [0,1] \times [1, \infty] \to \mathcal {S} _{r}$ ${u} (m _{1}, \ldots, m _{d}, n,r)$
% % has the form of the projection to $[0,1]$ composed with $m _{j} $.
% % \item If $r \in \partial \overline{ \mathcal {R}} _{d}$, at each node $\neta$ of
% % $ \mathcal {S} _{r}$ 
% % \item For $r$ the gluing neighborhood of the boundary, the ``thin'' parts of the
% % surface  identified with $ [0,1] \times [0, l]$, for some variable $l$ are
% % mapped by $
% % {u} (m _{1}, \ldots, m _{d}, n,r)$ to 
% % 
% %  $l _{\alpha} = -log (d
% % _{\alpha}) /{\pi}$, in these coordinates $ret _{r}$ on $S _{\alpha}$, $S _{\beta}$ is the projection to $ [0,1]$ composed
% % with a diffeomorphism onto the lower edge of green, respectively red
% % region. 
%
% \item \label{propertie4} For $0 \leq k \leq d-1$, $r
%    \in \mathcal{R} _{d}$, the $k$'th component  of
%       $\partial \mathcal{S} _{r} $ is mapped to $s
%       (m _{k})$, and the $d$'th component is mapped  by ${u} (m _{1}, \ldots, m _{d}, n,r)$ to $t (m_d)$. 
% % \item 
% \end{enumerate} 
% \end {definition}    
% See Figure \ref{fig:naturality} for an example
% of a map satisfying partial naturality.
% \begin{figure}[h]
%   \includegraphics[width=3in]{naturality.pdf}
% % \scalebox{.4}{\input{naturality.pdf}}
%  \caption {Figure for a map $u (m _{0,1}, m _{1,2},2)
% $ satisfying partial naturality. The edges of the
%    surface labeled $0,1,2$ are mapped to the
%    vertices $0,1,2$ respectively. Colored ends are
%    collapsed onto correspondingly colored edges.} 
%    \label{fig:naturality}
% \end{figure} 
% So far we haven't put any special conditions for
% $r \in \partial \overline{\mathcal{R} } _{d}$,
% having to do with nodes.
% These conditions will be forced by certain
% additional naturality axioms,  which arise from
% various gluing conditions.    After imposing these
% additional axioms, Theorems
% \ref{lemmanaturalmaps}, \ref{thm:naturaltargetdependent} which will follow, state that such a natural system of maps exists and is
% unique up to suitable concordance.
%
%
%
% We start by explaining the natural gluing
% operations that appear.   
% \begin{notation}
%   \label{} In what follows we may
%    interchangeably use the notation $\mathcal{S}
% ^{\circ } (d), \mathcal{S} (d)  $ for $\mathcal{S}
% ^{\circ}_{d}, \mathcal{S} _{d}$. This is to
% avoid notation clash with various notations for
% fibers. 
% \end{notation}
%
% First denote by $ \mathcal {T} (m _{1},
% \ldots, m _{d}, n)$  the space of maps satisfying
% the partial naturality properties above. We have the natural gluing map 
% \begin{equation} \label {eq.composition}  St _{i}: \overline{\mathcal {R}}
% _{s_1} \times \overline {\mathcal {R}} _{s_2} \times [0,1) \to
% \overline{\mathcal {R}} _{s_1+s_2-1},
% \end{equation}
% whose value on $(r, r', \tau)$ is given by gluing
% the surfaces $\mathcal{S} _{r}, \mathcal{S} _{r'} $ at the
% root of $\mathcal{S} _{r}$  and at the $i$'th
% marked point of $\mathcal{S} _{r'}$, with gluing parameter
% $\tau \in [0,1)$, and then associating to this
% glued surface its isomorphism class in $ \overline{\mathcal {R}} _{s_1+s_2-1}$.
%  (When the value of the gluing parameter is 0, this is the composition map in the
%  Stasheff topological $A _{\infty} $ operad).
%
%
% % \begin{equation} \label {eq.composition} St: \overline{\mathcal {R}} _{k} \times
% % (\overline{\mathcal {R}} _{i_1} \times \ldots \overline{\mathcal {R}} _{i_k}) \to \overline{\mathcal {R}} _{\sum _{1 \leq j \leq k} i _{j}},
% % \end{equation}
% % and $u \in \mathcal {T} (m ^{1} _{1}, \ldots, m ^{1} _{i_1}, \ldots, m ^{k}
% % _{1}, \ldots, m ^{k} _{i_k},n)$,
% % \begin{equation*} \mathcal {S} _{\sum _{j} i _{j}} \to \Delta ^{n},  
% % \end{equation*}, 
% Given an element $u \in \mathcal {T} (m _{1},
% \ldots, m _{s_1}, n)$ and an element $$u' \in \mathcal {T} (m'_1, \ldots, m'
% _{i-1}, m_1 \cdot \ldots \cdot m _{s_1}=m' _{i} ,
% m' _{i+1}, \ldots, m' _{s_2}, n),$$ we have a
% naturally induced map $$(u \star _{i}  u')_0: { \mathcal
% {S}} ^{\circ} 
% ({s_1, s _{2}, 0}) \to  \Delta ^{n},$$ where $${ \mathcal
% {S}} ^{\circ} 
% ({s_1, s _{2}, 0}) \to
% \overline{\mathcal {R}} _{s_1} \times \overline{\mathcal {R}} _{s_2} $$ is the 
% pullback of the fibration $${\mathcal{S}} ^{\circ}
% ({d = s _{1} + s _{2}-1  }) \to
% \overline{\mathcal {R}} _{s_1+s_2-1}$$ by $St _{i}| _{\overline{\mathcal {R}}
% _{s_1} \times \overline {\mathcal {R}} _{s_2} \times \{0\}} 
% $. More specifically, the fiber $\mathcal{S} _{r
% _{1}, r _{2}}$  of ${
%    \mathcal
% {S}} ^{\circ} 
% ({s_1, s _{2}, 0})$ over $(r
% _{1}, r _{2})$ is the disjoint union of a pair of
% distinguished (possibly disconnected) subspaces
% ${S} _{1}, {S} _{2}$ 
% identified with $\mathcal{S} _{ r _{1} } (s _{1})  $,
% respectively
% $\mathcal{S} _{r _{2} } (s _{2}) $.  
% So under this identification, we apply $u$ to
% ${S} _{1}$  and
% $u'$ to ${S} _{2}$, and this is the map $(u \star
% _{i}  u')_0 | _{\mathcal{S} _{r _{1}, r _{2}}}$, 
% see Figure  \ref{fig:splitmap}.
% \begin{figure}[h]
%   \includegraphics[width=4in]{splitmap.pdf}
% % \scalebox{1}{\input{splitmap.pdf}}
%  \caption {The green shaded ends are collapsed
%    onto the same edge of $\Delta^{n} $.} \label{fig:splitmap}
% \end{figure} 
%
%
%
% We can extend $(u \star _{i}  u')_0$ to a map
% $$(u  \star _{i} 
% u') _{1}:  { \mathcal {S}}^{\circ} ({s_1,s _{2}, 1})   \to \Delta ^{n},$$ where 
% $${\mathcal {S}}  ^{\circ} ({s_1,s _{2}, 1}) \to \overline{\mathcal {R}} _{s_1} \times \overline{\mathcal {R}}
% _{s_2} \times [0, 1) $$ is the pullback of the
% family  
% $${\mathcal{S}} ^{\circ}
% ({d = s _{1} + s _{2}-1  }) \to
% \overline{\mathcal {R}} _{s_1+s_2-1}$$ by $St _{i}| _{\overline{\mathcal {R}} _{s_1} \times \overline{\mathcal {R}}
% _{s_2} \times [0, 1)} 
% $. 
%
% % restricted over $\overline{\mathcal {R}} _{s_1} \times \overline{\mathcal {R}}
% % _{s_2} \times [0, \epsilon]$. In other words $$\overline{ \mathcal {S}}
% % _{s_1,s _{2}, \epsilon} = St _{i}| _{\overline{\mathcal {R}} _{s_1} \times \overline{\mathcal {R}}
% % _{s_2} \times [0, \epsilon]}   ^{*} \overline{\mathcal{S}} ^{0} _{d}.
% To get this extension we specify  
% \begin{equation}
%    \label{eq:ustaru'taur}
%    (u \star _{i}  u') _{r,r', \tau}: = (u \star _{i}  u') _{1}| _{\mathcal{S}
% _{r,r',\tau} }, 
% \end{equation}
% for $\mathcal{S} _{r,r',\tau} $  the fiber of ${
%    \mathcal {S}} ^{\circ} ({s_1,s _{2}, 1})$ over
% $$(r, r', \tau) \in \overline{\mathcal {R}} _{s_1} \times \overline{\mathcal {R}}
% _{s_2} \times [0, 1), \quad \tau   \in (0,1).
% $$ Recall that the surface $\mathcal{S}_{r,r',\tau} $ glued from $\mathcal{S}_{r}, \mathcal{S} _{r'}  $ has a subdomain which we denote by 
% $thin=thin _{\tau,i} $ that has a determined conformal identification with a strip of the form $[0,1] \times
% (-\phi(\tau), \phi(\tau))$, for some function (not
% explicitly needed)  
% \begin{equation}
%    \label{eq:phiearly}
%   \phi: (0,1] \to (0,\infty),    
% \end{equation} 
% s.t. $\lim _{\tau \mapsto 0} \phi (\tau) = \infty$
% determined by
% the particular parametrization of the gluing operation. 
% % This identification is
% % determined by the strip coordinates structure at the corresponding ends.
% Now $\mathcal{S}
% _{r,r'} - thin $  is the disjoint union of 
% subspaces holomorphically identified with
% the regions $$Reg _{r} \subset
% \mathcal{S} _{r}, Reg _{r'} \subset \mathcal{S}
% _{r'},  $$ so that $Reg _{r} $ is identified with
% $\mathcal{S} _{r} - \image {e _{0}} $, where $e
% _{0}$ is the strip end chart for the root end of $\mathcal{S} _{r}$. And likewise, so that $Reg _{r'} $ is
% identified $\mathcal{S} _{r'} - \image e _{i}$,
% for $e _{i}$ the strip end chart for the $i$'th end of
% $\mathcal{S} _{r'}$. 
%  We then define $(u \star _{i}  u') _{r,r', \tau}$ to coincide
% with $u$, $u'$ on $Reg _{r} $ respectively $Reg _{r'} $, while on $thin$
% in the distinguished coordinates $$[0,1] \times (-\phi (\tau),
% \phi(\tau)),$$ $(u \star _{i}  u') _{r,r', \tau}$ is the
% map given by the projection $$[0,1] \times (-\phi (\tau), \phi(\tau)) \to [0,1]$$ followed by 
% the map $m' _{i} $, similarly to the Axiom
% \ref{axiom:partial2} of partial naturality.  This operation is well-defined
% for all $\tau \in (0,1) $  and so determines the
% needed extension.  
%
% In order state the naturality axioms we need more geometry.
% Let $N$ be a gluing normal neighborhood of the
% boundary $\partial \overline{\mathcal{R}} _{d} $,
% $d \geq 3$.    Recall that $\mathcal{R} _{d}$ has
% dimension $d-2$. Let $S ^{d-3} _{0} \subset N  $ be an embedded sphere of dimension $d-3$, not intersecting  $\partial
% \overline{\mathcal{R}} _{d} $, homotopic to the
% inclusion $(\partial
% \overline{\mathcal{R}} _{d} \simeq S ^{d-3}) \to N $. (A $0$-dimensional
% sphere is understood throughout as a pair of points.) 
%   Let $R ^{d-2} 
% _{0} $ be the dimension $d-2$ ball subdomain of
% $\overline{\mathcal{R}} _{d}$ bounded by $S ^{d-3}
% _{0}  $.  Finally, let $\mathcal{S} _{r} - ends $ denote the compact Riemann surface with
% boundary obtained from $\mathcal{S} _{r} $ by
% removing the ends. Specifically, for $1 \leq i
% \leq d$   we remove the  
% images of the charts $e _{i}: {[0,1] \times (0,
% \infty)} \to \mathcal{S} _{r}   $, and we remove the image of the chart $e
% _{0}: [0,1] \times (-\infty,
% 0) \to \mathcal{S} _{r}$.   
%
% For $r \notin
% \partial \overline{\mathcal{R}}_{d}  $ set $D
% ^{2}_r: = \mathcal{S} _{r} - ends \simeq D ^{2}$.
% And let $$G
% ^{2}  _{r} := (D ^{2} _{r},
% \partial D ^{2}_{r}) \simeq (D ^{2}, \partial D
% ^{2}) $$ denote the
% pair, where $\simeq$ is homeomorphism.
% % Note that the pair $(\mathcal{S} _{r} - ends,
% % \partial (\mathcal{S} _{r} - ends))$ is
% % homeomorphic to the pair $(D ^{2}, \partial D
% % ^{2}) $.
% % associated quotient $(\mathcal{S}
% % _{r} - ends)/\partial (\mathcal{S} _{r} - ends)$
% % is homeomorphic to $S ^{2} $. 
%
% So we get a fiber
% bundle of pairs $$(D ^{2}, \partial D ^{2})
% \hookrightarrow (\widetilde{F}, A)  \to R ^{d-2} _{0},
% $$ with fiber over $r$: $G ^{2} _{r}  $. And where
% $A$ is just the corresponding sub-bundle with
% fiber over $r$: $\partial D ^{2}_{r} $.  
% % Let $$   S ^{2}
% % \hookrightarrow F \to R ^{d-2} _{0}$$ denote the
% % fibration obtained from $\widetilde{F} $ by
% % taking the fiber-wise quotients. That is the fiber
% % of $F$ over $r$ is   $S ^{2} _{r}: = D ^{2}_{r}/
% % \partial D ^{2}_{r} $. 
% % The fibration $F$ has a distinguished section we
% % call $s _{\infty}$  and is defined as follows.  
% %  Let $$\infty _{r} \in S ^{2} _{r} = (D ^{2}_{r}/
% % \partial D ^{2}_{r})   $$ denote the
% %  image of $\partial D ^{2}_{r}$ in
% %  the quotient. 
% %  Then $$s _{\infty} (r) = \infty _{r}, $$
% %  is the corresponding section. 
% %  so that its structure group reduces to $S ^{1} \subset SO (3) $, and so the classifying map % $c: S ^{d-3} \to BSO (3) $ of $F$ factors through $BS ^{1} $. Since $BS ^{1} \simeq \mathbb{CP} ^{\infty} $ is a $K (\mathbb{Z}, 2)$ space $c$ is null-homotopic,  and $F$ is trivial, unless $d=5$.
% %  and since 
% % with $S ^{2}$ in a homotopy natural in $r$ way, (see explanation below.)
% %
% %  We will denote by
% % $u \star u' _{\epsilon} $ the restriction of $u \star u' $ to $$ \overline
% % {\mathcal {S}} _{s_1, s_2, \epsilon} \equiv \overline {\mathcal {R}} _{s_1}
% % \times \overline {\mathcal {R}} _{s_2} \times [0,\epsilon].$$
% % element $$u'' \in \mathcal {T} (m'_1, \ldots, m' _{i-1},
% % m_1, \ldots, m_d, m' _{i+1}, \ldots, m'_k).$$ This gives a 
%
% We are almost ready to state our axioms.
% Let $C _{s} (\Delta ^{n})$ denote the set of composable chains $(m_1, \ldots , m_s)$ of length
% $s \geq 2$ in $\Pi (\Delta ^{n} )$. 
% \begin{definition}\label{def:minimaldimension}
%    For $(m _{1}, \cdots, m _{s})  \in C _{s} (\Delta ^{n}) $ let $D (m_1, \ldots, m
% _{s})$ denote the minimal dimension of a subsimplex of $\Delta ^{n}  $
%       which contains the edges corresponding to the morphisms $m _{i} $.
% \end{definition}
%  A system of maps $\mathcal{U}$ is an element of:
% \begin{equation} \label{eq:productU}
% \prod _{n \in \mathbb{N}} \prod _{s \in \mathbb{N}
%    _{\geq 2} } \prod
% _{(m_1, \ldots , m_s) \in C _{s} (\Delta ^{n} )}
% \mathcal{T} (m_1, \ldots m_s, n).
% \end{equation}
% Given a system $\mathcal{U}$ its projection onto $(n, s, (m_1, \ldots , m_s))$
% component will be denoted by $u (m_1, \ldots ,
% m_s, n)$.   To put this another way, let $$B= \{(m_1, \ldots
% m _{s}, n) | s \in
% \mathbb{N} _{\geq 2}, n \in \mathbb{N},   (m
% _{1}, \ldots, m _{s}) \in   C _{s} (\Delta^{n} )    \},$$
% then set theoretically the product
% \eqref{eq:productU} above is the set of certain
% set maps $\mathcal{U}$ with domain $B$.  Then in
% this language $u (m_1, \ldots , 
% m_s, n) = \mathcal{U} (m_1, \ldots ,
% m_s, n). $ 
% %  such that $\mathcal{U} (m _{1},
% % \ldots,  m _{s},n ) \in \mathcal{T} (m_1, \ldots m_s, n).
% %   $    
% \begin{definition} \label{defNaturality}
% We say that $\mathcal{U}$ 
% is \textbf{\emph{natural}} if it satisfies the following axioms:
% \begin{enumerate}
%   \item  \label{axiom:1} For all $s_1, s _{2}$ and for all $i$ if $m _{i}' = m _{1} \cdot \ldots \cdot m _{s _{1} }   $ then the map 
%   \begin{equation} \label {eq.delta1} 
%      ({u} (m _{1},
% \ldots, m _{s _{1} }, n) \star _{i} u (m'_1,
%      \ldots, m'_{s_2}, n)) _{1}
% \end{equation}
% % induced by $u (m_1, \ldots, m_{s_1})$ and $u (m'_1, \ldots, m' _{i-1}, m_1 \cdot
% % \ldots \cdot m _{s_1}, m' _{i+1}, \ldots, m'_{s_2})$,   as described above,
% coincides with the composition   
% \begin{equation} \label {eq.delta2}
%    {\mathcal {S}} ^{\circ}_{s_1, s _{2},
%    1}
% \xrightarrow{St _{i,*}} {\mathcal {S}}
%    _{s_1+s_2-1} ^{\circ} \xrightarrow{u (m'_1,
% \ldots, m' _{i-1}, m _{1},
% \ldots, m _{s _{1} }, m _{i+1}, \ldots,   m'_{s_2}, n)} \Delta
% ^{n},
% \end{equation}
% for $St _{i,*}$ the bundle map induced by $St _{i}
%       $. ($St _{i,*}$ is the universal map in the push-out
%       square.) 
% % For every $u (m ^{1} _{1},
% % \ldots, m ^{1} _{i_1}, \ldots, m ^{k} _{1}, \ldots, m ^{k} _{i_k},n)$ with $$ cardinality(m
% % ^{1} _{1}, \ldots, m ^{1} _{i_1}, \ldots, m ^{k} _{1}, \ldots, m ^{k} _{i_k}) \leq d$$
% % %  $\sum _{1 \leq
% % % j \leq r}\sum _{l \leq l \leq i _{j} } 1 \leq d$
% %  the induced elements $u _{j}$,
% % $u _{k}$ are $u ( m ^{j} _{1}, \ldots, m ^{j} _{i _{j}}, n)$, respectively $$u
% % (\prod _{1 \leq l \leq i _{1}} m ^{1} _{l}, \ldots, \prod _{1 \leq l \leq i _{k}} m ^{k}
% % _{l},n).$$ 
% % \item The pair of maps \eqref{eq.delta1}, \eqref{eq.delta2} also agree for a
% % sufficiently small non zero $\epsilon$ on the ``thin part'' of ${S} _{s
% % _{1}, s _{2}, \epsilon}$, (see the thick thin decomposition in preliminaries.)
% % \textcolor{blue}{is this really necessary?} 
% \item \label{axiom:naturality2}   Let $f: \Delta ^{n} \to \Delta ^{m}$  be a  simplicial map, which recall is an affine map preserving the orders of the vertices. Then there is an induced functor 
%       $$f: \Pi
%       (\Delta ^{n} ) \to \Pi (\Delta ^{m})
%       $$ 
%       % \begin{equation*}
% % \xymatrix {\Delta ^{n+k}
% %    \ar [d]  ^{\Sigma _{0} }   \ar [r] ^{pr}   & \Delta ^{n} \ar [ld]^{\Sigma _{1} } \\
% %                                     X _{}}, 
% % \end{equation*}
% % for $k>0$ be a morphism 
% %       $\Sigma _{0} \to \Sigma _{1}  $ in $\Delta/X _{\bullet} $
% %       with $\Sigma _{1} $ non-degenerate, 
% and  we ask that
% \begin{equation*}
% f \circ u(m_1, \ldots, m_s, n) = u (f(m_1),
%    \ldots, f (m_s), m). 
% \end{equation*}
% % \begin{equation*} f \circ u
% % (m_1, \ldots, m_s, n-1) = u (f(m_1), \ldots, f (m_s), n).
% % \end{equation*}
% % \item
% % Let
% % \begin{equation*}
% %    pr: \Delta ^{n+k}
% %  \to   \Delta ^{n},
% % \end{equation*}
% % $k>0$ be a simplicial map, then there is an induced functor $$pr: \Pi
% %       (\Delta ^{n+k} ) \to \Pi (\Delta ^{n} )
% %       $$ % \begin{equation*}
% % % \xymatrix {\Delta ^{n+k}
% % %    \ar [d]  ^{\Sigma _{0} }   \ar [r] ^{pr}   & \Delta ^{n} \ar [ld]^{\Sigma _{1} } \\
% % %                                     X _{}}, 
% % % \end{equation*}
% % % for $k>0$ be a morphism 
% % %       $\Sigma _{0} \to \Sigma _{1}  $ in $\Delta/X _{\bullet} $
% % %       with $\Sigma _{1} $ non-degenerate, 
% % and 
% % \begin{equation*}
% % pr \circ u(m_1, \ldots, m_s, n+k) = u (pr(m_1), \ldots, pr (m_s), n). 
% % \end{equation*}
% \item \label{axiom:naturality3} 
%   Let $(m _{1}, \ldots, m _{d}) \in C _{d}
%   (\Delta^{d} ) $ for $d \geq 3$.    Suppose that $D (m_1, \ldots, m _{d} ) = d$ then
% by the Lemma \ref{lemma:inducedmapofpairs} below, $u (m _{1}, \ldots, m _{d}, d)$ induces a map of pairs:
% \begin{equation}  \label{eq:widetildeu}
% \widetilde{u}:  
%    (\widetilde{F}, A  \sqcup \widetilde{F} | _{S _{0}
%    ^{d-3}})   \to (\Delta ^{d}, \partial \Delta
%    ^{d}),
% \end{equation}  
% %  And so $\widetilde{u}$  induces a map with same name:
% % \begin{equation} \label{eqMapInduced}
% %    \widetilde{u}:  (F,  {F} | _{S _{0}
% %    ^{d-3}}) \to  (\Delta
% %       ^{d}, \partial \Delta^{d})  
% % \end{equation}
% and we ask that $\widetilde{u} $ be a homological degree 1 map.
% \end{enumerate}
% \end{definition}   
%
% \begin{remark}
%    \label{remark:}
% Note that when $d=2$ we may still state a natural
%    version of Axiom \ref{axiom:naturality3}, but
%    it would be immediate from the partial naturality
% properties of Definition
%    \ref{def:partialnaturality}.
% \end{remark}
% \begin{lemma}
% \label{lemma:inducedmapofpairs} 
%    % $(R ^{d-2} _{0}   \times S ^{2}, 
%    % S ^{d-3} _{0}   \times S ^{2}) \simeq $
%    Let
% $\mathcal{U} $ satisfy the first pair of axioms
% in the Definition \ref{defNaturality}.
% For $(m _{1}, \cdots, m _{d})  \in C _{s} (\Delta
%    ^{d} ) $, $d \geq 3$,  
% suppose that $D (m_1, \ldots, m _{d}) = d$. Then 
% $\widetilde{u}:=u (m _{1}, \ldots, m _{d}, d)$ is
% a map of pairs:
% \begin{equation*} 
% \widetilde{u}:  
%    (\widetilde{F}, A  \sqcup \widetilde{F} | _{S _{0}
%    ^{d-3}})   \to (\Delta ^{d}, \partial \Delta
%    ^{d}).
% \end{equation*}  
% \end{lemma}
% \begin{proof} 
%    The map $\widetilde{u} $ maps $A$ 
%    to $\partial \Delta^{d} $ by the partial
%    naturality properties 2,3 and 4 in Definition
%    \ref{def:partialnaturality}. In fact $A$ is
%    mapped to the edges of $\Delta^{d} $.  
%   
%    %  For $u \in \mathcal{T} (m_1, \ldots m_s, n)$,
%    %  with $(m, \ldots m _{s}) $ and $n$   general,
%    % set $u _{r}:= u| _{\mathcal{S} _{r}}$.
%    Since
%    $S _{0} ^{d-3}$ is contained $N$, by Axiom
%    \ref{axiom:1}  for each $r \in S _{0}
%    ^{d-3}$, ${u} _{r} = u (m _{1}, \ldots, m
%    _{d},n,r)   $  has the form
%    $(u _{1} \star _{i} u
%    _{2}) _{\tau, r _{1}, r _{2}} $ for some $i$,
%    for some $$u _{1} \in
%    \mathcal {T} (m _{1}, \ldots, m _{s _{1}}, d)$$
%    for some $$u _{2} \in \mathcal {T} (m'_1,
%    \ldots, m' _{i-1}, m_1 \cdot \ldots \cdot m
%    _{s_1}=m' _{i} , m' _{i+1}, \ldots, m' _{s_2},
%    d),$$ and some $\tau, r _{1}, r _{2}$.
%   
%    Now 
%    $s _{1} <d$ and $s _{2} <d$    since
%    $s_1+s_2-1=d$, and since $s _{1}, s _{2} \geq
%    2$. It then follows by this and by Axiom
%    \ref{axiom:naturality2} that the image of $u _{1}$  and $ u_{2}$  is contained in  proper
%    faces of $\Delta^{d} $. Consequently, the image of
%    $\widetilde{u}| _{S _{0} ^{d-3}}  $ is
%    contained in $\partial \Delta^{d} $    and so
%    we are done. 
% \end{proof}
% % Note that $\widetilde{u}$, of
% % \eqref{eqMapInduced}, factors through $S ^{d-2} \wedge S ^{2} \simeq S ^{d}$. To see this, 
% % % fix a trivialization:
% % % \begin{equation*}
% % %    t: F \to S ^{d-2} \times S ^{2},
% % % \end{equation*}
% % note that by the
% % lemma above $\widetilde{u} (s
% % _{\infty} ) \subset \partial \Delta ^{d}   $ for
% % $s _{\infty} $ the section of $S ^{d-2} \times S
% % ^{2}  $ as above. Also if $\infty \in S ^{d-2} $
% % denotes the image of $S ^{d-3} _{0}   $ in the
% % quotient $R ^{d-2} _{0}/S ^{d-3} _{0}$ then the
% % fiber over $\infty$ is likewise mapped to
% % $\partial \Delta ^{d} $ by Lemma \ref{lemma:inducedmapofpairs} above.
% % 
% % can be seen as follows let $\infty \in S ^{2} $ denote the point corresponding to the collapsed boundary.
% % \begin{remark} 
% %     Using an inductive procedure as in the proof
% %     of the following theorem, it should be
% %     possible to show that the final axiom 3 actually
% %     follows by the previous axioms 1,2.
% %    % At least
% %    % it would be rather 
% % \end{remark}
% % \begin{definition}
% %    A system $\mathcal{U}$ will be said to be \emph{\textbf{natural}} if it is $d$-natural
% % for all $d$. \textcolor{blue}{why d?} 
% % \end{definition}
% \begin{theorem} \label{lemmanaturalmaps}
%    A natural system $\mathcal{U}$ exists. 
%    % and is unique up to homotopy
%    % (through natural systems).
% \end{theorem}
%  % Take a neighborhood $N$ of the boundary,
% % corresponding to the gluing parameters lying in $ [0,1)$. That is for $r \in N$,
% % $ \mathcal {S} _{r}$ is obtained by gluing a nodal surface with gluing
% % parameters in $ [0,1)$.
% % For a face $f _{i}$  or corner $r_i$ of $ \overline{\mathcal {R}} _{4}$, we define $N
% % (f_i), N (r _{i})$ to be the gluing normal neighborhoods of $f_i$, respectively
% % $r _{i}$ corresponding to all gluing parameters being in $[0,1)$, (with 0
% % meaning don't glue.)
% %  intersect, the intersection is contained in the epsilon neighborhood of some corner, and so that this neighborhood is contained
% % in the gluing normal neighborhood. Then rescale the pentagon so that
% % $\epsilon=1$.
% % \subsubsection {Setting up for the proof of Proposition \ref{lemmanaturalmaps}}
% We first give a not explicitly constructive proof in all generality, and afterwards describe a partial explicit construction.
% \begin{proof}  [Proof of \ref{lemmanaturalmaps}]
% % The above construction of maps $f _{r} (m _{1}, \ldots, m _{k}  )$ and $ret _{r} $ can be readily extended
% % to all $k,d$ and $r \in \overline{\mathcal{R}} _{d}  $. % The proof is rather tedious, and mostly self evident so the
% % reader may omit at first reading.
% % For each $r \in \overline{\mathcal {R}} _{d}$
% % we have an infinite dimensional family $\mathcal{X} _{r} $ of maps satisfying the (suitably
% % generalized) conditions as the map $ret _{r} $ above, 
% % however this family is clearly contractible. 
% %  Moreover  the associated fibration $
% % \mathcal {X} \to \overline{\mathcal {R}} _{d}$ is easily seen to be a  Serre fibration with
% % contractible fiber. 
% To construct our maps $u (m_1, \ldots, m_s,n)$ we will proceed by 
% induction. 
% When $n=0$ there is nothing to do, as we have unique
% maps for all $s \geq 2$, and they trivially
% satisfy Axioms 1,2,3.  However, we will have to be
% careful with the basis of the induction. 
%    % In case of  Axiom 3, it is
%    % vacuous since there is no chain $(m_1, \ldots m
%    % _{d}) $ 
% %  since $\overline{\mathcal {R}} _{2} = pt$ 
% % in this case for every $ (m_1, m_2)$ we simply fix $u (m_1, m_2, n)$ as
% % any $u \in \mathcal {T} (m_1, m_2,n)$ of the form $u=f \circ
% % ret$, with $ret$ satisfying enumerated conditions above, and $f$ the obvious
% % analogue of $f _{r}$ above. This will vacuously satisfy the first pair of
% % naturality condition. 
% % for $r \leq 2$, (the second naturality condition is not vacuous.)
%
% Given a composable sequence $(m _{1}, \ldots, m _{s}  )$ of morphisms in $\Pi (\Delta
% ^{n}) $, for some $n$, let $D (m_1, \ldots, m_s)$,
% called the  \emph{$D$-number}, 
% denote the least dimension of a non-degenerate
%    subsimplex of $\Delta ^{n} $ that contains
% the edges corresponding to $\{m _{i} \}$. Let $S
%    (N) $ be the following statement: 
% there are maps
% \begin{equation} \label{eqInductionMaps}
% u (m_1, \ldots, m_s, n),
% \end{equation} for all $s \geq 2$ and all $n \leq
% N$ and every composable chain $ (m_1, \ldots, m_s)$ with $D (m_1, \ldots, m_s)
%    = s \leq n$
%    such that  
% %    $$ are of the form $f _{r} \circ ret _{r}$, with $ret _{r}$
% % satisfying (suitably generalized) enumerated conditions above, and 
%    axioms 1,3 are satisfied and such that 
%    \begin{equation}
%       \label{eq:axiom2sigma}
%       \sigma
%    \circ u (m_1, \ldots, m _{s}, n) = 
%     u (\sigma (m_1), \ldots, \sigma(m _{s}), n)
%    \end{equation}
%     for $$\sigma: \Delta ^{n} \to \Delta ^{m}  $$
%    an injective simplicial map. That is we satisfy
%    Axiom \ref{axiom:naturality2} only partially and only for restricted $(m_1, \ldots,
% m _{s} )$ at the moment.  We will also denote by
% $S (N) $ a corresponding collection of maps
% \eqref{eqInductionMaps} with the requisite property.
%  It can be seen directly that $S (N) $ holds for
%    $N=0,1,2,3$. $N=0$ is the trivial case already
%    considered above.   Indeed,   when $N=4$
%    such a construction is given in the following
%    Section \ref{sec:Outline}, and this implies the
%    cases of $N=1,2,3$. 
% We intend to prove:
% \begin{equation*}
%    (N \geq 3) \land S (N) \implies S (N+1),
% \end{equation*}
% moreover the collection of maps $S (N+1) $  can be
% chosen to extend  the collection of maps $S (N) $.
% % We thus suppose that we have constructed maps    
% % $ u (m_1, \ldots, m_s, n)$ for all $s \geq 2$, $n
% %    \leq N$, $N \geq 4$  and all composable chains $
% % (m_1, \ldots, m_s)$ with $D (m_1, \ldots, m_s)
% %    = s$  so that  
% % %    $$ are of the form $f _{r} \circ ret _{r}$, with $ret _{r}$
% % % satisfying (suitably generalized) enumerated conditions above, and 
% %    axioms 1,3 are satisfied and so that 
% % \begin{equation} \label{eqAxiom3rest}
% % \sigma \circ u (m_1, \ldots, m _{s}, n) =  u ( \sigma (m_1), \ldots, \sigma(m _{s}), n),
% % \end{equation}   
% % for $\sigma$ an injective simplicial map.
% %  Set  $u (m_1, \ldots, m_s, n,r) = u (m_1, \ldots, m_s, n)| _{ \mathcal
% % {S} _{s,r}}$.
% %  We first construct
% % maps $ u (m_1, \ldots, m_s, n)$ for all $s \geq 2$, $n \leq N+1$ and all composable chains $
% % (m_1, \ldots, m_s)$ with $D (m_1, \ldots, m_s)
% %    = s$  so that  
% % %    $$ are of the form $f _{r} \circ ret _{r}$, with $ret _{r}$
% % % satisfying (suitably generalized) enumerated conditions above, and 
% %    axioms 1,3 are satisfied and so that 
% % \begin{equation} \label{eqAxiom3rest}
% % \sigma \circ u (m_1, \ldots, m _{s}, n) =  u ( \sigma (m_1), \ldots, \sigma(m _{s}), n),
% % \end{equation}   
% % for $\sigma$ an injective simplicial map.
% % Hence, by Piano induction $\forall N: S (N) $, and we will construct
% % our natural system  from the corresponding data.
% % We could start the induction at $N=2$. 
% %    number of $m _{i}$, $0 \leq i \leq s$, corresponding to distinct
% % edges of $\Delta ^{N+1}$. 
%
% Note that the condition \ref{eq:axiom2sigma} and the maps \eqref{eqInductionMaps}
% uniquely determine 
% \begin{equation} \label{eqMapsInduced}
%  u (m_1, \ldots, m_s, N+1)
% \end{equation} 
%  for all $ \forall  s \geq 2, \forall  (m_1,
%  \ldots, m_s)$ with  $D (m_1, \ldots, m_s) =s \leq N$.
% %    less than the number of edges of
% % $\Delta ^{N}$: $E (\Delta ^{N})$. 
% We need an extension in the case $D (m_1, \ldots, m_s) = s = N+1$. 
%
% By assumption $N+1>3$.  Let then $(m ^{0} _{1}, \ldots, m _{N +1} ^{0}    )$, be a chosen
%    composable sequence with $D (m ^{0}_1, \ldots, m ^{0} _{N+1}  ) = N+1$.
% Then
% %  Suppose that we have 
% % chosen an element 
% % \begin{equation*}
% %    u \in \prod _{s \in \mathbb{N} _{\geq 2}} \left (\prod _{\{(m_1, \ldots, m_s) \in C_s
% %       (N+1)| \, N
% % \leq D (m_1, \ldots, m_s) \leq D' < N + 1\}} \mathcal{T} (m_1, \ldots
% % m_s, N+1) \right )
% % \end{equation*}
% % so that its projections $u (m_1, \ldots , m_s,
% % N+1)$ onto $(s, (m_1, \ldots , m_s))$ component  have the form $$ u (m_1, \ldots , m_s,
% % N+1, r) =
% % f _{r} \circ ret _{r}$$ and satisfying naturality for all $d$. 
% % For all 
% % $$(s, (m' _{1}, \ldots, m'_s)) \in {\mathbb{N} _{\geq 2}} \times {\{(m_1, \ldots, m_s) \in C_s
% %       (N+1)| \,  D (m_1, \ldots, m_s) \leq D'+1)\}}, $$ 
% gluing as in the Axiom \ref{axiom:1} of naturality and the maps \eqref{eqMapsInduced} naturally determine a map
% \begin{equation} \label{eq:subs}
%    {u}= {u} (m ^{0} _1, \ldots,
% m ^{0}  _{N+1} , N+1): Sub _{N+1}   \to \Delta ^{N+1},
% \end{equation}
%  where $Sub _{N+1} = \rho ^{-1} (\partial \overline{\mathcal{R}} _{N+1} )  $, and where as before
%  $$\rho: { \mathcal {S}} _{N+1} ^{\circ}  \to
%  \overline{\mathcal{R}} _{N+1}.  $$ 
%  % $$.
% % Specifically,
% % for some (not all) $r \in \partial \overline{\mathcal{R}} _{s} $ the corresponding fiber
% % $\mathcal{S} _{r} $
% % is a nodal surface with ends $e_0, \ldots , e _{s} $, with smooth components
% % $\mathcal{S} ^{j}  _{r} $ having non-nodal ends a subsequence of $\{e _{s _{k} } \}$ of $e_0, \ldots , e _{s} $
% % s.t. the corresponding subsequence $\{m _{s _{k} } \}$ is contained in
% % $\Pi (\Delta ^{N} )$. Then $Sub _{s} $ is the restriction over the subspace of
% % $\partial \overline{\mathcal{R}} _{s}$ corresponding to such $r$'s. Since we
% % have a constraint only over part of the boundary of $\overline{\mathcal{R}} _{s}  $, there
% % are no topological obstructions, and we
% % can extend $u (m _{1}, \ldots, m _{s}, N+1  )$ arbitrarily over all of
% % $\overline{ \mathcal {S}} _{s} ^{\circ}$, so that $u (m _{1}, \ldots, m _{s},
% % N+1 ) \in  \mathcal {T} (m _{1},
% % \ldots, m _{s}, N+1) $ and so that naturality is satisfied.
% % In this case,
% % %  Suppose that we have 
% % % chosen an element 
% % % \begin{equation*}
% % %    u \in \prod _{s \in \mathbb{N} _{\geq 2}} \left (\prod _{\{(m_1, \ldots, m_s) \in C_s
% % %       (N+1)| \, N
% % % \leq D (m_1, \ldots, m_s) \leq D' < N + 1\}} \mathcal{T} (m_1, \ldots
% % % m_s, N+1) \right )
% % % \end{equation*}
% % % so that its projections $u (m_1, \ldots , m_s,
% % % N+1)$ onto $(s, (m_1, \ldots , m_s))$ component  have the form $$ u (m_1, \ldots , m_s,
% % % N+1, r) =
% % % f _{r} \circ ret _{r}$$ and satisfying naturality for all $d$. 
% % % For all 
% % % $$(s, (m' _{1}, \ldots, m'_s)) \in {\mathbb{N} _{\geq 2}} \times {\{(m_1, \ldots, m_s) \in C_s
% % %       (N+1)| \,  D (m_1, \ldots, m_s) \leq D'+1)\}}, $$ 
% % gluing and the maps from the induction hypothesis, naturally determine
% % a map
% % \begin{equation} \label{eq:subs2}
% % u=u (m_1, \ldots, m_{N+1}, N+1): \overline{ \mathcal {S}} _{N+1} ^{\circ} \vert _{
% %       \partial \overline{\mathcal{R}} _{N+1}}     \to \Delta ^{N+1}.
% % \end{equation}
%
% Let $U$  be a gluing normal neighborhood of
% $\mathcal{R} _{N+1}$. (We previously used the letter
% $N$, but just for this proof we use   the letter
% $U$.)  
% Extend ${u}$ in any way to $\rho ^{-1} (U) $, so
% that Axiom \ref{axiom:1} of naturality
% is satisfied.    We then need to further extend ${u}$ to ${ \mathcal {S}} _{N+1} ^{\circ} $ so that
%  Axiom \ref{axiom:naturality3} of naturality is satisfied.  
%
% Let $S ^{N-2} _{0}  \subset U $
% be an embedded sphere in $U$ not intersecting
% $\partial \overline{\mathcal{R}} _{N+1} $,
% homotopic to the inclusion $\partial
% \overline{\mathcal{R}} _{N+1} \to U$.
% Then $u$ induces a map
% of a pair
% \begin{equation} 
%    g: (\widetilde{F} | _{S _{0}
%    ^{N-2}}, 
% A  | _{S _{0}
%    ^{N-2}})   \to (\partial \Delta ^{N+1} \simeq S ^{N}, loop ),
% \end{equation}
% where $loop$ is a topologically embedded $S ^{1} $ in $\partial \Delta ^{N+1} $
% that is the image of the loop
%  $$\gamma =m ^{0}  _{1} \cdot \ldots \cdot m ^{0} _{N+1} \cdot m  _{s
% (m ^{0} _1), t (m ^{0} _{N+1})} ^{-1},$$ where $\cdot$ is concatenation of paths and the
% order of composition is diagrammatic.
%  The map $g$ is just the restriction of the map
%  $\widetilde{u} $,  
% \eqref{eq:widetildeu}.
% \begin{lemma} The map $g$ is homological degree 1.  \end{lemma}
% \begin{proof}
% As $N > 2$ $loop$ has codimension greater than 1, so that the meaning of
% homological degree is unambiguous, as the pair $(S ^{N}, loop) $ has a well-defined fundamental class by the homology long exact sequence for a pair.
% Moreover, approximating $g$ by a smooth map we may compute the homological
% degree via the smooth
% degree, (denote the approximation still by $g$).
%    That is let $f$ be a $N$-face of $\Delta ^{N+1} $ and $p \in interior (f)$ a
% regular image point of $g$. The homological degree of $g$ is
% then the count of elements of $g
% ^{-1} (p) $ with signs given by whether $dg _{k} $, $k \in g ^{-1} (p) $, is
% orientation preserving or reversing. 
%
%
%    Suppose without loss of generality that the vertices of $f$ are $0, \ldots
% ,N$.  As the degree of $g$ is clearly independent of the choice of $S _{0} ^{N-2}  $ we may assume that $S _{0} ^{N-2}$  is chosen so that for some $\epsilon > 0$:
% \begin{equation} \label{eq:ST1}   St _{1}: \left(R _{0} ^{N-2} \subset  \overline{\mathcal {R}}
% _{N} \right) \times \overline {\mathcal {R}} _{2} \times \{\epsilon\} \to
% \overline{\mathcal {R}} _{N+1}
% \end{equation}
% is an embedding into $S _{0} ^{N-2}  $. Note that   the map
%    $$St _{1}: \left(R _{0} ^{N-2} \subset  \overline{\mathcal {R}}
% _{N} \right) \times \overline {\mathcal {R}} _{2}
%    \times \{0\} \to
% \overline{\mathcal {R}} _{N+1}$$ may be understood
% as a ``face map'' for the polyhedral model of
%    $\mathcal{R} _{N+1}$.  
% We may in addition suppose that $S _{0} ^{N-2} -T  $ is covered by such embeddings corresponding to the various other ``faces'' of $\overline{\mathcal{R}} _{N+1}  $, where the region $T \subset S _{0} ^{N-2} $, is such that $g$ maps $\rho ^{-1} (T)$ into the union of $(N-1)$-faces of $\Delta ^{N+1} $. The image of the map \eqref{eq:ST1} will be denoted by $V$.
%
%
%    Then by the naturality Axiom \ref{axiom:naturality3} the face $f$ is covered by the image of $$\kappa=({u} (m ^{0}  _{1},
% \ldots, m ^{0}  _{N}, N+1) \star _{1}  u (m ^{0} _1 \cdot
%    \ldots \cdot m ^{0} _{N}, m ^{0} _{N+1}, N+1))  _{1}   \vert _{\widetilde{V}},$$  
% %    where $m _{i} $ are the
% % unique morphisms from the vertex $i$ to $i+1$, and 
% where $$\widetilde{V}  = \rho ^{-1} (V).
%    $$ 
% %    and where $V$ is the image of
% % \begin{equation*}   St _{1}: \left(R _{0} ^{N-2} \subset  \overline{\mathcal {R}}
% % _{N} \right) \times \overline {\mathcal {R}} _{2} \times \{1\} \to
% % \overline{\mathcal {R}} _{N+1},
% % \end{equation*}
% % for $R _{0} ^{N-2}$ as in the naturality axioms. 
%
%  
% By construction the smooth
% degree of $g \vert _{\widetilde{V}}$ is the smooth
%    degree of $\kappa $. But then, by the naturality
%    Axiom \ref{axiom:naturality3}, $\kappa$ has smooth degree
% one. And again, by naturality and the assumption on the form of  $S _{0} ^{N-2}
% $, as described above, no other point of $\widetilde{F} | _{S _{0}
%    ^{N-2}}   $ is in $g ^{-1} (p) $. It follows that $g$ is smooth degree one and so is homological degree one.
%
%
% \end{proof}
% \begin{lemma}
%    \label{lemma:degree1}
%   There is a degree one extension: 
% \begin{equation*} 
% \widetilde{g}:  (\widetilde{F}, A  \sqcup \widetilde{F} | _{S _{0}
%    ^{N-2}})   \to (\Delta ^{N+1}, \partial \Delta ^{N+1}),
% \end{equation*}
% with $\widetilde{g} (A) = loop$.
% \end{lemma}
% \begin{proof} This is all very elementary topology
%    so that we will not give exhaustive detail.
%  Note that $\widetilde{F} \simeq B ^{N-1} \times D
% ^{2}$, where $B ^{N-1}$ denotes the unit ball in
%    $\mathbb{R} ^{N-1} $. So to find the necessary degree one extension  it is
% enough to show that given a degree one map 
%    $h: (S ^{d-2} \times D
%    ^{2}, S ^{d-2} \times \partial D ^{2}) \to  (S
%    ^{d}, loop)$, there is a degree one extension
% $$\widetilde{h}: (B ^{d-1} \times D
%    ^{2}, \partial B ^{d-1} \times  D ^{2}) \to  (B
%    ^{d+1}, \partial B ^{d+1}).$$ 
% First note that a degree one map 
%    $$\widetilde{t}:  (B ^{d-1} \times D
%    ^{2}, \partial B ^{d-1} \times  D ^{2}) \to  (B
%    ^{d+1}, \partial B ^{d+1}),$$ 
%  exists by an elementary topology construction.
%    (Start with the slicing diffeomorphism $[0,1]
%    ^{d-1} \times [0,1] ^{2} \to [0,1] ^{d+1} $.) 
%    We may in addition ensure that the restriction
%    of $\widetilde{t} $ to the boundary $S ^{d-2} \times D
%    ^{2}$ 
%    is a map of a pair
%    $$t: (S ^{d-2} \times D
%    ^{2}, S ^{d-2} \times \partial D ^{2}) \to  (S
%    ^{d}, loop). $$ 
%  We may now apply  
%    an analogue of the classical theorem of Hopf, which says that
%    homotopy classes of maps of closed $d$-manifolds to $S
%    ^{d}$ are classified by degree. (We say
%    analogue because we are working with maps of pairs.) In our case
%    Hopf's theorem readily implies that $t$ is
%    homotopic to $h$ through maps of pairs. 
%    Then use classical homotopy extension theorem,
%    to get a homotopy from $\widetilde{t} $ to the
%    needed map $\widetilde{h} $, extending $h$.
%    \end{proof}
%
%
% Continuing with the proof of the theorem, given  $\widetilde{g} $ as in the lemma above,  we may readily construct our extension $u:
%  {\mathcal {S}} _{N+1} ^{\circ} \to \Delta ^{N+1} $ so
% that $$u \in \mathcal {T} (m ^{0}  _{1},
% \ldots, m ^{0} _{N+1}, N+1)$$ and so that the last naturality
% axiom is satisfied. 
%
% Now given any other composable sequence $(m_1,
% \ldots, m _{N+1} )$ with $$D (m_1, \ldots, m
% _{N+1})=N+1$$ let $$\sigma: \Delta ^{N+1} \to
% \Delta ^{N+1}  $$ be the unique simplicial
% bijective map taking $(m_1, \ldots, m _{N+1} )$ to $(m ^{0} _1,
% \ldots, m ^{0}  _{N+1} )$, (it is unique because the action is determined by the
% action on the vertices. 
% ) And we define
% \begin{equation*}
% u (m_1, \ldots, m _{N+1}, N+1): { \mathcal {S}} _{N+1} ^{\circ} \to
%    \Delta ^{N+1}
% \end{equation*}
% by
% $$u (m_1, \ldots, m _{N+1}, N+1 ) = \sigma ^{-1}  \circ u.
% $$ 
% % Observe that $u (m_1, \ldots, m _{N+1}, N+1 )| _{Sub _{N+1} }$ coincides with
% % $$\underline{u} (m_1, \ldots, m _{N+1}): Sub _{N+1} \to \Delta ^{N+1}, $$ 
% % with the latter defined as \eqref{eq:subs}, by the fact that the maps
% % \eqref{eqInductionMaps} satisfy \eqref{eqAxiom3rest}. 
% We then obtain a collection of maps,
% $$\{u
% (m_1, \ldots, m _{s'}, n)\} _{(m _{1}, \ldots, m
% _{s'} ), n \leq N +1, D (m _{1}, \ldots, m _{s'})
% =  s' \leq n}, $$
% and these maps satisfy the condition of
% the statement $S (N+1)
% $ by   the
% construction and the inductive hypothesis.  We
% thus complete the inductive step. Moreover, by
% construction the
% collection of maps $S (N+1) $  extends the
% collection $S (N) $. 
%
% By recursion, we may then define a sequence of
% systems $\{S _{N}\} _{N \geq 0}$,  so that $S (N+1)  $
% extends $S (N) $, for each $N$. 
% We then obtain the total
% collection:
% \begin{align*}
%     \mathcal{U}' =\bigcup _{N} S (N) =   \{u
% (m_1, \ldots, m _{s'}, n) | n \in \mathbb{N},  (m _{1}, \ldots, m
%    _{s'} ) \in C _{s'} (\Delta^{n} ), \\  \text{ such
%    that } D (m _{1}, \ldots, m _{s'}) =  s'\}.
% \end{align*}
%
%
% % for this resulting partial system of maps 
% % $$\{u
% % (m_1, \ldots, m _{s'}, n)\} _{(m _{1}, \ldots, m _{s'} ), n \leq N +1, D (m _{1}, \ldots, m _{s'}) =  s'}, $$ 
% % but this is immediate by construction and the inductive hypothesis.
% % Consequently we complete the induction step.  
%
% It remains to extend the partial system
% $\mathcal{U}' $ above to a full natural
% system $\mathcal{U} $, that is we need to remove restrictions on the $D$-number. % Proceeding inductively we may define maps
% % \begin{equation*}
% % u (m_1, \ldots, m _{N+1}, N+1): \overline{ \mathcal {S}} _{N+1} ^{\circ} \to
% %    \Delta ^{N+1} 
% % \end{equation*}
% % Thus we have defined maps $u (m_1, \ldots, m _{s}, N+1 )$ for all $(m _{1},
% % \ldots, m _{s}  )$ with $s \leq N+1 $.
% Given $(m _{1},
% \ldots, m _{s}  )$, a composable sequence in $\Pi (\Delta ^{n} )$ with $s > n$, we may write $m _{i} = pr \circ
% \widetilde{m}_{i} $ for $(\widetilde{m}_{1}, \ldots, \widetilde{m} _{s}    )$
% a composable sequence in $\Delta ^{s} $ s.t. $D(\widetilde{m}_{1}, \ldots,
% \widetilde{m} _{s}    )= s$, for $pr: \Delta ^{s} \to \Delta ^{n}  $ surjective
% simplicial map. 
% We then define $$u (m_1, \ldots, m _{s},n ) := pr \circ u (\widetilde{m}_{1}, \ldots, \widetilde{m} _{s},s  ).
% $$  We should check that the above is well-defined.
% Let $(\widetilde{m}'_{1}, \ldots, \widetilde{m}'
% _{s})$ be another choice of a composable sequence
% with $pr (\widetilde{m}' _{i}  ) = m _{i} $.
% There is a unique simplicial bijective map $\sigma: \Delta ^{s} \to
% \Delta ^{s}  $ fixing the image $i (\Delta ^{n}
% )$, for $i: \Delta ^{n} \to \Delta ^{s}  $
% inclusion of face, s.t. $pr \circ i = id$, and
% s.t.  $\sigma (\widetilde{m} _{i}  ) =
% \widetilde{m}' _{i}  $.
%
% Then we have 
% \begin{equation*} pr \circ u (\widetilde{m}'_{1}, \ldots, \widetilde{m}' _{s},s  ) =  pr \circ \sigma \circ u (\widetilde{m}_{1}, \ldots, \widetilde{m} _{s},s  ).
% \end{equation*}
% But $pr \circ \sigma = pr $ since $\sigma$ fixes $i (\Delta ^{n} )$. 
% So we obtain that
% \begin{equation*}
% pr \circ u (\widetilde{m}'_{1}, \ldots, \widetilde{m}' _{s},s  ) = pr \circ u (\widetilde{m}_{1}, \ldots, \widetilde{m} _{s},s  ), 
% \end{equation*}
% so that $u (m_1, \ldots, m _{s},n )$ is well-defined.
% So we have constructed our  system of maps satisfying all the axioms of naturality.
% % To prove uniqueness  up to homotopy, note that by our axioms a system of natural
% % maps is completely determined by all the maps $u (m_1, \ldots, m _{s}, n)$ with $$D (m_1,
% % \ldots, m _{s}) =s =n.
% % $$ We then again proceed by induction. 
% % Suppose that we have a pair of natural systems
% % $\mathcal{U} _{1}, \mathcal{U}_{2} $. 
% % Let $H (N) $ be the statement:   there is
% % continuous family of maps $u _{t} (m_1, \ldots, m _{s}, n)$, $t \in
% % [0,1]$, for all $n \leq N$, $N \geq 0$, $(m _{1}, \ldots, m _{s}  )$, with 
% % $$D (m_1,\ldots, m _{s}) =s =n,$$ so that $$u
% % _{t=i} (m_1, \ldots, m _{s}, n) = u _{i} (m_1,
% % \ldots, m _{s}, n) $$ for $i=0,1$, and so that for
% % each $t$ $u _{t} (m_1, \ldots, m _{s}, n)$ satisfy
% % the naturality axioms. As for $S (N) $  we also
% % denote by $H (N) $  the corresponding collection
% % of homotopies.
% % The statement $H (N) $, for $N=0,1,2$ holds
% % trivially.  
% % We prove:
% % \begin{equation*}
% %    (N \geq 2) \land H (N)   \implies H (N+1),
% % \end{equation*}  
% % moreover the collection $H (N+1) $  can be chosen
% % to extend $H (N) $.  This then readily implies our
% % claim.
% % 
% % So suppose  $(N \geq 2) \land H (N)$. 
% % Given some composable sequence $(m_1, \ldots, m _{s})$ in $\Pi (\Delta ^{N+1} )$, with $$D (m_1, \ldots, m _{s}) = s=N+1,$$
% % we have
% % continuous in $t$
% % families of induced (as before) maps:
% % \begin{equation*}
% % u _{t} (m_1, \ldots, m _{s}, N+1): Sub _{s} \to \Delta ^{N+1},
% % \end{equation*}
% % with $Sub _{s} $ as before, and so that $$u _{t=1} (m_1, \ldots, m _{s}, N+1): Sub _{s} \to \Delta ^{N+1}$$ coincides with the map 
% % $$u _{1} (m_1, \ldots, m _{s}, N+1): Sub _{s} \to \Delta ^{N+1}.$$
% % Use the homotopy extension property to get a
% % homotopy:
% % \begin{equation*}
% %    \widetilde{u}  _{t} (m_1, \ldots, m _{N+1}, N+1): {\mathcal{S}} _{d} ^{\circ}   \to \Delta ^{N+1}
% % \end{equation*}
% % of the map ${u}_{0} (m_1, \ldots, m _{N+1}, N+1) $. Now $$\widetilde{u}  _{1}
% % (m_1, \ldots, m _{N+1}, N+1)$$ and $${u}  _{1}
% % (m_1, \ldots, m _{N+1}, N+1)$$ do not necessarily coincide on
% % ${\mathcal{S}} _{d} ^{\circ}$ but they coincide on $Sub _{s} $ since
% % we used homotopy extension, and both these maps are ``degree one'' (that is they both
% % satisfy the last naturality axiom).
% % Then similarly to the proof of Lemma
% % \ref{lemma:degree1}, we may then construct a
% % homotopy relative to $Sub _{s} $, between the pair
% % of maps $\widetilde{u}  _{1}
% % (m_1, \ldots, m _{N+1}, N+1)$, ${u}  _{1}
% % (m_1, \ldots, m _{N+1}, N+1)$.  Taking the
% % composite homotopy we clearly obtain the needed
% % homotopy. As the choice of $(m_1, \ldots m _{s})
% % $  satisfying $D (m_1, \ldots, m _{s}) = s=N+1,$
% % was general, this concludes the proof of the
% % claim.
% % so that the restriction of $u$ to the components of
% % $\mathcal{S} _{r} $ are elements of 
% % restriction of $\overline{\mathcal{R}} _{s}$ 
% %
% % On part
% % because not all nodal degenerations of the surface reduce the ``$D$'' number.
% % \textcolor{blue}{a bit vague} 
% %  By construction these maps restricted to $ \mathcal {S} _{r}$ for $r \in
% %  \partial \overline{\mathcal {R}} _{s}$ will satisfy the enumerated condition above. 
% %  We may then use that $ \mathcal {X}$ above is a Serre fibration
% % with contractible fiber, and classical obstruction theory to extend the map $u
% % (m'_1, \ldots, m' _{s}, N+1)$ to all of $ \overline{\mathcal {S}} ^{\circ}  _{s}$ so
% % that for every $r$, $u(m'_1, \ldots, m'_{s}, N+1,r)$ is of the form $f _{r}
% % \circ ret _{r}
% % $. The first three naturality axioms for the new collection of maps
% % follow by construction, the last axiom is a property of the specific chosen
% % construction of the maps $f _{r} $, and is elementary topology, we omit explicit
% % verification.
% \end{proof} 
% \subsection{Target dependent natural systems} 
% Our natural systems $\mathcal{U}$ can be made
% dependent on particular simplices
% $\Sigma: \Delta ^{d} \to X$.  This is useful
% for proving invariance later on. More
% specifically, for $X$ a smooth manifold, a \textbf{\emph{target
% dependent system}}   $\mathcal{U} (X) $
% is an element of
% \begin{equation} \label{eq:productUX}
% \prod _{\Sigma ^{n} \in \Delta (X)} \prod _{s \in \mathbb{N}
%    _{\geq 2} } \prod
% _{(m_1, \ldots , m_s) \in C _{s} (\Delta ^{n} )}
% \mathcal{T} (m_1, \ldots m_s, n),
% \end{equation} 
% where $\Sigma ^{n}$ is an $n$-simplex for  $n \in
% \mathbb{N} $. 
% As before given a system $\mathcal{U} (X) $ its
% projection onto $(\Sigma ^{n}, s, (m_1, \ldots , m_s))$
% component will be denoted by $u (m_1, \ldots,
% m_s, \Sigma ^{n})$.   The superscript $n$  may be
% omitted, when the degree $n$  is not explicitly needed. 
% \begin{definition}   Similarly to the
%    Definition \ref{defNaturality},  
% we say that $\mathcal{U} (X) $ 
% is \textbf{\emph{natural}} if it satisfies the following axioms:
% \begin{enumerate}
%   \item  \label{axiom:NATX1} For all $\Sigma
%      ^{n}$, $s_1, s _{2}$ and for all $i$ if $m _{i}' = m _{1} \cdot \ldots \cdot m _{s _{1} }   $ then the map 
%   \begin{equation} 
%      ({u} (m _{1},
% \ldots, m _{s _{1} }, \Sigma ^{n}) \star _{i} u (m'_1,
%      \ldots, m'_{s_2}, \Sigma ^{n})) _{1}
% \end{equation}
% % induced by $u (m_1, \ldots, m_{s_1})$ and $u (m'_1, \ldots, m' _{i-1}, m_1 \cdot
% % \ldots \cdot m _{s_1}, m' _{i+1}, \ldots, m'_{s_2})$,   as described above,
% coincides with the composition   
% \begin{equation} 
%    {\mathcal {S}} ^{\circ}_{s_1, s _{2},
%    1}
% \xrightarrow{St _{i,*}} {\mathcal {S}}
%    _{s_1+s_2-1} ^{\circ} \xrightarrow{u (m'_1,
% \ldots, m' _{i-1}, m _{1},
% \ldots, m _{s _{1} }, m _{i+1}, \ldots,
%    m'_{s_2}, \Sigma ^{n})} \Delta
% ^{n},
% \end{equation}
% for $St _{i,*}$ the bundle map induced by $St _{i}
%       $.  
% % For every $u (m ^{1} _{1},
% % \ldots, m ^{1} _{i_1}, \ldots, m ^{k} _{1}, \ldots, m ^{k} _{i_k},n)$ with $$ cardinality(m
% % ^{1} _{1}, \ldots, m ^{1} _{i_1}, \ldots, m ^{k} _{1}, \ldots, m ^{k} _{i_k}) \leq d$$
% % %  $\sum _{1 \leq
% % % j \leq r}\sum _{l \leq l \leq i _{j} } 1 \leq d$
% %  the induced elements $u _{j}$,
% % $u _{k}$ are $u ( m ^{j} _{1}, \ldots, m ^{j} _{i _{j}}, n)$, respectively $$u
% % (\prod _{1 \leq l \leq i _{1}} m ^{1} _{l}, \ldots, \prod _{1 \leq l \leq i _{k}} m ^{k}
% % _{l},n).$$ 
% % \item The pair of maps \eqref{eq.delta1}, \eqref{eq.delta2} also agree for a
% % sufficiently small non zero $\epsilon$ on the ``thin part'' of ${S} _{s
% % _{1}, s _{2}, \epsilon}$, (see the thick thin decomposition in preliminaries.)
% % \textcolor{blue}{is this really necessary?} 
% \item \label{axiom:naturalityX2}   Let $f: \Sigma
% ^{n} \to   \Sigma  ^{m}$  be a
% morphism in $\Delta (X)$, then there is an induced functor 
%       $$f: \Pi
%       (\Delta ^{n} ) \to \Pi (\Delta ^{m}),
%       $$ 
%       % \begin{equation*}
% % \xymatrix {\Delta ^{n+k}
% %    \ar [d]  ^{\Sigma _{0} }   \ar [r] ^{pr}   & \Delta ^{n} \ar [ld]^{\Sigma _{1} } \\
% %                                     X _{}}, 
% % \end{equation*}
% % for $k>0$ be a morphism 
% %       $\Sigma _{0} \to \Sigma _{1}  $ in $\Delta/X _{\bullet} $
% %       with $\Sigma _{1} $ non-degenerate, 
% and we ask that
% \begin{equation*}
% f \circ u(m_1, \ldots, m_s, \Sigma ^{n}) = u (f(m_1),
%    \ldots, f (m_s), \Sigma ^{m}),
% \end{equation*}
% where $f$  on the left is the corresponding
% map $f: \Delta^{n} \to \Delta^{m} $, cf.
%       \eqref{eq:morphismovercategory}. 
%
% % \begin{equation*} f \circ u
% % (m_1, \ldots, m_s, n-1) = u (f(m_1), \ldots, f (m_s), n).
% % \end{equation*}
% % \item
% % Let
% % \begin{equation*}
% %    pr: \Delta ^{n+k}
% %  \to   \Delta ^{n},
% % \end{equation*}
% % $k>0$ be a simplicial map, then there is an induced functor $$pr: \Pi
% %       (\Delta ^{n+k} ) \to \Pi (\Delta ^{n} )
% %       $$ % \begin{equation*}
% % % \xymatrix {\Delta ^{n+k}
% % %    \ar [d]  ^{\Sigma _{0} }   \ar [r] ^{pr}   & \Delta ^{n} \ar [ld]^{\Sigma _{1} } \\
% % %                                     X _{}}, 
% % % \end{equation*}
% % % for $k>0$ be a morphism 
% % %       $\Sigma _{0} \to \Sigma _{1}  $ in $\Delta/X _{\bullet} $
% % %       with $\Sigma _{1} $ non-degenerate, 
% % and 
% % \begin{equation*}
% % pr \circ u(m_1, \ldots, m_s, n+k) = u (pr(m_1), \ldots, pr (m_s), n). 
% % \end{equation*}
% \item \label{axiom:naturalityX3} 
%   Let $(m _{1}, \ldots, m _{d}) \in C _{d}
%   (\Delta^{d} ) $ for $d \geq 3$.    Suppose that
% $D (m_1, \ldots, m _{d} ) = d$. Then, as in
% Definition \ref{defNaturality},
% $u (m _{1}, \ldots, m _{d}, \Sigma ^{d})$ induces a map of pairs:
% \begin{equation}  
% \widetilde{u}:  
%    (\widetilde{F}, A  \sqcup \widetilde{F} | _{S _{0}
%    ^{d-3}})   \to (\Delta ^{d}, \partial \Delta
%    ^{d}),
% \end{equation}  
% %  And so $\widetilde{u}$  induces a map with same name:
% % \begin{equation} \label{eqMapInduced}
% %    \widetilde{u}:  (F,  {F} | _{S _{0}
% %    ^{d-3}}) \to  (\Delta
% %       ^{d}, \partial \Delta^{d})  
% % \end{equation}
% and we ask that $\widetilde{u} $ be a homological degree 1 map.
% \end{enumerate}
% \end{definition}  
% \begin{theorem} \label{thm:naturaltargetdependent}
% Given a smooth manifold $X$, a natural $\mathcal{U}
%    (X) $ exists and is unique up to concordance.
%    (We shall say what this means in the proof.) 
% \end{theorem} 
% \begin{proof}
% Existence is simple. Pick a natural system $\mathcal{U} $ guaranteed by Theorem
%    \ref{lemmanaturalmaps}. This induces a target
%    dependent natural system $\mathcal{U} (X) $ defined by:
% $$\forall \Sigma ^{n}:  u (m _{1}, \ldots, m _{s},
%    \Sigma ^{n}) = u (m _{1}, \ldots, m _{s}, n),  $$
% where the maps $u (m _{1}, \ldots, m _{s}, n)$ correspond to
%    $\mathcal{U} $.  
%   
%   Now uniqueness up to concordance means that
%    given a pair $\mathcal{U} _{1} (X), \, \mathcal{U}
%    _{2} (X) $ of natural systems, there is a
%    natural system $\widetilde{\mathcal{U} } (X
%    \times I) $ whose restriction to $X \times
%    \{i\}$ is $\mathcal{U} _{i} (X)  $, $i=0, 1$.        
% The proof of this is totally analogous to
%    the inductive construction in the
%    proof of Theorem
%    \ref{lemmanaturalmaps}.
% \end{proof}
% 
% \subsection {Outline of an explicit construction} \label{sec:Outline}
% This section is not logically necessary, but in
% order to give the reader more intuition we now
% give a partial explicit construction of a natural
% system $\mathcal{U} $ (not its target dependent
% analogue).    To be more formal, we give an  explicit
% construction of the system satisfying condition $S (4) $, as in the proof
% of Theorem \ref{lemmanaturalmaps}. However, we
% will not formally check all the properties.
% (Although this would be straightforward.) 
% This construction could be in principle extended to all generality but at the cost of much complexity. 
% % , in the case of $\overline{R} _{4}  $ and $\Delta ^{4} $, and which can then be extended without issues (but at the cost
% % of much complexity) to the general
% % case. We then give a not explicitly constructive topological proof in complete
% % generality. As the explicit construction is not used by us, the reader may also
% % skip ahead.
%
% Fix a geometric model for $ \overline{\mathcal {R}} _{d}$, for
% example as the Stasheff associahedra.
% When $d=4$ this is a pentagon.
% Recall that to each corner of $ \overline{ \mathcal {R}}_4 $ we have a uniquely
% associated nodal Riemann surface with 3 components and 5 marked points, one of which is called the
%  root. 
% %  We also have a chosen topological embedding into the plane, which endows
% %  marked points with a cyclic ordering, going clockwise. 
% Recall that we label the root component by $\omega$, the
%   next component by $ \beta$ and the component furthest from root by $\alpha$.
%   (With respect to the linear ordering described earlier.)
% Denote by $M _{\alpha}$ the collection of marked points, different from the root
% $e _{0} $, on $\alpha$, likewise
% with $\beta, \omega$. This determines
% a sub-composable sequence $mor (S _{\alpha})$ of a composable sequence $ (m
% _{1} , \ldots, m _{4})$,
% and likewise with $\beta, \omega$, (note that $M _{\omega}, M _{\beta} $ could be empty).
%  
%
% %  The $, is contained in $[0,4] \times [0,1]$ and it's basic
% %  structure is as depicted in figure \textcolor{red}{figure here}. 
% % The top edge is the domain of concatenation  
% % We need to specify the corners on the right side. Let $L _{\alpha}$, $L
% % _{\beta}$, $L _{\gamma}$ denote the cardinality of $S _{\alpha}, S
% % _{\beta}, S _{\gamma}$ respectively, and 
% % and $$L (I _{\alpha}) = L _{\gamma} + L _{\beta} +L _{\alpha} -(L _{\alpha}-1)
% % (1-d _{\alpha}).$$ 
% % Assuming the top left corner has coordinates $ (0,0)$, the
% % second vertex from the top has coordinates $$ (L (I _{\alpha}), I _{\alpha}).$$
% % Let $I _{\beta} = I _{\alpha}+(1- I _{\alpha})(1 - d _{\beta})/2$, and $$L (I
% % _{\beta}) = L (I _{\alpha}) - 2(L (I _{\alpha}) - L _{\gamma} -1)/(1-I
% % _{\alpha}) (I _{\beta} - I _{\alpha}).$$ The third vertex from the top then has
% % coordinates $ (L (I _{\beta}), I _{\beta})$. 
% Let $r$ be in the gluing normal neighborhood of some corner,
% corresponding to
% non-zero gluing parameters $d _{\alpha, \beta} $, $d _{\beta, \omega} $. 
% We now construct a map
% $$f _{r} = f _{r} (m _{1}, \ldots, m _{4}): [0,4] \times [0,1] \to \Delta ^{4}$$
% % It is completely determined by the stable Riemann surface associated to our
% % corner, the strip coordinate charts  and the gluing parameters. 
% % We could in principle construct for all possibilities
% % at once, but to 
% In what follows by \emph{concatenation} of a collection of paths we mean their
% product in the Moore path category of $\Delta ^{4} $, the notation for
% composition will be assumed to be diagrammatic.
% This is the category with
% objects: points of $\Delta ^{4} $, and morphisms from $x _{0} $ to $x _{1} $:
% continuous paths $[0,T] \to \Delta ^{4} $, $T>0$, between $x _{0}, x _{1}  $,
% with composition the natural concatenation of
% paths. Note that this is quite different from our previously defined groupoid $\Pi (\Delta ^{4} )$.
%
% For a morphism $m$ in $\Pi (\Delta ^{4})$
% let $s(m)$ and $t({m})$ denote the source respectively target of $m$. Let $H
% ^{m}: \Delta ^{4} \times [0,1] \to \Delta ^{4}$ denote  the natural deformation
% retraction of $\Delta ^{4}$ onto the edge determined by $s (m), t (m)$, with
% time $1$ map the orthogonal affine projection onto this edge (for the standard
% metric on $\Delta ^{n} $).
%  Set $H
% ^{m} _{\tau} = H ^{m}| _{\Delta ^{n} \times \{\tau\}}$. Next, for a general
% piece-wise affine path $p: [0,T] \to \Delta ^{4}$, with end points $s (m), t
% (m)$, we have a homotopy $H ^{m} _{\tau}  \circ
% p$, $ \tau \in [0,1]$, from $p$ to a path $$ \widetilde{p}: [0,T] \to \Delta
% ^{4}$$ with image in the edge determined by $m$.
% Let $D (p, \tau)$, $ \tau \in [0,1]$, be the concatenation of $H ^{m} _{\tau}  \circ
% p$ with the homotopy $G
% _{\tau}$, $\tau \in [0,1]$, of paths with fixed end points, from $ \widetilde{p}$ to the map $$
% \widetilde{m}: [0,T] \to \Delta ^{4}$$ linearly parametrizing the edge determined by $m$. This second homotopy
% $G _{\tau}$ can be chosen in a way that depends only on $
% \widetilde{p}$. This can be done explicitly, using piece-wise linearity of $p$.
%
%  The map $f ^{t} _{r}$ from
% the $y=t$ slice $ [0,4] \times \{t\}$ is constructed 
% as follows.  Set $$I _{\alpha}= (1 -d
% _{\alpha, \beta})/2,$$ % set $$sc (t) = \frac{L (t) -L  _{\gamma} - L _{\beta}} {L
% % _{\alpha} },$$ 
% set $f _{\alpha, r}$ to be the concatenation  of the morphisms
% in $mor(M _{\alpha})$. That is if $$M _{\alpha} = (m ^{\alpha} _{1}, \ldots, m
% ^{\alpha} _{k}) $$ then $$f _{\alpha, {r} } =  m ^{\alpha} _{1} \cdot \ldots \cdot
% m ^{\alpha} _{k}.
% $$ Then for  $t \in [0, I
% _{\alpha}]$
% set $ f ^{t} _{\alpha,r} = D ( f
% _{\alpha, r},  2t) $. 
% % rescaled by $sc (t)$ to have domain $ [0,  L (t) -L
% % _{\gamma} - L _{\beta} ]$. 
% Then set $f ^{t} _{r}$ to be the concatenation of morphisms in $mor(M
% _{\beta})$, $mor(M _{\gamma})$ and of $ f ^{t} _{\alpha, r}$, in that order,
% although 
% note that the
% order of the morphisms in the concatenation is uniquely determined by the end
% point conditions, this holds further on as well.
%
% Next set  $$I _{\beta} = I _{\alpha}+(1- I _{\alpha})(1 - d _{\beta})/2.
% $$ % \subsubsection* {case I} Suppose our vertex is as in
% % \textcolor{red}{figurecase1} 
%  If $\alpha$ and
% $\beta$ components have a nodal point in common
% % , set $$sc (t) = \frac{L (t) -L _{\gamma}}{L (I
% % _{\alpha}) - L _{\gamma}} ,
% we set $$f _{\beta,r}: [0,4] \times \{t\} \to \Delta ^{4}$$ to be the
% concatenation of $f ^{I _{\alpha}} _{\alpha, r} $ with morphisms in $mor(M
% _{\beta})$, and for $t \in [I _{\alpha}, I _{\beta}]$
% we set $$ f ^{t} _{\beta,r}  = D( f _{\beta,r},  {\frac{2(t - I
% _{\alpha})}{1 - I _{\alpha}}}  ).$$
% % rescaled by $sc (t)$ to
% % have domain $ [0, L (t) -L _{\gamma}]$. 
% And then for $t \in [I _{\alpha}, I _{\beta}]$ set $f ^{t} _{r}$ to be the
% concatenation of morphisms in  $mor(M _{\omega})$ and of $ f ^{t}
% _{\beta,r}$. 
%
% Finally, % set $$sc (t) = \frac{L (t)}{L (I _{\beta})} 
%  set $f  _{\omega,r} $ to be the
%   concatenation of $f  ^{I _{\beta}} _{\beta, r}$ with morphisms in $mor(M
%   _{\omega})$, and for $ t \in [I _{\beta}, 1]$ 
%  set $${f} ^{t} _{r} = D( {f _{\omega,r}},
%   \frac{2(t - I _{\beta})}{1-I _{\beta}}).$$ 
% %   rescaled by $sc (t)$ to have domain $ [0, L
% %   (t)]$.
%
% When $\alpha$  has a nodal point in common with
% the $\omega$ component  set $f _{\beta,r}$ 
% to be the concatenation of  morphisms in $mor(M _{\beta})$, and for $t \in [I _{\alpha}, I _{\beta}]$
% set $$ f ^{t} _{\beta,r}  = D({f _{\beta,r}},
% {\frac{2(t - I _{\alpha})}{1 - I _{\alpha}}}).$$
% % rescaled by $sc (t)$ to
% % have domain $ [0, L (t) -L _{\gamma}]$. 
% Then for $t \in [I _{\alpha}, I _{\beta}]$
% set $f ^{t} _{r}$ to be the
% concatenation of morphisms $f ^{I _{\alpha}} _{r}$ and $ f
% ^{t} _{\beta,r}$, and $mor (M _{\omega})$ (although $ mor (M _{\omega})$ in this
% particular case is empty, we add this so that the degenerate case $M
% _{\alpha} = \emptyset$, $M _{\beta}=\emptyset$ makes sense, see the discussion
% below).
% Finally, for $ t \in [I _{\beta}, 1]$ 
% % set $$sc (t) = \frac{L (t)}{L (I _{\beta})} 
%  set $$f  ^{t} _{r} = D({f _{r} ^{I _{\beta}}},
% {\frac{2(t - I _{\beta})}{1 - I _{\beta}}}).$$ 
%
%
% When $r \in \overline{\mathcal {R}} _{4}$ is in the gluing neighborhood of a face but not
% of a corner the construction of $f_r: [0,4] \times [0,1] \to \Delta ^{4}$ is similar, in fact we
% can think of it as a special case of the above by setting $d _{\beta}=1$, $M
% _{\beta} = \emptyset$. 
% When $r \in \overline{\mathcal {R}} _{4}$ is not in the gluing neighborhood of the
% boundary we can also think of this as a special case of the above, with
%  $M _{\alpha} = \emptyset, M _{\beta} = \emptyset$, $d _{\alpha}=1$, $d _{\beta}=1$ in the
% construction above. 
%
% % \subsubsection {Flattening $\{f _{r} \}$}
% % Given our family $\{f _{r} \}$, we may 
% % We now slightly rig our family of maps. Fix an embedding
% % $ i: \overline {\mathcal {R}} _{4} \to \overline{\mathcal {R}} ' _{4} \simeq
% % \overline{\mathcal{R}} _{4} $, so that the boundary of embedded domain is
% % contained in the normal gluing neighborhood $N$ of the boundary of $
% % \overline{ \mathcal {R}} _{4}'$,   and set $g:
% % \overline{ \mathcal {R}} _{4}' \to \overline{\mathcal {R}} _{4}'$ to be the
% % smooth retraction onto  $ i(\overline {\mathcal {R}} _{4})$. The family of
% % maps $\{f _{r}\}$, $r \in \overline{ \mathcal {R}} _{4}$ then gives us a family
% % $ \{f _{i ^{-1} g (r)}\}$, $r \in \overline{ \mathcal {R}} _{4}'$. Let us
% % identify $ \overline{\mathcal {R}} _{4}'$ back with $ \overline{\mathcal {R}}
% % _{4}$ and rename $ \{f _{i ^{-1} g (r)}\}$ by $\{f _{r} \}$. This 
% % procedure is just meant to flatten out the family of maps $\{f _{r}\}$
% % near the boundary of $ \overline{ \mathcal {R}} _{4}$, so that axiom (1) will be satisfied.
% % Given our fibration ${u}: \mathcal {U} (\{x _{i_j}\}) \to \overline{\mathcal {R}} _{4} $,
% % we construct a fibration $ \widetilde{u}: \widetilde{\mathcal {U}} (\{x
% % _{i_j}\}) \to \overline{\mathcal {R}} _{4}  $, with fiber over $r \in \overline{\mathcal {R}} _{4}$ a
% % Hamiltonian fiber bundle over $S _{r}$. 
% % We go back to the case $r=4$. 
% % By assumptions each  $ \mathcal {S} _{r}$ for $r$ in the
% % 1-neighborhood of corners of $ \overline{\mathcal {R}} _{4}$ is specified by a pair of
% % gluing parameters in $ [0,1)$ (which in our previous notation are $d
% % _{\alpha}$, $d _{\beta}$) and a stable 4 leafed tree. 
% % The
% % corresponding glued surface $ \mathcal {S} _{r}$ is schematically depicted in \textcolor{red}{figure.glued} below. 
% % The colored segments $\alpha, \beta$ are determined by the gluing operation, (they are the segments along which we glue). 
% \subsubsection {Retracting $\mathcal{S} _{r} $ onto $[0,4] \times [0,1]$}
% We now construct a smooth $r$-family of maps $$ret _{r}:  { \mathcal {S}} _{r} \to
% [0,4] \times [0,1],$$ $r \in {\mathcal {R}}  _{4}$,
%  suitably compatible with the maps $$ {f} _{r}: [0,4]
% \times [0,1] \to \Delta ^{4}.
% $$ 
% In figure \ref{coloredtrees} $ (a)$, $ (b)$, $ (c)$ represent cases where $ (c)$: $r$ is
% not within gluing normal neighborhood of boundary, $ (b)$: $r$ is in a gluing neighborhood 
% of a side but not a corner and $ (a)$: $r$ is within gluing neighborhood of a
% corner, (we picked a particular corner and side for these diagrams). The color shading
% will be explained in a moment. In each case $ (a), (b), (c)$ we first color
% shade $ [0,4] \times [0,1]$ as in figure \ref{squareabg}, the green region is
% the domain of $ f ^{t} _{\alpha, r}$ contained in $ [0,4]
% \times [0, I _{\alpha}]$, in the blue regions the map $f _{r}$ is vertically
% constant, the red region is the domain of $ f ^{t} _{\beta,r}$
% contained in $ [0,4] \times [I _{\alpha}, I _{\beta}]$ and yellow region is the rest of the domain of
% $f _{r}$. The maps $ret _{r}$ are defined
% for each $r$ by taking color shaded areas to color shaded areas, so that 
% the following holds. 
% \def\svgwidth{4in} 
% \begin{figure}
% \centering 
% \includegraphics[width=4in]{figuresigm.pdf}
%  % \input{figuresigm.tex}
% %   \caption{Diagram for $S_d$. Solid black border is boundary, while
% %   dashed red lines are open ends. 
% %   The connection $\mathcal {A} ({ r,\{L_i\}})$ preserves
% %   Lagrangians $L_i$ over boundary components labeled $L_i$.}
%  \caption{The uncolored enclosed regions labeled 
%  $S _{\alpha}$, $S _{\beta}$
%  surrounding segments $m _{\alpha}$, $m _{\beta}$ are ``thin''.}   
%  \label {coloredtrees}
% \end{figure}
% 
%  \def\svgwidth{2in} 
% \begin{figure}   
% \centering 
% \includegraphics[width=2in]{squareabg.pdf}
% % \input {squareabg.pdf} 
%    \caption{Diagram for $S_d$. Solid black border is boundary, while
%    dashed red lines are open ends.
%    The connection $\mathcal {A} ({ r,\{L_i\}})$ preserves
%    Lagrangians $L_i$ over boundary components labeled $L_i$.}
%  \label {squareabg}
% \end{figure}
% \begin{enumerate} 
%   \item The ends $e _{i}$, $i=1, \ldots, 4$  of
%   $ \mathcal {S} _{r}$, colored in purple, are
%       identified in strip end coordinates as $
%       (0, \infty)
%   \times [0,1]$ and in these coordinates $ret _{r}$ is the composition of the projection $  (0,
%   \infty) \times [0,1] \to [0,1]$, with the map
%       $\underline {m} _{i} $  to the boundary of $
%       [0,4] \times [0,1]$,  characterized as
%       follows:  $f
%       _{r} \circ \underline {m} _{i} $ parametrizes the  morphism $m _i \in \Pi (\Delta
%       ^{4})$.   Similarly for the $e
%       _{0}$ end.
%
%
% \item The boundary of $\mathcal{S} _{r}$ goes either to the boundary of $ [0,4]
% \times [0,1]$ or to the vertical boundary lines between colored regions.
% \item The unshaded ``thin'' regions labeled $S _{\alpha}$, $S _{\beta}$ come
% from the gluing construction and are identified with $ [0,1] \times (-\phi (\tau _{\alpha} ), \phi (\tau _{\alpha} ))$, respectively $ [0,1] \times (-\phi (\tau _{\beta} ), \phi (\tau _{\beta} ))$. In these coordinates
% $ret _{r}$ on $S _{\alpha}$, $S _{\beta}$ is the projection to $ [0,1]$ composed
% with a diffeomorphism onto the lower edge of
%       the green, respectively the red
% region, (affine in respective natural coordinates). 
% \item The unshaded part of $\mathcal{S}_{r} $ is collapsed onto the horizontal line
% bounding yellow region of $ [0,4] \times [0,1]$.
% % \item The restriction of $ret _{r}$ to the segments $m _{\alpha}$, $m _{\beta}$
% % is a smooth embedding onto the lower edge of green, respectively red
% % region. 
% \item Blue shaded regions are identified in strip
%    end coordinates $ (0,
% \infty) \times [0,1] \to \Sigma _{r}$, as $ [0,1] \times [0,1]$, and are mapped
% to the corresponding blue regions in $ [0,4] \times [0,1]$.
% \end{enumerate}
% (The above prescription naturally extends to the boundary $ \overline{\mathcal
% {R}} _{4}$.)
%
% We then set $$u (m _{1}, \ldots, m _{4}, \Delta ^{4}) | _{\mathcal{S} _{r} } = f _{r} \circ ret _{r}.  $$ 
% These are almost the maps we want, but we need to
% ``collar them'' near the boundary of
% $\overline{\mathcal{R}} _{4}  $, so that Axiom 1 of
% naturality is satisfied.   We omit the details.
%
% \section {Auxiliary data $\mathcal{D}$} 
% \label{section:dataD} 
% % Formally, the data $\mathcal{D} $ will be an
% % element of the product
% %
% % For simplicity we shall work with unobstructed and monotone Lagrangians, see
% % \cite{citeFukayaOhEtAlLagrangianintersectionFloertheory.Anomalyandobstruction.II.}.
% % \textcolor{blue}{necessary?} 
%  Let $$M \hookrightarrow P
%  \xrightarrow{\pi} X$$ be a Hamiltonian fiber bundle, with model fiber $(M, \omega)$
%  that we shall assume here to be 
%  a closed, monotone: $$ [\omega] = const \cdot 2 c _{1} (TM),$$
%  $const > 0$, symplectic manifold. The constant
%  $const$  is called the monotonicity constant.  
%
% We now discuss geometric-analysis theoretic data
% needed for the construction of the functor $F
% _{P}: \Delta^{} (X) \to A _{\infty}-Cat ^{unit}$, as
% outlined in Section \ref{sec:IntroAfunctor} of the introduction. Essentially, this
% data $\mathcal{D} = \mathcal{D} (P)  $ specifies a
% choice of a (target dependent) natural system
% $\mathcal{U}$   and various choices of
% Hamiltonian connections, as well as certain
% choices of almost complex structures.  These
% choices are to be made for each $\Sigma \in Simp
% (X)  $, while being suitably compatible, so that
% we obtain our functor $F _{P}$.  
%
%
%
%
% For $(M, \omega) $   as above, we say that a Lagrangian submanifold $L \subset M$ is
% \emph{monotone} if
% the homomorphisms given by symplectic area and Maslov class
% $$[\omega]: H_2(M, L) \to \mathbb{R}, \quad \mu: H _{2} (M, L) \to \mathbb{Z} $$
% are proportional: $$[\omega] = const \cdot \mu.$$ 
%
% For an $x: pt \to X$, define 
% \begin{equation}
%    \label{eq:Fx}
%    F (x) =F _{P} (x),
% \end{equation}
% to be the set of
% oriented, spin, monotone Lagrangian submanifold
% $L$  in $$(P _{x}=x ^{*}P, \omega _{x}) \simeq (M,
% \omega)$$ with minimal
% (positive) Maslov
% number at least 2, and such that the inclusion map $\pi _{1} (L) \to \pi _{1} (M)
% $ vanishes. We call elements of $F _{P} (x)$
% \textbf{\emph{objects}}. These will in fact be
% objects of a  certain $A _{\infty}$ category to be
% constructed.
%
% Let $L \in F _{P} (x)$, and $j$ be an almost complex
% structure on $P _{x}  $ tamed by $\omega
% _{x}$, meaning that $$\forall v \in TP _{x}, v \neq
% 0:   \omega _{x} (v, jv)   > 0. $$   Let $\mathcal{M}
%      (L,j)$ denote the moduli space of Maslov
% number 2 $j$-holomorphic discs in $P _{x}  $, with
% one marked point on the boundary, with boundary of
% the disk in $L$.
% It is well known, c.f. \cite [Section 2.3]
% {citeNickSheridanOntheFukayaCategory} (which also
% contains a number of additional references)  that for a generic such $j$, $\mathcal{M}
% (L,j)$ is regular, is a transversely cut out
% $n$-dimensional manifold and is compact. The
% compactness is due to the following fact.  If $\mathcal{M}
% (L,j)$  were not compact, then by Gromov
% compactness there would be a sequence of curves in $$\mathcal{M}
% (L,j)$$
% degenerating to a nodal curve with at least a pair
% of components. One of these components has Maslov
% number at least $2$, by our assumption on the minimal Maslov
% number. And the other component 
% contributes positively to the total Maslov number
% of the nodal curve.   (The monotonicity, and
% energy positivity preclude negative Maslov/Chern number components.) 
% This would clearly be a contradiction, by
% the additivity of the Maslov number.
%
% Then we have map corresponding to the evaluation at the marked point:
% \begin{equation*}
% ev: \mathcal{M} (L, j) \to L, 
% \end{equation*}
% and we define $\omega (L) \in \mathbb{Z}$ as the degree of $ev$.
%
%
% Given a smooth $$\Sigma: \Delta ^{n} \to X, $$  
%  set $x _{i}: = \Sigma (i) \in X $, $i \in
% \{0, \ldots, n\}$ a vertex of $\Delta^{n} $.  Also
% denote by $x
% _{i}$   the corresponding inclusion map $x _{i}:
% pt \to X$.  Set 
% \begin{equation}
%    \label{eq:FPSigma}
%    F _{P} (\Sigma):= \bigsqcup _{i} F _{P} (x
%    _{i}),  
% \end{equation}
% with elements likewise called objects, at the moment this is just a set of Lagrangians,
% but later on this will be the set of objects of a
% certain $A _{\infty}$ category, (with the same name). 
%
% Given a pair of objects  $$L _{0}, L _{1}
% \in F _{P} (\Sigma ),$$  satisfying 
% $$\omega (L _{0} ) = \omega (L _{1} ),$$
% and such that $L _{0} \in F
% _{P} (x _{i}), L _{1} \in F _{P} (x _{j}),  $ 
% let $$m = m _{L _{0}, L _{1}}: [0,1] \to \Delta
% ^{n}$$ denote the edge
% between $i,j$ corners of $\Delta ^{n} $, $i,j \in
% \{0, \ldots, n\}$.  We then set $$
% \overline{m} := \Sigma \circ m.
% $$  
% For each such $\Sigma$, and for each $L _{0}, L _{1}$
% as above, the data $\mathcal{D}$
% prescribes 
% % \subset P _{x _{i} }, L _{1} \subset P _{x _{j} },   $$
% % (including $i=j$) 
% a  Hamiltonian connection $$ \mathcal {A} (L _{0}, L _{1})=
% \mathcal {A} (L _{0}, L _{1}, \overline{m} )$$  on
% $ \overline{m} ^{*} P$. (See Section
% \ref{sec:HamiltonianBundles} for definition of
% Hamiltonian connections.) 
%
% Denote by $\mathcal{A} (L _{0}, L _{1}  ) (L _{0}
% )$ the Lagrangian $\phi (L _{0})  \subset P _{x _{j} }
% $, for $\phi$ the $\mathcal{A} (L _{0}, L _{1}  )
% $-parallel transport map over $[0,1]$.
% Then we require that
% $\mathcal{A} (L _{0}, L _{1}  ) (L _{0} )$ be transverse to
% $L _{1} $.
% \begin{definition}
%    \label{notation:SL0L1}  
%    Let $$S (L _{0}, L _{1}  ) = S (L _{0}, L _{1},
%    \mathcal{A} (L _{0}, L _{1}) )  $$ denote the space of $ \mathcal {A} ({L_0,
%  L_1})$-flat sections with boundary on $L _{0}, L
%    _{1}  $, over $0$ respectively over $1$.  In other words elements of $S (L _{0},
%    L _{1}  ) $ are sections $$\gamma: [0,1] \to
%    \overline{m} _{L _{0}, L _{1}} ^{*}P, $$
%    tangent to the 
%    $\mathcal{A} (L _{0}, L _{1}) $-horizontal
%    distribution and satisfying $\gamma (0)  \in  L
%    _{0}$ and $\gamma (1) \in L _{1}$.   By a \textbf{\emph{starting
%    position}}  
%    of an element $\gamma \in S (L _{0}, L _{1}  ) $ we
%    mean $\gamma (0) \in L _{0}$. Likewise by
%    an \textbf{\emph{ending position}}  of an element $\gamma \in S (L _{0}, L _{1}  ) $ we
%    mean $\gamma (1) \in L _{1}$.
% \end{definition}
%
%
%
% % or  $\mathcal{A} (L _{0}, L _{1}  ) (L _{0} ) = L
% % _{1} $, in this second case $\mathcal{D}$ also fixes a metric $g =g (L _{0}, L _{1}  )$, 
% % and a Morse-Smale function $f = f (L_0, L_1)$, on the space $S (L _{0}, L _{1}  ) \simeq L _{i} $ of $ \mathcal {A} ({L_0,
% % L_1})$-flat sections, with boundary on $L _{0}, L _{1}  $
% % We shall refer to the two possibilities as \emph{Morse case} and \emph{Bott case} respectively.
% % So for simplicity we deal only
% % with the two extremes of the possible general Morse-Bott setting.
% \begin{definition} \label{def:jtadmissible}
%  Let $m = m _{L_0, L _{1}} $ be as above,  and let 
% $$\{j _{t} \} _{t \in [0,1]} = j  (L _{0}, L _{1},
%    \overline{m}   ) $$ be a family of fiber-wise
%    almost complex structures on $ \overline{m}
%    ^{*} P$, s.t. for each $t$ $j _{t}$ is tamed by the
%    symplectic form $\omega
%    _{\overline{m} (t) }$ on $P _{\overline{m} (t)
%    }$.  Then $\{j _{t} \}$  is said to be \textbf{\emph{admissible with respect to $\mathcal{A}(L _{0}, L _{1})$}} if the following holds. 
% \begin{itemize}
%   \item  
% For each $t$,  Chern number $1$ $j _{t}  $-holomorphic spheres in $P _{
% \overline{m} (t) } \subset  \overline{m} ^{*} P  $
% do not intersect any of the images of any elements of $S (L _{0}, L _{1}  ) $. 
% \item The moduli spaces $\mathcal{M} (L _{0}, j _{0}  )$, $\mathcal{M} (L _{1}, j
% _{1} )$ are regular, and the evaluation map
% \begin{equation*}
% ev _{0} : \mathcal{M} (L _{0}, j _{0}  ) \to L _{0}
% \end{equation*}
% does not
% intersect the set of starting positions of elements of $S (L _{0}, L _{1}  )$.
% Likewise, the evaluation map
% \begin{equation*}
% ev _{1} : \mathcal{M} (L _{1}, j _{1}  ) \to L _{1}
% \end{equation*}
% does not
% intersect the set of ending positions of elements of $S (L _{0}, L _{1}  )$.
% % In the Bott case the evaluation $ev _{0} $
% % does not
% % intersect the set of starting positions of elements of $S (L _{0}, L _{1}  )$,
% % corresponding to critical points of $f (L _{0}, L _{1}  )$.
% % Likewise the evaluation map $ev _{1} $,
% % does not
% % intersect the set of ending positions of elements of $S (L _{0}, L _{1}  )$,
% % corresponding to critical points of $f (L _{0}, L _{1}  )$.
% \end{itemize}
% \end{definition}
% Such a family $j  (L _{0}, L _{1},
%    \overline{m}   )$ is easily seen to exist, cf.
% \cite{citeNickSheridanOntheFukayaCategory}. Our
% data $\mathcal{D} $  then fixes a choice of such
% $j  (L _{0}, L _{1},
%    \overline{m}   )$ for each $\Sigma, m, L _{0},
% L _{1}$ as above.
% % We also fix for every  $L _{0}, L _{1}  $ and  $m$ as above a family
% % $\{j _{t} \} = \{j _{t} (L _{0}, L _{1}, \overline{m}   ) \}$ of fiber-wise
% % almost complex structures on $ \overline{m} ^{*} P$ so that:
% % \begin{itemize}
% %   \item 
% % For each $t$,  Chern number $1$ $j _{t}  $-holomorphic spheres in $P _{
% % \overline{m} (t) } \subset  \overline{m} ^{*} P  $ do not intersect any of the
% % elements of $S (L _{0}, L _{1}  ) \simeq L _{i} $ that denotes the space of $ \mathcal {A} ({L_0,
% %  L_1})$-flat sections with boundary on $L _{0}, L _{1}  $.
% % Here $P _{ \overline{m} (t) }$ denotes the fiber of $\overline{m} ^{*} P $ over $\overline{m} (t) $.
% % \item The moduli spaces $\mathcal{M} (L _{0}, j _{0}  )$, $\mathcal{M} (L _{1}, j
% % _{1} )$ are regular, and the evaluation map
% % \begin{equation*}
% % ev _{0} : \mathcal{M} (L _{0}, j _{0}  ) \to L
% % \end{equation*}
% % does not
% % intersect the set of starting positions of elements of $S (L _{0}, L _{1}  )$.
% % Likewise the evaluation map
% % \begin{equation*}
% % ev _{1} : \mathcal{M} (L _{1}, j _{1}  ) \to L
% % \end{equation*}
% % does not
% % intersect the set of ending positions of elements of $S (L _{0}, L _{1}  )$.
% % % In the Bott case the evaluation $ev _{0} $
% % % does not
% % % intersect the set of starting positions of elements of $S (L _{0}, L _{1}  )$,
% % % corresponding to critical points of $f (L _{0}, L _{1}  )$.
% % % Likewise the evaluation map $ev _{1} $,
% % % does not
% % % intersect the set of ending positions of elements of $S (L _{0}, L _{1}  )$,
% % % corresponding to critical points of $f (L _{0}, L _{1}  )$.
% % \end{itemize}
% % Such a family $\{j _{t} \}$ is easily seen to exist, cf
% % \cite{citeNickSheridanOntheFukayaCategory}. Let us call
% % such a $\{j _{t} (L _{0}, L _{1}, \overline{m}   ) \}$
% %  \emph{admissible with respect to $\mathcal{A}(L _{0}, L _{1})$}.
% Next $\mathcal{D}$ makes a choice of a target
% dependent natural system $\mathcal{U} (X) $. 
% Finally, $\mathcal{D}$
% will specify a certain natural system of
% Hamiltonian connections and a system  of complex structures that
% we now describe. 
% This is to be done for all choices of
% certain Lagrangian labels. This involves some
% necessarily complicated notation, but there is nothing deep
% going on, once we have the geometric input of the
% system $\mathcal{U} (X) $.
% Loosely speaking, $\mathcal{D} $ is just a system of compatible perturbations in
% the sense of Seidel
% ~\cite[Section
% 9i]{citeSeidelFukayacategoriesandPicard-Lefschetztheory}, but relative to
% $\mathcal{U} (X) $. 
% \subsection {From a Hamiltonian fibration over $X$
% to Hamiltonian fibrations over surfaces} 
% % , and for $\Sigma$ as above a choice of a system $\mathcal{F}$ of Hamiltonian connections compatible with $\mathcal{U}$, and with the system of choices $\{\mathcal{A} (L _{0}, L _{1}, \overline{m}   )\}$ given above.
% % We shall write $\mathcal{D} (\Sigma)$ for all the data of these choices.
% %
% % Note that given a morphism $i: \Delta ^{k} \to \Delta ^{n}     $,
% % $\mathcal{D} (\Sigma ^{n} )$ induces some data $\mathcal{D}' (\Sigma ^{n}
% % \circ i ) $,
% % by the naturality properties. 
% % We shall say that $\mathcal{D} (\Sigma ^{n} )$ is \emph {functorial} if 
% % $\mathcal{D}' ( \Sigma ^{n}  \circ i) = \mathcal{D} (\Sigma ^{n} \circ i)$
% % We then say that $\mathcal{D}$ is \emph
% % {functorial} if
% % $\mathcal{D}' (\Sigma ^{k} ) = \mathcal{D} (\Sigma ^{k} )$ for all such $m$.
% %
% % \begin{proposition} A functorial $\mathcal{D}$ exists.
% % \end{proposition}
% % \begin{proof} This readily follows by the following lemma.
% % \begin{lemma} Suppose we have specified the data sets $\mathcal{D} (i _{j}  \circ \Sigma ^{n} )$,
% % for $i _{j} : \Delta ^{k _{j} } \to \Delta ^{n}  $, $k _{j} \neq n $  which are
% % compatible.
% %
% %
% % some collection of face
% % maps, such that if $\Delta ^{k _{j _{0} } },  $
% %    
% % \end{lemma}
% %    
% % \end{proof}
% Let $M \hookrightarrow P \to X$ be as before, and
% let a natural system $\mathcal{U} (X) $   be
% chosen. 
% Given a
% composable chain $ (m_1, \ldots, m_d)$ and a map
% $$u (m_1, \ldots, m_d, \Sigma ^{n}):
% {\mathcal {S}} _{d} ^{\circ}  \to \Delta ^{n} \text{ that is part of a natural system
% $\mathcal{U} (X) $},$$ we
% have an induced fibration $$M \hookrightarrow \widetilde{S} (m_1, \ldots, m_d,
% \Sigma ^{n}) \to {\mathcal {S}} _{d} ^{\circ}  $$ by pulling back $M
% \hookrightarrow P \to X$ first by $\Sigma ^{n}: \Delta ^{n} \to X$ and then by
% $u (m_1, \ldots, m_d, \Sigma ^{n})$. We have a natural projection 
% $$  \widetilde{S} (m_1, \ldots, m_d, \Sigma ^{n}) 
% \to \overline{\mathcal{R}} _{d},  $$
% and we denote the fiber over $r \in \overline{{ \mathcal {R}}} _{d}$ by $ \widetilde{S} (m_1, \ldots,
% m_d, \Sigma ^{n},r)$, or simply by $\widetilde{\mathcal{S}}_{r}  $ where there
% can be no confusion. So $\widetilde{\mathcal{S}}_{r}  $
% is naturally a Hamiltonian $M$-fibration over the surface $ \mathcal
% {S} _{r}$, smooth over smooth components.  To
% state this another way, $$\widetilde{\mathcal{S} } _{r} = (u (m
% _{1}, \ldots, m _{d}, \Sigma ^{n}, r) \circ \Sigma) ^{*} P,$$
% where 
% \begin{equation} \label{eq:unr}
%   u (m
% _{1}, \ldots, m _{d}, \Sigma ^{n},r) := u (m
% _{1}, \ldots, m _{d}, \Sigma ^{n})| _{\mathcal{S} _{r}}.
% \end{equation} 
% \subsubsection {Distinguished trivializations} 
% By the partial naturality properties  of
%    the maps $u (m _{1}, \ldots, m _{d}, \Sigma ^{n},r) $, at
%    each $e _{i}$ end, $1 \leq i \leq d$,   of 
%    $ \mathcal {S} _{r}$, we have natural
%       trivializations $$(0, \infty) \times
%       \overline{m} _{i}  ^{*} P \to
%       \widetilde{\mathcal{S} } _{r}.$$ Similarly, at
%       the $e _{0}$ and.  For $r \in \mathcal{R}
% _{d}$,  
% we also have natural trivializations of
% $\widetilde{\mathcal{S} } _{r} $  over the
% $i$'th boundary component of $\mathcal{S} _{r}$,
% $0 \leq i \leq d-1$, as $\mathbb{R} ^{} \times P
% _{s (m _{i+1})}$. 
%   Or as $ \mathbb{R} ^{}  \times P _{t (m_d)} $
% over $d$'th boundary component.  The ordering is
% as described in the preamble of Section \ref{sec:systemOfmaps},
% see also Figure \ref{fig:sidenumberedsurface}. 
% There are analogous natural trivializations also for general $r \in
% \overline{\mathcal{R} } _{d}$.   
% We shall call these \textbf{\emph{distinguished
% trivializations/coordinates}}.   And the structure
% of these trivializations will be called the
% \textbf{\emph{distinguished trivialization
% structure}}. 
% \subsection {Lagrangian labels and
% admissible connections} 
% Given $r \in \mathcal {R} _{d}$, 
% and given choices $$L _{i} \in F (\overline{m} _{i+1}
% (0)), \text{ for $0 \leq i \leq d-1$, } \quad L
% _{d} \in F (\overline{m} _{d} (1)),$$ 
% such that $\omega (L _{i}) = \omega (L _{d} )$ for
% all $i =0, \ldots, d$,
% a \emph{labeling} is just an assignment of $L
% _{i}$ to the $i$'th component of $\partial \mathcal{S}
% _{r}$.   
% Extend the labeling construction above naturally to
% $\mathcal{S} _{r}$, with   $r \in \partial
% \overline{ \mathcal {R}} _{d}$.  In other words,
% for such an $\mathcal{S} _{r}$,  we
% label the boundary components in such a way that if
% we glue at some node of $\mathcal{S} _{r}$ then
% each boundary component of the glued surface inherits a consistent label.  
% See Figure \ref{fig:label} below. 
% \begin{figure}[h]
%   \includegraphics[width=3in]{compatibleLabel.pdf}
% % \scalebox{.9}{\input{compatibleLabel.pdf}}
%  \caption {} \label{fig:label}
% \end{figure} 
% For the moment we do
% not specify any dependence of the labels on $r$.
%
% Let us from now on omit the superscript $n$ in
% $\Sigma ^{n}$  where there is no need to
% disambiguate.
% % To emphasise the labels
% % are not $r$ dependent, (given the natural
% % interpretation.)  
% \begin{definition}
%    \label{def:admissibleConnection} We say that a
%    Hamiltonian connection (cf. Section \ref{sec:HamiltonianBundles}) $ \mathcal {A}$ on the
%    Hamiltonian fibration $$
% M \hookrightarrow \widetilde{ \mathcal {S}} (m_1, \ldots, m_d, \Sigma,r) \to \mathcal {S}
%    _{r},$$ $r \in \overline{ \mathcal{R}} _{d}$,  is \emph { \textbf{admissible}} with respect
%    to a labeling $L _{0}, \ldots, L
% _{d}$ if:
% \begin{itemize}
%   \item  Parallel transport by $ \mathcal {A}$ over the boundary component(s) of $\partial \mathcal{S} _{r}$ labeled $ L _{i}$ preserves the Lagrangian $L _{i}$, $0 \leq i \leq d$, in the distinguished coordinates.  
% %       This condition is unambiguous,
% % as by construction and by the partial
% %       naturality properties of Definition
% %       \ref{def:partialnaturality}, over such a 
% %       boundary component, $ \widetilde{ \mathcal
% %       {S}} _{r}$  has a distinguished
% %       trivialization as $
% %       [0,1] \times P _{s (m _{i+1})} $, $0 \leq i \leq
% % d-1$,  or as $ [0,1] \times P _{t (m_d)} $ in the
% % case of $L _{d}$. Note that we have such
% % trivializations, over the boundary components, also when $r
% % \in \partial \overline{\mathcal{R} } _{d}$.   So
% % that the above condition also makes sense for
% %       nodal $\mathcal{S} _{r}$.
% \item  
% For $1 \leq i \leq d $,  
% in the distinguished coordinates
%       $(0, \infty) \times \overline{m} _{i}  ^{*} P$,
%       at each $e _{i}$ end,
% $$\mathcal{A} = \widetilde{\mathcal{A}} (L _{i-1}, L
%       _{i}, \overline{m} _{i}  ) := pr ^{*} \mathcal {A} (L _{i-1}, L
%       _{i}, \overline{m} _{i}  ) $$ for $$pr: (0,
%       \infty) \times \overline{m} _{i}  ^{*} P \to
%       \overline{m} _{i}  ^{*} P $$ the natural
%       bundle map projection.   
% Here   $\mathcal {A} (L _{i-1}, L
%       _{i}, \overline{m} _{i}  )$ are part of our
%       data $\mathcal{D} $ as previously discussed. 
%
% Likewise, at the $e _{0}$  end in the
%       distinguished coordinates
%       $(-\infty,   0) \times \overline{m} _{0}
%       ^{*} P$,  $\mathcal{A} = pr ^{*} \mathcal
%       {A} (L _{0}, L
%       _{d}, \overline{m} _{0}  ) $ for $pr$
%       similarly defined.   
%      
% % 
% % in the distinguished
% % trivialization at the $e _{i}$ end, $ \mathcal {A}$ has
% % the form of the canonical, trivial in the 
% %       variable $[1, \infty)$, and hence
% %       flat, extension of the
% % connection $ \mathcal {A} (L _{i-1}, L _{i}, \overline{m} _{i}  )$. 
% %       $$\forall (z = (t,s))  \in  [0,1]
% % \times [1, \infty):  \Nabla _{v} ^{\mathcal{A}} () j _{(t,s)} = {j} _{t}  (L _{i-1}, L _{i},
% % \overline{m} _{i}  ).$$
% \end{itemize}
% \end{definition} 
% We do not yet impose any conditions at the nodes,
% but certain conditions will be forced by
% the additional 
% properties of the Definition \ref{def:natural}
% below.  For later, we mention that $\mathcal{D} $ will
% make a choice of such a connection for all
% possible $L _{0}, \ldots, L _{d}$   as above.
%  
% We also have a Lagrangian sub-fibration 
% of $$
% \widetilde{ \mathcal {S}} _{r}  \to \mathcal {S}
% _{r}$$
% over the boundary of
% $\mathcal{S}_{r} $,
% whose fiber over an element of
% the boundary component labeled $L _{i} $ is $L
% _{i} $, in the distinguished
% coordinates. (This naturally
% extends to the case $\mathcal{S} _{r}$ is nodal.)
%
% We name this sub-fibration by
% \begin{equation} \label{eq:subfib}
% {\mathcal{L}}  (\mathcal{U},  L _{0}, \ldots , L _{d}, r  ).
% \end{equation} 
% In particular, by construction, if $\mathcal{A}$ is
% admissible with respect to $L _{0}, \ldots, L
% _{d}$ as above then it preserves this sub-fibration.
% % This definition naturally extends to the compactification 
% \begin{notation}
%    \label{} 
% Denote by $$ \mathcal {T} (m _{1}, \ldots, m _{d};L _{0}, \ldots, L _{d},
% \Sigma,r) = \mathcal {T} (L _{0}, \ldots, L _{d},
% \Sigma,r)$$ the space of
% Hamiltonian connections on $$ M \hookrightarrow  \widetilde{ \mathcal {S}} (m_1, \ldots, m_d,
% \Sigma,r) \to \mathcal{S} _{r}$$
% admissible with respect to $L _{0}, \ldots, L
%    _{d}$. Note that this implicitly requires a
%    chosen system $\mathcal{U} (X) $, which will
%    not be indicated.
% \end{notation}
%   
%
% % \textcolor{blue}{does not
% % belong}  and set
% % \begin{equation*}
% %    \mathcal{T} (L_0, \ldots , L _{s}, \Sigma ^{n}  ) = \prod _{r \in
% %    \overline{ \mathcal{R}}_{s} } \mathcal{T} (L_0, \ldots , L _{s}, \Sigma ^{n}, r).
% % \end{equation*}
% % As with spaces $ \mathcal {T} (m _{1},
% % \ldots, m _{d}, n)$  
% % We have the natural 
% % pregluing map 
% % \begin{equation*}  \overline{\mathcal {R}} _{k} \times (\overline{\mathcal {R}} _{i_1} \times
% % \ldots \overline{\mathcal {R}} _{i_k}) \to \overline{\mathcal {R}} _{\sum _{1 \leq j \leq k} i _{j}},
% % \end{equation*}
% % and $ \mathcal {A} \in \mathcal {T} (L ^{1} _{0}, \ldots, L ^{1} _{i_1}, \ldots,
% % L ^{k} _{0}, \ldots, L ^{k} _{i_k},n)$.
% % \begin{equation*} \mathcal {S} _{\sum _{j} i _{j}} \to \Delta ^{n},  
% % \end{equation*}, 
% % there are obvious induced elements $u _{j}$ in $ \mathcal {T} (m ^{j} _{1},
% % \ldots, m ^{j} _{i _{j}}, n)$ and an element $u _{k} \in \mathcal {T} (\prod _{1
% % \leq l \leq i _{1}} m ^{1} _{l}, \ldots, \prod _{1 \leq l \leq i _{k}} m ^{k}
% % _{l}, n)$.
% %  We say that a choice of families of Hamiltonian connections $ 
% %  \{ \mathcal {A} _{r} (L _{0}, \ldots, L _{s}) \}$,
% % over $r \in \mathcal {R} _{s}$ and all $s$ is \emph{natural} if:
% \subsection {Gluing admissible connections.}
% \label{sec:gluingconnections} 
% % Let $u \in \mathcal {T} (m _{1},
% % \ldots, m _{s_1}, n)$ and $$u' \in \mathcal {T} (m'_1, \ldots, m'
% % _{i-1}, m_1 \cdot \ldots \cdot m _{s_1}=m' _{i} ,
% % m' _{i+1}, \ldots, m' _{s_2}, n),$$ be given. 
% Given an element $$ \mathcal {A} \in \mathcal {T}
% (m _{1}, \ldots, m _{s _{1}}; L_0,
% \ldots, L _{s_1} , \Sigma,r)$$ and an element $$ \mathcal {A}' \in \mathcal {T}
% (m _{1}, \ldots, m _{s _{2}};L'_0, \ldots, L' _{i-1}, L' _{i} , L' _{i+1}, \ldots, L'
% _{s_2}, \Sigma, r'),$$  s.t. $m_1 \cdot \ldots \cdot m _{s_1}=m' _{i}$ 
% and s.t. 
% $L' _{i-1}= L _{0}, L
% ' _{i}= L _{s _{1} }     $, we have a naturally
% induced glued connection $(\mathcal{A} \star _{i}
% \mathcal{A}') _{\tau}$ 
% on $$\widetilde{\mathcal{S} } _{r,r', \tau}:= (u \star _{i}
% u') _{r,r', \tau} ^{*} \circ \Sigma ^{*} P, $$
% where $(u \star _{i}  u') _{r,r', \tau}$  is as in
% \eqref{eq:ustaru'taur}.  The construction of
% the connection $(\mathcal{A} \star _{i}
% \mathcal{A}') _{\tau}$  is analogous to the construction of the maps
% $(u \star _{i}  u') _{r,r', \tau}$, see also
% Figure \ref{fig:gluedconnection}.
% \begin{figure}[h]
% \includegraphics[width=2in]{gluedconnection.pdf}
% % \scalebox{.4}{\input{gluedconnection.pdf}}
%  \caption{The green region is identified with a
%    subregion of the surface $\mathcal{S} _{r}$,
%    the red region is identified with a
%    subregion of the surface $\mathcal{S} _{r'}$. 
%    We have similar identifications  of the
%    fibration  $\widetilde{\mathcal{S} } _{r,r',
%    \tau}$ over these regions with (sub-fibrations
%    of) of $\widetilde{\mathcal{S} } _{r},
%    \widetilde{\mathcal{S} } _{r'} $.  
% Then with respect to this identification $(\mathcal{A} \star _{i}
% \mathcal{A}') _{\tau}$ 
%    over the green region is the connection
%    $\mathcal{A} $, and over the red region it is the connection
%    $\mathcal{A}' $. Over the black region $(\mathcal{A} \star _{i}
% \mathcal{A}') _{\tau}$ 
%   is the connection $pr ^{*} \mathcal {A} (L
% _{i-1}, L _{i}, \overline{m}' _{i}  )$, discussed
%    ahead.} \label{fig:gluedconnection}
% \end{figure} 
% % Using the second property in the Definition \ref{def:admissibleConnection}
% % we can naturally extend this to a family of connections
% % This is because 
% % naturally induced by the gluing maps $St _{i} $,
% % denoted by 
% %  $$St _{i} (\mathcal {A}, \mathcal {A}', 0) \in \mathcal {T} (L'_0, \ldots, L'
% % _{i-2}, L_0, \ldots, L _{s_1}, L' _{i+2}, \ldots, L'
% % _{s_2}, \Sigma, St _{i} (r, r', 0)).$$ 
% The pair $\mathcal{A}, \mathcal{A}'$ as above will be called \emph{composable}. 
% Thus, applying Axiom 1 of naturality
% of $\mathcal{U} (X) $, for a composable pair $\mathcal{A}, \mathcal{A}'$
% as above we get induced connections $\{St _{i}
% (\mathcal {A}, \mathcal {A}', \tau)\} _{0 \leq \tau <1} $, so that $$St _{i} (\mathcal {A}, \mathcal {A}', \tau) \in \mathcal {T} (L'_0, \ldots, L'
% _{i-2}, L_0, \ldots, L _{s_1}, L' _{i+2}, \ldots, L'
% _{s_2}, \Sigma, St _{i} (r, r', \tau)),$$
% for $0 \leq \tau < 1$, and so that in addition we have the following. 
% % Outside of the thin region $thin _{\tau,i} $ of
% % $\mathcal{S} _{St _{i} (r,r', \tau) } $ the
% % connection $St _{i} (\mathcal {A}, \mathcal {A}',
% % \tau)$ is naturally identified with $St _{i}
% % (\mathcal {A}, \mathcal {A}', 0)$.
%  Over the thin region $thin _{\tau,i} \subset \mathcal{S} _{St _{i} (r,r', \tau) } $, for $\tau>0$,
% $\widetilde{\mathcal{S}}_ {St _{i} (r,r', \tau) }
% $ is naturally isomorphic to the fibration $$ (-\phi (\tau), \phi
% (\tau)) \times (\overline{m}' _{i})
% ^{*} P \to (-\phi (\tau), \phi
% (\tau)) \times [0,1] $$ by
% Axiom \ref{axiom:partial2}, \ref{axiom:partial3} of
% partial naturality, and Axiom \ref{axiom:1} 
% naturality. Here $\phi$  is as in
% \eqref{eq:phiearly}. We likewise call this
% isomorphism the \emph{distinguished coordinates/representation}
% extending the previous use of this term. 
% Then over $thin _{\tau,i} $, in the above
%   distinguished representation, $St _{i} (\mathcal {A}, \mathcal
% {A}', \tau)$ is the connection $pr ^{*} \mathcal {A} (L
% _{i-1}, L _{i}, \overline{m}' _{i}  )$, where
% $$pr:  (-\phi (\tau), \phi
% (\tau)) \times (\overline{m}' _{i})
% ^{*} P \to (\overline{m}' _{i})
% ^{*} P$$ is the natural projection.
% % (Analogously, to the second part of the Definition
% % \ref{def:admissibleConnection}).  
%
% \subsection {Admissible fiber almost complex structures} 
% \begin{definition} We say that a family $\{j _{z} \}$ of fiber-wise, $\{\omega
% _{z} \} $-compatible, almost complex
% structures on the Hamiltonian $M$-fibration $
% \widetilde{ \mathcal {S}} (m_1, \ldots, m_d, \Sigma,r) \to \mathcal {S}
% _{r}$ is \emph { \textbf{admissible}} with respect to $L _{0}, \ldots, L
% _{d}$ if:
% \begin{itemize}
% %   \item  Parallel transport by $ \mathcal {A}$ over the boundary component
% % labeled $ L _{i}$ preserves Lagrangian $L _{i}$. This condition is unambiguous
% % as by construction over this boundary component $ \widetilde{ \mathcal {S}}
% % (m_1, \ldots, m_d, n,r)$ is identified with $P _{s (m _{i+1})}$, $0 \leq i \leq
% % d-1$,  or $ P _{t (m_d)}$ in case of $L _{d}$. 
% \item At the $i$'th end of $ \mathcal {S} _{r}$,
%    $1 \leq i \leq d$ 
%    in the distinguished trivialization $$(0,
%       \infty) \times \overline{m} _{i}  ^{*} P  \to \widetilde{ \mathcal {S}}
% % <<<<<<< HEAD
% % (m_1, \ldots, m_d, \Sigma, r),$$ the family $\{j _{z} \}$ coincides
% %       with the canonical, in the factor
% %       $(0,\infty) $, $\mathbb{R}$-translation invariant extension of the
% % family $\{ {j} _{t}  (L _{i-1}, L _{i},
% % \overline{m} _{i}  )\}$, with the latter as
% % in Definition \ref{def:jtadmissible}. More
% % specifically, in the distinguished coordinates
% %       $ [0, \infty) \times \overline{m} _{i}  ^{*} P$, 
% %       $$\forall    (z = (t,s))  \in  [0,1]
% % \times (0, \infty):  j _{(t,s)} = {j} _{t}  (L _{i-1}, L _{i},
% % \overline{m} _{i}  ).$$
% % =======
% (m_1, \ldots, m_d, \Sigma, r),$$ we have $$\{j
%       _{z} \} = \widetilde{j} (L _{i-1}, L _{i})
%       :=  pr ^{*} j (L _{i-1}, L _{i}, 
%       \overline{m} _{i})$$ for   
% $$pr: (0,
%       \infty) \times \overline{m} _{i}  ^{*} P \to \overline{m} _{i}  ^{*} P$$ the projection. 
% Here $ {j}  (L _{i-1}, L _{i},
% \overline{m} _{i}  )$, is as
% in Definition \ref{def:jtadmissible}, and is part
% of our data $\mathcal{D} $  as previously
%       discussed.
% % 
% %       
% %       coincides
% %       with the canonical,  $\mathbb{R}$-translation invariant in the factor
% %       $(0,\infty) $, extension of the
% % family  More
% % specifically, in the distinguished coordinates
% %       $ (0, \infty) \times \overline{m} _{i}  ^{*} P$, 
% %       $$\forall    (z = (t,s))  \in  [0,1]
% % \times (0, \infty):  j _{(t,s)} = {j} _{t}  (L _{i-1}, L _{i},
% % \overline{m} _{i}  ).$$
% % >>>>>>> 44a62bd4e1f8ea087a167e2108751fbcab092843
% Similarly, at the $e _{0}$ end.
% \end{itemize}
% \end{definition}
% \subsection {Gluing admissible fiber almost
% complex structures} \label{sec:gluingalmostcomplex}
% Denote by $$ \mathcal {J} (L _{0}, \ldots, L _{s},
% \Sigma, r)$$ the space of
% families of fiberwise almost complex structures $\{j _{z} \}$ on $$ \widetilde{ \mathcal {S}} (m_1, \ldots, m_s,
% \Sigma,r)$$
% admissible with respect to $L _{0}, \ldots, L _{s}$.
% % \textcolor{blue}{does not belong} 
% % and set
% % \begin{equation*}
% %    \mathcal{J} (L_0, \ldots , L _{s}, \Sigma ^{n}  ) = \prod _{r \in
% %    \overline{ \mathcal{R}}_{s} } \mathcal{J} (L_0, \ldots , L _{s}, \Sigma ^{n}, r).
% % \end{equation*}
% % As with spaces $ \mathcal {T} (m _{1},
% % \ldots, m _{d}, n)$  
% % We have the natural 
% % pregluing map 
% % \begin{equation*}  \overline{\mathcal {R}} _{k} \times (\overline{\mathcal {R}} _{i_1} \times
% % \ldots \overline{\mathcal {R}} _{i_k}) \to \overline{\mathcal {R}} _{\sum _{1 \leq j \leq k} i _{j}},
% % \end{equation*}
% % and $ \mathcal {A} \in \mathcal {T} (L ^{1} _{0}, \ldots, L ^{1} _{i_1}, \ldots,
% % L ^{k} _{0}, \ldots, L ^{k} _{i_k},n)$.
% % \begin{equation*} \mathcal {S} _{\sum _{j} i _{j}} \to \Delta ^{n},  
% % \end{equation*}, 
% % there are obvious induced elements $u _{j}$ in $ \mathcal {T} (m ^{j} _{1},
% % \ldots, m ^{j} _{i _{j}}, n)$ and an element $u _{k} \in \mathcal {T} (\prod _{1
% % \leq l \leq i _{1}} m ^{1} _{l}, \ldots, \prod _{1 \leq l \leq i _{k}} m ^{k}
% % _{l}, n)$.
% %  We say that a choice of families of Hamiltonian connections $ 
% %  \{ \mathcal {A} _{r} (L _{0}, \ldots, L _{s}) \}$,
% % over $r \in \mathcal {R} _{s}$ and all $s$ is \emph{natural} if:
% Given an element $ \{j _{z} \}$ in $ \mathcal {J} (L_0,
% \ldots, L _{s_1} , \Sigma,r)$ and an element $$ \{ {j}' _{z} \}  \in \mathcal {J}
% (L'_0, \ldots, L' _{i-2}, L_0, L _{s_1}, L' _{i+1}, \ldots, L'
% _{s_2}, \Sigma, r'),$$ 
%   the pair $ \{j _{z} \}, \{j' _{z} \}$ will be called \emph{composable}.
% For such a composable pair, analogously to the
% definition of $St _{i} (\mathcal {A}, \mathcal
% {A}', \tau)$,  we have an induced 
% element:  $$St _{i} (
% \{j _{z} \}, \{j' _{z} \}, \tau) \in \mathcal {J} (L'_0, \ldots, L'
% _{i-2}, L_0, \ldots, L _{s_1}, L' _{i+2}, \ldots, L'
% _{s_2}, \Sigma, St _{i} (r, r', \tau)),$$ for each $ 0 \leq \tau <1$.
%
% \subsection {Combining admissible connections and
% fiber almost complex structures} 
%
% % , where $ \mathcal {S} _{s_1,s_2} \to
% % \overline{\mathcal {R}} _{d} \times \overline{\mathcal {R}} _{k} \times [0,1)$ is the natural family of Riemann surfaces coming from gluing. 
% % We will denote by $ \mathcal {S} _{s_1, s_2, \epsilon}$ the restriction of $ \mathcal {S} _{s_1,s_2}$ to $
% % \overline {\mathcal {R}} _{s_1} \times \overline {\mathcal {R}} _{s_2} \times
% % [0,\epsilon)$.
% % element $$u'' \in \mathcal {T} (m'_1, \ldots, m' _{i-1},
% % m_1, \ldots, m_d, m' _{i+1}, \ldots, m'_k).$$ This gives a 
% \begin{definition} \label{def:natural} Let $M
%    \hookrightarrow P \to X$  be as above.
%  A \textbf{system} $\mathcal{F} = \mathcal{F} (P)
%    $ of connections, and almost complex structures
%    \textbf{\emph{relative}} to   
%  a system $\mathcal{U} (X) $ is an element of  
%  \begin{align*}
% \prod _{\Sigma \in Simp (X) } \prod _{s \geq 2}  
% \prod _{\{(L_0, \ldots , L_s) \vert L _{i} \in
%     F (\Sigma) \}} \prod _{r \in
%    \overline{ \mathcal{R}}_{s}}    \mathcal{T} (L _{0}, \ldots , L _{s}, \Sigma, r ) \times  \mathcal{J} (L _{0}, \ldots , L _{s}, \Sigma, r ),
% \end{align*} 
% (the system $\mathcal{U} (X) $ is implicit in the
%    above.) 
%  The projection of $\mathcal{F}$ onto the $(\Sigma
%    ,s,  (L_0, \ldots , L_s), r
% )$
% component  will be
% denoted by $\mathcal{F}   (L_0, \ldots , L_s,
%    \Sigma,r  )$.   To phrase this   in functional
%    language, 
% let 
% \begin{align*}
%          O= \{(L_0, \ldots
%   L _{s}, \Sigma, r) | s \in
% \mathbb{N} _{\geq 2},  \Sigma
%    \in Simp (X), 
%     \\ L _{i}
%    \in   F (\Sigma),  
%    r \in \overline{\mathcal{R} } _{s}
%        \},
%    \end{align*}   
% then set theoretically the product
% above is the set of certain 
% maps $\mathcal{F}$ with domain $O$.   Then in
% this language $\mathcal{F}  (L_0, \ldots, 
% L_s, n,r)$ is just the value $\mathcal{F} (L_0, \ldots, 
% L_s, n,r), $  of the map $\mathcal{F} $. 
% \end{definition}
% Let $pr _{i}$, $i=1,2$, denote the projections 
% \begin{align*}
%    & pr_1: \mathcal{T} (L _{0}, \ldots , L _{s},
%    \Sigma, r ) \times  \mathcal{J} (L _{0}, \ldots
%    , L _{s}, \Sigma, r ) \to  \mathcal{T} (L _{0},
%    \ldots , L _{s}, 
%    \Sigma, r ) \\
%     & pr_2: \mathcal{T} (L _{0}, \ldots , L _{s},
%    \Sigma, r ) \times  \mathcal{J} (L _{0}, \ldots
%    , L _{s}, \Sigma, r ) \to  \mathcal{J} (L _{0}, \ldots
%    , L _{s}, \Sigma, r ).  
% \end{align*}      
%  For shorthand, in what follows, we say that a Hamiltonian connection $\mathcal{A}\in
% \mathcal{F}$  if it is of the form $pr _{1}   \mathcal{F} (L_0, \ldots , L_s, \Sigma, r
% )$, for some $(L _{0}, \ldots, L _{s}, \Sigma, r)
%    $. 
%  
% %    $\mathcal{F} (L_0, \ldots , L_s, \Sigma, r  )   $
% % onto the component of $\mathcal{T} (L _{0}, \ldots , L _{s}, \Sigma, r )$, respectively $\mathcal{J} (L _{0}, \ldots , L _{s}, \Sigma
% % , r )$.
% % \textcolor{blue}{remove} 
% % respectively onto the component of $\mathcal{J} (L _{0}, \ldots , L _{s}, \Sigma
% % ^{n}, r )$. 
% % we say that $\{J _{z} \} \in \mathcal{F}$ if it is of the
% % form $pr _{2} \circ  \mathcal{F} (L_0, \ldots , L_s, \Sigma ^{n},
% % r )  $.
%
% \begin{definition} \label{def:naturalF}
% We say that $\mathcal{F}$, relative to a natural
%    $\mathcal{U} (X) $, 
% is \textbf{\emph{natural}} if:
% \begin{enumerate}
%    \item \label{axiom:connection1} The families of connections/almost
%       complex structures are smooth in the
%       parameter $r$,  over smooth components   of
%       the
%       surfaces $\mathcal{S} _{r}$. 
%       (At the
%       nodes the behavior will be 
%       characterized the Axioms
%       \ref{axiom:connection2},
%       \ref{axiom:connection3}, below.) 
%       % 
%       % this has the same meaning as
%       % for compatible systems of perturbation of
%       % Seidel,
%       % \cite{citeSeidelFukayacategoriesandPicard-Lefschetztheory}.
%       % This can be understood as just being smooth
%       % in $ since ,
%       % and this is what is meant by smoothness.
%       % This basically means that everything is
%       % smooth in $r$ over smooth components, and so
%       % that the gluing operation is  
% \item  \label{axiom:connection2} For a composable pair $\mathcal{A},
%      \mathcal{A}' \in \mathcal{F}$ as above  the connection $St _{i} ( \mathcal
% {A}, \mathcal {A}', 0)$ coincides with $$ pr _{1}  \mathcal {F}  (L'_0, \ldots, L'
% _{i-2}, L_0, \ldots, L _{s_1}, L' _{i+1}, \ldots, L'
% _{s_2}, \Sigma, St _{i} (r, r',0)).$$ 
% % And
% % likewise 
% % for a composable pair $\{J _{z} \},
% %      \{ J' _{z} \} \in \mathcal{F}$ as above $St _{i} ( \{J _{z} \}, \{J '_{z} \}, \epsilon)$ coincides with $$ \mathcal {A}  (L'_0, \ldots, L'
% % _{i-2}, L_0, \ldots, L _{s_1}, L' _{i+2}, \ldots, L'
% % _{s_2}, \Sigma ^{n}, St _{i} (r, r',\epsilon)),$$ for $\epsilon = 0$
% % \ldots 
% %   map 
% %   \begin{equation} \label {eq.delta1} \mathcal {S} _{s_1,s_2, \epsilon} \to
% %   \Delta ^{n},
% % \end{equation}
% % induced by $u (m_1, \ldots, m_{s_1})$ and $u (m'_1, \ldots, m' _{i-1}, m_1 \circ
% % \ldots \circ m _{s_1}, m' _{i+1}, \ldots, m'_{s_2})$, as described above,
% % coincides with the composition 
% % \begin{equation} \label {eq.delta2} \mathcal {S} _{s_1, s _{2}, \epsilon} \to
% % \mathcal {S} _{s_1+s_2-1} \xrightarrow{u (m'_1, \ldots, m' _{i-1}, m_1,\ldots, m _{s_1}, m'
% % _{i+1}, \ldots, m'_{s_2})} \Delta ^{n},
% % \end{equation}
% %   where
% % the first map is the universal map, and $\epsilon=0$.
% % For every $u (m ^{1} _{1},
% % \ldots, m ^{1} _{i_1}, \ldots, m ^{k} _{1}, \ldots, m ^{k} _{i_k},n)$ with $$ cardinality(m
% % ^{1} _{1}, \ldots, m ^{1} _{i_1}, \ldots, m ^{k} _{1}, \ldots, m ^{k} _{i_k}) \leq d$$
% % %  $\sum _{1 \leq
% % % j \leq r}\sum _{l \leq l \leq i _{j} } 1 \leq d$
% %  the induced elements $u _{j}$,
% % $u _{k}$ are $u ( m ^{j} _{1}, \ldots, m ^{j} _{i _{j}}, n)$, respectively $$u
% % (\prod _{1 \leq l \leq i _{1}} m ^{1} _{l}, \ldots, \prod _{1 \leq l \leq i _{k}} m ^{k}
% % _{l},n).$$  
% \item  \label{axiom:connection3}
%    The pair of connections,  $$St _{i} ( \mathcal
% {A}, \mathcal {A}', \tau),$$  $$ pr _{1}  \mathcal {F}  (L'_0, \ldots, L'
% _{i-2}, L_0, \ldots, L _{s_1}, L' _{i+1}, \ldots, L'
% _{s_2}, \Sigma, St _{i} (r, r', \tau)),$$
%    also agree for all
% $0 < \tau < 1$ on the ``thin region'' $thin
%       _{\tau,i}$ of $ \mathcal {S}
% _{ St _{i} (r, r',\tau)}$. 
% \item  \label{property:natfacemap}
% % Given the functor $$pr: \Pi
% %       (\Delta ^{n+k} ) \to \Pi (\Delta ^{n} ),
% %       $$ induced by a surjective simplicial projection,
% %       % \begin{equation*}
% % % \xymatrix {\Delta ^{n+k}
% % %    \ar [d]  ^{\Sigma _{0} }   \ar [r] ^{pr}   & \Delta ^{n} \ar [ld]^{\Sigma _{1} } \\
% % %                                     X _{}}, 
% % % \end{equation*}
% % % for $k>0$ be a morphism 
% % %       $\Sigma _{0} \to \Sigma _{1}  $ in $\Delta/X _{\bullet} $
% % %       with $\Sigma _{1} $ non-degenerate, 
% % by the third axiom of naturality of $\mathcal{U}$ we have:
% % \begin{equation*}
% % pr \circ u(m_1, \ldots, m_s, n+k) = u (pr(m_1), \ldots, pr (m_s), n),
% % \end {equation*}   
% % and we ask that
% Given a morphism in $Simp (X) $,  $f: \Sigma
%  _{1} ^{n} \to \Sigma_{2} ^{m}$,  by Axiom
%    2 of naturality of $\mathcal{U} (X) $, 
% the Hamiltonian bundle $\widetilde{
% \mathcal {S}} (m_1, \ldots, m _{d}, \Sigma ^{n}_{1},
% r) $ is expressed as a certain pull-back  of 
% $\widetilde{ \mathcal {S}} (f(m_1), \ldots, f(m
%    _{d}), \Sigma _{2} ^{m},r)$, where $f$ on the right denotes the corresponding
%    simplicial map $f: \Delta^{n} \to \Delta^{m}
%    $. 
% So that there is a natural bundle map of  Hamiltonian $M$-fibrations $$p: \widetilde{
% \mathcal {S}} (m_1, \ldots, m _{d}, \Sigma _{1}
% ^{n},  r) \to
% \widetilde{ \mathcal {S}} (f(m_1), \ldots, f(m
%    _{d}), \Sigma _{2} ^{m},r), $$  preserving the
%    distinguished trivalization structure.
% Then we ask that $$p ^{*}pr _{1} \mathcal {F} (L_0, \ldots, 
% L_d, \Sigma _{2} ^{m},r) = pr _{1} \mathcal {F}
%    (L_0, \ldots, L _{d}, \Sigma ^{n}_{1}, r).$$
% \item There are analogous conditions on the families of almost complex
% structures $pr _{2}   \mathcal{F} (L_0, \ldots , L_s, \Sigma,
% r
% )$ that we will not state.
% \end{enumerate}
% \begin{notation}
%    We will sometimes write by abuse of notation $\mathcal{F} ( \ldots )$, for either the connection
%  $pr _{1}  \mathcal{F} ( \ldots ) $, or the family of almost complex
%  structures  $pr _{2}  \mathcal{F} ( \ldots ) $, since there usually can be no
%  confusion.
%
%    %
%    % and $\mathcal{A} ( \ldots )$ may refer
%    % to Hamiltonian connections but if we use $\mathcal{F}$ than we want to
%    % emphasize that the connection is part of a system.
% \end{notation}
% \end{definition}
% % \begin{itemize}
% %   \item   For every $ \mathcal {A} (L ^{1} _{0}, \ldots, L ^{1}
% % _{i_1}, \ldots, L ^{k} _{0}, \ldots, L ^{k} _{i_k}, St (r ^{k}, r ^{i_1},
% % \ldots, r ^{i _{k}}) )$, (see \eqref{eq.composition})
% %   with $$ cardinality(L ^{1} _{0},
% %  \ldots, L ^{1} _{i_1}, \ldots, L ^{k} _{0}, \ldots, L ^{k} _{i_k})  = d$$
% % %   $\sum _{1 \leq
% % %  j \leq r}\sum _{l \leq l \leq i _{j} } 1 \leq d$
% %  the  connection $ \mathcal {A} _{j}$ on $ \widetilde{ \mathcal {S}} _{r
% %  ^{i_j}}$, and the connection $ \mathcal {A} _{k}$ on $ \widetilde{ \mathcal
% %  {S}} _{r ^{k}}$ induced by the composition map  \eqref{eq.composition}, is
% %  an element in $  \mathcal {T} (L ^{j} _{0},
% %  \ldots, L ^{j} _{i _{j}},r ^{i _{j}})$, respectively $\mathcal
% %  {T} ( L ^{1} _{0}, L ^{0} _{i _{1}}, \ldots,  L ^{k} _{0}, L ^{k}
% %  _{i _{k}})$ and these are exactly the connections $ \mathcal { \mathcal {A}}
% %  _{r ^{i_j}} (L ^{j} _{0}, \ldots, L ^{j} _{i _{j}})$, respectively $\mathcal
% %  {A} _{r ^{k}} ( L ^{1} _{0}, L ^{0} _{i _{1}}, \ldots,  L ^{k} _{0}, L ^{k})$. 
% % \item And if given a face map $f: \Delta ^{n-1} \to \Delta ^{n}$, 
% % \begin{equation*} f \circ u
% % (m_1, \ldots, m_d, n-1) = u (f(m_1), \ldots, f (m_d), n). 
% % \end{equation*}
% % \end{itemize}
% % Let $Nat$
% % denote the space of natural systems
% % $\mathcal{U} $.   
% % Let $Sys _{\mathcal{U}}$ denote the set of natural
% % systems $\mathcal{F}$  compatible with a
% % given $\mathcal{U}$.  Set $$Sys:= \bigcup
% % _{\mathcal{U} } Sys _{\mathcal{U} }.$$
% % $Sys$ can be given a natural topology, which we
% % will only indicate. First we may define a metric
% % $d$ on $Nat$ by measuring distance between the
% % associated maps $${u} (m _{1}, \ldots, m _{d}, n):
% % {\mathcal {S}} ^{\circ}  _{d}  \to \Delta ^{n}.
% %  $$ (In other words we express $Nat$ as a
% % countable product of metric spaces and use the
% % standard induced metric.)  This can be extended to
% % a metric on $Sys$ as a pa 
% \begin{theorem} \label{lemma:naturalF}  A natural 
%    system $\mathcal{F} $ relative to   
%    any given natural system $\mathcal{U} (X)  $ exists.
%    % Moreover,   any pair $\mathcal{F} _{1},
%    % \mathcal{F} _{2}$ of such systems we omit the details as
%    % this would be rather lengthy.
%    % Let $\rho: Sys \to Nat 
%    %  $
%    %  be the natural map, then the fiber of $\rho$
%    % is 
%    % non-empty and $Sys$ is path connected.   
% \end{theorem}
% \begin{proof}  
% % First  we construct a Serre fibration $ \mathcal {H} (L _{0}, \ldots, L _{d},n)
% % \to \overline{ \mathcal {R}} _{d}$, with fiber over $r$ a certain subspace of $
% % \mathcal {T} (L _{0}, \ldots, L _{d}, n,r)$. Once again for simplicity we only
% % describe the case $d=4$. In this 
% % The proof is rather tedious, and mostly self evident so the
% % reader may omit at first reading.
% %  For each $r \in \overline{\mathcal {R}} _{d}$
% % we have an infinite dimensional family of maps satisfying the itemized condition above,
% %  however this family is clearly contractible, and the associated projection map to $ \overline{\mathcal {R}} _{d}$ is a smooth submersion. 
% %  Moreover it is easy to check that the associated fibration $
% % \mathcal {X} \to \overline{\mathcal {R}} _{d}$ is a Serre fibration with
% % contractible fiber. (This is not automatic from submersion property as the
% % projection is not proper.)
% Restricting to a single $\Sigma: \Delta ^{0} \to X$  this is the classical Fukaya category case, and the
% proof of existence of a natural system  is given in Seidel~\cite[Section
% 9i]{citeSeidelFukayacategoriesandPicard-Lefschetztheory} in the
% language of what Seidel calls compatible system of
% perturbations, which is completely analogous to
%    the language  of connections used here. Although in Seidel's book only
% the case of exact Lagrangians in exact
% symplectic manifolds is considered, this readily
% extends to our context, since we are not yet
% concerned with compactness or regularity
%  properties. 
%   
%    % This is an
% % inductive argument: given a system of compatible perturbations on lower stratum
% % it can be extended to higher strata.
% %  since $\overline{\mathcal {R}} _{2} = pt$ 
% % in this case for every $ (m_1, m_2)$ we simply fix $u (m_1, m_2, n)$ as
% % any $u \in \mathcal {T} (m_1, m_2,n)$ of the form $u=f \circ
% % ret$, with $ret$ satisfying enumerated conditions above, and $f$ the obvious
% % analogue of $f _{r}$ above. This will vacuously satisfy the first pair of
% % naturality condition. 
% % for $r \leq 2$, (the second naturality condition is not vacuous.)
%
%   
%    In what follows, as usual, we write $\Sigma
%    ^{n}$ for a degree $n$ 
% element of $Simp (X) $, i.e. of the form $\Sigma
% ^{n}: \Delta^{n} \to X$.  We proceed by induction. 
% Let $S (N) $ be the statement: 
% there is an element 
% \begin{align*}
%  \mathcal{F} ^{N} \in \prod _{r \in
%    \overline{ \mathcal{R}}_{s} }     \prod _{\{(L_0, \ldots , L_s) \vert L _{i} \in
%     F (\Sigma ^{n} )\}}  \prod _{s \geq 2} \prod
%    _{\{\Sigma ^{n} \vert n \leq
%    N\}}  
%     \mathcal{T} (L _{0}, \ldots , L _{s}, \Sigma
% ^{n}, r ) \times  \mathcal{J} (L _{0}, \ldots , L _{s}, \Sigma
% ^{n}, r )
% \end{align*}
% satisfying naturality condition of Definition
%    \ref{def:naturalF}, where the fourth  axiom is
%    only required to hold on $Simp ^{N} (X) $, the
%    latter
%    denoting the subcategory of simplices of degree
%    up to $N$. $S (N) $ will also denote the
%    corresponding system $\mathcal{F} ^{N}$.   
% % \begin{equation}
% %  \prod _{\{\Sigma ^{n}|n \leq N\}}  \prod _{s \geq 2}   \prod _{\{(L_0, \ldots , L_s)| L _{i} \in
% %     \obj F (\Sigma ^{n} )\}} \mathcal{T} (L _{0}, \ldots , L _{s}, \Sigma
% % ^{n}),  
% % \end{equation}
% %  so that their
% % restrictions $$u (m_1, \ldots, m_s, n,r) = u (m_1, \ldots, m_s, n)| _{ \mathcal
% % {S} _{r}}$$ are of the form $f _{r} \circ ret _{r}$, with $ret _{r}$ satisfying enumerated
% % conditions above, and so that naturality is satisfied. 
%
% $S (0) $ is already explained above. We prove $$S (N) \implies S (N+1), $$ in addition
% the corresponding system
% $S (N+1) $ can be assumed to extend $S (N) $. 
% % 
% % We need to extend this to an element of 
% %  \begin{align*}
% %     \prod _{r \in
% %    \overline{ \mathcal{R}}_{s} }     \prod _{\{(L_0, \ldots , L_s)\vert L _{i} \in
% %     F (\Sigma ^{n} )\}}  \prod _{s \geq 2} \prod _{\{\Sigma ^{n} \vert n \leq N +1\}} 
% %     \mathcal{T} (L _{0}, \ldots , L _{s}, \Sigma
% % ^{n}, r ) \times  \mathcal{J} (L _{0}, \ldots , L _{s}, \Sigma
% % ^{n}, r ),
% % \end{align*}
% % also satisfying naturality.
%
% Let $\Sigma ^{N+1}: \Delta^{N+1} \to X$ be given. 
% Let $L _{0}, \ldots,  L _{s} \in F (\Sigma ^{N+1})
% $, so that each $L _{i} \in F (x _{i}) $  for $x
% _{i} = \Sigma ^{N+1} (v _{i}) $ for some vertex $v _{i}
% \in \Delta^{N+1} $. 
% In particular the set $\{L _{0}, \ldots,  L _{s}
% \}$ determines the set of 
% vertices $\{v _{0}, \ldots, v _{s} \}$ of
% $\Delta^{N+1} $.    
% Denote by $D (L_0, \ldots, L_s )$ the least dimension of a subsimplex of $\Delta
% ^{N+1} $ with vertices $\{v _{0}, \ldots, v _{s}\}$. 
% %
% %
% % number $D$ of $m _{i}$, (edge or vertex determined by
% %  $ (L _{i-1}, L _{i})$),  $0 \leq i \leq s$ corresponding to distinct edges of
% %  $\Delta ^{N+1}$. 
% There is a unique extension of $\mathcal{F}$ to an element 
% \begin{equation} \label{eqInducedF}
% \begin{split}
%  \mathcal{F} \in \prod _{r \in
%    \overline{\mathcal{R}}_{s} } \prod _{\{(L_0, \ldots , L_s) \mid \, N
% \geq D (L _{0} , \ldots, L_s)}  \prod _{s \geq 2} \prod _{\{\Sigma ^{n} \mid  n
% \leq N +1\}}  \\
%     \mathcal{T} (L _{0}, \ldots , L _{s}, \Sigma
% ^{n}, r ) \times  \mathcal{J} (L _{0}, \ldots , L _{s}, \Sigma
% ^{n}, r )
% \end{split}
% \end{equation}
% satisfying the naturality condition.
% %
% %
% %
% %    \prod _{\{\Sigma ^{n}|n \leq N+1\}}\prod _{s \in \mathbb{N} _{\geq 2}} \left
% %    (\prod _{\{(L_0, \ldots, L _{s} )  | \, E (\Delta ^{n-1})
% % \geq D (L _{0} , \ldots, L_s)} \mathcal{T} (L_0, \ldots
% % L_s, \Sigma ^{n} ) \right )
% We need to extend to the case $N+1
% = D (L _{0} , \ldots, L_s)$ and 
% so that naturality is satisfied.
% For all $ (L_0 , \ldots, L_s)$ with $$D (L_0,\ldots, L_s) = N+1,$$ and given $\Sigma ^{N+1} $,  the naturality condition and $\mathcal{F}$ from \eqref{eqInducedF} determine $$ \mathcal {F} (L_0, \ldots, L _{s}, \Sigma^{N+1},r)$$ for
%  $r$ in the boundary of $ \overline{ \mathcal {R}} _{s}$, see the discussion following \eqref{eq:subs}.
%
%
% Set
% % For each $r$ the space of such connections on $P _{x} \times \mathcal {S} _{r}$ is non-empty,  
% $$\mathcal{P} := \bigcup _{r \in \overline{
%    \mathcal{R}}
% _{s}} \mathcal{T} (L _{0}, \ldots , L _{s}, \Sigma
% ^{n}, r ) \times  \mathcal{J} (L _{0}, \ldots , L _{s}, \Sigma
% ^{n}, r ). $$   So we have a natural fibration   $\mathcal{P} \to
% \overline{\mathcal{R}} _{s}  $, with the fiber
% over $r \in \overline{\mathcal{R}} _{s} $  denoted
% by $\mathcal{P} _{r}$. 
% The topology on $\mathcal{P} $ is the natural
% metric ``Gromov topology'',
% constructed using gluing operations of Sections
% \ref{sec:gluingconnections},
% \ref{sec:gluingalmostcomplex}.    We will only
% describe this briefly. First,
% constructing  
%  a metric $d$  on
% $\mathcal{P}| _{\mathcal{R} _{s}}$  can be reduced
% to constructing a metric on the spaces of
% connections/almost complex structures on a fixed
% Hamiltonian fibration $M \hookrightarrow
% \widetilde{S} \to S$, as $\mathcal{R} _{s}$
% is contractible.  In other words it is enough
% to construct $d$ on the fiber $\mathcal{P}
% _{r}$, $r \in \mathcal{R} _{s}$.  
% Since   $\mathcal{P} _{r}$  is naturally a Frechet
% manifold we just suppose that $d$ on $\mathcal{P}
% _{r}$ is the metric
% inducing the corresponding topology. 
% Given $\mathcal{S} _{r}$, $r \in \partial
% \overline{\mathcal{R}} _{s} $  there is,
% corresponding to each gluing parameter $0 <\tau
% \leq 1$,  a ``glued'' non-nodal surface $$gl _{\tau}
% (\mathcal{S} _{r}) \simeq \mathcal{S} _{gl _{\tau}
% (r) \in \mathcal{R} _{s}}, \quad \text{$\simeq$ being
% holomorphic isomorphism.}  $$   In other words
% we glue at each node of $\mathcal{S} _{r}$ with
% gluing 
% parameter $\tau$.
% Similarly, using the gluing
% operations of Sections
% \ref{sec:gluingconnections},
% \ref{sec:gluingalmostcomplex}, given $e \in
% \mathcal{P} _{r}$, $r \in \mathcal{R} _{s}$  there is
% an element $gl
% _{\tau} (e) \in \mathcal{P} _{gl _{\tau} (r)} $. 
% Now, for $r _{1} \in \partial
% \overline{\mathcal{R}} _{s} $,    $r _{2} \in
% {\mathcal{R}} _{s} $  and for elements $e _{1} \in
% \mathcal{P} _{r _{1}}, e _{2} \in \mathcal{P}
% _{r _{2}}$  we define:
% \begin{equation*}
%    d (e _{1}, e _{2}) := \lim _{\tau \mapsto
%    0} d (gl _{\tau} (e _{1}), e _{2}).  
% \end{equation*}
% Define this similarly in the case $r _{1}, r _{2}
% \in \partial \overline{\mathcal{R} } _{s} $. 
%
%
% % In other words $p
% % _{1} \in \mathcal{P} _{r _{1}}$, $r _{1} \in
% % \mathcal{R} _{s}$  is close to $p _{2} \in
% % \mathcal{P} _{r _{2}} $, $r _{2}
% % \overline{\mathcal{R}} _{s}$, if the     
% The fibers
% of $\mathcal{P} $ are non-empty, the corresponding statement for
% just connections follows by \cite[Lemma
% 3.2]{citeAkveldSalamonLoopsofLagrangiansubmanifoldsandpseudoholomorphicdiscs}.
% The fibers are contractible, for the connection
% component this is just because the relevant space
% is naturally affine. For the almost complex
% structure component, this is basically classical
% by work of Gromov ~\cite{citeGromovPseudo}.
% Moreover,   $\mathcal{P} $  is a Serre fibration,
% this is only non-obvious at boundary points of
% $\overline{\mathcal{R}} _{s}$, but there the
% corresponding lifting property for cubes can be
% easily verified
% directly, again using the gluing operations of
% Sections  \ref{sec:gluingconnections},
% \ref{sec:gluingalmostcomplex}.
%
%
% % Example of a construction of such connections is
% % given in \cite[Lemma
% % 3.2]{citeAkveldSalamonLoopsofLagrangiansubmanifoldsandpseudoholomorphicdiscs.}.
% %  This is used
% %   further on as well, but we will no longer discuss it.
% To summarize we have a Serre fibration
% $\mathcal{P} \to \overline{\mathcal{R}} _{s}$ with non-empty
% contractible fibers.   We have a section of $\mathcal{P} $ over
% $\partial \overline{\mathcal{R}} _{s}$ 
% corresponding to the partially constructed family $$ \{ \mathcal {F} (L_0, \ldots, L _{s},
%  \Sigma^{N+1},r) \} _{r \in
%  \partial {\overline{\mathcal{R}}} _{s}}$$ above.
% By the classical obstruction theory, there
% is an extension to a section $\zeta$   over $\overline{
%    \mathcal {R}} _{s}$. We may need to
% homotopically adjust the
% section $\zeta$  to satisfy the Axiom \ref{axiom:connection3} of naturality, but
% this is straightforward. So that this completes
% the proof of the inductive step.  
%
% By recursion, we may then define a sequence of
% systems $\{S _{N}\} _{N \geq 0}$,  so that $S (N+1)  $
% extends $S (N) $, for each $N$, 
% we then
% set $\mathcal{F} := \bigcup_{N} S (N). $ 
% And this completes the proof.
%
% % By
% % construction the maps restricted to $ \mathcal {S} _{r}$ for $r \in \partial \overline{\mathcal {R}}
% % _{s}$ will satisfy the enumerated condition above. We may then use that $ \mathcal {X}$ above is a Serre fibration
% % with contractible fiber, and classical obstruction theory to extend the map $u
% % (m_1, \ldots, m _{s}, n+1)$ to all of $ \overline{\mathcal {S}} _{s}$ so
% % that for every $r$, $u(m_1, \ldots, m _{s}, n+1,r)$ satisfies enumerated
% % conditions above. The second naturality property for the new collection of maps
% % will follow by properties of maps $ret _{r}$. 
% \end{proof}
% \subsection {The summary of the perturbation data
% $\mathcal{D} (P)$}   \label{sec:summaryperturbation}
% Let $M \hookrightarrow P \to X$ be as above. To summarize, the
% perturbation data $\mathcal{D} = \mathcal{D} (P)  $ consists of a 
% choice of a natural system
%  $\mathcal{U} (X)$, and a choice of a natural system
% $\mathcal{F} = \mathcal{F} (P)   $ of connections/almost complex
% structures relative to $\mathcal{U} (X)  $.   
% \begin{theorem}
%    \label{thm:concordanceMain}
% Any pair $\mathcal{D} _{0} (P) , \mathcal{D} _{1}
%    (P) $  of
% perturbation data are concordant. Concordant
%    means that
%    there is data $\widetilde{\mathcal{D}} (I
%    \times P)  $, for
%    $P \times I$ the pull-back of $P$ by the
%    projection  $X \times I \to X$, so that
%    $\widetilde{\mathcal{D} } (I \times P)  $
%    restricted over $X \times \{0\}$ is
%    $\mathcal{D} _{0} $ and restricted over $X
%    \times \{1\}$    is $\mathcal{D} _{1}$.
%    (Interpreted naturally.) 
% \end{theorem}
% \begin{proof}
%    Theorem \ref{thm:naturaltargetdependent} tells
%    us that $\mathcal{U} _{0} (X), \, \mathcal{U} _{1}
%    (X)   $ are concordant,  where the latter
%    are the systems corresponding to
%    $\mathcal{D} _{0}, \mathcal{D} _{1}$. Let
%    $\widetilde{\mathcal{U}} (X \times I)  $    denote the
%    corresponding system. 
% The proof of Theorem \ref{lemma:naturalF} then
%    readily gives a natural system
%    $\widetilde{\mathcal{F}} (P \times I) $, relative to
%    $\widetilde{\mathcal{U} } (X \times I)  $,
%    restricting to  $\mathcal{F}
%    _{1} (P), $ on $P
%    \times \{0\}$, respectively to $\mathcal{F}
%    _{2} (P)$ on $P \times \{1\}$.
%    Here $\mathcal{F}
%    _{1} (P), \, \mathcal{F} _{2} (P) $  correspond
%    to $\mathcal{D} _{1} (P) , \mathcal{D} _{2} (P) $. 
% \end{proof}
%
%
% % \begin{proposition}
% %    \label{prop:DataD} 
% % Any pair $\mathcal{D} _{0} (P), \mathcal{D} _{1}
% %    (P)  $
% %    of perturbation data for $P$  are homotopic.
% %    That is there is a family $\{\widetilde{
% %       \mathcal{D}}
% %    _{t}\}$, $t \in [0,1] $, smooth in $t$, with
% %    $\widetilde{\mathcal{D} } _{i} = \mathcal{D}
% %    _{i}$, $i=0,1$.  
% %     \end{proposition}
% % \begin{proof}
% %    This is just a direct consequence of Theorem
% %    \ref{lemmanaturalmaps} and Lemma \ref{lemma:naturalF}.
% % \end{proof}
% \section {The functor $F$} \label{section:functor} 
%  Let $A_{\infty}-Cat$ denote the
%  category of small $\mathbb{Z} _{2} $ graded $A _{\infty}$ categories over
%  $\mathbb{Q}$,  with morphisms fully-faithful
%  embeddings, as defined below, that are in addition quasi-equivalences. 
% % Note that we can work in principle over a completely
% %  general ring, but this can give no extra information in our setting, as will be explained further on. \textcolor{blue}{explain this} 
% %  as in the construction of the Fukaya categories any torsion is annihilated.
% \begin{definition} \label{defFullyfaithful} We say that an $A _{\infty} $
% functor $G$ is a
% \emph {\textbf{fully-faithful embedding}}, if $G$ has vanishing higher order components, is
% injective on objects and if the first component map on hom spaces is an isomorphism
% of chain complexes. In other words $G$ above is just an identification map of
% a full $A _{\infty} $ sub-category.
% \end{definition}
% %  Note that such a map in particular is a
% % monomorphism.
%
%  We now describe the construction of the functor $$F _{P, \mathcal {D}}: Simp(X) \to A_{\infty}-Cat$$
%  associated to a Hamiltonian fibration $P$  and the chosen data $
%  \mathcal {D}$, described in the previous section.
%  In what follows we usually drop $ \mathcal {D}$
%  and $P$ from notation. For a point $x: pt \to X $
%  the associated category will be constructed
%  following
%  Sheridan~\cite{citeNickSheridanOntheFukayaCategory}.
%  In fact the analysis does not change for the case
%  of higher dimensional simplices $\Delta^{n} \to
%  X$, the geometry however needs to be
%  substantially generalized. 
%
% %  For this construction we shall need a certain natural system of maps from the universal curve over the moduli space of marked disks to the standard topological simplices $\Delta ^{n} $.
%
% % Fix a Hamiltonian connection $
% % \mathcal {A}$ on $P$.
% \subsection {$F$ on a point} \label{section:Fonpoint} 
% For $x: pt \to X$, $F (x)$ is defined to be a
% certain 
% Fukaya type $A _{\infty}$ category, 
% whose set of objects is the set $F (x) $ discussed
% in Section \ref{section:dataD}, cf. \eqref{eq:Fx}.
% % For technical reasons, we also
% % assume that inclusion maps of these Lagrangians into $P _{x}$ vanishes on $\pi _{1}$. }.
% % This category will depend
% % on certain auxiliary choices, which we fix for every $x$, but these are choices
% % are not meant to vary over $X$ in any continuous fashion.
% % Although we could accommodate the entire algebraic/analytic machinery of
% % Fukaya-Oh-Ohta-Ono, if necessary.
% For a pair $L _{0}, L _{1} \in F (x) $, with $\omega (L _{0} ) \neq \omega (L
% _{1} )$ we set $hom (L _{0}, L _{1}   ) =0$, (to avoid dealing with curved $A
% _{\infty} $ categories), otherwise
% % uniformly
% % monotone Lagrangians (as defined in  \cite[Section
% % 2.1]{citeBiranCorneaLagrangiancobordism.I.}), or in the
% % Language of \cite{citeFukayaOhEtAlLagrangianintersectionFloertheory.Anomalyandobstruction.II.} weakly unobstructed pair with the same charge:
% we set $$hom
% (L_0, L_1) = CF ({L _{0}, L
% _{1}}, \mathcal{D}),$$ where the latter is a $\mathbb{Z} _{2} $ graded Floer chain complex
% over $ \mathbb{Q}$ that is defined as follows. 
% %  But we have to slightly reformulate the usual definition
% % of $CF  (L_0, L_1)$ and of the multiplication maps, for our setup.
% % More specifically we fix a homotopy $h ({x}, L_0, L_1)$ (remembered for latter)
% % from the constant connection on $P _{x}\times [0,1]$ to a generic Hamiltonian
% % connection
%
% Let $ \mathcal {A} ({L_0, L_1})$ be the Hamiltonian connection on
% $P _{x} \times [0,1]$ determined by the chosen data $\mathcal{D}$, and likewise let $j (L
% _{0}, L _{1}  )$ to be the family of almost complex structures determined by
% $\mathcal{D}$.
% % The $ \mathcal {A} ({L_0, L_1})$ parallel
% % transport map from $P _{x,0} $ (the fiber of over 0) to $P _{x,1} $ (the fiber
% % over 1) 
% % (The $C ^{2}$
% %  condition can be removed if one chooses to remember a specific homotopy class of
% %  the trivial connection to $ \mathcal {A} ({L_0, L_1})$).
% Then $CF ({L _{0}, L
% _{1}}, \mathcal{D})$
% % : $ \overline{ \mathcal {A}} (L
% % _{i-1}, L _{i}, \overline{m} _{i}  )$
% is the vector space over $\mathbb{Q} $,
% freely generated
% by elements of $S (L _{0}, L _{1}) $, where the
% latter is as in Definition \ref{notation:SL0L1}.   
% To quickly recall, $S (L _{0}, L _{1}) $ is the
% space of  $\mathcal {A} ({L_0, L_1})$-flat
% sections $\gamma$ of $P _{x} \times [0,1]$, with boundary on the pair of Langrangians $$L_0 \subset P _{x} \times
% \{0\},  L_1 \subset P _{x} \times \{1\}.$$  
%  
% These $\gamma$ are called \textbf{\emph{geometric
% generators}}. To relate this with more classical
% Lagrangian Floer homology generators,  we point out that
% there is a natural set isomorphism:
% \begin{equation*}
%   \phi:   S
% (L _{0}, L _{1}) \to  (\mathcal{A} (L _{0}, L _{1}) L _{0}) \cap L
% _{1}
% \end{equation*}
% where $\mathcal{A} (L _{0}, L _{1}) L
% _{0}$ is as in the paragraph prior to the
% Definition \ref{notation:SL0L1}.     
% The map $\phi$ is given by $$\phi (\gamma) =
% \gamma (1).  $$ 
%
% % 
% % where as before $\mathcal{A} (L _{0}, L
% % _{1}) L _{0}$   denotes the image of $L _{0}$  by the
% % $\mathcal{A} (L _{0},L _{1}) $-parallel
% % transport map over $[0,1] $  
%
% Then the $\mathbb{Z} _{2} $ grading of a generator
% $\gamma \in S (L _{0}, L _{1}) $ is given 
% by the sign of the intersection point $\phi
% (\gamma ) $. 
% % of $\mathcal{A} (L _{0},L _{1}  ) (L _{0}
% % )$ with $L _{1} $, corresponding to $\gamma $. 
% % where $\mathcal{A} (L _{0},L _{1}  ) (L _{0}
% % )$ as before denotes the image of $L _{0} $ by the $\mathcal{A} (L _{0},L _{1}  )$-parallel transport map over $[0,1]$.
% % , or  in the Bott case for $\gamma$ the
% % corresponding critical point of $f (L _{0}, L _{1}  ) $
% % by $-1 ^{\mu _{CZ} (\gamma)}$  
% % for $$ \mu _{CZ} (\gamma) = \mu _{CZ} (S (L _{0}, L _{1}  )) - n/2 + morseindex _{f} (\gamma),
% % where $\mu _{CZ} (S (L _{0}, L _{1}  ))$ is the generalized Maslov index of
% %  an element of $S (L _{0}, L _{1}  )$, as construced in
% %  \cite{citeRobbinSalamonTheMaslovindexforpaths.}.
% % So to keep the notation
% % simpler we fix usage of the letter  $\gamma$ for morphisms, irrespectively of
% % the underlying objects (sections, or critical points). 
% \subsubsection {Differential on $CF ({L _{0}, L
% _{1}}, \mathcal{D})$}
% For $\gamma _{0}, \gamma _{1}$ geometric generators of
% $$ CF (L _{0}, L _{1}, \mathcal{D} ),  $$ 
% let $ {\mathcal{M}} (\gamma _{0}; \gamma _{1}  )$
% denote the space of holomorphic (to be further
% explained)  sections of $$ ([0,1] \times
% \mathbb{R}) \times P
% _{x} \to [0,1]
% \times \mathbb{R},$$  with boundary on the
% Lagrangian sub-bundles $$\{0\} \times \mathbb{R} \times L_0
%   \to \mathbb{R},  \{1\} \times
% \mathbb{R} \times L
% _{1} \to \mathbb{R} ^{},
% $$ and asymptotic to $\gamma _{0} $,
% respectively to $\gamma _{1}$,  at the $\infty$, respectively $- \infty$ ends. 
% Here, {\emph{asymptotic}}  means that $$\lim _{s
%    \mapsto \infty}  {\sigma| _{[0,1] \times
%    \{s\} }} = \gamma _{0}, $$ 
% and $$\lim _{s
%    \mapsto -\infty}  {\sigma| _{[0,1] \times
%    \{s\} }} = \gamma _{1}, $$ 
% where the limit
%    is $C ^{\infty}$ limit.
% And let $\overline {\mathcal{M}} (\gamma _{0}; \gamma _{1}  )$
% denote the natural Gromov-Floer compactification of the quotient ${\mathcal{M}} (\gamma _{0}; \gamma _{1}  )/\mathbb{R}$, where $\mathbb{R} $ acts by translation on the domain.
% % Although we already have well defined term asymptotic let us also introduce the term \emph{confluent}, which in the present Morse case just means asymptotic, but will have a slightly different meaning in the Bott case.
%
% \begin{terminology}
%    \label{terminology:holomorphic} Here and elsewhere  the term \emph {
% \textbf{holomorphic section}} of various Hamiltonian fibrations over Riemann
% surfaces $S$ will mean the following. Our Hamiltonian
%    fibrations $\widetilde{S} \to S$ always come with choices
% of 
%  a Hamiltonian connection $\mathcal{A}$, and a family of
%  fiber-wise almost complex structures $\{j _{z} \}
%    _{z \in S}
%  $,  determined by the perturbation data
%  $\mathcal{D}$. This gives an induced almost complex structure
%   $J ( \mathcal {A}, \{j _{z} \} )$ on $\widetilde{S} $  restricting to $\{j _{z} \}$ on the fibers,  having a holomorphic projection map to the base, and preserving the horizontal distribution of $ \mathcal {A}$.
% Holomorphic then means that the section has $J (
%    \mathcal {A}, \{j _{z} \} )$-complex linear
%    differential. 
%    % As a shortcut we may call such an
%    % almost complex structure $J (\mathcal{D} ) $. 
% \end{terminology} 
% In the above case,  let $\mathcal {A} (L _{0}, L _{1}, \overline{m}  _{0})$,
% ${j}  (L _{0}, L _{1}, \overline{m}  _{0})$ be part of our
% data $\mathcal{D} $, where 
% $\overline{m}  _{0} = x \circ m _{0}$,  
% $m _{0}: [0,1]  \to \Delta^{0}
% $, and so $\overline{m}  _{0}: [0,1] \to X$ is the constant map
% to $x$.  
% Then
% ``holomorphic'' is with respect
% to the almost complex structure induced by: 
% $$\widetilde{\mathcal{A}} (L _{0}, L _{1}) :=  pr
% ^{*} \mathcal {A} (L _{0}, L _{1}, \overline{m}
% _{0}), \quad
% \widetilde{j}  (L _{0}, L _{1}):=pr ^{*}j (L
% _{0}, L _{1}, \overline{m}  _{0})$$ for   
% $$pr: ([0,1] \times
% \mathbb{R}) \times P
% _{x} \to [0,1]
% \times P _{x} $$ the projection.
%
% % \textcolor{blue}{notation}
% % \textcolor{blue}{notation}
%
% For a generic pair  $\mathcal{A} (L_0, L _{1} ), j (L _{0}, L _{1}  )$,  all the  moduli spaces $\overline {\mathcal{M}} (\gamma _{0}; \gamma _{1}  )$ are transversely cut out for all $\gamma _{i} $,
% ~\cite{citeNickSheridanOntheFukayaCategory} but
% these kinds of transversality results go much
% further back,
% see
% for
% example Oh~\cite{citeOhFloerCohLagrangianIntersections}.  
%
%
%  The differential   $$\mu ^{1}: CF ( L _{0}, L _{1},
%  \mathcal{D} )  \to CF ( L _{0}, L _{1},
%  \mathcal{D} ) $$
% is  defined as usual by $$\mu ^{1} (\gamma _{i} ) = \sum _{i} \# 
% \overline{\mathcal{M}} (\gamma _{i}  ; \gamma _{j}  ) \gamma _{j}   ,$$ 
% for $\{\gamma _{i} \}$ a basis of geometric generators for $CF ( L _{0}, L _{1},
%  \mathcal{D} )$. Here $\# 
% \overline{\mathcal{M}} (\gamma _{i}  ; \gamma _{j}
% )$ is defined to be zero, unless the virtual
% dimension of $\overline{\mathcal{M}} (\gamma _{i}
% ; \gamma _{j}  )$ is zero and in that case it is the signed count of points. The sum is finite by the monotonicity condition.
% % when $
% % {\mathcal{M}} (\gamma _{i}  ; \gamma _{j}  )$ has dimension 0, and hence is
% % compact in the monotone case as index bounds give energy bounds.
% % \subsubsection {Bott case}
% % In the Bott case when $\mathcal{A} (L _{0}, L _{1}  ) = L _{1}  $, our Floer
% % complex is formally the same as the pearl complex of Biran-Cornea
% % \cite{citeBiranCorneaquantumstructures}, see also
% % \cite{citeBiranCorneaRigidityAndUniruling}, as well \cite{citeFrolPSSpackage}
% % dealing with orientations. It is also closely related to the Morse-Bott Floer complex of flow lines with cascades of Fraunfelder
% % \cite{citeUrsArnoldGivental}, and the Morse-Bott contact homology complex
% % of Bourgeois \cite{citeFredericBourgeois}.
% % Given the function $f $ as before,
% % $\gamma _{0}, \gamma _{1}  $
% % a pair of geometric generators, that is a pair of critical points of $f$.
% % We define the space $ \widetilde{ {\mathcal{M}}} ^{1}  (\gamma _{0}; \gamma _{1}  )$ as the space of holomorphic sections $\sigma$ of $P _{x} \times ([0,1] \times \mathbb{R})$,  with boundary on the
% % Lagrangian sub-bundles $L_0
% % \times \mathbb{R}, L _{1} \times \mathbb{R}  
% % $, with $\sigma$ asymptotic to flat sections $\gamma _{ \pm \infty } \in S (\mathcal{A} _{0}, \mathcal{A} _{1}  ) $ at the $\pm \infty$ ends respectively, and such that $\gamma _{\infty} $ is in the unstable manifold of $\gamma _{0} $ for the negative $g$-gradient flow of $f$, and $\gamma _{-\infty} $ is in the stable manifold of $\gamma _{1} $ for the negative $g$-gradient flow of $f$.
% % For $\sigma$ as above, we shall say that it is \emph{confluent in backward time} to
% % $\gamma _{0}  $ at the $\infty$ end, and \emph{confluent in forward time} to $\gamma
% % _{1} $ at the $-\infty$ end.
% %
% % The space $ \widetilde{ {\mathcal{M}}} ^{1}  (\gamma _{0}; \gamma _{1}  )$ is
% % the space of flow lines with one cascade in the sense of Fraunfelder
% % \cite{citeUrsArnoldGivental}.
% % We may analogously define $\widetilde{ {\mathcal{M}}} ^{m}  (\gamma _{0}; \gamma _{1}  )$ - the space of flow lines with $m$ cascades confluent to $\gamma _{0} , \gamma _{1} $ at the $\infty$ respectively $-\infty$ ends.
% % And define ${ {\mathcal{M}}} ^{m}  (\gamma _{0}; \gamma _{1}  )$  to be the quotient by the natural reparametrization group action.
% %
% % We set $$\mathcal{M} (\gamma _{0}; \gamma _{1}  ) = \sqcup _{m} { {\mathcal{M}}}
% % ^{m}  (\gamma _{0}; \gamma _{1}  ),  $$ and set $\overline{\mathcal{M}} (\gamma
% % _{0}; \gamma _{1}  )$, to be the natural compactification by broken cascade flow
% % lines as in \cite{citeUrsArnoldGivental}.
% %
% % As in the Morse case we may find a regular pair $\mathcal{A} (L_0, L _{1} )$,
% % $j (L _{0}, L _{1}  )$ so that all the  moduli spaces $\overline {\mathcal{M}} (\gamma _{0}; \gamma _{1}  )$ are transversely cut out for all $\gamma _{i} $.
% % See for instance \cite{citeBiranCorneaquantumstructures}, for details in a completely analogous setting of
% % the pearl complex.
% % The differential $\mu ^{1} $ is then defined formally as before.
% % We well sometimes suppress mentioning the  perturbation, i.e. the connection $A _{L_0, L_1}$.
% % \subsection {Multiplication maps $\mu r$}
% \subsubsection {Section classes}
% \label{sec:SectionClasses}
% Let $M \hookrightarrow \widetilde{S} \to S$ be a Hamiltonian
% fibration over a Riemann surface with boundary
% and end structure $\{e _{i}\} _{i=0} ^{i=d}$.
% Suppose we have distinguished 
% trivializations $\widetilde{e}  _{i}: [0,1] \times (0, \infty)
% \times M \to \widetilde{S} $,   over $e _{i}: [0,1] \times (0,\infty) \to S
% $, $0< i \leq d$. And $\widetilde{e}  _{0}: [0,1]
% \times (-\infty, 0)
% \times M \to \widetilde{S} $,  over $e _{0}$.
% And
% suppose we have a Lagrangian sub-fibration
% $\mathcal{L}$  over the
% components of the boundary $\partial S$, analogous
% to the sub-fibrations \eqref{eq:subfib}. Let
% $\sigma $ be a section of $\widetilde{S} $, so
% that $\sigma (\partial S) \in  \mathcal{L}  $.
% Suppose in addition that $\sigma$ is continuous and is $C ^{0}$
% asymptotic at each end, to a section $\widetilde{\sigma } $  
% which is translation invariant in the $(0, \infty)
% $  factor (in the distinguished trivialization).
% Here asymptotic just means $C ^{0}$
% convergence, in the distinguished trivialization,
% $$\lim _{s \to \infty} \sigma |
% _{[0,1] \times \{s\} } = \gamma,$$ for $\gamma =
% \widetilde{\sigma }| _{[0,1] \times s}
% $. As $\widetilde{\sigma } $ is translation
% invariant,  the right-hand side is well-defined.   
% In this case, as may be apparent, we may define the homology
% class of $\sigma $, relative to the boundary and
% relative to the asymptotic constraints.  
%
% The above
% extends to the case ${S}$  is disconnected, of the form
% $\mathcal{S} _{r} $ for $r \in \partial
% \mathcal{R} _{s}$. In this case we ask that our
% sections $\sigma $ also have matching asymptotic
% constraints, at the corresponding nodal ends $n
% _{j, \pm} $, cf. Section
% \ref{sec:preliminariesRiemannSurfaces}.         
% Given this, we may again define a relative homology
% class of $\sigma $.
% We will
% not give exhaustive detail of this, as this is
% very standard. We just have a language change,
% instead of maps of surfaces to a manifold $M$, we have
% sections of $M$-fibrations over surfaces.   Let us denote such
% relative homology
% classes by letters $A$. 
% % \begin{remark}
% %    \label{remark:} The reader may observe that in
% %    what follows, we don't really need the data of
% %    homology classes $A$,  but instead just the
% %    data of path components of certain moduli space
% %    of sections. However these path 
% % 
% % \end{remark}
%
% \subsubsection {Higher multiplication maps}
% \label{sec:multiplicationMapsPoint}
% The  multiplication maps
% \begin{equation}  \label {eq.mult}
% \begin{split} \mu ^{d}: hom  (L_0, L_1) \otimes hom  (L_1, L_2)
% \otimes \ldots \\ \otimes hom (L _{d-1}, L _{d}) \to hom (L_0, L_d),
% \end{split}
% \end{equation}
% $d>1$ are defined as follows.
% \begin{notation}
%    \label{notation:generators} In the rest of the
%    paper we
%    use the notation $\{\gamma ^{j} _{i}\} _{i \in
%    I ^{j}} \in CF (L
%    _{j-1}, L _{j}) $ for the basis of geometric
%    generators. (The set $I ^{j}$ will usually be omitted
%    from notation.) 
%    So the superscript in this notation refers to the
%    vector space.  Similarly, $\gamma ^{0} _{i} \in CF (L
%    _{0}, L _{d})$ will likewise denote the
%    generators.  If the subscript is not specified
%    then we just mean general geometric generator.
% \end{notation}
%
% For generators $\gamma ^{j} \in CF (L
% _{j-1}, L _{j}, \mathcal{D} ) $, $1 \leq j \leq d$, $\gamma ^{0} \in CF (L _{0}, L
% _{d}, \mathcal{D}) $,  
% % For $y \in Y ( \{i _{k}\})$
% % denote the corresponding hom set  of $F ( \Sigma)$ by $CF _{y}$.
% %  The
% % concatenation of $a _{\alpha}$ will be denoted by $a _{c}: [0,1] \to \Delta ^{n}$. 
% we define the  moduli
% space 
% \begin{equation} \label{eq:modulispacepoint}
% \mathcal {M} (\{\gamma ^{j} \}; \gamma ^{0},   x,
%    \mathcal{D},  A) 
% \end{equation} 
% as follows.
% The  elements  are pairs $(\sigma, r) $, for
% $\sigma $ a relative class
% $A$ (to be further explained),   $\mathcal
% {F}  (\{L _{i} \},  x, r)$-holomorphic (cf.
% Terminology \ref{terminology:holomorphic}) section of
% the trivial fibration
% $$ {\mathcal{S}}_{r} \times P _{x} \to \mathcal{S}
% _{r},
% $$ 
% where $r \in {\mathcal{R}} _{d}  $, $\mathcal{F} $
% is the system determined by $\mathcal{D} $. 
% % To avoid
% % too much time with details let us avoid specifying  The
% % $L ^{2}$ energy is with respect to  condition is to force
% % $\sigma$ to be asymptotic   
% And s.t. each pair $(\sigma,r) $ satisfies:
% \begin{itemize}
% % respectively complex structures, 
% \item 
%    $\sigma (\partial \mathcal{S} _{r})  
%    \subset {\mathcal{L}}  (\mathcal{U},  L _{0}, \ldots , L _{d}, r  ), $
%  see \eqref{eq:subfib}.
% %  such that for $i
% %  \notin J$ this connection is trivial over the boundary component between the
% %  $i$'th and the $i-1$'st (marked)-end. For $j \in J \subset I$ the connection over the
% %  boundary component between the $j-1$'st and $j$'th (marked)-end  is trivial
% %  outside a compact subset $I _{j}  \simeq [0,1]$ of the component, and such that for $c_0, c_1$ in the
% %  boundary component with $c_0, c_1 \notin I _{j} $ the parallel transport map from the
% %  fiber over $c_0$ to the fiber over $c_1$ maps $L _{j-1} $ to $L _{j} $. 
%  % ${\mathcal{L}}  (\mathcal{U},  L _{0}, \ldots , L _{d}, r  ) $,
%  % see \eqref{eq:subfib},
%  %   of $P _{x} \times \mathcal{S} _{r}  
%  % $ over the boundary of $\mathcal{S}_{r} $.
%
% %  
% %  
% %  $\mathcal{L} _{r} $ is invariant under $\mathcal
% % {F}  (\{L _{i} \},  \Sigma ^{n},r)$-parallel transport,
% %  and coincides with the constant sub-fibration with fibers $L _{j}$, near the $j$ end of the
% %  $i-j, j$ boundary component and with the constant sub-fibration with fibers $L _{i-1} $ the
% %  $i$ end of the $i-1,i$ boundary component. The section ${\sigma} $ above is
% %  required to have boundary on ${\mathcal{L}} _{r}  $. 
% \item By assumptions, 
% at each $e _{i}$ end of $ \mathcal {S} _{r}$, $i \neq
% 0$, in the distinguished coordinates $$ [0,1]
%       \times  (0, \infty) \times (M \simeq P
% _{x})   \to 
% \widetilde{\mathcal {S}} _{r},$$ the data $\mathcal
% {F}  (\{L _{j} \},  \Sigma, r)$ is $
% \mathbb{R}$-translation invariant in the $(0,
% \infty) $  factor.
% % has
% % the form of the canonical, flat,
% % at the $e _{i} $ end, for each $r$ the pullback $(e _{i,r} ^{t}) ^{*}  \mathcal
% % {F}  (\{L _{i} \},  x,r)$ is
% %    the connection $\mathcal {A} ({L _{i-1}, L _{i}})$, respectively $ \mathcal
% %    {A} (L_0, L _{d})$ when $i=0$,  for
% %    all $t$ sufficiently large, and such that $(e _{i,r}) ^{*}  \mathcal
% % {F}  (\{L _{i} \},  x,r)$ is flat for all sufficiently large $t$.
%    % and  such that $\mathcal {A} (r, { \{L_i\}}) $ preserves Lagrangians $L_i$ on the
%   % corresponding, labeled  boundary components, as in figure \ref{figure.labeled}.
% Then we ask that
% $\sigma$ be asymptotic  to
%    $\gamma ^{j} $. Here, \textbf{\emph{asymptotic}}  means that $$\lim _{s
%    \mapsto \infty}  {\sigma| _{[0,1] \times
%    \{s\} }} = \gamma ^{i}, $$ where the limit
%    is $C ^{\infty}$ limit.       Likewise, in the distinguished coordinates 
%  \begin{equation*}
%       [0,1]
%       \times  (-\infty, 0) \times (M \simeq P
% _{x})  \to 
% \widetilde{\mathcal {S}} _{r},
%    \end{equation*} at the $e _{0}$  end,
% we ask that $\sigma $ be
% asymptotic to $\gamma ^{0} $.
% % \item By assumptions, 
% % at the $e _{i} $ end of $ \mathcal {S} _{r}$, in the strip coordinate charts $$e
% % _{i} ^{1}: [0,1] \times [1, \infty) \to \mathcal {S} _{r},$$ the data $\mathcal
% % {F}  (\{L _{i} \},  x,r)$ is $ \mathbb{R}$-translation invariant, and
% % % at the $e _{i} $ end, for each $r$ the pullback $(e _{i,r} ^{t}) ^{*}  \mathcal
% % % {F}  (\{L _{i} \},  x,r)$ is
% % %    the connection $\mathcal {A} ({L _{i-1}, L _{i}})$, respectively $ \mathcal
% % %    {A} (L_0, L _{d})$ when $i=0$,  for
% % %    all $t$ sufficiently large, and such that $(e _{i,r}) ^{*}  \mathcal
% % % {F}  (\{L _{i} \},  x,r)$ is flat for all sufficiently large $t$.
% %    % and  such that $\mathcal {A} (r, { \{L_i\}}) $ preserves Lagrangians $L_i$ on the
% %   % corresponding, labeled  boundary components, as in figure \ref{figure.labeled}.
% % when $i \neq 0$ we ask that
% % $\sigma$ is asymptotic to
% %    $\gamma ^{i} $, a geometric generator of $\hom _{F (x)}  (L _{i -1 }, L _{i}
% %    )$, or at the $e _{0} $ asymptotic to  $\gamma _{0}$ a geometric generator of
% %    $ \hom _{F (x)}  (L _{0}, L _{d}  )$.
% \item The pair of the conditions  above mean that
% $\sigma $ determines a relative homology class,
% as in Section \ref{sec:SectionClasses}, and we ask
% that all the $\sigma $  are in the same class $A$.
% % \item $\overline{\sigma}$ specifies  spherical, respectively disk
% %     $\{j _{x} \}$-holomorphic (trees of) bubbles (connected to $\sigma$) in the
% %     fibers over interior, respectively fibers over
% % points in the boundary, in total homology class $B$. 
% %   \item $$B + C + \sum _{i} D _{i} =A. $$
% \end {itemize}   
%
%
%   %  Fix a family of Hamiltonian connections $\{\mathcal {A} (r, { \{L_i\}}) \}$, on 
%   %  $\{P _{x} \times \mathcal {S} _{r}\}$, $ r \in \overline{\mathcal {R}} _{d}$,   
% %   Extend the
% %   homotopies $ h (x, L _{i}, L _{i+1})$ to a homotopy of the trivial connection
% %   on $P _{x} \times S _{n}$ to a  Hamiltonian connection
%
% Given geometric generators $ \{\gamma
% ^{j} \in \hom _{F
% (x)}(L _{j-1}, L _{j}  ) \} $, $1 \leq j \leq d$, $d
% \geq 2$,  and a geometric generator $ \gamma ^{0}
% \in \hom _{F
% (x)}(L _{0}, L _{d}  ) $,
% assuming that $\mathcal
% {F}  (\{L _{j} \},  x,r)$ is regular 
% we define $\mu ^{d} $ by duality as:
% \begin{equation} \label{eq:mud1}
% \langle \mu ^{d} (\gamma ^{1}, \ldots, \gamma ^{d}    ), \gamma ^{0}   \rangle = \sum _{A} \# \mathcal {M} (\gamma ^{1}, \ldots, \gamma ^{d}; \gamma ^{0},  x, \mathcal{D}, A),
% \end{equation}
% when the above moduli spaces  have dimension 0, for
% $ \langle ,  \rangle $ the natural inner
% product pairing induced by our choice of basis
% (consisting of geometric generators).
% The sum is finite by monotonicity.
%
%
% % Let \begin{equation} \label{eq:modulipoint}
% % \mathcal{M} (\gamma ^{0,1}  _{j _{1} }, \ldots , \gamma ^{i-1,i}  _{j _{d} }  , \gamma ^{0,d}  _{j}),
% % \end{equation}
% % be \textcolor{blue}{notation change}  the space of pairs
% % $ (r, u)$, $r \in \overline{\mathcal {R}} _{d}$, and $u$ a holomorphic section
% % of $P _{x} \times \mathcal {S} _{r}$,  with boundary on \textcolor{blue}{add this} 
% %
% % asymptotic over the strips to generators
% % of  $hom (L _{i-1}, L _{i})$, $ hom  (L_0, L_d)$, 
% % where ``asymptotic'' has a different meaning depending on whether $\mathcal{A} (L _{i-1}, L _{i}  )$, $\mathcal{A} (L _{0}, L _{d}  )$ are Bott type or Morse type. 
% % In the latter case the geometric generators for the morphisms complex are flat sections and asymptotic has the natural meaning. 
% % In the former case the geometric generators for the morphisms complex are critical points of the auxiliary Morse-Smale functions $f$, and if $\gamma \in \hom (L _{i-1}, L _{i}  )$ is such a generator, $u$ is asymptotic to $\gamma$ means that $u$ is asymptotic in the natural sense to $\gamma _{\infty} \in S (L _{i-1}, L _{i}  ) $ such that $\gamma _{\infty} $ is in the unstable manifold of $\gamma$ for the negative $g$-gradient flow of $f$.
% % If $\mathcal{A} (L _{0}, L _{d}  )$ is Bott type and $\gamma \in \hom (L _{0}, L _{d}  )$ is a geometric generator then $u$ is asymptotic to $\gamma$ means that $u$ is asymptotic to $\gamma _{-\infty} \in S (L _{0}, L _{d}  ) $ such that $\gamma _{-\infty} $ is in the stable manifold of $\gamma$ for the negative $g$-gradient flow of $f$.
% % Then \eqref{eq:mud} is defined by $\# \mathcal{M} (\gamma ^{0,1}  _{j _{1} }, \ldots , \gamma ^{i-1,i}  _{j _{d} },   \gamma ^{0,d}  _{j})$ when the dimension of this moduli space is $0$, \eqref{eq:mud} is defined be $0$ otherwise.
%
%
%  \subsubsection {Compactification  regularity, and associativity} \label{section:regularitycompactness} 
% Given a certain dictionary,
% the moduli spaces $$ \mathcal {M} (\{\gamma ^{i} \}; \gamma ^{0},   x, \mathcal{D},  A) 
%  $$  are  identical to the moduli spaces in Sheridan
% \cite{citeNickSheridanOntheFukayaCategory}, with
% respect to a system, determined by
% $\mathcal{F}$, of Hamiltonian perturbations. 
%
%
% To be more explicit, a Hamiltonian connection on a  trivial $M$ bundle over a surface
% $S$ is the same as the data of a 1-form on $S$ with values in $C ^{\infty} _{0}
% (M) $: smooth functions with mean 0. This is the same as the data of a Hamiltonian perturbation. So in our case we just have a language change, the reason for which will be obvious when we shall construct the value of $F$ on higher dimensional simplices of $X$.
% Consequently the compactification and regularity story is word for word
% identical to  Sheridan
% \cite{citeNickSheridanOntheFukayaCategory}.
% (Again, given the right dictionary.) 
% We say a bit more about compactification.
% The compactification
% $$\overline{ \mathcal {M}} (\{\gamma ^{i} \}; \gamma ^{0},   x, \mathcal{D},  A) 
% ,$$ 
% is obtained by allowing $r \in \overline{\mathcal{R}} _{d}$
% and allowing broken holomorphic sections over
% the disconnected surfaces $\mathcal{S} _{r}$,
% $r \in \partial \overline{ \mathcal{R}} _{d}$.
% (This is in addition to the usual stable map
% compactification.)  
% A broken holomorphic section of
% $\widetilde{\mathcal{S} } _{r} \to \mathcal{S}
% _{r} $ is a holomorphic
% section over each smooth component of $\mathcal{S}
% _{r}$, so that at the node ends $n _{j, \pm}$, the
% corresponding sections are asymptotic to the same
% geometric generator $\gamma $. (In the natural
% bundle trivializations at the ends.) 
%
% % In the Bott case of the type that we consider, 
% % the Floer degenerations look more complicated, however this is still well
% % understood as the cluster complex approach to Fukaya category of Cornea-Lalonde
% % \cite{citeCorneaLalondeCluster}. We should note also that if we are willing to use the
% % more technical language of virtual fundamental cycles, then the fully general
% % Morse-Bott type Fukaya category is constructed by Fukaya-Oh-Ono-Ohta in
% % \cite{citeFukayaLagrangianIntersectionFloertheoryAnomalyandObstructionIandII},
% % \footnote{Strictly speaking they construct an $A _{\infty} $ algebra, but this
% % is an inessential distinction.
% % } and that construction could also be used here.
% % We shall therefore limit ourselves to outlining only basic features of the compactness story.
% % We are already allowing nodal degenerations of the surfaces $\mathcal{S} _{r} $, additionally two other things can happen for a sequence $\{(u _{k}, r _{k}  )\}$ with $r _{k} \mapsto r \in \mathcal{R} _{d}  $.
% % We may have vertical bubbling, that is for a subsequence $\{(u _{k _{j}, r _{k _{j} }  } )\}$ of $\{(u _{k}, r _{k}  )\}$,  $||D ^{vert} u _{k _{j} } (s _{j} ) ||  \mapsto \infty $, $s _{j} \mapsto s \in \Sigma _{r}  $ for $D ^{vert} u _{k _{j} }$ the vertical part of the differentials.
% %  If $s $ is in the interior of $\Sigma _{r} $, then  the standard rescaling argument gives that there is a rational curve bubbling off.
% % This curve must be vertical, that is in a fiber of $\widetilde{\mathcal{S}}_{r}  $ as by construction the projection map $\widetilde{\mathcal{S}} _{r} \to \mathcal{S} _{r}   $ is holomorphic.
% % If $s $ is in the boundary of $\Sigma _{r} $, then  the standard rescaling argument gives that there is a holomorphic disk bubbling off.
% % Again the disk must be vertical, and have boundary on $\mathcal{L} _{r}| _{s}  $.
% % Additionally we may have a Floer degeneration for a sequence $\{(u _{k}, r  )\}$
% % this is completely analogous to the kind of Floer degenerations, which  appear
% % in in the cluster complex construction of the Fukaya category in \cite{cite}.
% \subsubsection {$A _{\infty} $ associativity} 
% The maps  $\mu ^{d} $ satisfy the $A
% _{\infty}$-associativity equations (stated over $\mathbb{F} _{2} $ for
% simplicity)
% \begin{equation} \label {eq.Ainfty} \sum _{n,m} \mu ^{d-m+1}  (\gamma
% ^{1}, \ldots, \gamma ^{n}, \mu ^{m}  (\gamma ^{n+1}, \ldots, \gamma ^{m+n}), \gamma
% ^{n+m+1}, \ldots ,\gamma ^{d}) = 0, 
% \end{equation}
% This is shown as usual by considering the boundary of
% the one dimensional moduli spaces, of the form:
% $ \overline{ \mathcal {M}} (\{\gamma ^{i} \};
% \gamma ^{0},   x, \mathcal{F}, A) $.
%
% %
% % \begin{proof} We show that 
% %  \begin{equation} \label{eq:ainftyvanish}
% %  \langle \sum _{n,m} \mu ^{d-m+1} _{\Sigma} (\gamma
% % _{1}, \ldots, \gamma _{n}, \mu ^{m} _{\Sigma} (\gamma _{n+1}, \ldots, \gamma _{m+n}), \gamma
% % _{n+m+1}, \ldots ,\gamma _{d}), \gamma _{0} \rangle  = 0,
% %  \end{equation}  
% %  for every generator $\gamma _{0}$, which will do it.   
% %  Note first that
% %    the left hand side of \eqref{eq:ainftyvanish} is the sum of the count of
% %    boundary points of the compact 1-dimensional manifold $ \overline{\mathcal {M}}
% % (\{\gamma _{i} \},  \Sigma ^{n}, A)$, over all $A$ such that the dimension of
% % this moduli space is $1$. Indeed for a weakly exact $M$ this is immediate: by naturality properties of
% % $\mathcal{F}$, points in the boundary of $ \overline{\mathcal {M}}
% % (\{\gamma _{i} \},  \Sigma ^{n}, A)$ give contributions to left
% % hand side of \eqref{eq.Ainfty}. On the other hand the classical gluing argument
% %  shows that all contributions to the left hand side come from boundary points.
% % The same argument works in our monotone, unobstructed setting,
% % since in this case contributions from additional boundary points of $ \overline{\mathcal {M}}
% % (\{\gamma _{i} \},  \Sigma ^{n}, A)$ corresponding to disk bubbling cancel each
% % other out.
% % \end{proof}
% % The main ingredient for the regularity is classical following \cite{citeMcDuffSalamon$J$--holomorphiccurvesandsymplectictopology}.
% % Note, as already mentioned, the 
% % above fixed choices of Hamiltonian connections are not required to depend
% % continuously on $P _{x}$, in any sense. \textcolor{blue}{compactification} 
% % \subsubsection {1-simplices}
% % We now fix a smooth
% % Hamiltonian connection $  \mathcal {A}$ on $M \hookrightarrow P \to X$.
% % 
% % % so that for the resticted holomorphic charts $ \phi ^{r}_{\pm}: [0,1] \times (r,
% % % \infty) $ at the open ends of $ \Sigma _{1}$, $ (\phi _{\pm} ^{r}) ^{*} (P_1,
% % % \mathcal {A} _{L ^{x_1}, L ^{x_2}, \Sigma _{1}})$ is Hamiltonian diffeomorphic
% % % to the pair $(m ^{*} P \times (r, \infty),  pr ^{*}{\widetilde{A}} _{m})$ for $pr:
% % % [0,1] \times (r, \infty) \to [0,1]$ the projection map, and for $r$
% % % sufficiently large. We then count
% % For $L _{0}, \ldots, L _{d-1} \in F
% % (x_1) \subset F (m)$ and for $L _{d} \in F (x_2) \subset F (m)$ the composition
% % map is defined by:
% % % Then $F (x_1)$ naturally acts on $F (m)$ on the left, that is we have  maps (not necessarily chain maps)
% % \begin{equation*}
% % \begin{split}
% % \mu ^{d}: hom  (L_0, L_1) \otimes \ldots \otimes hom
% %  (L _{d-2}, L _{d-1}) \otimes \\hom (L _{d-1}, L _{d}) \to hom  (L_0,  L _{d}),
% % \end{split}
% % \end{equation*}
% % defined similarly to maps \eqref{eq.mult}.
% % % For $d \geq 2$ let $ \mathcal {H}
% % % _{d} \subset R _{d}$ denote the subspace corresponding to the moduli space of
% % % complex structures on the disk with $d$ marked points on the boundary removed,
% % % where 2 of the marked points labeled $ (0,d)$, $ (d-1, d)$ are fixed.
% % Fix any $r$ family of smooth
% % embeddings $i _{r}: [0,1] \to S_r$, $r \in \mathcal
% % {R} _{d}$, and let $I _{r}$ denote the
% % corresponding region in $S _{r}$.
% % % mapping the ends on $S _{1}$ to $L
% % % _{0}$, $L _{d}$, and $L _{d-1}$, $L _{d}$ end respectively.
% % % The boundary
% % % components are cyclically labeled by $L _{0}, \ldots, L _{d-1}, L _{d}$, such that  
% %$L _{0}$,     $L _{d}$ end corresponds to $ (0,d)$ marked point and $L _{d-1}, L _{d}$ end corresponds to $ (d-1, d)$.
% % Fix an $r$-family of smooth retractions $ret _{r}: S _{r} \to I _{r}$.  Such
% % that in coordinates $\phi (r, L _{0}, L _{d})$, $ \phi (r, L _{d-1}, L _{d})$ at
% % the corresponding ends, this retraction corresponds to the natural projection $
% % [0,1] \times (0, \infty) \to [0,1]$. And such that the boundary component $L
% % _{d}$ retracts onto $i _{r} (1)$ and all other boundary components retract onto
% % $i _{r} (0)$.
% % 
% % % \def\svgwidth{2.5in}
% % % \begin {figure}
% % % \input {figuresigm3.pdf}
% % %   \caption{Diagram for embedding $[0,1]$ into $S _{r}$.}
% % %  \label {figuresigm3}
% % % \end{figure}
% % Set $P _{r,m} = ret _{r} ^{*} \widetilde{m} ^{*} P$
% % % and use homotopy extension to extend the
% % % homotopies $h (x_1, L _{i}, L _{i+1})$ to a homotopy of $ ret ^{*} \mathcal {A}
% % % (L ^{x_1}, L ^{x_2})$
% % and let $ \mathcal {A} ( r,\{L_i\}, m)$ be a Hamiltonian connection
% %  on $P _{r,m}$, such that for large $t$, for the $t$
% % restricted charts  $\phi ^{t} (r, {L_i, L _{i+1}})$, $ \phi ^{t} (r, {L _{0},
% % L_{d}})$,  at the corresponding ends, the
% % pull-back of $\mathcal {A} (r, \{L_i\},m)$ is the translation invariant
% % connection $ \mathcal {A} (L _{i}, L _{i+1})$, $ \mathcal {A} (L _{0}, L
% % _{d})$ respectively. (Note that $ (\phi
% % ^{t} (r, {L_i, L _{i+1}}) ^{*}P _{r,m}$,  $0 \leq i \leq d-2$, is canonically
% % trivialized as $P _{x_1} \times [0,1] \times (t, \infty)$ by construction.)
% % 
% %  The bundle $P _{r,m}$ is canonically trivialized over the boundary
% %  components as either $P _{x_1} \times \mathbb{R}$ or $ P _{x_2} \times
% %  \mathbb{R}$, and the connection  $ \mathcal {A} ( r,\{L_i\}, m)$ is asked to
% %  preserve $L_i$,  over the respectively labeled boundary components.
% % 
% % We then count isolated pairs $ (r, u)$, $r \in \overline{\mathcal {R}} _{d}$ and $u$
% % a holomorphic section of $P _{r, m}$, asymptotic to elements
% % of $$hom (L _{i}, L _{i+1}), \quad hom (L _{0},  L _{d}),$$ under
% % identifications.
% % % \def\svgwidth{3in}
% % % \begin{figure}
% % % \centering
% % % \input {figuresigm2.pdf}
% % %   \caption{}
% % %  \label {figuresigm2} 
% % % \end{figure}
% % The other composition maps are defined similarly. The $A
% % _{\infty}$-associativity equation is proved in a completely analogous way as for
% % classical Fukaya category.
% \subsection {$F$ on higher dimensional simplices}
% % This discussion extends in a fairly straight-forward way to higher
% % dimensional singular simplices, except that the definition of compositions and
% % the proof of the $A _{\infty}$-associativity equation becomes much more interesting.
% % Let us think inductively, so that we may assume we have defined $A
% % _{\infty}$-categories $F (\Sigma ^{k})$ for $\Sigma ^{k}$ singular
% % $k$-simplices, $0 \leq k \leq n-1$.
% % Let us just say a few words about the case of
% % 2-simplices, as this is very natural. 
%
% Let $\Sigma:
% \Delta ^{n} \to X$ be smooth. The
% category $F (\Sigma)$  will have objects
% $\bigsqcup _{i} \obj F ({x}_i)$, where $x _{i}: pt
% \to X $ is as before, the composition of the
% $i$th vertex inclusion $\Delta ^{0} \to \Delta ^{n}  $, with the map $\Sigma$.
% We also write $x _{i} $ for $x _{i} (pt) \in X $. 
% % $ \{x_i\}$ denote the  corners of $\Delta ^{n}$, and $ \overline{x}_i$ denotes
% % $\Sigma (x_i)$.
% % If $\{\Sigma_j\}$ denote the simplices corresponding to   restrictions of
% % $\Sigma$ to various faces of $\Delta$, then 
% % \textcolor{blue}{the following needs to move} 
% % The morphism sets are  defined so that
% % the inclusions $F ( \overline{x} _{i}) \to F (\Sigma)$ are fully-faithful
% % embeddings of categories, (see Definition \ref{defFullyfaithful}) 
% % and so that $hom _{F (\Sigma)}(L_i, L_j)$, with $L_i \in F
% % ( \overline{x} _{i}), L _{j} \in F ( \overline{x} _{j})$, $x_i \neq x_j$ is
% % given as follows.
%
%
%
% Let $$m: [0,1] \to \Delta ^{n}$$ be the edge between $i,j$ corners of $\Delta ^{n} $ and set $$
% \overline{m} = \Sigma \circ m.
% $$ Given a pair of objects  $L _{0} \in F ({x
% _{i} }) \subset F (\Sigma ), L _{1}   \in  F (x
% _{j})  \subset F (\Sigma )  $, (including $i=j$) and given the Hamiltonian connection $$ \mathcal{A} (L _{0}, L
% _{1} ) =
% \mathcal {A} (L _{0}, L _{1}, \overline{m} )$$ on $
% \overline{m} ^{*} P$, determined by $\mathcal{D}$, we define as before
% $hom _{F (\Sigma)}(L _{0}, L _{1})$  to be the $\mathbb{Z} _{2} $ graded chain complex over $\mathbb{Q}$ generated by
% the elements of $S (L _{0}, L _{1}, \mathcal {A}(L
% _{0}, L _{1})) $, cf. Definition
% \ref{notation:SL0L1}. The grading is defined as
% before.
% % flat sections $\gamma$ of $(\overline{m} ^{*} P, \mathcal {A}(L _{0}, L _{1}))$, with
% % boundary conditions: $\gamma (0) $  on the Lagrangian submanifolds $L _{0} \in \overline{m} ^{*} P \vert_0$,
% % $L _{1} \in \overline{m} ^{*} P| _1$.
%
% The differential $\mu ^{1}$ is defined identically to the
% differential on morphism spaces of categories $Fuk(P _{x})$.
% The only difference is that $\overline {m} ^{*} P $ may no longer be naturally
% trivialized. 
%
%
% % We take the 
% % $ \mathbb{R}$-translation invariant extension of the
% % connection  $ \mathcal {A} (L _{i}, L _{j})$, on $ \overline{m} ^{*} P$, to a connection on $ P _{1,m} =
% % \overline{m} ^{*} P \times \mathbb{R}$.
% % % on $$M \hookrightarrow (P _{1,m} = pr ^{*} \widetilde{m} ^{*} P) \to S
% % % _{1}=[0,1] \times \mathbb{R},$$ for $ pr:  S _{1} \to [0,1]$ the projection,
% % % coinciding with $ \mathcal {A} (m)$ on the slices $ [0,1] \times \{t\}$, and which is trivial in the $t$ direction.
% % Denote this connection by $ \overline{\mathcal
% % {A}} (L _{i}, L _{j})$.
% % classes of holomorphic sections of $ P _{1,m}$ ``asymptotic'' (in the sense of \eqref{eq:modulipoint}) to flat sections of $
% % ( \overline{m} ^{*} P, \mathcal {A} (L _{i}, L _{j}))$ as $t \mapsto
% % \infty$, $t \mapsto -\infty$.
% This completely describes all objects and morphisms of $F (\Sigma)$.
% We now need to describe
% the $A _{\infty}$ structure. 
% % , but not the $A
% % _{\infty}$-structure, which we now do.
% %  For the construction we will have to make
% % some further auxillary choices, but these will only need to be fixed once, so
% Given $\{L _{\rho (k)} \in F (x _{\rho (k)})\} _{k=0} ^{k=d}  $, $$\rho: \{0,
% \ldots, d\} \to \{0, \ldots,n \}, $$ with $\omega (L _{\rho ({k}) } )=
% \theta \in \mathbb{Z}$,
% we  need to define the
% higher composition maps
% \begin{equation} \label {eq.comp} \mu ^{d} _{\Sigma} : hom (L _{\rho (0)} , L
%    _{\rho (1)} ) \otimes
% \ldots \otimes hom (L _{\rho (d-1)} , L_{\rho (d)}) \to hom (L _{\rho (0)} , L
% _{\rho (d)} ).
% \end{equation}
% % Let $Y$ be the set of pairs $ (i _{j-1}, i _{j})$ $1\leq j \leq d$, for which
% % $i _{j-1}, i _{j}$ are distinct, and $c$ its cardinality. 
%
% Note that by construction, to each morphism of $F (\Sigma)$ naturally corresponds
% either an edge or a vertex of $\Delta ^{n}$, in either case we may naturally associate to
% these a morphism in the groupoid $\Pi (\Delta ^{n})$.
% %  In fact the construction of maps
% % \eqref{eq.comp} and of the $A _{\infty}$-structure will as an upshot
% % determine a new topological $A _{\infty}$-category we call $\Pi ^{simp}
% % (\Delta ^{n})$ with the same objects and morphisms as $\Pi (\Delta ^{n})$.
% % This category has the following property: for a composable sequence of morphisms
% % $ (m_1, \ldots, m_d)$ in $ \Pi (\Delta ^{n})$, $\mu ^{d} _{r, \Pi ^{simp} (\Delta
% % ^{n}) } (m_1, \ldots, m_k)$ is a map $ [0,1] \to \Delta ^{n}$ with image in the
% % edge determined by $s (m_1), t (m_k)$, with the latter denoting source and target respectively, and
% % where $r \in \overline{\mathcal {R}} _{k}$.
% %  However we will not mention it explicitely to
% % keep language simpler. 
% % , and $Y$ is just the set of those hom spaces above that correspond to edges. 
%  The collection $ \{x_{\rho(k)}\}$ then clearly determines a composable chain $(m_1, \ldots, m_d)$
% of morphisms in $\Pi (\Delta ^{n})$.  
%
%
% So let $\gamma ^{j} \in hom _{F (\Sigma ) } (L
% _{\rho(j-1)}, L _{\rho(j)}) $, $1 \leq j \leq d$,
% $\gamma ^{0} \in hom _{F (\Sigma )} (L _{\rho(0)}, L
% _{\rho(d)}) $, be geometric generators. 
% Given the map $$u (m_1, \ldots, m_d, \Sigma):
% {\mathcal {S}} _{d} ^{\circ}  \to \Delta ^{n} ,$$   
% \text{ that is part of a natural system
% $\mathcal{U} (X) $ determined by $\mathcal{D} $ } and given the system $\mathcal{F}$ determined by $\mathcal{D}$,
% we define the moduli space $$ { \mathcal {M}}
% (\{\gamma ^{j} \}; \gamma ^{0},   \Sigma, 
% \mathcal{D}, A) $$ analogously to \eqref{eq:modulispacepoint}. 
% The elements of this moduli space are pairs 
%  $(\sigma,r)$, $r \in {\mathcal{R}} _{d}  $,   and
%  $\sigma$ a class $A$ (to be further explained),   
%  $\mathcal
% {F}  (\{L _{\rho (j) } \},  \Sigma,r)$-holomorphic section
% of
% $$ \widetilde{S} _{r} =  \widetilde{S} (m_1, \ldots,
% m_d, \Sigma ,r) \to \mathcal{S} _{r}. $$  
% In addition each pair $(\sigma,r) $  satisfies:
% \begin {itemize}
% 
% \item $\sigma (\partial \mathcal{S} _{r})  
%    \subset {\mathcal{L}}  (\mathcal{U},  L
%    _{\rho(0)}, \ldots , L _{\rho(d)}, r  ) $,
%  see \eqref{eq:subfib}.
% Recall that the right-hand side is a sub-fibration of $\widetilde{S} _{r}   $ over the boundary of $\mathcal{S}_{r} $.
% %  
% %  
% %  $\mathcal{L} _{r} $ is invariant under $\mathcal
% % {F}  (\{L _{i} \},  \Sigma ^{n},r)$-parallel transport,
% %  and coincides with the constant sub-fibration with fibers $L _{j}$, near the $j$ end of the
% %  $i-j, j$ boundary component and with the constant sub-fibration with fibers $L _{i-1} $ the
% %  $i$ end of the $i-1,i$ boundary component. The section ${\sigma} $ above is
% %  required to have boundary on ${\mathcal{L}} _{r}  $. 
% \item By assumptions, 
% at the $i$'th end of $ \mathcal {S} _{r}$, $i \neq
% 0$, in the distinguished coordinates $$  (0,
% \infty) \times \overline{m}
% _{i} ^{*}P    \to 
% \widetilde{\mathcal {S}} _{r},$$ the data $\mathcal
% {F}  (\{L _{\rho (j) } \},  \Sigma, r)$ is $
% \mathbb{R}$-translation invariant, in the $(0,
% \infty) $  factor.
% % has
% % the form of the canonical, flat,
% % at the $e _{i} $ end, for each $r$ the pullback $(e _{i,r} ^{t}) ^{*}  \mathcal
% % {F}  (\{L _{i} \},  x,r)$ is
% %    the connection $\mathcal {A} ({L _{i-1}, L _{i}})$, respectively $ \mathcal
% %    {A} (L_0, L _{d})$ when $i=0$,  for
% %    all $t$ sufficiently large, and such that $(e _{i,r}) ^{*}  \mathcal
% % {F}  (\{L _{i} \},  x,r)$ is flat for all sufficiently large $t$.
%    % and  such that $\mathcal {A} (r, { \{L_i\}}) $ preserves Lagrangians $L_i$ on the
%   % corresponding, labeled  boundary components, as in figure \ref{figure.labeled}.
% Then we ask that
% $\sigma$ be asymptotic  to
%    $\gamma ^{j} $ a geometric generator of $$\hom
%    _{F (\Sigma )}  (L _{\rho(j -1) }, L _{\rho(j)}
%    ),$$ where asymptotic is as in   Section
%    \ref{sec:multiplicationMapsPoint}.   Likewise, in the distinguished coordinates 
%  \begin{equation*}
%      (-\infty, 0) \times  \overline{m}
%     _{0} ^{*}P   \to 
% \widetilde{\mathcal {S}} _{r},
%    \end{equation*}
% we ask that $\sigma $ be
% asymptotic to $\gamma ^{0} $ a geometric
% generator of $\hom _{F (\Sigma )}  (L _{\rho(0)},
% L _{\rho(d)})$.
% \item The pair of the conditions  above mean that
% $\sigma $ determines a relative homology class,
% as in Section \ref{sec:SectionClasses}, and we ask
% that all the $\sigma $  are in the same class $A$.
% % \item $\overline{\sigma}$ specifies  spherical, respectively disk
% %     $\{j _{x} \}$-holomorphic (trees of) bubbles (connected to $\sigma$) in the
% %     fibers over interior, respectively fibers over
% % points in the boundary, in total homology class $B$. 
% %   \item $$B + C + \sum _{i} D _{i} =A. $$
% \end {itemize}   
% \subsubsection {Compactness and regularity} We do not need to reinvent the wheel
% proving compactness and regularity results for the above moduli spaces.
% (Although it obviously works the same way.)
% Instead pick a Hamiltonian trivialization of $$M \times \Delta ^{n} \xrightarrow{tr} \Sigma ^{*} P, $$
% then using this our system $\mathcal
% {F}$ can be made to correspond to a system
% %    (\{L _{i} \},  \Sigma, r)$, and the systems of almost complex structures
% % $\mathcal{J} ( \{L _{i} \}, \Sigma )$ to $M \times \Delta ^{n} $. 
% % Then in the coordinates of $M \times \Delta ^{n} $, these
% % systems of connections and complex structures are essentially equivalent to
% % systems of
% compatible perturbations, in the sense of 
% Seidel~\cite[Section
% 9i]{citeSeidelFukayacategoriesandPicard-Lefschetztheory},
% and
% Sheridan~\cite{citeNickSheridanOntheFukayaCategory}. Since
% as previously mentioned in Section
% \ref{section:regularitycompactness}, for a trivial Hamiltonian
% $M$-fibration over a surface the data of a
% Hamiltonian connection (of the type that appears
% in our context) is equivalent to the data
% of a Hamiltonian perturbation.   
% Consequently, compactness and
% regularity works the same way as described in Section
% \ref{section:regularitycompactness},
% which is
% based on
% the work of Sheridan~\cite{citeNickSheridanOntheFukayaCategory}.
% We do not give extensive detail as this is likely
% fairly evident.
%
%
% % The moduli space $ { \mathcal {M}} (\{\gamma ^{i} \},  \Sigma ^{n}, \mathcal{F}, A) $,
% % has a 
% %  natural Gromov-Floer compactification. We shall
% % give only basic detail of this and of associated regularity,
% %  as analytically the story indistinguishable from the case of the analogous moduli spaces \eqref{eq:modulipoint}, what is somewhat different is the geometry as we  are now dealing with Hamiltonian connections on non-trivialized bundles.
% %  \subsection {Compactification of $ { \mathcal {M}} (\{\gamma ^{i} \},  \Sigma ^{n}, \mathcal{F}, A) $}
% % We are already allowing nodal degenerations of the surfaces $\mathcal{S} _{r} $, additionally two other things can happen for a sequence $\{(u _{k}, r _{k}  )\}$ with $r _{k} \mapsto r \in \mathcal{R} _{d}  $.
% % We may have vertical bubbling, that is for a subsequence $\{(u _{k _{j}, r _{k _{j} }  } )\}$ of $\{(u _{k}, r _{k}  )\}$,  $||D ^{vert} u _{k _{j} } (s _{j} ) ||  \mapsto \infty $, $s _{j} \mapsto s \in \Sigma _{r}  $ for $D ^{vert} u _{k _{j} }$ the vertical part of the differentials.
% %  If $s $ is in the interiour of $\Sigma _{r} $, then  the standard rescalling argument gives that there is a rational curve bubbling off.
% % This curve must be vertical, that is in a fiber of $\widetilde{\mathcal{S}}_{r}  $ as by construction the projection map $\widetilde{\mathcal{S}} _{r} \to \mathcal{S} _{r}   $ is holomorphic.
% % If $s $ is in the boundary of $\Sigma _{r} $, then  the standard rescalling argument gives that there is a holomorphic disk bubbling off.
% % Again the disk must be vertical, and have boundary on $\mathcal{L} _{r}| _{s}  $.
% % Additionally we may have a Floer degeneration for a sequence $\{(u _{k}, r  )\}$ this is completely identical to the kind of Floer degenerations, which may appear in \eqref{eq:modulipoint}.
% %
% %
% %
% % \subsection {Regularity} To get regularity for the family $\{\mathcal
% % {F}  (\{L _{i} \},  \Sigma ^{n},r)\}$ \ldots 
% % In the case of general symplectic manifolds 
% % We shall want a natural $\mathcal{F}$ so that this moduli space is regular. 
% % %  Recall  that geometrically generators of $hom (L, L')$ are
% % % Hamiltonian chords of connections $\{ \mathcal {A} (L _{i-1}, L _{i}) \}$, respectively $ \mathcal {A} (L_0, L _{d})$, with
% % % boundary on $ \{L _{i-1}, L _{i}\}$ respectively $L _{0}, L _{d}$
% %  Regularity can be obtained as in
% % \cite{citeFukayaOhEtAlLagrangianintersectionFloertheory.Anomalyandobstruction.II.},
% % however we do not need abstract perturbations with our
% % monotonicity/unobstructedness assumptions,
% % this is also described in \cite{citeBiranCorneaLagrangiancobordism.II.}.
% % The regularity  can be obtained via classical
% % methods $J$-holomorphic curves, (no virtual moduli cycle is necessary) as done
% % by Seidel in \cite{citeSeidelFukayacategoriesandPicard-Lefschetztheory.} and adopted to monotone setting by Biran-Cornea
% % \cite{cite}.  
% \subsubsection {Composition maps in the $A
% _{\infty}$ category $F (\Sigma ) $ }
% \label{section.degenerate}  For $\{L _{\rho (j)
% }\}$, as above,  given geometric generators $
% \gamma ^{j}  \in \hom _{F
% (x)}(L _{\rho(j-1)}, L _{\rho(j)}  ) $, $1 \leq j \leq d$, $d
% \geq 2$, and a geometric generator  
% $ \gamma ^{0}  \in \hom _{F
% (x)}(L _{\rho(0)}, L _{\rho(d)}  ) $,
% assuming that $\mathcal
% {F}  (\{L _{\rho(j)} \},  \Sigma,r)$ is regular 
% we define $\mu ^{d} _{F (\Sigma ) } (\gamma
% ^{1}, \ldots, \gamma ^{d})  $ by the pairing:
% \begin{equation} \label{eq:mud2}
% \langle \mu ^{d} _{F (\Sigma ) } (\gamma
% ^{1}, \ldots, \gamma ^{d}),
% \gamma ^{0}   \rangle = \sum _{A} \# \mathcal {M} (\gamma ^{1}, \ldots , \gamma ^{d}; \gamma ^{0},  \Sigma, \mathcal{D}, A),
% \end{equation}
% when the above moduli spaces are of dimension 0, for
% $ \langle ,  \rangle $ as before the inner product
% pairing induced by our basis choice. Again the sum
% is finite by the monotonicity.
% % \eqref{eq.comp} as usual by \begin{equation*} \langle \mu ^{d} _{\Sigma}(\gamma _{1}, \ldots, \gamma _{d}),
% % \gamma _{0} \rangle = \sum _{A}\int _{ \overline{ \mathcal {M}} (\{\gamma _{i}
% % \}, \Sigma, \mathcal{F},    A)} 1,
% % \end{equation*}
% % where the integral on the right just means signed count of points in the moduli
% % space. 
% \subsubsection {Associativity} This works as before.
% % It remains to say what to do with degenerate simplices. For this we may proceed
% % recursively,  as $\Delta ^{0} \to X$ is by definition always non-degenerate.
% % Suppose we have
% %  \begin{equation*}
% % \xymatrix {\Delta ^{n} \ar [rd]  ^{\Sigma ^{n}} \ar[r] ^{p}& \Delta ^{n-1}
% % \ar[d] ^{\Sigma ^{n-1}} \\ & X}, \\
% % \end{equation*}
% % with $\Sigma ^{n-1}$ non-degenerate, then we set $F (\Sigma ^{n}) = F (\Sigma
% % ^{n-1})$.
% %  The simplicial map $p$ is injective when restricted to some face, and
% % we denote by  $v$  the  vertex of $v: \Delta ^{0} \to \Delta ^{n}$ not lying in this
% % face. We then have a commutative diagram 
% %  \begin{equation*} 
% % \xymatrix {F (\Sigma ^{n-1}) \ar [d] ^{id}\ar [rd] & & F ( \Sigma ^{n-1} \circ p
% % \circ v) = F (\Sigma \circ v)\ar[ld] \ar [d] \\ \ar [d] &  F (\Sigma ^{n})
% % \ar [ld] \ar [rd] & \ar [d] \\ F (\Sigma ^{n-1}) &  &  F (\Sigma ^{n-1})},
% % \end{equation*}
% % where the bottom diagonal arrows are both. 
% % Simplest can be formalized by
% % describing $F (\Sigma )$ as a colimit of a certain category of diagrams. But as
% % this is a large colimit and is not guaranteed to exist we can just describe it
% % explicitely. The objects of $F (\Sigma)$ are $\obj F (\Sigma \circ v) \sqcup
% % \obj F (\Sigma ^{n-1})$. Note that we have an identification of $i: \obj F
% % (\Sigma \circ v) \to \obj F (\Sigma ^{n-1} \circ p \circ v)$ and we set 
% % $ hom _{F (\Sigma) }(L, i L) = \mathbb{K}$, $ hom _{F (\Sigma) }(iL, L) =
% % \mathbb{K}$ with composition map $ hom _{F (\Sigma) }(L, i L) \otimes
% % \mathbb{K}$, $ hom _{F (\Sigma) }(iL, L) \to hom (L,L)$
%
% %  We have that $F (\Sigma \circ v) = F (\Sigma ^{n-1} \circ p \circ v)$ and
% % so by previous argument there is a natural map $N: F (\Sigma \circ v) \to F (\Sigma ^{n-1})$.
% % We can write down $F (\Sigma ^{n})$ directly but this is tedious. The main point
% % is that $F (\Sigma ^{n})$  is canonically constructed from $F (v)$ and $F
% % (\Sigma ^{n-1})$ and fits into the diagram above. Then the category $F (\Sigma)$, has objects $\obj F (v) \sqcup \obj F (\Sigma ^{n-1})$. 
% \begin{lemma} \label{lemma:functorialF} The assignment $\Sigma
% \mapsto F (\Sigma)$  extends to a natural functor $$F: Simp (X) \to
% A_{\infty}-Cat.
% $$ 
% \end{lemma}
% \begin{proof} Given a face map $f: \Delta ^{n-1} \to \Delta ^{n}$ and 
% $\Sigma^{n}: \Delta ^{n} \to X$,  by the
%    naturality Axiom \ref{property:natfacemap} of our
% connections there is a canonical functor $F (\Sigma ^{n} \circ f) \to F
% (\Sigma ^{n})$ that is by construction  a fully-faithful embedding. 
% It follows via iteration that a morphism $\sigma: \Sigma ^{k} \to \Sigma ^{l}$, with
% $\Sigma ^{k}, \Sigma ^{l} \in Simp (X)$, $k<l$ induces a fully-faithful
% embedding: $$F (\sigma): F (\Sigma ^{k}) \to F (\Sigma ^{l}),$$ and this assignment is clearly
% functorial. Note that $F (\sigma)$ is essentially surjective on the cohomological level,
% which follows by a classical continuation argument, cf. \cite[Section
% 10a]{citeSeidelFukayacategoriesandPicard-Lefschetztheory}, and so each $F (\sigma)$ is a
% quasi-equivalence. 
% \end{proof}
% Let us call the functor $F _{P, \mathcal{D}}: Simp (X) \to A _{\infty}-Cat  $, as
% constructed geometrically in
% this section, a \emph{geometric functor} to emphasize the origin.
% \subsection {Unital replacement of $F$} \label{sectionReplacement}
% Let $A _{\infty} -Cat ^{unit}$ denote the subcategory of $A _{\infty}-Cat $
% consisting of strictly unital $A _{\infty} $ categories and unital functors. 
% By \emph{unital replacement} for $$F: Simp (X) \to A _{\infty}-Cat $$ we mean a
% functor  $$ {F} ^{unit} : Simp (X) \to A _{\infty} -Cat ^{unit}$$ together with a
% natural transformation $$N: F \to {F} ^{unit}, $$ which is object-wise
% quasi-equivalence. 
% \begin{lemma} \label{lemma:unitalreplacement} Any functor $F: Simp (X) \to A
% _{\infty}- Cat$ has a unital replacement.
% \end{lemma}
% \begin{proof} 
% To obtain this we proceed inductively: for each 0-simplex $x \in Simp  
% (X)$, since each $F (x)$ is c-unital we may fix a formal diffeomorphism $\Phi
% _{x}: F (x) \to F (x)$, with first component maps
%    $\Phi ^{1} _{x}  $ the identity maps, such that
%    the induced $A_\infty$-structure
%    $$ {F} ^{unit}  (x) = (\Phi_x)_* (F (x))$$ is strictly unital, \cite[Lemma
% 2.1]{citeSeidelFukayacategoriesandPicard-Lefschetztheory}. Let $$N _{{x}}:
%   {F} (x) \to  {F} ^{unit}  (x)$$
%   denote the induced $A _{\infty}$ functor.  Let $F _{k} $ denote the 
%    restriction of $F$ to $Simp ^{\leq k}  (X)$ with $Simp ^{\leq k} (X)$ denoting the sub-category of $ Simp (X)$, consisting
% of simplices whose degree is at most $k$.
% And 
% suppose that the maps $N_{{x}}$ can be extended to a natural transformation $N
% _{k}: F _{k} \to F ^{unit}_{k} $ of functors $$F _{k}: Simp ^{ \leq k} (X) \to A _{\infty} -Cat,$$ $${F}
%    ^{unit} _{k} : Simp ^{\leq k} (X) \to A _{\infty} -Cat ^{unit},$$ 
% $k>0$ with the following property.   $$\forall
%    \Sigma:  N _{k} (\Sigma):  {F}  (\Sigma)
% \to  {F}  ^{unit}   (\Sigma)$$ is induced by a formal diffeomorphism $\Phi
% _{\Sigma}: F (\Sigma) \to F (\Sigma) $, whose first component maps are the
% identity maps.
%
%
%
%
% We construct  an extension $N _{k+1}$.
% For each given 
% $\Sigma ^{k+1}: \Delta ^{k+1} \to X$ and $i: \Sigma ^{k} \to \Sigma ^{k+1}  $,
% a morphism in $Simp (X)$,
% by assumption $F (i) $ is a fully-faithful embedding. Identifying 
% $F  (\Sigma ^{k} ) $ with a full subcategory of $F (\Sigma ^{k+1} )$,
% we may clearly construct, as in the proof of \cite[Lemma
% 2.1]{citeSeidelFukayacategoriesandPicard-Lefschetztheory}, a formal diffeomorphism $$\Phi _{\Sigma ^{k+1}}: F
% (\Sigma ^{k+1} ) \to F (\Sigma ^{k+1} )$$ 
% with $\Phi ^{*} _{\Sigma ^{k+1} }  (\Sigma ^{k+1}   ) $ unital, 
% and so that its restriction to $F (\Sigma ^{k} )$ coincides with the formal diffeomorphisms
% $\{\Phi _{\Sigma ^{k+1} \circ i} \}$, for each $i: \Sigma ^{k} \to \Sigma ^{k+1}
% $.
% The result then follows.  
%  \end {proof} 
% Let us write $F ^{unit}  $ for the particular unital
% replacement of $F $
% as constructed in the proof of the lemma above.
% % , which still has
% % the property that all morphisms are taken to fully-faithful embeddings
% % The property of being a pre-$\infty$-functor is crucial for some algebraic consideration.
% % This notion may seem at the moment somewhat ad hock, but will become crucial in
% % part III, where it will be important to work directly with the
% % quasi-category $A _{\infty}-Cat _{\bullet}  $, see also the discussion in the
% % introduction on connections to Toen's derived Morita theory.
% %  For a pre-$\infty$-functor
% % coming from the analytic construction of section \ref{section:construction} we
% % may also use the name \emph{analytic pre-$\infty$-functors} to emphasize the origin.
% % On the other hand given a degeneracy morphism $\Sigma ^{m} \to
% % \Sigma ^{n}$, with $n <m $, by construction of $F (\Sigma ^{m}) $, 
% %  there is a canonical identification isomorphism $F (\Sigma ^{m}) \to F (\Sigma ^{n})$.  
% % As 
% % On the
% % other hand given a commutative diagram 
% %  \begin{equation*} 
% % \xymatrix {\Delta ^{n} \ar [rd] ^{\Sigma}\ar[r]& \Delta ^{n-1} \ar[d]
% % ^{\Sigma ^{n-1}}\\ & X},  \\
% % \end{equation*}
% % $\Sigma$ is degenerate and by our construction there is a tautological morphism
% % $F (\Sigma ^{n}) \to F (\Sigma ^{n-1}).$ The functoriality of $F$ is then evident
% % from construction.
% %  \begin{proposition} The functor $F _{P, \mathcal {D}}: Simp (X) \to A
% %  _{\infty}-Cat$ constructed above is a pre-$\infty$-functor as defined in Section
% %  \ref{section.algebraic}. 
% %  \end{proposition}
% % \begin{proof} This follows by the proof of Proposition 
% % \ref{propostion.simplicialmap}. 
% % \end{proof} 
% % \begin{remark} A different way to obtain a unital repl
% %    
% % \end{remark}
% % \subsection {Naturality} 
% % Given a smooth embedding $f: Y \to X$ and $M \hookrightarrow P \to X$ a
% % Hamiltonian bundle as before, there is an induced functor $$f _{*}: Simp (Y ) \to Simp (X
% % ).
% % $$ And consequently there is an associated pullback functor $$f ^{*} F _{P, \mathcal {D}}: Simp (Y) \to A_{\infty}-Cat,$$ where $F _{P, \mathcal {D}}: Simp (X) \to A_{\infty}-Cat$ is the geometric functor as above. On the other hand we may pullback by $f$ the bundle
% % as well as the perturbation data $ \mathcal {D}$ to get another
% % functor $F _{ f ^{*} P, f ^{*}\mathcal {D}}: Simp (Y)
% % \to A_{\infty}-Cat$. The following is immediate from construction. 
% % \begin{lemma}  \label{natural.1} $F _{ f ^{*} P, f ^{*}\mathcal {D}} = f ^{*}
% % F _{P, \mathcal {D}}$.
% % \end{lemma}
% \subsection {Concordance classes of functors $F: Simp (X) \to A _{\infty}-Cat$}
% % Before we proceed to global Fukaya category, we note that there is a more
% % fundamental invariant that we may assign to $M \hookrightarrow P \to X$. This is
% % The \emph{concordance class} of the pre-$\infty$-functor $F _{P, \mathcal {D}}:
% % Simp (X) \to A_{\infty}-Cat$, defined as follows. 
% We say that a pair of functors $F_0, F_1: Simp (X) \to
% A_{\infty}-Cat $ are
% \emph{concordant} if there is a functor $$T: Simp (X \times I) \to
% A_{\infty}-Cat  ,$$ restricting to $F _{0},F _{1}  $ over $Simp (X \times \{0\})$,
% respectively over $Simp (X \times \{1\})$.
% Note that
%  by the proof of Lemma \ref{lemma:unitalreplacement} if $F _{1}, F _{2}  $ are
%  concordant then so are $F _{1} ^{unit}, F _{2} ^{unit}    $.
% %  this can be made compatible
% %  with unital replacements, and so we would immediately obtain the required
% %  concordance.
% % We say that a pair of functors $F_0, F_1: Simp (X) \to A_{\infty}-Cat$ are
% % \emph{concordant} if there is a functor $T: Simp (X \times I) \to A_{\infty}-Cat
% % $, natural isomorphisms $N_i: F_i \to T| _{Simp (X \times \{i\})}$. We expect
% % that concordance implies quasi-concordance but we will not need this. 
%  \begin{theorem} \label{thm.concordance} Let $M
%     \hookrightarrow P \to X$ be a smooth Hamiltonian
%     fibration. For a given pair of data $\mathcal{D}
%     _{1}, \mathcal{D} _{2}  $ for $P$, the 
%     functors $$F _{P, \mathcal {D} _{1} }: Simp (X) \to A_{\infty}-Cat, $$
%     $$F _{P, \mathcal {D} _{2}}: Simp (X) \to A_{\infty}-Cat $$ are concordant.
%  \end{theorem}
% \begin{proof}  
% The pair $\mathcal{D} _{1}, \mathcal{D} _{2}$  are
% concordant by Theorem \ref{thm:concordanceMain}.
%    Let $\widetilde{\mathcal{D} } (P \times I)  $
%    denote the corresponding data. Then clearly 
% $$F _{P \times I, \widetilde{\mathcal{D} } (P
%    \times I)}: Simp (X) \to A_{\infty}-Cat, $$
% gives the required concordance.   
% % The pair $ \mathcal {D}_1, \mathcal
% % {D}_2 $ gives us partial data
% % $\widetilde{\mathcal{D} }  =
% % (\widetilde{\mathcal{U} },
% % \widetilde{\mathcal{F} }) $  for the
% % Hamiltonian fibration $\widetilde{P}=  
% % P \times I \to X \times I$.   Here by partial
% % we mean that it is defined for all
% % $\Sigma: \Delta^{n} \to X \times I$ with
% % image in $X \times \partial I$.    
% % 
% % 
% %    Extend
% % $\widetilde{\mathcal{D} } $  in any way 
% % to a complete perturbation data $\mathcal{H} $
% % for $P$.  To see that such an 
% % extension exists, first note 
% % extension of $ \widetilde{\mathcal{U} }  $
% % over $X \times I$  
% % exists by the same inductive argument as in the
% %    proof of Theorem \ref{lemmanaturalmaps}.
% %    Likewise an extension of
% %    $\widetilde{\mathcal{F} }$ over $P \times
% %    I$ exists by the proof of Lemma
% %    \ref{lemma:naturalF}.    \textcolor
% %    {blue}{add general U} 
% % 
% %  Consequently we may define the functor $T$ giving a concordance
% % between $$F _{P, \mathcal{D} _{1} }, F _{P, \mathcal{D} _{2} }  $$ to be the geometric
% % functor   $$F
% % _{\widetilde{P}, {\mathcal{H} }   }: Simp (X \times I) \to A _{\infty}-Cat.
% % $$
% % 
% %
% %    $\{\mathcal{D} _{t} \}$ 
% %  
% %  we may clearly construct perturbation data
% %  $\mathcal{D}$ for the construction of:
% %  \begin{equation*}
% %  {F _{\mathcal{D}}}: Simp (X \times I) \to A _{\infty}-Cat ^{} , 
% %  \end{equation*}
% %  which extends the perturbation data $\mathcal{D}_{i} $ over $Simp (X \times
% %  \{i\})$, as this amounts to choices of extension of Hamiltonian connections. 
% %  
% %  we construct $F _{ \mathcal
% % {D}_1, \mathcal {D}_2}: Simp (X) \to A_{\infty}-Cat$, with a pair of
% % natural transformations $N_i: F  _{ \mathcal {D}_i}  \to F _{ \mathcal
% % {D}_1, \mathcal {D}_2} $, which are object-wise quasi-equivalences. The
% % construction of $F _{ \mathcal {D}_1, \mathcal {D}_2}$ is formally identical to
% % that of functors $ F _{ \mathcal {D}}$ and so will only be sketched. As before it will be helpful to first define
% % an auxiliary category $\Pi (\Delta ^{n} \times [0,1])$ over which
% % the category $F _{ \mathcal
% % {D}_1, \mathcal {D}_2} (\Sigma)$ ``fibers'', for each $\Sigma: \Delta ^{n} \to
% % X$. \footnote {The word ``fibers'' is used here informally, however it will be
% % somewhat justified after we apply the dg-nerve construction as then we will get
% % a (co)-Cartesian fibration of quasi-categories, as we shall see.} 
% % % Fibers here just mean that we have a surjective functor $$F _{ \mathcal
% % % {D}_1, \mathcal {D}_2} (\Sigma) \to \Pi (\Delta ^{n}) \times [0,1]).$$
% %  The
% % objects of $\Pi (\Delta ^{n} \times [0,1])$ are the vertices of $\Delta ^{n} \times [0,1]$, by which we mean the points $
% % \{x_i \times 0\}$, $ \{x _{i} \times 1\}$ for  $ \{x _{i}\}$ the vertices of
% % $\Delta ^{n}$. The morphisms of $\Pi (\Delta ^{n} \times [0,1])$ are geodesic
% % paths $ [0,1] \to \Delta ^{n} \times [0,1]$ for the Euclidean metric, beginning
% % and ending on the corresponding points. The composition map as before is completely
% % determined as there is a unique morphism between any pair of objects. Then the
% % objects of $F _{ \mathcal
% % {D}_1, \mathcal {D}_2} (\Sigma)$ are oriented Lagrangian submanifolds in the
% % fibers of the pull-back by $$ \Delta ^{n} \times [0,1] \xrightarrow{\pi} \Delta
% % ^{n} \xrightarrow{\Sigma} X$$ of $P$, over vertices of $ \Delta ^{n} \times
% % [0,1] $. The hom spaces are defined analogously by fixing a Hamiltonian
% % connection on $m ^{*} \pi ^{*} \Sigma ^{*} P$ for each $m \in hom (\Pi
% % (\Delta ^{n} \times [0,1]))$, and taking flat sections with Lagrangian boundary
% % conditions. And so on as before, except that we want the Hamiltonian
% % perturbation data to match over the pair of canonical embeddings of $\Pi (\Delta ^{n})$ into $\Pi
% % (\Delta ^{n} \times [0,1])$, with the data used to obtain $F _{ \mathcal
% % {D}_i} (\Sigma)$, so that we have canonical natural transformations  
% %  $F _{ \mathcal
% % {D}_i} \to F _{ \mathcal{D}_1, \mathcal {D}_2} (\Sigma)$. These natural
% % transformations are by construction object-wise fully-faithful embeddings and
% % essentially surjective on the cohomological level, and thus object-wise quasi-equivalences.
% \end{proof}
% \begin{remark} 
% \label{sub:homotopy_groups}
% Concordance
% relation is an equivalence relation (in the special case above). Although we will not show this here. The concordance
% class of the functor $F _{P, \mathcal{D}} $ is
%    then the most fundamental invariant of the
%    Hamiltonian fibration $P$ that is constructed in this paper, however calculating with it may be very
% difficult.
% \end{remark}
%
%
%
%
% % subsection homotopy_groups (end)
%
%
% % \textcolor {red}{to be done}
% % The reader may note that for our construction we needed a lot less then an
% % actual smooth Hamiltonian bundle $M \hookrightarrow P \to X$. The data we need
% % is a simplicial Hamiltonian fibration 
%
% %  The natural deformation retraction of $\Delta ^{n}$ onto the $x
% % _{0}, x _{d}$ edge $a _{0,d}$ (or corner if these are not distinct), induces a
% % homotopy $h: I ^{2} \to \Delta ^{n}$, $h| _{ [0,1] \times \{1\}}=a _{c}$, $h|
% % _{ [0,1] \times \{0\}}=a _{0,d}$. The fibration $h ^{*} \Sigma ^{*} P$ is
% % essentially what we are looking for, except that for illustrative convinience
% % we convert this to a fibration over $pol  _{c}$, by collapsing the fibration
% % over $ \{0\} \times [0,1]$, and $ \{1\} \times [0,1]$ (it is trivialized there)
% % and bending the edges as in figure \ref{figure.polc}. 
% % \textcolor{red}{add figure}
% % We now have to explain how to extend this construction over the whole $ \mathcal
% % {R} _{5}$ 
%
%
%
% % is defined by fixing an embedding $i: \Delta ^{2} \to S _{2}$, where  $S
% % _{2}$ denotes a disk with three punctures on the boundary, with boundary
% % components labeled  by $L _{i}$. Also fix a
% % retraction $ret: S_2 \to \Delta ^{2}$, so that the labeled boundary components are retracted to
% % corresponding corners of $\Delta ^{2}$, and so that the ends of $S _{2}$ retract
% % onto the edges in a controlled way at infinity. (More precisely in the strip
% % like coordinate charts at the ends, we want the retraction to correspond to  the
% % natural projection.)
% % , as in definition of $ret _{r}$ previously.
% %   Pick a suitably
% % generic Hamiltonian extension $ \mathcal {A} (\Sigma, \{L_i\})$ of the
% % connections $ \mathcal {A} (L_0, L_1), \mathcal {A} (L_1, L_2), \mathcal {A} (L_0, L_2)$ to $\Sigma ^{*} P$.
% % This gives a connection on $ ret ^{*} \Sigma ^{*} P$, with appropriate
% % asymptotics and we may   count isolated holomorphic sections of $  ret ^{*} \Sigma
% % ^{*} P$ asymptotic to generators of $hom (L_0, L_1), hom (L_1, L_2), hom
% % (L_0,L_2)$, under identifications. The higher composition maps are defined
% % similarly.
% %  \subsubsection {Value on a loop}
% % The free loop space of $X$ naturally maps into the morphism space from $\emptyset \to \emptyset$. For a loop $m: S ^{1}
% % \to X$, $F (m) = CF (m ^{*} P, \mathcal {A} (m))$, which is the Floer chain
% % complex generated by flat sections of $( \widetilde{m} ^{*} P, \mathcal {A}
% % (m))$, defined much as for intervals, but now by counting $ \mathbb{R}$-reparametrization classes of 
% % holomorphic sections for the translation invariant extension of the connection
% % $( \widetilde{m} ^{*} P, \mathcal {A} (m))$ to $ \widetilde{m} ^{*} P \times
% % \mathbb{R} \to S ^{1} \times \mathbb{R}$.
% 
% %  Or in mathematical terms we want to consider $Top$ enriched
% % tensor functors from the $Top$ enriched path groupoid of $X$ to the $Top$
% % enriched groupoid of Hermitian vector spaces and isometries between them. 
% % 
% % Given a Hermitian vector bundle $E \to X$, we have the
% % associated classifying map $X \to BU$. 
% % 
% % Given a Hamiltonian bundle $M \hookrightarrow P \to X$, and a suitable
% % symplectic manifold we get  collection of $A _{\infty}$-categories 
% % 
% % 
% % Kevin Costello in \cite{Costello.calabi} showed that every
% \section {Global Fukaya category} \label{section.GF}
% Let $M \hookrightarrow P \to X$ be as previously. In
% this section, we will associate to the
% previously constructed functors
% $F _{P, \mathcal{D}} $  a certain
% geometric-categorical object
% which we call the global Fukaya category. More
% specifically this
% will have the structure of an $\infty$-fibration
% over $X _{\bullet }$, which is our name for a
% categorical fibration over a Kan complex. 
% So one necessary ingredient for this story will
% be the notion of an $\infty$-category, or a
% quasi-category in the specific model here. As this
% model is fixed in the paper we will no longer
% mention this. An $\infty$-category is a simplicial set with an additional property, relaxing the notion of Kan complex.  
% % The latter
% % are fibrant objects in the Quillen model structure on the category
% % $sSet$ of simplicial sets, and play the same role in the category of simplicial
% % sets as CW complexes play in the category of topological spaces: they are the
% % fibrant objects in the corresponding Quillen equivalent model
% % categories.
% Whereas Kan complexes are fibrant objects in the Quillen model structure on the category
% $sSet$ of simplicial sets,  
% $\infty-categories$ are in turn the fibrant objects for a different non Quillen
% equivalent model structure on $sSet$ called the Joyal model structure. 
% For the reader's convenience we will review some of
% this theory of simplicial sets in the Appendix \ref{appendix.quasi}.
% 
% % We call this the Global Fukaya category.
% % as it is an 
% % object associated to $P$ which contains (in general only part of) the information of the category
% % $Fuk (M)$.
% % Using this we shall do our computation in part II.
% % \textcolor{blue}{ellaborate} 
%
% We will see in Section \ref{section:extension} how
% to enrich our construction so that our geometric
% functors $F _{P, \mathcal{D} }$ extend to functors 
% $$F _{P, \mathcal{D}}: \Delta (X) 
% \to A_{\infty}-Cat ^{unit}, $$ in other words so
% that degeneracies are included. This is purely
% algebraic and we assume this for now. 
% \begin{remark}
%    \label{remark:NaiveColimit} A naive idea for an
%    invariant of the Hamiltonian fibration $M
%    \hookrightarrow P \to X$ is to try to form the
%    colimit  directly:
% $$Fuk (P, \mathcal{D}) = colim _{\Delta (X) } F _{P,
% \mathcal{D} },$$ which one may hope is an $A
%    _{\infty}$-category. However, this has great
% technical difficulties. The colimit may not even
% exist, as in general colimits of diagrams of $A
% _{\infty}$ categories may not exist.  Our category
% $A _{\infty}-Cat ^{unit}$ is a very special
% sub-category of all $A _{\infty}$ categories, so
% that such co-limits may exist (this is perhaps open). But this is not good
% enough, as we need suitable invariance of $Fuk (P,
% \mathcal{D} ) $, say up to quasi-isomorphism,    under change of
% $\mathcal{D} $, which means that in our case we
% need some kind of homotopy colimit, which means
% that our $A _{\infty}-Cat ^{unit}$ needs to be some kind
% of model category. This is again a technical
% challenge particularly because $A _{\infty}-Cat
%    ^{unit}$
% is so special.  See however ~\cite{citeLefevre-HasegawaSurlesAinftycategories} where a
% kind of model structure is constructed on a more
% general  category of $A _{\infty}$  categories,
% (but with no co-limits!). 
% \end{remark}
% We are going to compose $F$ with  the nerve
% functor to land in the much more robust category of
% $\infty$-categories, and then take the colimit. The use of the nerve functor has some perhaps unexpected benefits. We get a certain
% rich additional structure for our invariant
% object closely tied the geometry, (an $\infty$-fibration structure)
% this is crucial for computations in Part II.   
%  %
% %
% %
% %
% %
% %
% %
% %
% %
% % We shall use notion of dg or more
% % appropriately $A _{\infty} $-nerve to associate to a functor $F: X _{\bullet} \to A
% % _{\infty}-Cat  $ an $\infty$-functor into
% % $X _{\bullet} \to \mathcal{S}$, the $\infty$-groupoid of $\infty$-categories, (which is just a
% % Kan complex, i.e. a space) as discussed in the introduction. 
% % And in fact if we work with (pre)-triangulated, 
% % graded, rational Fukaya categories we claim 
% % that the above mentioned pair of $\infty$-groupoids $\mathcal{S}$ and
% % $\mathcal{A} _{\infty} $ 
% % coincide anyway, assuming Lurie's folklore theorem discussed in Section
% % \ref{section:Toen}. 
% \subsection{The $A _{\infty} $-nerve} \label{sec:OutlineAinftyNerve}
% We have already briefly discussed the $A _{\infty}
% $-nerve in the Introduction, and from now on it
% will just be called nerve $N$.  This is a construction
% generalizing the classical nerve of categories due
% to Grothendieck.  
% % is an analogue for $A _{\infty} $ categories of the
% % classical nerve functor from the category of small
% % categories to the category of 
% % simplicial sets, (in-fact $\infty$-categories).
% % $N$ 
% % is a functor from the category of all (strictly-unital) $A _{\infty} $ categories, with
% % morphisms $A _{\infty} $ unital functors, to the
% % category of $\infty$-categories
% % $\infty-\mathcal{C}at $.
% % The latter
% % are particular simplicial sets, generalizing the notion of a Kan complex, and are one model
% % for $(\infty,1)$-categories, that is weak infinity categories in which
% % $n$-morphisms with $n>1$ are invertible. 
% % More precisely Lurie
% % discusses the case of dg-categories and only indicates the case 
% % of $A _{\infty}$ categories. A complete description of the nerve
% %  construction for $A _{\infty}$ categories is contained in the thesis of Tanaka,
% %  \cite{citeLeeAfunctorfromLagrangiancobordismstotheFukayacategory},
% %  where it plays a central role, and is also carefully worked out in Faonte \cite{citeFaonteSimplicialNerve}, where a number of properties are proved. 
% % We will reproduce it here for
% % the reader's convenience.
% %
% %
% %
% %
% %
% %
% %
% %
% %
% % The total space of this (co)-Cartesian fibration or the \emph{global Fukaya
% % category} 
% % will be
% % constructed as a certain colimit in $sSet$.   
% % It should be noted that in the Lagrangian cobordism approach to Fukaya category
% % in Nadler-Tanaka
% % \cite{citeNadlerTanakaAstableinfinity-categoryofLagrangiancobordisms}
% % a stable $\infty$-category $ \mathcal {Z}$
% % is constructed directly. The category $ \mathcal {Z}$ is expected to be closely
% % related to  the nerve of the triangulated envelope of the Fukaya category.
% % \subsubsubsection {Outline of the
% % nerve construction}
% % \label{sec:OutlineAinftyNerve} 
% We want a certain natural functor $$N: A _{\infty}-Cat ^{unit} \to
% \infty-\mathcal{C}at.$$ A full construction is in
% Appendix \ref{appendix:nerve}, but here is an
% outline.
% Let ${C}$ be a strictly unital $A
% _{\infty}$ category. The 2-skeleton of the nerve $N ({C})$, has objects of
% ${C}$ as 0-simplices, morphisms of ${C}$ as $1$-simplices and the 2-simplices consist of a triple of objects $X, Y, Z$, a
% triple of morphisms $$f \in hom  _{  {C}} (X,Y), g \in hom _{ 
% {C}} (Y, Z), h \in hom _{  {C}} (X, Z),$$
% a morphism $e \in \hom  _{{C}} (X, Z) _{1}$,
% (subscript $1$ corresponds to the degree) with $de= h - f \circ g$.  
%
% % We will describe the full nerve construction in
% % the Appendix \ref{appendix.quasi} following Tanaka \cite{citeLeeAfunctorfromLagrangiancobordismstotheFukayacategory}.  
% % \subsubsection {Colimit}
% % \begin{remark} A diagram $F:\Delta/ X _{\bullet} \to A_{\infty}-Cat ^{unit}$, induces a functor $N \circ F:
% % \Delta/ X _{\bullet} \to \mathcal {C}at ^{\Delta} _{\infty}$, with the latter being
% % simplicially enriched category with objects quasi-categories, and hom object
% % $hom _{ \mathcal {C}at ^{\Delta} _\infty} ( \mathcal {C}_1, \mathcal {C}_2)$
% % the maximal Kan sub-complex of the quasi-category underlying the simplicial
% % mapping space. This in turn induces a map of simplicial sets 
% % $F _{\infty}: N (\Delta/ X _{\bullet}) \to \mathcal {C}at _{\infty}$, where $N
% % (\Delta/ X _{\bullet})$ denotes the classical nerve of 
% % the category $\Delta/ X _{\bullet}$, and $ \mathcal {C}at _{\infty}$ denotes the
% % quasi-category of quasi-categories. More specifically the latter is the simplicial nerve of $
% % \mathcal {C}at _{\infty} ^{\Delta}$. There is then  an $\infty,1$-categorical
% % colimit $$colim _{N (\Delta/ X _{\bullet})} F _{\infty} \in \mathcal {C}at _{\infty},$$
% % \cite[chapter 4]{citeLurieHighertopostheory.}. The above colimit is not actually unique but all such
% % colimits form a natural contractible Kan subcomplex of $ \mathcal {C}at
% % _{\infty}$. For the pre-$\infty$-functor $F$ particular representatives of the colimit
% % represent equivalence classes for our global Fukaya category.
% %  Despite the formal elegance of the above description, 
% % it is a little awkward to deal with a whole space of possible colimits, even if
% % it is contractible, moreover to show invariance of the equivalence class of the
% % colimit with respect to deformations of auxiliary data, to connect it to colimits which are likely more familiar, and to have a more computable form, 
% %  it will be helpful to express the colimit using the language of model
% %  categories and homotopy colimits, which moreover will yield a ``canonical''
% %  representative. 
% % \end{remark} 
% % The category $\Delta/X _{\bullet} $ as the category of simplices of a simplicial set is
% % Reedy, with fibrant/cofibrant constants, see for instance \cite[Chapter
% % 5]{citeHoveyModelcategories.}. The functor category $sSet ^{\Delta/ X _{\bullet} } $, has the Reedy model structure
% %  induced by the Joyal model category structure on $sSet$,
% % which is reviewed in the Appendix \ref{appendix.reedy}. Weak equivalences  in
% % $sSet ^{\Delta/ X _{\bullet}} $ are precisely natural transformation, which are object-wise
% % weak-equivalences,  that is natural transformations $N$
% % so that $N (\Sigma)$ is a weak equivalence for each $\Sigma \in \Delta/ X _{\bullet}$. 
% % For the Joyal model structure weak equivalences are
% % the so-called categorical equivalences. 
% %
% % % The homotopy colimit of $N \circ F: Simp (X) \to sSet$, 
% % % in the so called global sense is the derived functor of the colimit functor for
% % % the Reedy model category structure on the  category $sSet ^{Simp (X)} $ of $Simp
% % % (X)$ shaped diagrams. 
% % % This homotopy colimit exists as $Simp (X)$ is Reedy with fibrant constants, in
% % % fact it is a directed category, which is a special case, see \cite[22.10]{DHKS},
% % %  The Reedy model category structure on $
% % % A_{\infty}-Cat ^{Simp (X)} $ has as weak equivalences,
% % %  respectively fibrations natural transformations $N$ that are object-wise weak
% % %  equivalences, respectively fibrations, cofibrations in general are not
% % %  object-wise. 
% % The homotopy colimit is the total left derived functor of $$colim: 
% % {sSet} ^{\Delta/ X _{\bullet}} \to {sSet},$$
% % \begin{equation*} hocolim = \textbf{L}colim: \ho   {sSet} ^{\Delta/ X _{\bullet}}  \to \ho  {sSet}.
% % \end{equation*}
% % The most important part of the above  description is that
% % hocolim is in particular a homotopy functor, which is needed for the last part
% % of Theorem \ref{prop.quasicat}.
% %  
% % This derived functor exists see Theorem \ref{thm.reedy}.   
% % % as for the Reedy
% % % model structure on $sSet ^{Simp (X)}$, $colim$ is part of a Quillen adjunction 
% % % \begin{equation*} colim: sSet ^{Simp (X)} \longleftrightarrow sSet: c ^{*}, 
% % % \end{equation*}
% % % This is exactly the condition of a Reedy
% % % category (here $Simp (X)$) having fibrant constants.
% % To compute $hocolim$ on a particular $G \in 
% % {sSet} ^{\Delta/ X _{\bullet}}$, we may take a Reedy cofibrant replacement and apply  the
% % usual colimit functor. Thankfully  the functors $N\circ  F _{P, \mathcal {D}}$
% % that we construct are always Reedy cofibrant, which follows from the fact that $N$ takes fully-faithful embeddings to monomorphisms in
% % $sSet$, and from the characterization of cofibrations in the Reedy model
% % structure, described in the Section \ref{appendix.reedy}.  
% %
% %
% % % For any model structure on $ A_{\infty}-Cat$ as
% % % above everything is cofibrant, moreover our $F \in A_{\infty}-Cat ^{Simp (X)} $ is also cofibrant
% % %  for the Reedy model structure, this follows immediately from Hovey \cite[Theorem
% % % 5.1.2]{}, using the fact that our simplices are all non-degenerate. Consequently
% % % the homotopy colimit for $F$ would just be the colimit. 
% % % \begin{remark} We feel that the abst might be better to take the above solution
% % % as evidence that an explicit theory of weak colimits of diagrams of $A _{\infty}$-categories can
% % % be developed as asked above, as there is too much geometric intuition to be lost
% % % with pure abstraction. We hope this will be partly illustrated by the example of
% % % the global Fukaya category of a Hamiltonian fibration over $S ^{1}$, which is
% % % coming up soon.
% % % \end{remark}
% %
% % % their associated category of $A_\infty$-modules. A functor $F: Simp (X) \to A_{\infty}-Cat$ induces a functor, 
% % % $F ^{mod}: Simp (X) ^{op} \to dg-cat$ the category of $dg$-categories by sending
% % % an $A _{\infty}$-category $A$ to $A-mod= Funct (A, Ch _{dg})$ the dg-category of
% % % cohomologically unital $A _{\infty}$-functors to the differential graded
% % % category of chain-complexes. The category $dg-cat$ is an actual model category,
% % % and the category of functors $dg-cat ^{Simp (X) ^{op}}$ also has a model
% % % category structure called the Reedy model structure. This is because the
% % % category $Simp (X) ^{op}$ is an inverse category with cofibrant constants. 
% % % \subsection {Yoneda embedding of the diagram: $F: Simp (X) \to A_{\infty}-Cat$}
% % % The $A _{\infty}$ Yoneda embedding $ \mathcal {Y}$ takes an $A
% % % _{\infty}$-category $A \in A_{\infty}-Cat$ to a dg-category $ \mathcal {Y} (A)
% % % \subset mod (A)$, with the right hand side denoting the dg-category of c-unital
% % % $A _{\infty}$-functors 
% % % \begin{equation*} A ^{op} \to Ch _{dg} ( \mathbb{F}_2),
% % % \end{equation*}
% % % where $Ch _{dg} ( \mathbb{F}_2)$ denotes the dg-category of chain complexes over
% % % $ \mathbb{F}_2$.
% %
% %
% %
% % % , i.e. this is functor $hocolim$ making the diagram
% % % below commute up to a natural 
% % % \begin{equation*}  
% % % \xymatrix{Funct (Simp (X), A_{\infty}-Cat) \ar [d] \ar[r]   &
% % % A_{\infty}-Cat \ar [d]
% % % \\ 
% % % ho(Funct (Simp (X
% % % _{\bullet}), A_{\infty}-Cat)) \ar [r] ^{} & ho (A_{\infty}-Cat)},
% % % \end{equation*}
% % % with top and bottom arrows $colim$, $hocolim$, and vertical arrows the
% % % localization functors. 
% % % We will explicitely describe one such homotopy colimit a
% % % bit further on.
% \subsection {Definition of the global Fukaya
% category}
% \label{sec:GlobalFukaya}
% \begin{definition}  We define:
% \begin{equation*} Fuk _{\infty} (P, \mathcal {D})
%    :=    colim _{\Delta (X)
%    } N \circ F ^{unit} _{P, \mathcal{D} }  \in sSet.
% \end{equation*}  
% % where the second equivalence is due to $N \circ F _{P , \mathcal {D}}$
% % being Reedy cofibrant as mentioned above.
% \end{definition} 
% % The above colimit is easy to describe explicitly. The set of $k$-simplices of
% % $colim _{Simp (X)} N \circ F _{P, \mathcal {D}}$ is the set $\sqcup _{\Sigma
% % \in }$
% An explicit construction of the colimit is given in Lemma \ref{lemma.colimit}.  
% In principle the above definition could be very impractical since general
% objects in $sSet$ are difficult to deal with, while taking fibrant replacements
% for the Joyal model category structure
% could obfuscate all the original geometry contained in the Fukaya category.
% Thankfully none of this is necessary as we have a couple of miracles coming from the underlying 
% geometry to save us.  The content of these
% miracles is the
% following theorem, to be proved in Section \ref{section.algebraic}.
% % More specifically, in the case of our geometric functors the above
% % colimit  is always a quasi-category, and moreover
% % has a certain fibration property.
%
% \begin{theorem} \label{prop:quasicatfibration} As defined $Fuk _{\infty} (P,
% \mathcal {D}) \in \infty-\mathcal {C}at$, i.e. is
%    a $\infty$-category moreover
% there is a natural $\infty$-fibration $$N
%    (Fuk (M, \omega )) \hookrightarrow Fuk
% _{\infty} (P, \mathcal {D}) \to X _{\bullet},$$
%    whose  concordance equivalence (Definition
%    \ref{def:concordanceCartesian})  class
% % in the over category $sSet/X _{\bullet} $ 
%    is independent of the choice of $
% \mathcal {D} $.
% \end{theorem}
% %  \begin{remark} $Fuk _{\infty} (P, \mathcal {D})$ is a representative for
% %  $$colim _{N (Simp (X))} F _{P, \mathcal {D}, \infty} \in \mathcal {C}at
% %  _{\infty},$$ where $$F _{P, \mathcal {D}, \infty}: N (Simp (X)) \to \mathcal
% %  {C}at _{\infty}$$ is the map of simplicial sets induced by
% %   $F _{P, \mathcal {D}}$.  This follows by proof of the Corollary 4.2.4.8 in
% %   Lurie \cite{citeLurieHighertopostheory.}, we don't elaborate on this as it will not play any
% %   practical role. 
% % \end{remark}
% % \begin{proof} This follows immediately by Lemma \ref{natural.1}. 
% % \end{proof}
% % \subsection {Connection with Hochschild cohomology groups} \label{sectionFukayaHochschild}
% % Given an $A _{\infty} $ category $C$ we define certain groups $HH _{geom}
% % ^{2-i} (C)   $, as $\pi _{i} (\infty-\mathcal{C}at, NC) $. 
% % We have outlined the
% % relationship of these group with the classical Hochschild cohomology in the
% % introduction.
% \subsection {Universal construction via diffeological
% spaces} \label{sectionFukayaHochschild}
% Let $E _{M} \to BHam (M, \omega) $ be the associated Hamiltonian
% $M$-bundle $$E _{M} = E \times _{Ham (M,  \omega)
% } M,$$ for $E$   
% the universal principal $Ham (M,
% \omega)$-bundle $E \to BHam (M,  \omega) $.
% $B Ham (M,
% \omega)$ or the classifying space of any smooth Lie group, based on Milnor's
% construction \cite{citeMilnoruniversalbundles}, 
% admits a well-defined notion of smooth maps into it from smooth manifolds. To be precise it has a natural diffeology, see Magnot-Watts~\cite{citeMagnoWattsDiffeology} and likewise the universal $M$-bundle $$E _{M}  \to BHam (M,
% \omega)$$ has a natural diffeology, so that for a diffeological map $$f: B \to BHam (M,
% \omega)$$ the pull-back bundle $f ^{*} E _{M}  $ is naturally diffeological. If
% $B $ is in addition a smooth dimension $k$ manifold then $f ^{*} E _{M}  $ is a diffeological
% space locally (diffeologically) diffeomorphic to
% $U \times M$,  for $U \subset \mathbb{R} ^{k} $. 
% But the
% latter is clearly locally diffeomorphic to $\mathbb{R} ^{k+2n} $, for $2n$ the dimension
% of $M$. Thus $f ^{*} E _{M}  $ is a diffeological space locally diffeomorphic to
% $\mathbb{R} ^{k+2n} $ and hence is a smooth manifold. More formally, $f ^{*} E _{M}  $ is contained in the
% full subcategory of the category of diffeological spaces corresponding to smooth
% manifolds.
% 
%
% In the case $B = \Delta ^{k}
% $  is the $k$-simplex, and given an open $U$ with $\Delta ^{k} \subset U \subset \mathbb{R} ^{k}
% $, by a smooth map $\Sigma: \Delta ^{k}  \to BHam
% (M, \omega)$ we mean a map with a diffeological extension
% $\widetilde{\Sigma }: U  \to BHam (M, \omega)$, with $U$ given the 
% diffeology induced from $\mathbb{R} ^{k} $ .  
% Then we may conclude as above that $\widetilde{\Sigma }  ^{*} E _{M}  $ is naturally a smooth bundle.  
%
% So to each smooth in the sense
% above map $\Sigma:
% \Delta^{k} \to BHam (M,  \omega)  $  we have a
% naturally corresponding smooth bundle $\Sigma
% ^{*}  E _{M}$ over $\Delta^{k} $. The construction of the
% previous section then works as before, associating
% to $\Sigma $ an $A _{\infty}$ category $F _{E
% _{M}, \mathcal{D}} (\Sigma
% ) $.   We may then define $B Ham (M,  \omega)
% _{\bullet } $ as the simplicial set, with $B Ham (M,  \omega)
% _{\bullet } (k) $   the set of
% diffeological, collared maps $\Sigma: \Delta ^{k}
% \to BHam (M, \omega)$. $B Ham (M,  \omega)
% _{\bullet } $ is readily seen to be a Kan complex.
%
%
%
% \begin{proposition}
%    \label{prop:functorSimpBham} There is a natural 
% functor 
%    $$F _{E
% _{M},\mathcal{D} }:
% Simp(BHam (M,  \omega)) \to A _{\infty} -Cat
%    , $$
% and so an induced functor
%    $$F ^{unit} _{E
% _{M},\mathcal{D} }:
% \Delta(BHam (M,  \omega)) \to A _{\infty} -Cat
%    ^{unit}, $$
% for $Simp(BHam (M,  \omega))$ the category of 
% smooth (diffeological simplices) as above.
% \end{proposition}
% The proof is omitted since this is just a
% summary of what we have already discussed. 
% \subsection {Universal construction via smooth
% simplicial sets}
% \label{section:SmoothSimplicialSet}
% A more abstract but technically more elementary
% approach to the universal construction can
%    be extracted from 
% ~\cite{citeSavelyevSmoothSimplicial}. There an
% abstract Kan complex 
% $BG ^{\mathcal{U} } _{\bullet} $ \footnote {The
% notation in ~\cite{citeSavelyevSmoothSimplicial}
% omits $\bullet$ subscript.}  is
% constructed for any Frechet Lie group and for each choice of a particular
% Grothendieck universe $\mathcal{U} $.  The  
% Kan complex $BG ^{\mathcal{U} } _{\bullet} $ 
% has a certain additional structure called a smooth
% structure. Concretely, this smooth structrure implies that for every
% $k$-simplex $\Sigma \in BG
% ^{\mathcal{U} } _{\bullet } (k) $, there is a
% canonically associated smooth $G$-fibration $P
% _{\Sigma } \to \Delta^{k} $. Using this, we
% immediately 
% obtain
% a functor using our construction:
% \begin{equation}
%    \label{eq:Fabastract}
%    F _{E _{M}, \mathcal{D} }:
% Simp(BHam (M,  \omega) ^{\mathcal{U} } _{\bullet
%    }) \to A _{\infty} -Cat,
% \end{equation}
% where  $Simp(BHam (M,  \omega) ^{\mathcal{U} } _{\bullet
%    })$ denotes the simplex category of the
% simplicial set $BHam (M,  \omega) ^{\mathcal{U} } _{\bullet
%    }$, cf. Section
%    \ref{section:simplexCategory}.    
%    Now, as a particular case of 
%    ~\cite[Theorem
%    7.5]{citeSavelyevSmoothSimplicial},  $$|BHam (M,
% \omega) ^{\mathcal{U} } _{\bullet }| \simeq BHam
% (M,  \omega)  $$  for $|\cdot|$  the geometric
% realization, and $\simeq$ homotopy equivalence.  
% In particular, to prove  Theorems
%    \ref{thmInfinityUniversal},
%    \ref{thmGroupHomoIntro} we may also start with
%    \eqref{eq:Fabastract}. See the proof below. One
%    advantage of this simplicial approach is that
%    the connection with homotopy groups becomes
% elementary. 
%
% % Now, formally all our constructions  are
% % based on these smooth bundles. So that the construction of $Fuk _{\infty}(P _{U} )  $ proceeds exactly the same way  if we define the category of smooth simplices
% % in $BHam (M, \omega)$ to be the category of diffeological simplices.
% % In particular using Theorem \ref{prop:quasicatfibration} for the universal fibration $P _{U} $ we obtain the proof of Theorem \ref{thmInfinityUniversal} of the Introduction.  % Given a class in $\pi _{i-1} (Ham (M, \omega, \id)$ via the
%    % clutching construction, we get a Hamiltonian
%    % equivalence class of an $M$-fibration over $S ^{i} $, with one fiber
%    % identified with $(M, \omega)$.
% \subsubsection* {Proof of Theorems
% \ref{thmInfinityUniversal},
% \ref{thmGroupHomoIntro}}
% \label{sec:proofTheoremsMain} % the group of equivalence classes of (co)-Cartesian
% % fibrations over $S ^{i} _{\bullet}  $ with fibers equivalent to $NC$, and with
% % one fiber identified with $NC$, see the discussion following Corollary
% % \ref{corollaryStraightening}.
% % \begin{theorem} \label{thmGroupHomo} There is a natural (and later in a specific
% %    case shown to be non-zero) group homomorphism 
% %    $$ \pi _{i-1} Ham (M, \omega) = \pi _{i} BHam (M, \omega) \to HH ^{geom} _{i-2} (N Fuk (M)).
% %    $$
% % \end{theorem}
% % \begin{proof} 
% Given a smooth Hamiltonian fibration $M \hookrightarrow P
%    \to X$, by Theorem \ref{prop:quasicatfibration}
%    we obtain a well-defined concordance class of
%    an $\infty$-fibration $Fuk
%    _{\infty} (P) \to X _{\bullet}$. By Theorem
%    \ref{corollaryStraightening} this is
%    classified by a homotopy class of a map:
%    \begin{equation*}
%       cl _{P}: X _{\bullet} \to (\mathbb{S}, N
%       Fuk (M,  \omega)).   
%    \end{equation*}
%   
%   
%   
%    Likewise, given the functor  $$F ^{unit} _{E
%    _{M}, \mathcal{D}}:
% \Delta (BHam (M,  \omega)) \to A _{\infty} -Cat
%    ^{unit}, $$
% by Theorem \ref{prop:quasicatfibration} we
%    obtain an  $\infty$-fibration 
%  \begin{equation}
%       \label{eq:universalCocartesian}
%       N
%    (Fuk (M, \omega )) \hookrightarrow Fuk
% _{\infty} (E _{M}, \mathcal {D}) \to BHam (M,  \omega)  _{\bullet}.
%    \end{equation}
%    By Theorem \ref{corollaryStraightening},
%    there is then a uniquely
%    determined (simplicial) homotopy class of
%    the ``classifying'' simplicial map 
%    \begin{equation*} 
% cl = cl (Fuk _{\infty} (E _{M})  ): B Ham (M,
%       \omega) _{\bullet}    \to (\mathbb{S}, NFuk (M)),
%    \end{equation*}
%    of the $\infty$-fibration
%    \eqref{eq:universalCocartesian}.  Then we
%    obtain a group homomorphism  of simplicial
%    homotopy groups $$cl _{*}: \pi _{i} (B Ham (M,
%    \omega) _{\bullet}, x _{0})   \to \pi
%    _{i} (\mathbb{S}, NFuk (M)).
%    $$
%   If we knew that $B Ham (M,  \omega) _{\bullet}$  is weakly equivalent to the usual continuous
%    singular set of $B Ham (M,  \omega) $,  then we
%    would obtain a group homomorphism
% $$cl _{*}: \pi _{i} (B Ham (M,
%    \omega), x _{0})   \to \pi
%    _{i} (|\mathbb{S}|, NFuk (M)).
%    $$ This is probably true, but I don't know if a
%    ready reference exists.
% % See ~\cite{citeOhTanakaSmoothType} where this kind
% %    of weak equivalence is defined.  
% Alternatively, we can use the map $$cl: B Ham (M,
%    \omega) ^{\mathcal{U}} _{\bullet}   \to
%    (\mathbb{S}, NFuk (M)),
%    $$
% induced by \eqref{eq:Fabastract}.  Since $|BHam (M,
% \omega) ^{\mathcal{U} } _{\bullet }| \simeq BHam
% (M,  \omega)  $ we immediately  obtain the
% homomorphism
% $cl _{*}: \pi _{i} (B Ham (M,
%    \omega), x _{0})   \to \pi
%    _{i} (|\mathbb{S}|, NFuk (M)).
%    $ 
% And this fully proves Theorem \ref{thmGroupHomoIntro}.
%
% Now if $M \hookrightarrow P \to X$  is a smooth
%    Hamiltonian fibration then $P
%    \simeq f _{P} ^{*}E _{M}$ for some
%    diffeological smooth map $f _{P}: X \to B Ham
%    (M,  \omega) $.   Then by Theorem \ref{thmNaturality} 
% $$Fuk _{\infty} (P) = f _{P, \bullet} ^{*} Fuk _{\infty} (E
%    _{M}), $$  
%   with $f _{P, \bullet }: X _{\bullet } \to B Ham
%    (M,  \omega) _{\bullet }$ denoting the induced
%    simplicial map.  
%    In particular $Fuk _{\infty} (P) $
%    is classified as $\infty$-fibration by the map
%    $cl \circ f _{P}$.    And so by Theorem
%    \ref{corollaryStraightening}  $cl _{P} \simeq
%    cl \circ f _{P}$. And so we have proved Theorem
%    \ref{thmInfinityUniversal}.  (We could also
% have proceeded via smooth simplicial sets for this
% part.) 
% % Similarly given $F ^{unit}: \Delta /B Ham (M, \omega) _{\bullet} \to A
% %    _{\infty}-Cat ^{unit}$ induced by $P _{U} $, we have an induced continous
% %    map: 
% %    \begin{equation*}
% %       B Ham (M, \omega) \simeq |\Delta /B Ham (M, \omega) _{\bullet}| \to |(A
% %       _{\infty}-Cat, Fuk (M))|,
% %    \end{equation*}
% %    with the latter as in Theorem \ref{thmToen}. And so an induced map 
% % $$\pi _{i} B Ham (M, \omega)   \to HH ^{2-i} (Fuk (M)),$$ by Theorem \ref{thmToen}.
% \qed
% % The following theorem shows that only the topological type of the Hamiltonian
% % fibration is detected by the associated (co)-Cartesian fibration.
% % \begin{theorem} \label{theoremTopologicalInvariant}
% %  Let $M \hookrightarrow P _{1}  \to X$, $M
% %    \hookrightarrow P _{2}  \to X$ be smooth Hamiltonian
% %    fibrations, with $X$ a smooth
% %    finite dimensional manifold. Suppose that there is a continuous Hamiltonian bundle map $P _{1}
% %    \to P _{2}  $. Then $$[cl (P _{1} )] = [cl (P _{2} )],
% %    $$
% %    with  $cl (P _{i}) $  the classifying
% %    simplicial maps $cl (P _{i}): X _{\bullet } \to
% %    (\mathcal{S}, NFuk (M, \omega ) ) $, and with  
% %    $[\cdot] $  denoting  the simplicial homotopy
% %    class.
% % \end{theorem}
% % \begin{proof} By the main result of
% %    \cite{citeDiffeologyMagnotWatts}, $P _{i} $, $i=1,2$ are
% %    classified by diffeological smooth maps $f _{i}: X \to B Ham (M,
% %    \omega) $, which we may assume to be embeddings.  We then have:
% % \begin{lemma}
% %     \label{lemma:Pi}  $$Fuk _{\infty} (P _{i} )
% %     \simeq f _{i} ^{*}Fuk _{\infty} (E _{M}) $$
% %  \end{lemma}
% %  \begin{proof}
% %    This follows immediately by Lemma
% %    \ref{natural.1}. 
% %   \end{proof}
% %    On the other hand $f _{i} $ are homotopic so  the conclusion follows.
% % \end{proof}
% \subsection {Global Fukaya category and unital replacement, a remark} 
% In the construction of $Fuk _{\infty} (P)$ we had to take a
% unital replacement for the functor $F: \Delta/X _{\bullet}  \to A _{\infty}-Cat $.
% One may worry then that this algebraic step will obfuscate the ``geometry'' of
% simplices of $Fuk _{\infty} (P) $. This is not really the case.
% First the $A _{\infty} $ nerve $NC$ of a non-unital $A _{\infty} $ category $C$ still
% exists as a semi-simplicial set,  that is as 
% a co-functor $\Delta ^{inj} \to Set$, with $\Delta ^{inj} $ the subcategory of
% $\Delta$ consisting of injective morphisms. For a unital replacement equivalence $C 
% \to C ^{unit}  $ of $C$, constructed as in Section \ref{sectionReplacement},
% we then have an induced morphism of semi-simplicial sets
% $NC \to NC ^{unit}$, which by construction induces a bijection $NC ([n]) \to NC
% ^{unit} ([n]) $, for each $[n]$. So we may think without loss of geometric information, of simplices of $NC
% ^{unit} $ in terms of simplices of $NC$. (The former just
% have an extra formal algebraic structure.)
% %
% %
% %
% %
% % And 
% % the colimit
% % $$colim _{\Delta /X
% %    _{\bullet} } N \circ F, $$ exists as a semi-simplicial set.
% %
%
%
% \section {Extending $F$ to degeneracies} \label{section:extension}
% % We first construct an abstract algebraic extension, in our geometric setting this extension also has a geometric interpretation that we later describe.
% % % denotes the simplex of $N
% % %   (\Pi (\Delta ^{n} ))$ corresponding to the length $n$ chain of morphisms
% % %   $(m_1, \ldots , m_n)$ $m_i$ is the morphism from the vertex $i-1$ to vertex
% % %   $i$.
% % Let $$F: Simp(X _{\bullet}) \to A _{\infty}-Cat ^{unit} $$ be a functor.
% % We extend this to a functor: 
% % \begin{equation} \label{eq:extension}
% % F ^{ext} : \Delta/X _{\bullet} \to A _{\infty}-Cat ^{unit},
% % \end{equation}
% % although later on we use the same name $F$ for $F ^{ext} $.
% %
% %
% % Suppose we are given a diagram 
% % % in $\Delta/N (\Pi (\Delta ^{n} ))$
% % \begin{equation*}
% % \begin {tikzcd}
% % \Delta ^{0}  \ar [r, "j+1"]  \ar [dr, "j_*"]   & \Delta ^{n+1}
% %    \ar [d, "s _{j}
% %    (\Sigma)"]   \ar [r, "pr _{j}"]  & \Delta ^{n} \ar [ld, "\Sigma"] \\
% %                                    & X, 
% % \end {tikzcd}
% % \end{equation*}
% % where 
% % $$pr _{j}: \Delta ^{n+1} \to \Delta ^{n}, \quad j \in [n] $$
% % is induced by the unique surjection $[n+1] \to [n]$, hitting $j$ twice, and
% % where $\Sigma$ is $ndc$.
% % Here $j_{*} = \Sigma \circ j $, and
% % ${j} $ also denotes the  
% % map $pt \to \Delta ^{n} $ corresponding to this vertex. 
% % % Let $m _{j} \in \Pi
% % % (\Delta ^{n+1} ) $ denote the morphism from $j$ to $j +1$.
% %
% %
% %
% % % If $s _{j}
% % % (\sigma) $ is the degenerate $n+1$
% % % simplex obtained from an $n$ simplex $\sigma \in N (\Pi (\Delta ^{n} ))$ via the projection
% % % $$pr _{j}: \Delta ^{n+1} \to \Delta ^{n}, \quad j=0,...,n $$ 
% % % obtained by collapsing the edge from $j$ to $j+1$ to
% %  Then 
% % \begin{equation} \label{eq:eqsj}
% % F   (s _{j} (\Sigma) )
% % \end{equation} 
% %   is defined to be the $A _{\infty} $ category with objects 
% % \begin{equation*}
% %    \obj  F   (\Sigma)) \sqcup \obj  F  (j _{*} ).
% % \end{equation*}
% % Note that there are then two embeddings $\obj F (j _{*} ) \to     
% % \obj F   (s _{j} (\Sigma) )$, one given by $$F(inc _{j}): \obj F (j _{*}  ) \to
% % \obj F (\Sigma),$$ $inc _{j}: j _{*}  \to
% % \Sigma  $ the  map in $Simp (X)$ corresponding to the vertex inclusion map of
% % $j$, and the other just being the tautological map $\tau$ to the summand $\obj F  (j _{*}
% % )$.
% %
% %
% %
% % The $hom$ sets are defined  by the conditions:
% % \begin{enumerate}
% %    \item  There are tautological strict, full embeddings of
% % $A_\infty$ categories
% % \begin{equation*}
% %    F  (\Sigma) \to  F (s _{j} (\Sigma)), 
% %    \quad F (j _{*} ) \to  F ( s _{j} (\Sigma)),
% % \end{equation*}
% % corresponding to the natural set embeddings:
% % \begin{align*}
% %   &  \obj  F   (\Sigma)) \to \obj F (s _{j} (\Sigma)) \\ 
% %  & \obj  F  (j _{*} ) \to \obj F (s _{j} (\Sigma)).
% % \end{align*}
% %
% % \item 
% %    \begin{equation*}
% %     hom _{F (s _{j} (\Sigma) )} (L', \tau (L)
% %  ) := hom _{F (\Sigma)} (L', F(inc _{j}) (L)),  
% %  \end{equation*}
% %  for $L'  \in \obj  F (\Sigma), $
% % % \subset obj \, s _{j} (\xi) (\sigma) $$ 
% % and  $L  \in \obj  F (j _*).$
% % \end{enumerate}
% %
% %
% % % and where $$s_j(L'), s_j (L _{\xi (j)})$$
% % % denote their images $s_j (\xi) (\sigma)$ under the canonical
% % %    embeddings above.
% % % \subset
% % % obj \, s _{j} (\xi) (\sigma).$$ 
% % The composition operations $\mu ^{d} _{F _{s _{j} (\Sigma) } }  $ are then defined so that 
% % the tautological projection $$ F (d _{j} ): \obj F   (s _{j} (\Sigma)) \to \obj F (\Sigma)$$ extends to a strict, fully-faithful $A _{\infty} $ functor, where $d _{j}: s _{j} \Sigma  \to \Sigma $ is the morphism induced by $pr _{j} $.
% %
% % % If we have a higher degeneracy in the form of a commutative diagram:
% % % \begin{equation*}
% % % \begin{tikzcd}
% % % \Delta ^{n+k}
% % %    \ar [d]   \ar [r]  & \Delta ^{n} \ar [ld, "\Sigma"] \\
% % %                                     X, 
% % % \end{tikzcd}
% % % \end{equation*}
% % % for $\Sigma$ $ndc$, we may rewrite it as a composition of simple degeneracies as above, and define 
% % % the extension  inductively using the above prescription.
% %
% % \subsection {Extension for geometric functors $F _{\mathcal{D}} : Simp (X) \to A _{\infty}-Cat ^{unit}  $}
% % When $F _{\mathcal{D}} $  is a geometric functor defined via perturbation data
% % $\mathcal{D}$, it is possible to make sense of the above algebraic extension geometrically,
% % but this necessitates slightly extending the notion of perturbation data as
% % follows. This 
% % extension will be denoted by $\mathcal{D} ^{ext} $ in what follows.
% We have to construct our perturbation data $\mathcal{D}$ for all simplexes 
% in such a way that there is a natural functor:
% \begin{equation} \label{eq:extendedF}
%    F: \Delta(X) \to A _{\infty}-Cat ^{unit},
% \end{equation}
% extending the geometric functor
% $$F _{\mathcal{D}}: Simp(X) \to A _{\infty}-Cat ^{unit},
% $$ for this data $\mathcal{D}$
% as previously constructed. This perturbation data will be referred to as \emph{extended perturbation data}.
%
% % Let $\Delta ^{n} _{\bullet, reg}   \subset \Delta ^{n} _{\bullet}     $ denote \ldots 
% Suppose that we are given a commutative diagram: 
% \begin{equation*}
% \begin {tikzcd}
% \Delta ^{0}  \ar [r, "j+1"]  \ar [dr, "
%    x _{j}"]   & \Delta ^{n+1}
%    \ar [d, "\widetilde{\Sigma} "]   \ar [r, "pr"]  & \Delta ^{n} \ar [ld, "\Sigma"] \\
%                                    & X, 
% \end {tikzcd}
% \end{equation*}
% where 
% $$pr: \Delta ^{n+1} \to \Delta ^{n}, \quad j \in [n] $$
% is induced by the unique surjection $[n+1] \to
% [n]$ taking $j$ and $j+1$  to $j$.
% Here $x _{j}  = \Sigma \circ j $, for 
% ${j} $ also denoting the  
% map $pt \to \Delta ^{n} $   whose image is the
% vertex $j$.   In particular, we have a
% morphism $pr: \widetilde{\Sigma } \to \Sigma  $
% in $\Delta (X) $. 
%
%
% Let the system of natural maps $\mathcal{U} (X) $
% be fixed throughout in what follows.  And $\mathcal{D} = (\mathcal{U} (X),
% \mathcal{F}) $. Let $\mathcal{F} _{\Sigma} $
% correspond to $\Sigma$.  We first show how to construct certain induced perturbation
% data $\mathcal{F} _{\widetilde{\Sigma}} = pr ^{*}
% \mathcal{F} _{\Sigma} $. And using this we
% construct an $A _{\infty} $ category $F
% (\widetilde{\Sigma} ) $ as previously.  
%
% % \begin{equation*}
% % \xymatrix {\Delta ^{0}  \ar [r] ^{{j+1} } \ar [dr] ^{{j _{*} } }  & \Delta ^{n+1}
% %    \ar [d] ^{{s _{j}
% %    (\Sigma)}}  \ar [r] ^{pr _{j} } & \Delta ^{n} \ar [ld]^{\Sigma} \\
% %                                    & X _{}} , 
% % \end{equation*}
% % where 
% % $$pr _{j}: \Delta ^{n+1} \to \Delta ^{n}, \quad j \in [n] $$
% % is induced by the unique surjection $[n+1] \to [n]$, hitting $j$ twice, and
% We may as before define $$\obj F 
% (\widetilde{\Sigma} ) = \bigcup _{0 \leq i \leq
% n+1} \obj F (x _{i}), \quad x _{i} =
% \widetilde{\Sigma} \circ i,$$ $i: pt \to
% \Delta^{n+1} $ the inclusion map of $i$'th vertex. And in this case $pr$ clearly induces a map of sets of objects $$pr _{*}: \obj F  (\widetilde{\Sigma} ) \to \obj F (\Sigma).
% $$ 
% We need to specify our system $\mathcal{F}
% _{\widetilde{\Sigma } }$  of connections and
% almost complex structures corresponding to
% $\widetilde{\Sigma}$. We say what to do with
% connections, the case of almost complex structures
% is analogous. Given vertices $i,j$  of $\Delta
% ^{n+1} $ and objects $L \in {F}  (x
% _{i}), L' \in {F} (x _{j}  )$,  we set $$\mathcal{A} (L, L') =  \mathcal{A} (pr _{*} L, pr _{*} L'),
% $$ where the latter connection is determined by $\mathcal{F} _{\Sigma}  $, and where the equality is with respect to the natural identification 
% $$(\Sigma \circ pr \circ m
% _{i,j}) ^{*}  P =  (\widetilde{\Sigma} \circ m
% _{i,j}) ^{*} P.
% $$ 
% % While for $i=j _{0}, j= j_{0} +1   $ we set $$\mathcal{A} (L, L') = \mathcal{A} (pr _{*} L, pr _{*} L').
% % $$
%
% Likewise, given objects $L _{0}, \ldots, L _{s} \in F (\widetilde{\Sigma} ) $  we set 
% $$\mathcal{F}   (L_0, \ldots , L_s, \widetilde{\Sigma} ,r  ) = \mathcal{F}   (pr_*L_0,
% \ldots, pr_*L_s, \Sigma,r  ).$$ Here the equality
% is again with respect to the natural
% identification of the corresponding bundles,
% (based on the Axiom \ref{axiom:naturalityX2}  of $\mathcal{U} (X) $).
% % The pull-back operation $(pr \circ u (m _{1}, \ldots, m _{s},r, \Delta ^{n}))  ^{*}$ is induced be the natural pull-back operation of
% % connections, and families of fiberwise almost complex structures.
% % On the other hand if $pr$ is not injective on $L _{0}, \ldots, L _{s}  $ 
% All together this determines the (partial) data
% $\widetilde{D} _{\Sigma} = (\mathcal{U} (X),
% \mathcal{F} _{\widetilde{\Sigma } } )  $. 
% % Later on we call this \textbf{\emph{extended perturbation data}}, which is nothing more then perturbation data in the previous sense but for a degenerate simplex.
%
% Using this $\mathcal{D} _{\widetilde{\Sigma} } $ we then define an 
%  $A _{\infty} $ category denoted by $ F (\widetilde{\Sigma} ) $ as previously.
% By construction, the natural map on objects:
%  \begin{equation*} 
%  pr _{*}:  \obj F   (\widetilde{\Sigma} ) \to \obj F (\Sigma)
% \end{equation*}
% extends to a strict $A _{\infty} $ functor 
% $$F (pr): F (\widetilde{\Sigma}) \to F (\Sigma) $$ satisfying:
% \begin{equation*}
% F (pr) \circ F (\sigma) = id,
% \end{equation*} where $\sigma: \Sigma \to
% \widetilde{\Sigma}$ is induced by the map  $d
% ^{j+1}: [n] \to [n+1]   $, which is the unique
% injection in $\Delta$  whose image misses the
% vertex $j+1$. We are going to call the above the
% \textbf{\emph{extension construction for
% $\widetilde{\Sigma}, \mathcal{D}  _{\Sigma} $}}
% with the corresponding extended data denoted by $\mathcal{D} 
% _{\widetilde{\Sigma } }$, called \emph{extended
% perturbation data for $\widetilde{\Sigma} $ }. 
%
% We are now going to proceed by induction. Let $S
% (N) $ be the statement: there exists an extended
% perturbation data $\mathcal{D}$ for all simplices
% up to degree $N$, so that we we have an extension
% of $F| _{Simp ^{N} (X) } $ to a functor $$F ^{N}:
% \Delta ^{N}(X)  \to A _{\infty}-Cat ^{unit}, $$
% where $\Delta^{N} (X)  $ and $Simp ^{N} (X)$ are
% the subcategories of simplices of degree at most
% $N$. The case $S (0) $   is trivial since $Simp
% ^{0} (X)  = \Delta ^{N} (X) $, so that there is
% nothing to prove.  As usual for us, we also denote
% by $S (N) $  the corresponding partial
% perturbation data.
%
% We prove $S (N) \implies S (N+1) $, and 
% moreover $S (N+1) $    can be assumed to extend $S
% (N) $. Let ${\Sigma}'  $ be a general $(N+1)$-simplex
% of $X _{\bullet} $. If $\Sigma ' $ is degenerate,
% so that $\Sigma' = \Sigma \circ pr $ for some
% degeneracy morphism $pr$, for some $\Sigma \in
% Simp ^{N} (X) $, then define the data $\mathcal{D}
% _{\Sigma'} $ and $F ^{N+1} (\Sigma') $ as in the
% extension construction above for
% $(\widetilde{\Sigma } =  \Sigma'), \mathcal{D} _{\Sigma} $.
% Otherwise, if $\Sigma$ is a non-degenerate $(N+1)$-simplex then its faces are $N$-simplices for which we already have perturbation data $\mathcal{D}$,
% which is then extended arbitrarily to perturbation
% data $\mathcal{D} _{\Sigma} $ for $\Sigma$.
% The extension is obtained as in the proof of
% Lemma \ref{lemma:naturalF}. Using this data define
% $F ^{N+1} (\Sigma) $ as previously. 
%
% % we have a degeneracy of the form of the commutative diagram: % % in $\Delta/N (\Pi (\Delta ^{n} ))$
% % \begin{equation*}
% % \begin {tikzcd}
% % \Delta ^{0}  \ar [r, "j+1"]  \ar [dr, "j_*"]   & \Delta ^{N+1}
% %    \ar [d, "\Sigma'"]   \ar [r, "pr _{j}"]  & \Delta ^{N} \ar [ld, "\Sigma"] \\
% %                                    & X, 
% % \end {tikzcd}
% % \end{equation*}
% % and we may define $F ^{N+1} (\Sigma)$ in complete analogy to the definition of $F ({s _{j} \Sigma }) $ above.
% We thus complete the induction step.   By recursion, we may then define a sequence of
% systems $\{S _{N}\} _{N \geq 0}$,  so that $S (N+1)  $
% extends $S (N) $, for each $N$. 
% We then
% define that total extended data as $\mathcal{D} = \bigcup_{N}
% S (N)  $. And so  we obtain our  extension
% $$F:
% \Delta (X)  \to A _{\infty}-Cat ^{unit}, $$ by the
% previous construction, using the extended data
% $\mathcal{D} $. 
%
%
% % $F ^{ext} $ the name of the abstract extension of
% % $F$ to degenerate simplices constructed as above.
% % and $A _{\infty} $ relations are then tautological.
% % Finally we ask that for a degeneracy morphism $deg _{j}: s _{j} \Sigma \to
% % \Sigma$  the functor $$F (deg _{j} ): F   (s _{j} (\Sigma)) \to F   (\Sigma)$$ is just the
% % tautological projection.
% \section {Algebraic-topological considerations} \label{section.algebraic}
% In this section by equivalence of
% $\infty$-categories we always mean categorical equivalence. This and other
% categorical preliminaries needed for this section are discussed in the Appendix
% A.  
% We will prove here Theorem
% \ref{prop:quasicatfibration}. 
%
% \subsection {Colimit of $F$} 
% \begin{definition}\label{def:geometricFunctor}
%   A functor $F: \Delta(X) \to A _{\infty}-Cat
%    ^{unit}$ which is induced by a geometric
%    functor $F _{\mathcal{D}}: Simp
%    (X)  \to A _{\infty}-Cat ^{unit}$  as in Section
%    \ref{section:extension} will be called
%    \textbf{\emph{geometric}}. 
% \end{definition}
% \begin{remark}
%    \label{remark:} Instead of the above definition
%    it would be more ideal to extract
%    suitable minimal algebraic
%    axioms for our ``geometric functors''. However,
%    this may take us too far afield.  
% \end{remark}
%
% % By Proposition \ref{proposition:cofibrant} ${F} _{P} $ is Reedy cofibrant,
% Given a geometric functor $
% F: \Delta(X) \to A _{\infty}-Cat ^{unit}
% $, let 
% \begin{equation}
%    \label{eq:FukF}
%    Fuk _{\infty} (F)  := colim
% _{\Delta (X)} 
%    N{F}. 
% \end{equation}
% The category of simplicial sets is well known to
% be (co)-complete
% so that the limit certainly exists as a simplicial
% set. However, we shall show, in the following
% proposition, that this limit has
% additional structure of an $\infty$-fibration, and
% this will be rather crucial in Part II. \begin{remark}
%    \label{remark:}
% It is possible that the proposition below can be
% obtained as a consequence of more general
% principles, using
% general theory of colimits of
% $\infty$-categories.  However, I suspect that for a general
% functor of the form $G: \Delta (X) \to
% \infty-\mathcal{C}at $, we must first take a fibrant
% replacement of the functor $G$ (for the Joyal
%    model structure), if we want a similar
% structure on $colim _{\Delta (X)  } G$.
% (I am not however sure this is sufficient.)  \end{remark}
%
% % in addition show that this
% % colimit is a quasi-category, and that moreover  
% 
%  \begin{proposition} \label{propostion.simplicialmap}  There is a natural
%  projection of simplicial sets 
%  $$p: Fuk _{\infty} (F)   \to X _{\bullet}  ,$$ and this
%  is an $\infty$-fibration.
% \end{proposition}
% \begin{proof} 
% % Let us denote by $\hat {\Delta} ^{n} _{\bullet}$ the Kan subcomplex of
% % the singular set of the topological space $\Delta ^{n} $ with $i$
% % simplices vertex preserving linear maps $\Delta ^{i} \to \Delta ^{n}
% % $. 
% % Then clearly there is a natural embedding $\Delta ^{n} _{\bullet}
% % \to \hat {\Delta} ^{n} _{\bullet}  $. 
% Recall that a given  $\Sigma:
% \Delta ^{n} \to X $ could equally be thought of as an element of $X _{\bullet}
% (n)$ or as a natural transformation $\Delta^{n} _{\bullet} \to X _{\bullet}   $.
% Let us first give a more easily
% conceptualized presentation of the colimit $Fuk _{\infty}
% (F)$.  We should say that we are just
% simplifying the standard, ``level wise'' 
% construction, of colimits  of simplicial
% sets, in our specific context, so that the
% structure of an $\infty$-fibration becomes
%    apparent.
%
% %    This will be
% % important for the  
% % computation in Part II \cite{citeSavelyevGlobalFukayacategoryII}. 
%
%
% Define a partial
% order $<$ on the set of pairs $(f, \Sigma)$,  $f
%    \in NF(\Sigma) (k) $ a $k$-simplex, $k \geq 0$,  $ \Sigma
%    \in \Delta(X) $ as follows.  $$(f, \Sigma) < (f', \Sigma')$$ if
% there is a morphism $$\sigma: \Sigma \to \Sigma'$$
%    in $\Delta  (X)$ induced by injective $[n] \to [m]$ with
% $n \leq m$, i.e.
%    a \emph{face morphism}, s.t. $$NF (\sigma) (f) = f'.
%    $$ Clearly for every $(f, \Sigma)$ there is a unique least pair $$(f _{min}, \Sigma _{\min}) < (f, \Sigma).
% $$  Note that if
% $f$ is an $k$-simplex then $\Sigma _{min}$ is
% not necessarly a $k$-simplex. However,  once we
% impose the following equivalence relation, we
% get something similar, see Lemma \ref{lemma:canonicalrespresentative} below.
%   
%   Let $\widetilde{C}$ be the set of minimal pairs. Define an equivalence relation
% on $\widetilde{C} $ first by defining
% $$(f, \Sigma) \sim (f', \Sigma')$$ if there exists a degeneracy morphism $d: \Sigma \to
% \Sigma'$ induced by $[m] \to [n]$ with $m>n$,  such that $$NF (d) (f) = f'. 
% $$  And then by imposing symmetry and
% transitivity. Denote the equivalence class of $(f, \Sigma)$ by $[f, \Sigma]$.    
%   
% The following is not formally necessary, but it
%    might be helpful for visualization.
% \begin{lemma}
%       \label{lemma:canonicalrespresentative} 
%    Each class $[f, \Sigma] $ has a unique
%    natural  representative $(f _{c}, \Sigma _{c})$  so
%    that if $f$ is a $k$-simplex then $\Sigma
%    _{c}$ is a $k$-simplex.    
%    \end{lemma}
%   \begin{proof} Let $(f, \Sigma ) $ as above be
%      given,  with $\deg (f) =k$.  And
%      let $(f _{min}, \Sigma _{min}) \in
%      \widetilde{C} $  be as above. Then 
%      $\deg(\Sigma _{min}) \leq (n = \deg (f
%      _{min}))$.   Suppose that  $\deg(\Sigma
%      _{min}) <
%      n$. Then by the extension construction of Section \ref{section:extension} there is a degeneracy $$d _{c}:
%      \Sigma _{c} \to \Sigma _{\min},$$  
%    with $\deg (\Sigma _{c}) =k$ together with a $k$-simplex
%   $f _{c} \in NF (\Sigma _{c})  $   so that $NF
%      (d _{c}) (f _{c}) = f _{min}.  $  Moreover, $(d
%      _{c}, f _{c})$  are uniquely determined, (by
%      the extension construction). So
%      that we set $\Sigma  _{c} = \Sigma
%      _{min}\circ d _{c} $.  Also note that $\Sigma
%      _{c}$ is just $p _{\Sigma} (f _{min}) $, with
%      $p _{\Sigma}$ as in \eqref{eq:pSigma} ahead.
%    \end{proof}
%          
%   
% Continuing with the proof of the proposition, we then define $C =\widetilde{C}  /
% \sim$. This is naturally a simplicial set, with $$C (k) = \{[f, \Sigma] \in C \, \vert \, f \in NF
%    (\Sigma) (k)\}.
%     $$ 
%     % where $[\cdot]$ denotes the equivalence class.
%    For example $C (0)$ is naturally isomorphic
%    to $$\sqcup
% _{x \in X} Obj \, F  (x).$$  
%
%
% % Note that $X _{\bullet} $ 
%
%
% %    It is clear that
% % $p _{\Sigma}$ is simplicial, and that it commutes with face maps, i.e. maps
% %  $N \mathcal{F} ({\Sigma_{\Sigma}}_{1} ) \to N \mathcal{F}
% % ({\Sigma_{\Sigma}}_{2} )$, induced by a morphism ${\Sigma_{\Sigma}}
% % _{1} \to {\Sigma_{\Sigma}}
% % _{2}  $ in $Simp (\mathcal{A}_{\infty} )$. It follows that there is an induced simplicial projection $p:
% % \mathcal{P} 
% %  \to \mathcal{A} _{\infty} $.
% %  For each $k$ let $L ([k])$ be:
% % \begin{equation*} \{(f, \Sigma)|
% % f \in NF  (\Sigma) ( [k]), \Sigma = p _{\Sigma} ^{min} (f)  )\},
% % \end{equation*}
% % For $(f, \Sigma ) \in L ([n]) $, and $\sigma: [n] \to [k]$ a morphism in $\Delta
% %    ^{op} $,  set   $$L
% %    (\sigma)  (f, \Sigma) = (NF (\Sigma) (\sigma)  f, p _{\Sigma}  ^{min}   NF
% %    (\Sigma) (\sigma) f  ),$$ 
% %   (considering $NF (\Sigma)$ as a functor $\Delta ^{op} \to Set$).
% % % Likewise define $s _{i}  (f, \Sigma) = (s _{i}  f, \Sigma _{s
% % %    _{i} f } )$.
% % %
% % %
% % %
% % %
% % %    $d_i (f, \Sigma) = (d_i f,  \Sigma)$ if $f=s _{i-1} f'  $ for
% % % some $f'$, set $d_i (f, \Sigma) = (d_i f, d_i \Sigma)$ otherwise.
% % % And set $s _{i} (f, \Sigma) = (s _{i} f, \Sigma ) $.
% % %
% % Then $L$ is clearly a functor $\Delta ^{op} \to Set $, i.e. a simplicial set.
%
%
% The following in particular will give a direct proof
% that the colimit \eqref{eq:FukF} exists.
% \begin{lemma} \label{lemma.colimit} $C = colim
% _{\Delta (X)} 
%    N{F} 
%  $, 
%    with equality meaning natural
%    isomorphism.   
% \end{lemma}
% \begin{proof}   Note first that $C$ is a co-cone on the diagram $NF $. Indeed, for
% each $\Sigma$ define $\phi _{\Sigma}: NF   (\Sigma) \to C$ by $$\phi
%    _{\Sigma} (f) = [f _{min} , {\Sigma} _{\min}   ].$$ 
% %    for $i: p ^{min}  _{\Sigma} 
% % (f) \to \Sigma$ the canonical morphism in $Simp (X)$, and where $i  ^{*} (f) $
% %    means the unique element of $NF (p ^{min} 
% % _{\Sigma}  (f))$ which is mapped to $f$ under the embedding $NF (p ^{min} 
% % _{\Sigma}  (f)) \xrightarrow{NF (i)} NF (\Sigma)$.
%  It is easy to see that for a face morphism $i: \Sigma
% \to \Sigma'$ we have that the composition $$NF  (\Sigma) \xrightarrow{NF
%   (i)} NF  (\Sigma')
% \xrightarrow{\phi _{\Sigma'}} C,$$  coincides with $\phi _{\Sigma} $.
% Likewise for a degeneracy morphism 
% $d: \Sigma
% \to \Sigma'$ we have that the composition $$NF  (\Sigma) \xrightarrow{NF
%   (d)} NF  (\Sigma')
% \xrightarrow{\phi _{\Sigma'}} C,$$  coincides with $\phi _{\Sigma} $,
% because of  the equivalence relation $\sim$.
%
%
% The universal property is also easy to verify, for given another
% co-cone $C'$ with maps $\rho _{\Sigma}: NF   (\Sigma) \to C'$, $\Sigma \in
%    \Delta  (X)$ we can naturally define $U: C \to C'$ by $$U ([f, \Sigma]) = \rho  _{\Sigma}
% (f).$$ Then $U$ is clearly well-defined, by $C'$
%    being a co-cone. And moreover, $U$ is a map of
%    co-cones (all the relevant diagrams
%    commute). Since for a given $f \in NF
%    (\Sigma)$, we have  $$U (\phi
%    _{\Sigma} (f)) = \rho _{\Sigma _{\min} } (f _{\min} ) = \rho
%    _{\Sigma} (f),$$
%    where the last equality holds since $(C', \{\rho _{\Sigma}\})$ is a co-cone, and
%  since by construction there is a morphism  $$i: NF (\Sigma _{\min} ) \to NF (\Sigma),$$
%    with $F (i) f _{\min}  = f $. 
% \end {proof}
%
% Continuing with the proof of the proposition, for each $n$-simplex $\Sigma \in \Delta (X) $, we have a natural simplicial map 
% \begin{equation}
%    \label{eq:pSigma}
%    p _{\Sigma}:
%    N {F} (\Sigma) 
%    \to {\Sigma }_ * ({\Delta} ^{n} _{\bullet}) \subset X
%    _{\bullet},
% \end{equation}
% defined as follows. On
%    the vertices of $N {F} (\Sigma)$, $p _{\Sigma} $ is just the
%    obvious projection. 
% Given $k$-simplex $f$ in $N F (\Sigma) (k) $, we
%    get a composable chain of edges $f _{1},
%    \ldots, f _{k}$,
% with $f _{i}$ the edge of $f$  between the vertex $i-1,
%    i$, for $1 \leq i \leq k$.
% This  determines a list
% of vertices $v_0, \ldots, v_k \in NF (\Sigma) $ s.t. the source/target of $f_i$ is $e _{i-1}$
% respectively $v _{i} $.
% This in turn determines a list of vertices $\{p _{\Sigma} (v _{i}
% )\} \subset {\Sigma }_ * ({\Delta} ^{n} _{\bullet}) $,   and we set $p _{\Sigma} (f) $ to be the unique (possibly degenerate)
% $k$-simplex of ${\Sigma }_ * ( {\Delta} ^{n} _{\bullet})$ with
% these vertices. We will omit the verification that $p _{\Sigma} $ is
% simplicial. 
%   
%    The simplicial projection  $$p: C  \to X _{\bullet}$$ is then: send $ [f, \Sigma]$ to $p _{\Sigma} (f)$, which is readily seen to be well-defined.
% % For  $\Sigma$ as above, note that we have a natural simplicial map 
% %    $$p _{\Sigma}:
% %    N {F} (\Sigma) 
% %    \to {\Sigma }_ * ({\Delta} ^{n} _{\bullet}) \subset X
% %    _{\bullet} ,$$  defined as follows. On
% %    the vertices of $N {F} (\Sigma)$, $p _{\Sigma} $ is just the
% %    obvious projection. Now a
% % $k$-simplex $f$ in $N F (\Sigma)$ by definition determines a composable chain $(f_1,
% % \ldots, f_k)$ in $ NF (\Sigma)$,  and hence  determines a sequence
% % of vertices $e_0, \ldots, e_k$ s.t. the source/target of $f_i$ is $e _{i-1}$
% % respectively $e _{i} $.
% % This in turn determines a sequence of vertices $\{p _{\Sigma} (e _{i}
% % )\} $,   and we set $p _{\Sigma} (f) $ to be the unique (degenerate)
% % $k$-simplex of ${\Sigma }_ * ( {\Delta} ^{n} _{\bullet})$ with
% % these vertices. We shall omit the verification that $p _{\Sigma} $ is
% % simplicial.  We also define $$p ^{min} _{\Sigma} : NF (\Sigma) \to  {\Sigma }_ * ( {\Delta} ^{n}
% %    _{\bullet}),  $$ by
% %   $p ^{min} _{\Sigma}  (f) $  is the unique minimal 
% %    simplex of ${\Sigma}_ * ( {\Delta} ^{n} _{\bullet})$ with 
% %    $$f \in NF (p ^{min} _{\Sigma}  (f)),$$ meaning that $f $ is in the image of the
% %    embedding $NF (p ^{min} _{\Sigma}  f ) \to NF (\Sigma)$, 
% %     % _{\Sigma}
% %     and where minimal is with respect to the
% %    partial order on $Simp (X)$, $\Sigma _{1} < \Sigma _{2}  $ if there is a
% %    morphism in $Simp (X)$, $\Sigma _{1} \to \Sigma _{2}  $.
%
%  It is immediate from the definition of
% an inner fibration in Section \ref{sec:innerfibrations} that $p$ is an inner-fibration if
% and only if the pre-image of every simplex $\Sigma: \Delta ^{n} _{\bullet} \to X
%    _{\bullet} $ by $p$ is a $\infty$-category, where ``pre-image'' $p ^{-1} (\Sigma) $ is the 
%    pre-image by $p$ of the simplicial subset $\Sigma  (\Delta ^{n} _{\bullet} ) $.
%    % simplicial set with simplices natural transformations $\widetilde{\Sigma}:
%    % \Delta ^{n} _{\bullet} \to L    $, s.t. $p \circ \widetilde{\Sigma} =
%    % \Sigma$, for $p$ the natural transformation above. 
% In our case this follows by construction as the preimage of  $\Sigma$
% is clearly identified with $NF  (\Sigma)$, which
% is an $\infty$-category by properties of $N$.
%
% We now verify that in addition $p$ is a
% $\infty$-fibration.   By the Definition
% \ref{def:categoricalFibration} we
% need  to show that for every equivalence $m: a \to
% b$ in $\mathcal{X} = {X} _{\bullet} $  and
% for every object $a' \in Fuk _{\infty}(P) $ with
% $p(a') = a$, there exists an equivalence
% $\widetilde{m}: a'  \to b'$ in $\mathcal{E} = Fuk
% _{\infty}(P) 
% $ with $p(\widetilde{m}) = f$.
% % Let $m$ be an edge in $X _{\bullet} $ from $x _{0} $ to $x _{1} $ and
% % let $L _{m}  = p ^{-1} (m)$  and let $L_{i} = p
% % ^{-1} (x _{i})$, see definition of
% % pre-image above. To show that $p$ is a $\infty$-fibration, by \cite[Proposition
% % {2.4.1.5}]{citeLurieHighertopostheory}, which we 
% % review in Appendix \ref{}, 
% % it is enough to show that for every such
% % $m$ and $a \in L_1 $ there is an
% % equivalence $e _{a}  \in L _{m}$ with target $a$,
% % and with  $p (e _{a}) = m$ (this means in particular
% % that $e _{a}$ is not just a degenerate edge at
% % $a$). 
%  \begin{lemma} \label{lemma.weakequiv} The functor $N: A_{\infty}-Cat
%  ^{unit} \to \infty- \mathcal{C}at$,  takes quasi-  equivalences to weak
% equivalences in the Joyal model structure, i.e.
% categorical equivalences.  
% \end{lemma}
% \begin{proof} The proof of this is contained in the proof of Proposition
% 1.3.1.20, Lurie \cite{citeLurieHigherAlgebraa}. We can also prove this directly by first recalling
% that quasi-equivalences of $A _{\infty}
% $-categories $A,B$ are invertible (when working
% over a field with characteristic $0$), up
% to homotopy, and then  via the nerve construction translate this to a categorical
% equivalence of $N (A), N (B) $. 
% \end {proof} 
% Recall that the morphisms of $A _{\infty}-Cat$ are
% in particular quasi-equivalences, also recall
% Lemma  \ref{lemma:functorialF}.
% Since the inclusions $F (x _{i}) \to F (m)  $  are
% quasi-equivalences, it follows
% by the lemma above, and by the construction of
% $L$ that the inclusions of $L_{i} $ into
% $L_{m} $ are categorical equivalences of
%    $\infty$-categories, and so $\widetilde{m}$ as above must exist.
% %
% %
% % Let $\Sigma ^{2} $ be the 2-simplex in $\mathcal{A} _{\infty} $ with edges $m, m'$ and  $id
% % _{x_0} $, for some $m'$, where $id _{x_0} $ denotes the degenerate 1-edge at $x
% % _{0} $. This 2-simplex exists as $\mathcal{A} _{\infty} $ is a Kan complex.
% % % Take the
% % % degenerate
% % % 2-simplex $s_0 m$ in $\mathbb{S} $.
% %  By factorization axiom, in the $A
% % _{\infty} $ category $F (\Sigma ^{2}  )$ there is a factorization  of the morphism $id _{a}  \in F
% % (id _{x_0})$ as $\gamma _{1} \cdot \gamma _{2}  $, (in $\ho F (\Sigma ^{2} )$.
% % But then the nerve of $\gamma _{1} $ is
% % the required equivalence.
% % It follows by Lemma
% % \ref{lemma.weakequiv}, \ref{lemma.cartesian} that $p $ is (co)-Cartesian.
% \end {proof}
% % \begin{corollary} \label{corollary:cocartesian}
% %    For $F: Simp (X) \to A _{\infty} -Cat ^{unit}  $ a pre-$\infty$ functor the
% %    fibration $\mathcal{P} _{F} $, which is by construction the
% %    pull-back by the associated simplicial map $F _{\bullet}: X _{\bullet} \to
% %    \mathcal{A} _{\infty} $ of $\mathcal{P}$ is (co)-Cartesian.
% % \end{corollary}
% \begin{proof} [Proof of Theorem \ref{prop:quasicatfibration}] 
% By the discussion above we have an
%    $\infty$-fibration
% $$Fuk _{\infty} (P, \mathcal{D})   \to X _{\bullet}  .$$
% The first part of the theorem follows by the following general fact: for an inner fibration of
% simplicial sets $$p: P _{\bullet} \to X _{\bullet},$$ if $X _{\bullet}$ is a $\infty$-category 
% then $P _{\bullet}$ is a $\infty$-category. Let us prove this elementary point.
% Suppose we are given  $${\rho}: \Lambda ^{n} _{k} \to P _{\bullet}$$  for $0< k < n$.
% As $X _{\bullet}$ is a $\infty$-category there a simplex $$ \widetilde{\rho}: \Delta
%    ^{n} _{\bullet}  \to X _{\bullet}$$ extending $p \circ \rho$.  But then
%   $\rho$ maps into the $\infty$-category 
%  $p ^{-1} ( \widetilde{\rho})$, and consequently  there is an
%  extension of $\rho$, c.f. Proposition \ref{prop.innerfib2}.
%
%
% 
%  The final part of the theorem follows by the following.
% \begin{lemma} For the geometric functor $F _{P,
%    \mathcal {D}} $ the concordance class
%    of the $\infty$-fibration $p: Fuk _{\infty} (P, \mathcal {D}) \to
% X _{\bullet}$ is
% independent of the choice of $ \mathcal {D} $. 
% \end{lemma}
% \begin{proof} By Theorem \ref{thm.concordance} 
% given a pair $\mathcal{D} _{0}, \mathcal{D}
%    _{1}$     of perturbation data for $P$, 
% there is a geometric functor:
% \begin{equation*}
%    \widetilde{F}: \Delta (X \times I) \to A
%    _{\infty}-Cat ^{unit}, 
% \end{equation*} 
% which gives a concordance of the functors
% $${F} _{P, \mathcal{D} _{i}}: \Delta (X) \to A
%    _{\infty}-Cat ^{unit}. 
% $$ 
% Then by the Proposition
%    \ref{propostion.simplicialmap} there exists an 
%    $\infty$-fibration:
%    \begin{equation*}
%    \mathcal{T} \to X _{\bullet}  \times I _{\bullet},
%    \end{equation*}
%    whose restriction over $X _{\bullet}  \times \partial I _{\bullet} $ coincides with $$Fuk
%    _{\infty} (P, \mathcal {D}_{0} ) \sqcup Fuk
%    _{\infty} (P, \mathcal {D}_{1} ).$$ 
% % Then the result follows by Theorem
% %    \ref{corollaryStraightening}. 
% % The lemma then follows by Lurie's straightening theorem
% %    \ref{thm.straightening}, or more simply by Corollary
%    \end {proof}
%    \end {proof}
% \subsection {Naturality} \label{sectionNaturalityPullback} We
% may expect if our constructions are really natural
%    that the $\infty$-fibration $$N Fuk (M,
%    \omega) \hookrightarrow  Fuk _{\infty} (P,
%    \mathcal{D}) \to X _{\bullet}$$ is functorial with respect to pull-back and this is indeed the case. 
% Let $f: X \to Y$ be a smooth map, $P \to Y$ a smooth
% Hamiltonian fibration and $\mathcal{D} =
%    \mathcal{D} (P) $ extended
% perturbation data. We may then 
% define pull-back extended perturbation data $f ^{*} \mathcal{D} $ for $f ^{*}P
%    \to X$, as follows.
% First we have the ``pull-back''  natural system
%    $\mathcal{U} (X) $, defined by $$u (m _{1},
%    \ldots, m _{s}, \Sigma) := u (m _{1},
%    \ldots, m _{s}, \widetilde{\Sigma } = f \circ \Sigma
%    ),$$ where the maps $u$   on the right are part
%    of $\mathcal{U} (Y) $. 
% Next, let $$\widetilde{f}: f ^{*}P \to P$$
%    be the natural bundle map, then given $\Sigma \in X _{\bullet} (d) $ we set
% $$ \mathcal{F} (L_0, \ldots , L_s, {\Sigma},r) =
%    \mathcal{F} (\widetilde{f} (L_0), \ldots,
%    \widetilde{f} (L_s), \widetilde{\Sigma},r), $$ for $\widetilde{\Sigma}= f \circ \Sigma  $.
% This determines our data $f ^{*}\mathcal{D} $.
%
%    Let
%    ${f} _{\bullet} : X _{\bullet} \to Y _{\bullet}   $ be the induced map of
%    simplicial sets.
% \begin{theorem} \label{thmNaturality} $$Fuk _{\infty} (f ^{*} P, f ^{*}
%    \mathcal{D}) = {f} ^{*} _{\bullet}   Fuk
%    _{\infty} (P, \mathcal{D}), $$ where $f ^{*} \mathcal{D}$ is
%    as above, and where  ${f} ^{*} _{\bullet}   Fuk
%    _{\infty} (P,  \mathcal{D}) $  denotes the
%    standard pull-back of the
%    simplicial fibration by $f _{\bullet }$.  
% \end{theorem}
% \begin{proof} The proof is immediate.
% \end{proof}
% % \subsection {Pre-$\infty$ functors} \label{section:preinfinity}
% %    \textcolor{blue}{remove}  
% % This section is a remark. We say a bit here on the origin of the term pre-$\infty$ functor. By Theorem
% % \ref{prop:quasicatfibration} to every analytic pre-$\infty$ functor we have a naturally
% % associated equivalence class of a (co)-Cartesian fibration over $X _{\bullet} $. By Lurie's
% % straightening theorem this induces a class $[X _{\bullet},
% % \mathbb{S}]$. This
% % may be thought of as one justification for the term. In a future work
% % we intend to show
% % that the associated map $X _{\bullet} \to  \mathbb{S} $ can
% % be canonically factored in the homotopy category through a certain  Kan complex of $A
% % _{\infty}$ categories  with vertices unital $A _{\infty} $ categories and
% % simplexes defined as certain higher correspondences.  This would
% % be a better explanation for the term ``pre-$\infty$ functors''. However for
% % (pre)-triangulated, graded, rational $A _{\infty} $ categories  
% % it will
% % be shown that the corresponding hypothetical space $A _{\infty}-Cat
% % ^{\mathbb{K}, tr}$ is weakly equivalent to 
% % $ \widehat{S}$, the space of stable $\infty$-categories. We shall make use of this latter space in part
% % II.
% \appendix
% % \label {sec.details}
% \section {$\infty$-categories and Joyal model structure} \label{appendix.quasi}
% We don't need absolutely everything in this section, particularly we can avoid ever mentioning model categories, but the latter helps with the narrative. A very good concise reference for much of this material is Riehl  
% \cite{citeRiehlAmodelstructureforquasi-categories},
% which we will mostly follow. The material on
% various fibrations $\infty$-categories is taken
% from
% Lurie \cite[Section 2.4]{citeLurieHighertopostheory}. First let us recall the notion of a Kan
% complex, which may be thought of as formalizing
% the property of a simplicial set to be like the
% singular set of a topological space, defined in
% Section \ref{section:simplexCategory}.
%
%  Let
% $\Delta ^{n} $ be the standard representable $n$-simplex: $\Delta
%     (i) =
% \Delta ([i], [n])$. Previously we denoted this by $\Delta ^{n} _{\bullet}   $,
%    but as there are no topological simplices in this section we simplify the
%    notation, which is also consistent with above references.
%    Let $\Lambda ^{n} _{k} \subset \Delta ^{n}$ denote the
% sub-simplicial set corresponding to the ``boundary'' of $\Delta ^{n}  $ with the
% $k$'th face removed, $0 \leq k \leq n$. By $k'th$ face we mean the face opposite
% to $k$'th vertex.  This is called  the 
%  \textbf{\emph{$k$'th horn}} or just horn. 
%
% A simplicial set $S _{\bullet}$ is said
% to be a \emph{Kan complex} if for all $n,k$ given a diagram with solid arrows
% \begin{equation*} 
% \begin {tikzcd}
%    \Lambda ^{n} _{k} \ar [r]  \ar [d] & S _{\bullet} \\
%  \Delta ^{n} \ar [ur, dotted]  &, \\ 
% \end{tikzcd}
% \end{equation*}
% there is a dotted arrow making the diagram commute.
% % it can be completed to a commutative diagram
% % \begin{equation*} \xymatrix {\Lambda ^{n} _{k} \ar [r]  \ar [d] & S _{\bullet}
% % \\ \Delta ^{n} \ar [ur] & \\}.
% % \end{equation*}
%
% An \emph{$\infty$-category} is a simplicial set $S
% _{\bullet}$ for which the above extension property
% is only required to hold for \emph{inner horns}  $\Lambda ^{n} _{k}$,
% i.e. those horns with $0<k<n$.  A Kan
% complex is a simplicial model of an $\infty$-groupoid, 
% as the Kan condition can be interpreted as giving
% the condition that morphisms are invertible up to
% (coherent) homotopy. Likewise, the  defining
% property of an $\infty$-category tells us that
% while 1-morphisms/edges may not be invertible (up to
% homotopy),  the higher
% morphisms are.
% It would perhaps take us
% too far afield too further motivate $\infty$-categories
% here. 
% However, the introductory sections of Lurie~
% ~\cite{citeLurieHighertopostheory}   should be
% highly accessible. 
%
%
% A morphism between   $\infty$-categories is just a
% simplicial map.
% We will denote $\infty$-categories by
% calligraphic letters e.g. $ \mathcal {B} $.  In
% \cite[Chapter 3]{citeLurieHighertopostheory} an
% $\infty$-category of $\infty$-categories is
% constructed, with 1-morphisms simplicial maps, and we call this $\mathcal
% {C}at _{\infty}$. On the other hand the
% full-subcategory of  the category of simplicial
% sets with objects $\infty$-categories will be
% denoted by $\infty-Cat$.       
% \begin{notation}
% We denote the maximal Kan subcomplex of
%    $\mathcal{C}at _{\infty} $ by
% $\mathbb{S} $.
% \end{notation} 
%
% % The full-subcategory of $sSet$ with
% % objects $\infty$-categories will be denoted by $\infty- \mathcal {C}at $.
%
% \subsection {Categorical equivalences, morphisms and equivalences}
% \label{section:prelimQuasi} We have a
% natural functor $\tau: sSet \to Cat$, defined as
% follows. $\tau (S _{\bullet})$ is the category with objects 0-simplices of $S
% _{\bullet}$, 1-simplices as morphisms, degenerate 1-simplices as identities and
% freely generated composition subject to the relation $g= f \circ h$ if there is
% a 2-simplex $e$ with 0-face $h$, 2-face $f$ and
% 1-face $g$. 
% (Remembering our diagrammatic order for
% composition.) See the figure below.     
% \begin{figure}[h]
%   \includegraphics[width=2in]{simplex.pdf}
% % \scalebox{.3}{\input{simplex.tex}}
%  \caption {} \label{figure:simplex}
% \end{figure} 
% The category $\tau (S _{\bullet }) $  may be
% understood as the \emph{fundamental category} of
% $X _{\bullet }$    in analogy to the fundamental
% groupoid.
%
% % \begin{equation*}
% % \begin{tikzcd}
% %  \ar[r, ""] \ar [d, ""] &   \ar [d,""] \
% %  \ar [r, ""]   & 
% % \end{tikzcd}
% % \end{equation*}
%
% We also have a functor $\tau _{0}: sSet \to Set$ by sending $A _{\bullet}$ to the set of isomorphism classes of objects in $\tau
% (A _{\bullet})$.
%
% If $S _{\bullet} = \mathcal {X}$ is a $\infty$-category an edge $e
% \in \mathcal {X} $, i.e. a 1-simplex $e: \Delta
% ^{1} \to \mathcal{X} $,  is said to be an
% \emph{equivalence} if $\tau (e) $ is an isomorphism
% in $\tau \mathcal ({X}) $. We may use morphism
% notation, so that $e: a \to
% b$ signifies that the edge $e$ goes   from the
% vertex $a$ to
% $b$.
%
% The \emph{maximal Kan subcomplex}  of a $\infty$-category $
% \mathcal{X} $ is the maximal sub-simplicial set 
% $K (\mathcal{X} ) \subset \mathcal{X} $ with all
% edges equivalences.  (It can be constructed simply
% by removing edges that are not equivalences, and
% all simplices containing them.) 
% The fact that $K (\mathcal{X} ) $
% is forced to be a Kan
% complex can be readily verified using the
% $\infty$-category structure of $\mathcal{X} $.
% (This is an instructive exercise.) 
%
% \begin{definition}\label{def:connectedComponent}
% Let $\mathcal{X} $ be a Kan complex and $a \in
%    \mathcal{X} (0)  $. A \textbf{\emph{connected
%    component}}  of $a$  is the set of vertices sharing an
%    edge with $a$. We may sometimes
%    denote such a connected component by
%    $(\mathcal{X}, a) $.
% \end{definition}
%
%
% We define $sSet ^{\tau_0}$ to be the category with the same objects as $sSet$
% but with the morphisms given by $sSet ^{\tau_0} (A _{\bullet}, B _{\bullet}) =
% \tau ^{0} (B _{\bullet} ^{A _{\bullet}})$. 
% A map of simplicial sets $$u: A _{\bullet} \to B
% _{\bullet}$$ is
% said to be a \emph{categorical equivalence} if the induced map in $sSet ^{\tau
% _{0}}$ is an isomorphism. 
% % It is said to be a weak
% % categorical equivalence if the pull-back
% % map $$ sSet ^{\tau_0} (B _{\bullet}, X _{\bullet}) \to sSet ^{\tau_0} (A _{\bullet},
% % X _{\bullet})$$ induced by $u$ is an equivalence for all $X _{\bullet}$. A
% % categorical equivalence is necessarily a weak
% % categorical equivalence.
% 
%  We will say that a pair of $\infty$-categories
% are \emph{categorically equivalent} if
% there is categorical equivalence between them.   
% As we
% are following Riehl \cite{citeRiehlAmodelstructureforquasi-categories}, we refer the reader there for the following:
% \begin{theorem} \label{thm:joyal}[Joyal, Lurie,
%    Riehl] There is a
% model structure on $sSet$, called the Joyal
% model structure so that the fibrant objects are
% $\infty$-categories and a weak 
% equivalence between $\infty$-categories is a categorical equivalence.
% \end{theorem} 
% We do not formally need the above theorem, but we
% hope it places things into some perspective,  by
% making $\infty$-categories a less mysterious object.
% \subsection {Inner fibrations}
% \label{sec:innerfibrations}
% A map $p: \mathcal {A}\to \mathcal {B}$ of
% $\infty$-categories is said to be an \emph{inner fibration}
% if it has the lifting property with respect to all inner horn inclusions. More
% specifically, for $ 0< k < n$, whenever we are given a commutative diagram with 
% solid arrows:
% \begin{equation} 
% \begin{tikzcd} \label{diagram:inner}
%    \Lambda ^{n} _{k}  \ar [r]  \ar[d, hookrightarrow] &
% \mathcal {A} \ar [d, "p"]  \\
%  \Delta ^{n}  \ar [r] \ar [ur, dashrightarrow] & \mathcal {B}, \\
% \end{tikzcd}
% \end{equation}
% there exists a dashed arrow as indicated, making the whole diagram commutative. 
%
% For reference $p$ is said to be
% an \emph{Kan fibration} if the above extension property holds for all horns. A
% Kan fibration is an analogue in the simplicial world of Serre fibrations of
% topological spaces. The following is immediate from definitions. 
% \begin{proposition} \label{prop.innerfib2} A map
%    $p: \mathcal{A} \to \mathcal{B}$ is an inner fibration, if and only if the pre-image of every simplex
% of $\mathcal{B}$ is a $\infty$-category. 
% \end{proposition}
% % \begin{corollary} \label{corollary.innerfib} 
% % \end{corollary}
% \subsection {Categorical fibrations} These are
% the analogues of Serre fibrations for the Joyal
% model structure. 
%
% % (co)-Cartesian
% % We shall explain the
% % co-Cartesian version, as the Cartesian version is
% % just dual to the former.
% % 
% % Given $p: \mathcal{A} \to \mathcal{B}$, an edge $f: \Delta ^{1} \to 
% % \mathcal{A}$ is said to  
% % be  \emph{co-Cartesian} if whenever we are given a diagram with solid arrows: 
% %  \begin{equation} 
% % \begin {tikzcd}
% %      \Delta ^{n} _{0,1}  \ar [rd, "f"]   \ar [d, hookrightarrow]
% %   \\ \Lambda ^{n}_0 \ar [r] \ar [d] & \mathcal{A} \ar [d, "p"]  \\
% %  \Delta ^{n} \ar [r] \ar [ur, dashrightarrow] &
% %  \mathcal{B}, \\
% % \end{tikzcd}
% % %     \xymatrix { \Delta ^{n} _{0,1}  \ar [rd] ^{f}  \ar@{^{(}->}
% % %  [d] \\ \Lambda ^{n}_0 \ar [r] \ar [d] & A _{\bullet} \ar [d] ^{p} \\
% % %  \Delta ^{n} \ar [r] \ar@{-->} [ur] & B _{\bullet}}, \\
% % \end{equation}
% %  there is a dashed arrow as indicated making the diagram commutative. Here $\Delta ^{n}
% %  _{0,1}$ denotes the ``edge'' (sub-simplical complex) joining the vertexes
% %  $0,1$. 
% \begin{definition}\label{def:categoricalFibration} 
% We say that $p: \mathcal{E} \to \mathcal{X} $
% is a \textbf{\emph{categorical fibration}} if:
%   \begin{enumerate}
%       \item  The map $p$ is an inner fibration.
% \item For every equivalence $f: a \to b$ in
%    $\mathcal{X} $  and every object $a' \in
%         \mathcal{E} $ with $p(a') = a$, there
%         exists an equivalence
%         $\widetilde{f}: a'  \to b'$ in $\mathcal{E}
%         $ with $p(\widetilde{f}) = f$.
%    \end{enumerate} 
% \end{definition} 
% A categorical fibration over a Kan complex will be
% called an \textbf{\emph{$\infty$-fibration}}. 
% % The map $p: \mathcal{A} \to \mathcal{B}$ is said to be a \emph{co-Cartesian fibration}, if it is an inner 
% % fibration and if for every edge $e: \Delta ^{1}
% % \to \mathcal{B}$, with
% % co-domain $b$, and every $ \widetilde{b}$ lifting $b$ there is a
% % co-Cartesian lift $ \widetilde{e}: \Delta ^{1} \to
% % \mathcal{A} $, with co-domain
% % $ \widetilde{b}$.   
% % 
% % Denote by $coCFib ( \mathcal {B}) $ the
% % quasi-category of (co)-Cartesian fibrations over $ \mathcal {B}$, which by
% % definition is the full-subcategory of the over-category $ \mathcal {C}at
% % _{\infty}/ \mathcal {B}$, with objects (co)-Cartesian fibrations. 
% \begin{definition}
%    \label{def:concordanceCartesian} 
% We say that a pair of $\infty$-fibrations
% $p _{i}: \mathcal{P} _{i} \to \mathcal{X}  $,
%    $i=0,1$  over
% a Kan complex $\mathcal{X} $ are
% \textbf{\emph{concordant}} if the following
% holds. 
% There is a $\infty$-fibration $$\mathcal{Y}  \to \mathcal{X}
% \times \Delta ^{1},$$ whose
% pull-back by $i _{0}: \mathcal{X}   \to
% \mathcal{X}  \times \Delta ^{1}    $
% is identified with $\mathcal{P} _{0}$  and the
%    pull-back by $i _{1}: \mathcal{X}  \to
% \mathcal{X}  \times \Delta ^{1} _{\bullet}   $
% is identified with $\mathcal{P} _{1} $. Here the two maps $i_{0}, i_{1}$ correspond to the two vertex inclusions $\Delta ^{0} _{\bullet}  \to \Delta ^{1} _{\bullet}   $.
% \end{definition}
% % \begin{definition}\label{def:}
% %  A simplicial fibration $p: \mathcal{P} \to
% %    \mathcal{X} $ that is both  (co)-Cartesian and
% %    Cartesian will be called an $\infty$-fibration. 
% %   \end{definition}  
% % Similarly to the above, we may define
% %    concordances of $\infty$-fibrations. 
% \begin{notation} 
%    \label{} 
% We shall denote the set of concordance classes of
% $\infty$-fibrations over a Kan complex $\mathcal{X} $  by
% $Fib _{\infty}(\mathcal{X} ) $.
% \end{notation}
%
% %   \begin{theorem} \label{thm.straightening}
% % \cite[Theorem 3.2.01]{citeLurieHighertopostheory} \textbf{Straightening theorem}.  There is
% % equivalence of quasi-categories $Fun (\mathcal {B}, \mathcal {C}at _{\infty})
% %      \simeq coCFib ( \mathcal {B})$.
% % \end{theorem}
% % Stated more properly this combines \cite[Theorem
% % 3.2.01]{citeLurieHighertopostheory} and
% % \cite[Proposition 3.1.5.3]{citeLurieHighertopostheory}, both of which are statements on the level of
% % model categories. When $ \mathcal {B} $ is a Kan complex the notions of
% % Cartesian and co-Cartesian fibrations over $ \mathcal {B} $ coincide and
% % the model category
% % presenting $ coCFib ( \mathcal {B})  $ is just the over category $sSet/
% % \mathcal {B} $ with the induced Joyal model structure. The fibrant objects in  this model structure are the (co)-Cartesian fibrations over $ \mathcal {B} $. 
% % The term ``presents'' here  means that the underlying quasi-category is the
% % simplicial nerve of the Dwyer-Kan
% % \cite{citeDwyerKanSimpliciallocalizationofCategories} simplicial localization of the model
% % category.
% % Thus we simplify the above as follows for our needs in this paper. We say that a
% % pair of co-Cartesian fibrations $\mathcal{P } _{1} \to \mathcal{B} $,
% % $\mathcal{P} _{2} \to \mathcal{B} $ are \textbf{\emph{equivalent}} if there is
% % an (categorical) equivalence of $\infty$-categories $\mathcal{P} _{1} \to \mathcal{P} _{2}  $
% % over $\mathcal{B}$. The set of such equivalence classes is formally $\tau_0
% % CFib (\mathcal{B})$. 
%
% For convenience we recall the basic definition.
% \begin{definition}\label{def:simplicialhomotopyofmaps}
%    We say that a pair of maps of simplicial sets
%    $f,g: A _{\bullet } \to B _{\bullet }$ are
%    \textbf{\emph{homotopic}} if  there is map of
%    simplicial sets
%    \begin{equation*}
%       F: A _{\bullet } \times \Delta ^{1} \to B
%       _{\bullet},  
%    \end{equation*}
%   so that $F| _{A _{\bullet} \times
%    \{0\}} =f$  and $F_{\bullet}| _{B _{\bullet}
%    \times \{1\}} =g$. 
%    % the end
%    % points of $\Delta ^{1} $   is identified with
%    % $f$, respectively with $g$. 
% \end{definition}
% In \cite[Section
% 3.3.2]{citeLurieHighertopostheory} Lurie
%   constructs a universal  $\infty$-fibration over
% $\mathbb{S}$.  More specifically he constructs a
% universal Cartesian fibration over $
% \mathcal{C}at _{\infty}$. It's restriction over
% $\mathbb{S}$  is a $\infty$-fibration by ~\cite
% [Proposition 3.3.1.8.]{citeLurieHighertopostheory}.  
%   As a direct consequence we have the
% following theorem. 
% \footnote{The statement should be interpreted with care, since
% there are set theoretic issues. The most natural
% (and arguably standard)
% interpretation is via Grothendieck universes, 
% as for example done explicitly in
% ~\cite{citeSavelyevSmoothSimplicial}, in a very similar
% context. But we ignore these subtleties here.}
%  % The following lemma is implicit in
% % \cite{citeLurieHighertopostheory}, it is just an
% % expression of the fact that $\mathbb{S} $  is the base
% % of the universal (co)-Cartesian fibration over
% % Kan complexes. 
%
% \begin{theorem} [Lurie ~\cite{citeLurieHighertopostheory}]  \label{corollaryStraightening}
% For a Kan complex $\mathcal{X}$, there is a natural 
% isomorphism  $$Fib _{\infty} (\mathcal{X} ) \simeq
% [\mathcal{X},  \mathbb{S}],$$          
% with $[\mathcal{X},  \mathbb{S}]$  denoting the
%    ``set'' of homotopy classes of maps $\mathcal{X}
%    \to \mathbb{S}  $.
% %    Moreover if $X _{\bullet} $ is connected then the set of 
% % equivalence classes of (co)-Cartesian fibrations over $X _{\bullet} $, with a
% %    fiber of $x _{0} \in X _{\bullet}  $ identified with $\mathcal{C}$, is the
% %    set $[(X _{\bullet}, x _{0}),   (Cat
% %    _{\infty}, C)]$
% %    of homotopy classes of based maps $(X _{\bullet}, x _{0})  \to (Cat
% %    _{\infty}, C)
% %    $.
% \end{theorem} 
% % % On the other hand $[(X _{\bullet}, x _{0}),   (Cat
% % %  _{\infty}, C)]$ has a group structure of when $X _{\bullet} = S ^{i}
% % %  _{\bullet}  $, it is just $\pi _{i} (Cat _{\infty}, C )  $, the $i$'th homotopy
% % %  group of the maximal Kan subcomplex of $Cat _{\infty} $.  This induces a
% % %    group structure on the corresponding set of (co)-Cartesian fibrations,
% % %    geometrically corresponding to gluing fibrations over $S ^{i} _{\bullet}  $ along the fiber over $x _{0}
% % %    $.
% % \subsection {Semi-locality} The following is only
% % used in Part II,
% % ~\cite{citeSavelyevGlobalFukayacategoryII}. Suppose that $S _{\bullet} \subset X _{\bullet}$ is a Kan sub-complex, whose inclusion map is a weak equivalence. 
% % \begin{lemma}  \label{proposition:locality}
% % The restriction functor $$\tau
% % CFib (X _{\bullet}) \to \tau CFib (S _{\bullet}),$$ is an isomorphism, and so
% % the restriction map $$\tau _{0} 
% % CFib (X _{\bullet}) \to \tau _{0}  CFib (S _{\bullet}),$$ is a set-isomorphism.
% % \end{lemma}
% % \begin{proof} 
% %  By Lurie's
% % straightening theorem \ref{thm.straightening}, this is equivalent to the 
% % restriction functor 
% % \begin{equation}  \label {eq.isotau} 
% % \tau Func (X _{\bullet}, \infty -\mathcal {C}at) \to \tau Func (S _{\bullet}, \infty -\mathcal {C}at),
% % \end{equation}
% % being an isomorphism. But $X_{\bullet}$ and $S _{\bullet}$ are Kan complexes and so 
% % \begin{equation*} \label {eq.tau}  \tau Func (X
% % _{\bullet}, \infty-\mathcal {C}at) \simeq \tau Func (X
% % _{\bullet}, \mathbb{S})  \simeq \ho Top  (|X
% % _{\bullet}|, |\mathbb{S}|),
% % \end{equation*}
% %  where $|\cdot|$ denotes geometric realization
% % functor and where $\ho Top $ denotes the homotopy category of topological spaces.  The
% % last equivalence is due to the following.
% %   $  Func (X
% % _{\bullet}, \mathbb{S})$ is a Kan complex as it is the
% % mapping space of Kan complexes, then observe that for
% % Kan complexes $ \tau _{0} $ is just the functor of connected components, next
% %    use 
% %  that the geometric realization $|\cdot|$, and
% % singular set functors induce a derived Quillen equivalence between $\ho Top$
% % and $\ho sSet$. 
% % 
% %  Similarly $$
% % \tau Func (S _{\bullet}, \infty-\mathcal {C}at) \simeq \tau Func (S
% % _{\bullet}, \mathbb{S}) \simeq \ho Top  (|S_{\bullet}|,
% % |\mathbb{S}|).$$ 
% % % Recall that geometric realization is a
% % % right Quillen functor, with image CW complexes, and so preserves weak
% % % equivalences between fibrant objects. Hence t
% % The inclusion $|S _{\bullet}| \to |X
% % _{\bullet}| \simeq X$ is a homotopy equivalence, as the inclusion $S _{\bullet}
% % \subset X _{\bullet}$ is a weak equivalence by assumption and since geometric
% % realization has image in CW complexes. It follows that \eqref{eq.isotau} is an
% % isomorphism. 
% % 
% % 
% % \end{proof}
% % Using this we may compute the class of global
% % Fukaya category of $M \hookrightarrow P \to X$ in $ \tau_0
% % CFib (X _{\bullet})  $,  by restricting $F$ to
% % a sub-category $\Delta /  {S _{\bullet} } $, with $S _{\bullet} \subset X _{\bullet}$
% %  minimal Kan sub-complex generated by the sub-simplicial set of $X _{\bullet}
% %  $ corresponding to some smooth triangulation of $X$. This is in principle finite
% %  local data if $X$ is compact, and in general locally finite. This plays a role
% %  in the calculation in part II.
%  \subsection{ $A_\infty$-nerve}
% \label{appendix:nerve}  This section
% mostly follows Tanaka \cite[2.3]{citeLeeAfunctorfromLagrangiancobordismstotheFukayacategory}, except that for us
% everything will be ungraded, and for simplicity with $ \mathbb{F}_2$-coefficients. 
%
% For $ [n] \in \Delta$,  a \emph{length $s$ wedge decomposition} of
% $[n]$ is a collection of monomorphisms in $\Delta$
% \begin{equation*} j_i: [n_i] \to [n], \quad i=1,\ldots,s,
% \end{equation*} 
% satisfying  the following properties:
% \begin{itemize}
%       \item $\forall i:  n _{i} \geq 1.$
%       \item  $1 \in \image (j _{1}) $, $n \in
%          \image (j
%       _{s})$.  
% \item $\forall 2 \leq i \leq n: \max _{[n _{i}-1] }
%    j _{i-1} = \min _{[n _{i}] } j_{i}$ 
% \end{itemize}
% % 
% % such that the fiber product $$[n _{i}]  \times _{[n] } [n _{i+1}] \simeq [0]$$ and the canonical projection 
% % $$[n _{i}]  \times _{[n] } [n _{i+1}] \to
% % [n_i]$$ is the map $
% % [0] \to [n_i]$ sending $0$ to $n_i \in [n_i]$. Here we are thinking of $[n]$ as the
% % totally ordered finite set $ \{0, \ldots,n\}$. 
% % \textcolor {blue}{not clear} 
%
% We denote the set of all
% decompositions of $[n]$ by $D  [n]$.   We may of
% course equally understand a length $s$
% decomposition as a finite set  $\{J _{1}, \ldots J 
% _{s}  \}$ of subsets of $[n]$ decomposing $[n]$. 
% However, it is a bit simpler to formulate the following in terms of the maps $j _{i} $.
% \begin{definition}
% \label{def.nerve} For $  {A}$ a small unital $A _{\infty}$ category its
% nerve $N ({A})$ is a simplicial set with the set of vertices  the set
% of objects of ${A}$. A $n$-simplex $f$ of $N (  {A})$ consists of
% the following data:
% \begin{itemize}
%    \item  A map $[n] \to Objects ({A})$. We denote the corresponding objects $X _{0}, \ldots, X _{n}$. 
% \item For each mono-morphism $j: [n _{j}] \to [n]$
%    in $\Delta$, with $n _{j} \geq 1$,
% an element
% \begin{equation*} f _{j} \in hom _{  {A}} (X _{j (0)}, X _{j
% (n _{j})}).
% \end{equation*} 
% We may completely characterize each such $j $ by its image set, and will sometimes write $j $ for the corresponding set and vice
% versa, thus $f _{ [n]} $ corresponds to the identity $j: [n] \to [n]$.
% \item For a given $j: [n _{j}] \to  [n] $,
%    and $i \in [n _{j}] $ 
%    denote by $j - j(i): [n _{j} -1] \to [n]$ the
% unique morphism in $\Delta$  with image set $j-j(i) $. Then the collection of these $f _{j}$ is required to satisfy the
% following equation:
%  \begin{equation} \label{eq.simplex} \mu ^{1} (f _{ j}) = \sum _{0 < i < n
%  _{j}} f _{ j- j (i)} + \sum _{s \geq 2} \sum
%     _{decomp _{s} \in D [n _{j}]} \mu ^{s} (f _{
%        {j \circ j_1}},
%  \ldots, f _{{j \circ j _{s}}}),
% \end{equation}
% with $decomp _{s} \in D [n_j]$ denoting a length $s$ decomposition and ${j_i} $, $1 \leq i \leq s$,  its elements.
% This also corresponds to the discussion in
% Section \ref{sec:OutlineAinftyNerve} describing
%       the case of $2$-simplices of $N (A) $.   
% \end{itemize} 
% \end{definition}
% The simplicial maps are as follows. Given an injection $k: [m] \to [n]$  and an
% $n$ simplex $f $,  define an $m$-simplex $f'$ by
% $\{f' _{j} = f _{k \circ j}\}$, where $j: [l] \to
% [m] $ is an injection.
%
% On the other hand, given the unique surjection in $\Delta
% $:  $s _{i} : [n+1] \to [n]$, $s _{i} (i+1) = s _{i} (i)$,
% and given an $n $-simplex $f $, define an $(n+1)$-simplex $f' $ by setting 
% \[f' _{j} = \left \{ \begin{array}{ll}  
%          e _{X_i} & \mbox{if $j= \{i, i+1\}$ };\\
%         f _{s _{i}  \circ j} & \mbox{if $s _{i}|  _{j}$  is
%         injective}. \\ 
%         0 & \mbox {otherwise},
%           \end{array} \right.
%         \]
% for $j: [l] \to [n+1] $ an injection. It is straightforward but tedious to
% verify that the latter is indeed a face and that simplicial relations are
% satisfied. On the other hand, Faonte
% ~\cite{citeFaonteSimplicialNerve} given a
% conceptual construction of the above nerve so that
% the above is automatic.
%  \begin{proposition}
%  \cite[2.3.2]{citeLeeAfunctorfromLagrangiancobordismstotheFukayacategory},
%  \cite{citeFaonteSimplicialNerve}
% \label{proposition.quasicat} For $ {A}$ a unital $A _{\infty}$ category its nerve $ \mathcal {A}= N ( 
% A)$ is a $\infty$-category.
% \end{proposition}
% For the reader's convenience we outline the proof here. 
% \begin{proof} Suppose we have an inner horn $\rho_k: \Lambda ^{n} _{k} \to
%  NA$. In particular,  by the construction of the
%    simplices of $NA$, corresponding to the faces of
%    the horn, there are determined  $f _{j} \in hom (A) $,
%    for all $j: [n _{j}] \to [n] $ except $j=
%    [n] -\{k\} $ and $j= [n]$. Set $f _{[n]} =0$, and set
% \begin{equation*}  f _{ [n] - \{k\}} =  \sum _{0 < i <
% n; i \neq k} f _{ [n]- \{i \}} + \sum _{s \geq 2} \sum _{decomp _{s} \in D [n]} \mu ^{s} (f
% _{ {j_1}}, \ldots, f _{ {j _{s}}}),
% \end{equation*}
% % this determines a $(n-1)$-simplex  $\sigma$,
% %    where for $j: \Delta ^{n _{j}} \to \Delta
% %    ^{n-1}$,  with $n _{j} \neq n-1$, $f _{j} = f
% %    _{j} $ 
% % then by construction \eqref{eq.simplex} is satisfied for the collection of maps
% %    $ \{f _{j} \} $, $j: \Delta ^{i} \to \Delta
% %    ^{n-1}$,   
% Only thing left to check is that as defined the
% data $\{f _{j}\}$ determines  a $n$-simplex.
%
% This amounts to verifying a pair of identities:
% \begin{align*}
%    0 & =  \sum _{0 < i < n} f _{
%    [n] -i} + \sum _{s \geq 2} \sum _{decomp _{s} \in D
% [n]} \mu ^{s} (f _{ {j_1}},
%  \ldots, f _{{j _{s}}}),  \\
% \mu ^{1} (f _{n-\{k\}}) & = \sum _{0 < i < n-1; i
%    \neq k} f _{
%    [n] -k -i} + \sum _{s \geq 2} \sum _{decomp
% _{s} \in D ([n-1])} \mu ^{s} (f _{{j _{k} \circ j_1}},
%  \ldots, f _{{j _{k} \circ j _{s}}}),
% \end{align*} 
% where $j _{k}: [n-1] \to [n]  $   is the inclusion
% with image $[n] - \{k\}$.  
% %    actually determines the $k $'th face of our
% % simplex.  The only non-obvious part of this, is that
% % the condition \eqref{eq.simplex} is satisfied.
%
% The first identity is immediate by the definition of $f
%    _{[n] - \{k\}  }$.
% For the second identity, a direct calculation is
%    long but straightforward, using the $A
%    _{\infty} $ associativity
% equations. For $n=2$ this is automatic  and for
%    $n=3 $ this is can be checked in
% a few lines. However, for general $n$ it is certainly better to
% prove this using conceptual methods as is done in
% Faonte ~\cite{citeFaonteSimplicialNerve}. 
% \end{proof}
% For $F: A \to B$ an $A _{\infty} $ functor we define $NF: NA \to NB $ 
% via the assignment:
% \begin{equation*} f _{j} \mapsto \sum _{decomp_s \in D [n_j]} F ^{s} (f
% _{j_1}, \ldots,  f_{j_s}).
% \end{equation*}
% \begin{lemma} \cite{citeLeeAfunctorfromLagrangiancobordismstotheFukayacategory},
%  \cite{citeFaonteSimplicialNerve}
%    The assignment $A \mapsto NA$, and $F \mapsto NF$ as above, determines a
% functor $$N: A _{\infty} -Cat ^{unit}   \to \infty
%    -\mathcal{C}at.$$
% \end{lemma}
% % testing.
% The details on why this constitutes a functor $N$
% are omitted. 
% \begin{lemma} Let $C$ be an $A _{\infty}$-category, and $S \subset NC$ 
%    \end{lemma}
% \printbibliography
% \begin{lemma} \label{lemma.cartesian} Suppose alskdfj $p: \mathcal {P} \to G\mathcal {B}$
% is an inner fibration. Suppose further that
% the pullback $ \mathcal {P} _{m}  = m ^{*} \mathcal {P}$, for $m: \Delta ^{1} 
% \to \mathcal {B}$ has the property that the inclusion of fiber maps $ \mathcal
% {P}_0 \to \mathcal {P} _{M}, \mathcal {P} _{1} \to \mathcal {P} _{m}$ are
% equivalences,
%  for every morphism $m $, where $ \mathcal {P} _{i} $ are the
% fibers of $ \mathcal {P} _{m} $ over the vertices $0,1 $. Then $p: \mathcal {P}
% \to \mathcal {B}$ is a (co)-Cartesian fibration. Moreover $ \mathcal {B} $ is a
% Kan complex. 
% \end{lemma}
% \begin{proof} We show that $p $ is co-Cartesian. 
%  Since $ \mathcal {P}_i, \mathcal {P}_{m} $ are $\infty$-categories, 
%  the equivalences $E_{i}: \mathcal {P}_0 \to \mathcal {P} _{m}, \mathcal {P} _{1} \to \mathcal {P} _{m}$ 
%  are invertible in $sSet ^{\tau _{0}} $. Denote by $R _{i} $  their inverses. 
% %  Let $R _{i}:  \mathcal {P} _{m} \to
% % \mathcal {P} _{i}$ denote the associated inverse maps.
%  Thus, there is a  simplicial map
% \begin{align} T: \mathcal {P} _{m} \times \Delta ^{2} \to \mathcal {P}
% _{m}, 
% \end{align}
% with $T ^{0} \equiv T| _{\mathcal {P} _{m} \times \{0\}} = E_1 \circ R_1
% $ and $T ^1 = id$, $T ^{2}= E_1 \circ R_1 $ and $T ^{0,2} \equiv T| _{\mathcal
% {P} _{m} \times \{ \Delta ^{2} _{0,2}\}} $ being the composition of projection
%  $\mathcal {P} _{m} \times \{\Delta ^{2} _{0,2} \} \to \mathcal
% {P} _{m} $
%  with $ E_1 \circ R_1  $. 
% For every $a \in \mathcal {P} _{0}$, we set $e _{a}: \Delta ^{1}
% \to \mathcal {P} _{m}$ to be $T|_ {\{ E_0 a \} \times \Delta ^{2} _{0,1}} $, 
% with $\Delta ^{2} _{0,1} $ denoting the $0,1$ face. 
% The edge $e _{a} $ is an equivalence in $ \mathcal {P} _{m} $, by construction.
% To see this consider $T|_ {\{ E_0 a \} \times \Delta ^{2}} $.
% \end{proof}
%  \section {Reedy model structure} \label {appendix.reedy}
% Under favorable conditions on a small category $B$, the category of $B$ shaped
% diagrams in $C$: $C ^{B}$, i.e. the functor category, has a natural model
% category structure whenever $C$ is a model category. This is the Reedy model
% category structure, (note that this model structure depends on the model 
% category structure of $C$).
% A sufficient condition for this is that $B$ is a so-called 
% Reedy category. 
% \begin{definition} A Reedy category is a small category $B$, with a pair of 
% sub-categories $B ^{+}$, $B ^{-}$ such that:
% \begin{itemize} 
%   \item There exists a \emph{degree function} which assigns to every $b
% \in B$ a non-negative integer $\deg b$, such that  all non-identity maps of
% $B ^{+}$ raise the degree and all non-identity maps of $B ^{-}$ lower the
% degree.
% \item Every map $f \in hom B$ admits a functorial factorization $f= f^{-} f
% ^{+}$, with $f ^{+} \in hom B ^{+}$, $f ^{-} \in hom B ^{-}$.  
% \end{itemize}
% \end{definition}
%
% Let us summarize the main point in one theorem without full
% generalities. 
% \begin{theorem}
%    \cite[22.10]{citeDwyerHirschhornEtAlHomotopylimitfunctorsonmodelcategoriesandhomotopicalcategories.} \label{thm.reedy} The category $\Delta/X _{\bullet}$ as
% the category of simplices of $X _{\bullet}$ is Reedy. Moreover there is a natural model structure on $sSet
% ^{Simp (X)}$: the Reedy model structure, induced by the Joyal model structure
% and we have a Quillen adjunction
%  \begin{equation*} colim: sSet ^{Simp (X)} \longleftrightarrow
% sSet: c ^{*}, 
% \end{equation*}
% where $c ^{*}$ is the functor sending a simplicial set $S _{\bullet}$ to the
% constant diagram with vertices $S _{\bullet}$. In particular there is a left
% derived functor: 
% \begin{equation*} hocolim: \ho sSet ^{Simp (X)} \to\ho sSet.  
% \end{equation*}
% \end{theorem}
% To compute hocolimits in our case we will need explicit description of the
% cofibrations in the Reedy model structure.  For a model category $M$,  a
% Reedy category $B$, $b \in B$ define $\partial B ^{+}/b \subset B ^{+}/b$ to be
% the maximal subcategory, which does not contain the identity of $b$. 
% Define the \emph{latching functor} $LX$ as the composition
% \begin{equation*} M ^{B} \to M ^{B ^{+}/b} \xrightarrow{colim} M,
% \end{equation*}
% with the first map the restriction functor. 
%
% Then $X$ is cofibrant in the Reedy model structure if for every $b \in B$, the
% natural map (induced by the universal property) $LXb \to Xb$ is a cofibration in
% $M$. 
%
% Going back to the case $M=sSet$, $B = \Delta/X _{\bullet} $, the category
% $\partial B
% ^{+}/\Sigma ^{n}$, is identified with the sub-category of the over category of
% the object $\Sigma ^{n}: \Delta ^{n} \to X$,  corresponding to the boundary of
% $\Delta ^{n}$ with morphisms induced by face maps.   
%
%  \begin{proposition} \label{proposition:cofibrant} For $F$ a pre-$\infty$-functor $ NF \in sSet
% ^{\Delta/X _{\bullet} }$ is Reedy cofibrant. 
% \end{proposition}
% \begin{proof} By Proposition \ref{propostion.simplicialmap}
%  we have a simplicial projection 
% \begin{equation*}  colim_{\partial B^{+}/\Sigma ^{n}} \to  \Sigma ^{n} _{*} 
%    (\partial \Delta ^{n}_{\bullet}).
% \end{equation*}
% % with the right hand side the sub-simplicial set of $X _{\bullet}$ corresponding
% % to the boundary of the simplicial set $\Sigma ^{n} _{*} (\Delta ^{n} _{\bullet})
% % \subset X _{\bullet}$. 
%
% The inclusion $\Sigma ^{n} _{*} 
%    (\partial \Delta ^{n}_{\bullet}) \to \Sigma ^{n} _{*}  (\Delta ^{n} _{\bullet})
% \subset X _{\bullet}$ is a
% monomorphism. Moreover, since pre-$\infty$-functors $F$ take
% face maps to faithful embeddings, the functors $ N F$ take face
% maps to monomorphisms in $sSet$. In particular for $$b \in \partial B
% ^{+}/\Sigma, \quad b= (f: \Sigma ^{k} \to \Sigma ^{n}),$$
% It follows that the universal map (which is a map of fibrations)  
% $$colim_{\partial B^{+}/\Sigma ^{n}} \to NF (\Sigma),$$
% is a monomorphism and so is a cofibration in the Joyal model structure.
% \end{proof} 
% Since pre-$\infty$-functors.s $F$ take asdf \cite{Yang-Mills} 
% face maps to mono-ayvz9528morphisms, the functors $ \overline{F}= N F$ take face
% maps to monomorphisms in $sSet$. In particular for $$b \in \partial Simp (X)
% ^{+}/\Sigma, \quad b= (f: \Sigma ^{k} \to \Sigma ^{n}),$$
% $$ \overline{F} (b) = ( \overline{F} (f): \overline{F} (\Sigma ^{k}) \to
% \overline{F} (\Sigma ^{n} ))$$ is a monomorphism. 
%  By description above the universal map $LF \Sigma \to F
% (\Sigma)$ is then clearly also a monomorphism, i.e. a cofibration for the Joyal
% model structure. \textbf{expand this} 
%  with fibrant/cofibrant constants, see \cite[22.10]{DHKS}. 
% \subsection {Reedy categories} A category $C$ is said to be \emph{Reedy}
% ayvz9528 
%  \printbibliography
\bibliographystyle{siam}    
\bibliography{C:/Users/yasha/texmf/bibtex/bib/link} 
% \bibliography{/home/yasha/texmf/bibtex/bib/link}  
% \bibliography{/Users/yasha/texmf/bibtex/bib/link} 
% \bibliography{/home/yashasavelyev/texmf/bibtex/bib/link}  
% S5hTTpKqK
\end{document}
