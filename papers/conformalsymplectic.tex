\documentclass{amsart}  
%[12pt]
%\textwidth=125mm
%\textheight=185mm
%\headheight=10mm
\usepackage{graphicx}
\usepackage {appendix}
\usepackage{amsfonts}
\usepackage{url}
\usepackage{hyperref} 
\hypersetup{backref,pdfpagemode=FullScreen,colorlinks=true}
\usepackage{amsmath}
\usepackage{amssymb}
\usepackage{amscd}
\usepackage{color}
\usepackage{amsthm}
\usepackage{bm}
\usepackage{indentfirst}
\usepackage[hmargin=3cm,vmargin=3cm]{geometry}
\usepackage[all, cmtip]{xy}
\numberwithin{equation}{section}
\newtheorem{thm}[equation]{Theorem} 
\newtheorem{axiom}[equation]{Axiom} 
\newtheorem{theorem}[equation]{Theorem} 
\newtheorem{proposition}[equation]{Proposition}
\newtheorem{lma}[equation]{Lemma} 
\newtheorem{lemma}[equation]{Lemma} 
\newtheorem{cpt}[equation]{Computation} 
\newtheorem{corollary}[equation]{Corollary} 
\newtheorem{clm}[equation]{Claim} 
\newtheorem{conjecture}{Conjecture}
\newtheorem{definition}[equation]{Definition}
\theoremstyle{definition}
% \newtheorem{definition}[equation]{Definition}
\newtheorem{ft}{Fact}
\newtheorem{notation}{Notation}
\newtheorem{descr}{Description}[equation]

\theoremstyle{remark}
\newtheorem*{pf}{Proof}
\newtheorem*{pfs}{Proof (sketch)}
\newtheorem{remark}[equation]{Remark}
\newtheorem{example}{Example}
\newtheorem{question}{Question}

\newcommand{\R}{{\mathbb{R}}}
\newcommand{\Z}{{\mathbb{Z}}}
\newcommand{\Q}{{\mathbb{Q}}}
\newcommand{\D}{{\mathbb{D}}}
\newcommand{\HH}{{\mathbb{H}}}

\newcommand{\bs}{\bigskip}
\newcommand{\ra}{\rightarrow}
\newcommand{\del}{\partial}
\newcommand{\ddel}[1]{\frac{\partial}{\partial{#1}}}
\newcommand{\sm}[1]{C^\infty(#1)}

\newcommand{\delbar}{\overline{\partial}}
\newcommand{\Sum}{\Sigma}
\newcommand{\Pe}{\mathcal{P}}
\newcommand{\X}{\mathfrak{X}}
\newcommand{\J}{\mathcal{J}}
\newcommand{\A}{\mathcal{A}}
\newcommand{\K}{\mathcal{K}}

\newcommand{\ZZ}{\mathcal{Z}}
\newcommand{\eL}{\mathcal{L}}

\newcommand{\mone}{{-1}}
\newcommand{\st}{{^s_t}}
\newcommand{\oi}{_0^1}
\newcommand{\intoi}{\int_0^1}
\newcommand{\til}[1]{\widetilde{#1}}
\newcommand{\wh}[1]{\widehat{#1}}
\newcommand{\arr}[1]{\overrightarrow{#1}}
\newcommand{\paph}[1]{\{ #1 \}_{t=0}^1}
\newcommand{\con}{\#\;}
\newcommand{\codim}{\text{codim}}
\newcommand {\ham} {\text{Ham} (M, \omega)}
\newcommand {\isom} {\text{Isom} ^{h}  (M, \omega, j)}
\newcommand {\lham} {lie \text{Ham} (M, \omega)}
\newcommand {\hamcp} {\text{Ham} (\mathbb{CP} ^{r-1}, \omega )}
\newcommand{\overbar}{\overline}
\newcommand {\vM} {{(T^*)} ^{vert} \cM}
\newcommand{\om}{\omega}
\newcommand{\al}{\alpha}
\newcommand{\la}{\lambda}
\newcommand{\Om}{\Omega}
\newcommand{\ga}{\gamma}
\newcommand{\eps}{\epsilon}
\newcommand{\Cal}{\tex{Cal}}

\newcommand{\cA}{\mathcal{A}}
\newcommand{\cB}{\mathcal{B}}
\newcommand{\cC}{\mathcal{C}}
\newcommand{\cD}{\mathcal{D}}
\newcommand{\cO}{\mathcal{O}}
\newcommand{\cE}{\mathcal{E}}
\newcommand{\cF}{\mathcal{F}}
\newcommand{\cG}{\mathcal{G}}
\newcommand{\cH}{\mathcal{H}}
\newcommand{\cI}{\mathcal{I}}
\newcommand{\cJ}{\mathcal{J}}

%\newcommand{\cO}{\mathcal{O}}
\newcommand{\cS}{\mathcal{S}}

\newcommand{\cU}{\mathcal{U}}

\newcommand{\cQ}{\mathcal{Q}}
\newcommand{\cM}{\bm{M}}
\newcommand{\cP}{\bm{P}}
\newcommand{\cL}{\bm{L}}

\newcommand{\fS}{\mathfrak{S}}
\newcommand{\fk}{\mathfrak{k}}
\newcommand{\fg}{\mathfrak{g}}
% \newcommand{\fz}{\mathfrak{z}}
\newcommand{\fZ}{\mathfrak{Z}}
\newcommand\vol{\operatorname{vol}}
\newcommand {\hatcp}{\widehat{\mathbb {CP}} ^{r-1} }
\newcommand{\rJ}{\mathrm{J}}
\newcommand{\rB}{\mathrm{B}}
\newcommand{\rT}{\mathrm{T}}
\newcommand {\Hpm} {\mathcal{H}^{\pm}}
\newcommand{\bP}{\mathbb{P}}

\DeclareMathOperator {\period} {period}
\DeclareMathOperator {\sign} {sign}
\DeclareMathOperator {\Id} {Id}
\DeclareMathOperator {\floor} {floor}
\DeclareMathOperator {\ceil} {ceil}
\DeclareMathOperator {\mult} {mult}
\DeclareMathOperator {\Symp} {Symp}
\DeclareMathOperator {\Det} {Det}
\DeclareMathOperator {\comp} {comp}
\DeclareMathOperator {\growth} {growth}
\DeclareMathOperator {\energy} {energy}
\DeclareMathOperator {\Reeb} {Reeb}
\DeclareMathOperator {\Lin} {Lin}
\DeclareMathOperator {\Diff} {Diff}
\DeclareMathOperator {\fix} {fix}
% \newcommand{\M}{\mathbb{CP} ^{r-1} }
\DeclareMathOperator {\grad} {grad}
\DeclareMathOperator {\area} {area}
\DeclareMathOperator {\diam} {diam}
% \DeclareMathOperator {\rank} {rank}
\DeclareMathOperator {\dvol} {dvol}
\DeclareMathOperator {\quant} {Quant}
\DeclareMathOperator {\ho} {ho}
\DeclareMathOperator {\length} {length}
\DeclareMathOperator {\Proj} {P}
\renewcommand{\i}{\sqrt{-1}}
\DeclareMathOperator{\mVol}{\mathrm{Vol}(M_0,\omega_0)}
\DeclareMathOperator{\Lie}{\mathrm{Lie}}
\DeclareMathOperator{\lie}{\mathrm{lie}}
\DeclareMathOperator{\op}{\mathrm{op}}
\DeclareMathOperator{\rank}{\mathrm{rank}}
\DeclareMathOperator{\ind}{\mathrm{ind}}
\DeclareMathOperator{\trace}{\mathrm{trace}}
\DeclareMathOperator{\image}{\mathrm{image}}
\DeclareMathOperator{\Sym}{\mathrm{Sym}}
\DeclareMathOperator{\Ham}{\mathrm{Ham}}
\DeclareMathOperator{\Aut}{\mathrm{Aut}}
\DeclareMathOperator{\Quant}{\mathrm{Quant}}
\DeclareMathOperator{\Fred}{\mathrm{Fred}}
\DeclareMathOperator{\id}{\mathrm{1}}
\DeclareMathOperator{\lcs}{lcs}
\DeclareMathOperator{\lcsm}{lcsm}
% \DeclareMathOperator{\ker}{ker}
\DeclareMathOperator{\coker}{coker}
\begin{document}
\title{Conformal symplectic Weinstein conjecture and non-squeezing}
\author{Yasha Savelyev}
\thanks {Partially supported by PRODEP grant}
\email{yasha.savelyev@gmail.com}
\address{University of Colima, CUICBAS}
\keywords{locally conformally symplectic manifolds, conformal symplectic non-squeezing, Gromov-Witten theory, virtual fundamental class, Fuller index, Seifert conjecture, Weinstein conjecture}

\begin{abstract} We study here, from the Gromov-Witten theory point of view, some aspects of rigidity of locally conformally symplectic manifolds, or $\lcs$ manifolds for short, which are a natural generalization of both contact and symplectic manifolds.  As a first step we show that the classical Gromov non-squeezing theorem has a certain $C ^{0} $ rigidity or persistence with respect to $\lcs$ deformations. This is one version of $\lcs$ non-squeezing, another possible version of non-squeezing related to contact non-squeezing is also discussed.  In a different direction we study Gromov-Witten theory of the $\lcs$ manifold $C \times S ^{1} $ induced by a contact form $\lambda$ on $C$, and show that the extended Gromov-Witten invariant counting certain charged elliptic curves in $C \times S ^{1} $ is identified with the extended classical Fuller index of the Reeb vector field $R ^{\lambda} $, by extended we mean that these invariants can be $\pm \infty$-valued. Partly inspired by this, we conjecture existence of certain 1-d curves we call Reeb curves in certain $\lcs$ manifolds,  which we call conformal symplectic Weinstein conjecture, and this is a direct extension of the classical Weinstein conjecture.  Also using Gromov-Witten theory, we show that 
%    We find that the main new phenomenon (relative to the symplectic case) is the potential existence of holomorphic sky catastrophes, an analogue for pseudo-holomorphic curves of sky catastrophes in dynamical systems originally discovered by Fuller. We are able to rule these out in some  situations, particularly for certain $\lcs$ 4-folds, and as
% one application 
the CSW conjecture holds for a $C ^{3} $- neighborhood of the induced lcs form on $C \times S ^{1} $, for $C$ a contact manifold with contact form whose Reeb flow has non-zero extended Fuller index, e.g. $S ^{2k+1} $ with standard contact form, for which this index is $\pm \infty$. We show that this can be globalized provided there do not exist sky catastrophes for families of holomorphic curves in a $\lcs$ manifold. The latter phenomenon is not known to exist, but if it does, would be analogous to sky catastrophes in dynamical systems discovered by Fuller in the 1950's.  
% Furthermore, we show that either it holds for any regular Lichnerowicz exact $\lcs$ structure on $C \times S ^{1} $, homotopic through general $\lcs$ forms to the Hopf $\lcs$ structure, or %  This has some non-classical applications, and based on the story we develop, we give a kind of ``holomorphic Seifert/Weinstein conjecture'' which is a direct extension for some types of $\lcsm$'s of the classical Seifert/Weinstein conjecture. This is proved for $\lcs$ structures $C ^{\infty} $ nearby to the Hopf $\lcs$ structure on $S ^{2k+1} \times S ^{1}  $.
%  % As a more basic consequence we show here that classical Gromov-Witten
% theory of a symplectic manifold is invariant under a much larger class
% of deformations than
% is classically understood, (that is we may allow conformal
% deformations) this in principle gives more flexibility for doing
% surgery for example.
\end{abstract}
 \maketitle
 \tableofcontents 
 \section {Introduction}
% In the case of the example above, our Gromov-Witten invariants
% particularly something we call quantum Euler characteristic, turn out
% to be closely related to the Fuller index of the Reeb vector field,
% associated to a fixed free homotopy class of loops in $C$,
% and which in principle is defined under very general assumptions. In some
% very special cases, (finitely many Reeb orbits, and no multiply
% covered orbits in a fixed homotopy class), quantum Euler characteristic also coincides with
% the Euler characteristic of cylindrical contact homology, (when it is
% defined). In general it is
% tricky to relate quantum Euler characteristic to contact homology
% Euler characteristic. But hypothetically it may work like the following: if
% cylindrical contact homology was finite dimensional and functorially identified with rational homology of
% the quotient space $Y/T $, for a space $Y$
% equipped with a $T$ action, (so that the quotient is of orbifold type)  then quantum Euler characteristic
% that we define would be the orbifold Euler characteristic of the
% orbifold $T$ quotient of $Y$.
% s seem to
% contact homology. As  both invariants are
% ``counting'' Reeb orbits in some way.  The (mean) Euler characteristic  of the (linearized/cylindrical)
% contact homology is a rather powerful invariant, with a great
% advantage of being fairly computable. See for instance
% \cite{citeViktorYusufIteratedindex}, \cite{citeJacquelineOntheMean},
% \cite{citeOttoOpenBooksBoothby}.
% To push the Euler characteristic analogy further recall that for a
% oriented smooth manifold we may calculate the Euler characteristic in
% two ways: in terms of Betty numbers, and by counting zero's of a
% vector field. In some Morse-Bott situations  considered in
% this note further on, our calculation of the Gromov-Witten invariant proceeds exactly by counting
% zero's of a vector field on manifolds of Reeb orbits, whereas in some analogous calculations of the
% Euler characteristic of contact homology one  looks at the Betty
% numbers of manifolds of Reeb orbits. (At least roughly speaking.)
%
% However there there are also a couple big conceptual and technical
% differences. First, although we shall not make it explicit here, the
% above pairs of invariants appear to be quantitatively very different when moving
% away from perfect Morse-Bott situations, (even in perfect Morse-Bott
% situations the invariants are at least superficially different.) 
% Also, in our case we really do have Gromov-Witten invariants with all the
% inherent potential advantages: gluing sum formulas, degenerations of
% complex structures, etc., and disadvantages: can be hard to compute
% due to transversality,
% orientation issues, etc., although there are at least sometimes
% similar issues with the mean  Euler characteristic. Also, our invariants are defined for any contact
% manifold, while  there are restrictions on the contact
% homology side (one needs cylindrical homology to be defined, or we need
% augmentations, etc.). 
% We note that in the main example our invariants are 
% counting holomorphic tori, and consequences for existence of Reeb
% orbits are obtained.
% This is exciting, as although there is much research on
% enumerative geometry, counting genus 1 curves, geometric applications
% of Gromov-Witten theory
% in symplectic geometry have to my knowledge been mostly concerned with 
% genus 0 curve counts. In  this note genus 1 curve counts will have
% a deep geometric significance.
% I am not
% sure if something like this can happen in the contact homology picture,
% however moduli spaces of holomorphic tori do come in Floer-Novikov
% theory \cite{cite} and this might be related.
% \begin{definition} \label{def:locallyconformal} A locally conformally
%    symplectic manifold or $\lcs$ manifold $M ^{2n} $, is a smooth manifold
%    $M$, so that the frame bundle $\mathcal{F}M$ of the tangent bundle, has a
%    reduction of structure group to the group of conformal symplectic
%    automorphisms of $\mathbb{R}^{2n} $: $\mathcal{C} (2n)$, which
%    satisfies the local integrability condition. For every point $x \in
%    M$ there is a smooth chart $\phi: \mathbb{R} ^{2n} \to M $, so that 
%    diffeomorphism $\mathcal{F}M$ is
%    locally diffeomorphic to $$
%
% \end{definition}
% The following theorem is proved for a general compact almost complex manifold in \cite[]{citeMcDuffSalamon$J$--holomorphiccurvesandsymplectictopology}. \textcolor{blue}{add this} we shall give an elementary proof  in the case $J$ is compatible with an $\lcs$ form.
The theory of pseudo-holomorphic curves in symplectic manifolds as initiated by Gromov and Floer has revolutionized the study of symplectic and contact manifolds. What the symplectic form gives that is missing for a general almost complex manifold is a priori energy bounds for pseudo-holomorphic curves a fixed class.
On the other hand there is a natural structure which  directly generalizes both symplectic and contact manifolds, called locally conformally symplectic structure or $\lcs$ structure for short. A locally conformally symplectic manifold or $\lcsm$ is a smooth $2n$-fold $M$ with an $\lcs$ structure: which is a
non-degenerate 2-form $\omega$, which is locally diffeomorphic to $
{f} \cdot \omega _{st}  $, for some (non-fixed) positive smooth function $f$, with $\omega _{st}  $ the standard symplectic form on
$\mathbb{R} ^{2n} $. It is natural to try to do Gromov-Witten theory for such manifolds.
The first problem that occurs is that a priori energy bounds are gone, as since $\omega$ is not necessarily closed, the $L ^{2} $-energy can now be unbounded on the moduli spaces of $J$-holomorphic curves in such a $(M,\omega)$. Strangely a more acute problem is potential presence of holomorphic sky catastrophes - given a smooth family $\{J _{t} \}$, $t \in [0,1]$, of $\{\omega _{t} \}$-compatible almost complex structures, we may have a continuous family $\{u _{t} \}$ of $J _{t} $-holomorphic curves s.t. $\energy (u _{t} ) \mapsto \infty$ as $t \mapsto a \in (0,1)$ and s.t. there are no holomorphic curves for $t \geq a$. These are analogues of sky catastrophes discovered by Fuller \cite{citeFullerBlueSky} for closed orbits of dynamical systems.

We can tame these problems in certain situations and this is how we arrive at a certain $\lcs$ extension of Gromov non-squeezing.
Even when it is impossible to tame these problems we show that there can still be an extended Gromov-Witten type theory which is analogous to the theory of extended Fuller index in dynamical systems, \cite{citeSavelyevFuller}. In a very particular situation the relationship with the
Fuller index becomes perfect as one of the results 
of this paper obtains the (extended) Fuller index for Reeb vector fields on a contact manifold $C$ as a
certain (extended) genus 1 Gromov-Witten invariant of the Banyaga $\lcsm$ $C \times S ^{1} $, see Example \ref{example:banyaga}. The
latter also gives a conceptual interpretation for why the Fuller index is
rational, as it is reinterpreted as an (virtual) orbifold Euler number.

Inspired by this, we conjecture that certain $\lcsm$'s must poses certain  curves that we call Reeb curves, and this is a direct generalization of the Weinstein conjecture,  we may call this conformal symplectic Weinstein conjecture.
We prove this CSW conjecture for certain $\lcs$ structures $C ^{3} $ nearby to Banyaga type lcs structures on $C  \times S ^{1}  $. This partly uses the above mentioned connection of Gromov-Witten theory of $C \times S ^{1}  $ with the classical Fuller index. Note that Seifert \cite{citeSeifert} was likewise initially motivated by a $C ^{0} $ neighborhood version of the Seifert conjecture for $S ^{2k+1} $, which he proved. We could say that in our case there is more evidence for globalizing, since the original Weinstein conjecture is already proved, Taubes~\cite{citeTaubesWeinsteinconjecture}, for $C$ a closed contact three-fold. In addition to the $C ^{3} $ neighborhood version, we also prove a stronger result that relates the  CSW conjecture to existence of holomorphic sky catastrophes.

Finally, we should exclaim that the Gromov-Witten theory in this story plays a local (in the space of structures) role, unless addition global geometric control is obtained. (As is the case for us sometimes.) This is analogous to what happens with Fuller index in dynamical systems. 
A global $\lcs$ invariant, which takes the form of a homology theory, is under development, but many ingredients for this are already present here. (For example generators, and appropriate almost complex structures.)
\subsection {Locally conformally symplectic manifolds}
Let us give a bit of background on $\lcsm$'s.
% A locally conformally symplectic manifold is a
% smooth $2n$-fold $M$ with a
% non-degenerate 2-form $\omega$ which is locally diffeomorphic to $e
% ^{f} \omega _{0}  $, for some (non-fixed) function $f$, with $\omega _{0}  $ the standard symplectic form on
% $\mathbb{R} ^{2n} $. 
These were originally considered by Lee
in \cite{citeLee}, arising naturally as part of an abstract study of
``a kind of even dimensional Riemannian geometry'', and then further studied by
a number of authors see for instance, \cite{citeBanyagaConformal} and
\cite{citeVaismanConformal}.
This is a
fascinating object,  a $\lcsm$ admits all the interesting classical notions of
a symplectic manifold, like Lagrangian submanifolds and Hamiltonian
dynamics, while at the same time forming a much more
flexible class. For example Eliashberg and Murphy show that if a
closed almost complex $2n$-fold $M$ has $H ^{1} (M, \mathbb{R}) \neq 0
$ then it admits a $\lcs$ structure,
\cite{citeEliashbergMurphyMakingcobordisms}, see
also \cite{citeMurphyConformalsymp}.  

To see the connection with the first cohomology group, let us point out right away the most basic invariant of a $\lcs$ structure $\omega$:  the Lee class, $\alpha = \alpha _{\omega}  \in H ^{1} (M, \mathbb{R}) $. This has the property that on the associated $\alpha$-covering space $\widetilde{M} $, $\widetilde{\omega} $ is globally conformally symplectic. The class $\alpha$ may be defined as the following Cech 1-cocycle. 
Let $\phi _{a,b}$ be the transition map for $\lcs$ charts $\phi _{a}, \phi _{b}  $ of $(M, \omega)$. Then $\phi _{a,b} ^{*} \omega _{st} = g _{a,b} \cdot \omega _{st}  $ for a positive real constant $g _{a,b} $ and $\{\ln g _{a,b} \}$ gives our 1-cocycle. Thus an $\lcs$ form is globally conformally symplectic iff its Lee class vanishes.


Assuming $M$ has dimension at least 4, the Lee class $\alpha$ has a natural differential form representative, called the Lee form and defined as follows.  We take a cover of $M$ by open sets $U _{a} $ in which $\omega= f _{a} \cdot \omega _{a}   $ for $\omega _{a}  $ symplectic, and $f _{a} $ a positive smooth function.
   Then we have 1-forms $d (\ln f _{a} )$ in each $U _{a} $ which glue to a well defined closed 1-form on $M$, as shown by Lee. By slight abuse, we denote this 1-form, its cohomology class and the Cech 1-cocycle from before all by $\alpha$.
It is moreover immediate that for an $\lcs$ form $\omega$ $$d\omega= \alpha \wedge \omega,$$
for $\alpha$ the Lee form as defined above.

As we mentioned $\lcsm$'s can also be understood to generalize contact manifolds. This works as follows.
% However Gromov-Witten type theory of $\lcs$ manifolds, has not been
% considered, (beyond obvious questions on its existence.)
% This is
% tricky, because there often will not be
% any global compactness for spaces of pseudo-holomorphic curves in a $\lcsm$.
First we have a natural class of explicit examples of $\lcsm$'s, obtained
by starting with a symplectic cobordism (see \cite{citeEliashbergMurphyMakingcobordisms}) of a closed contact manifold
$C$ to itself, arranging for the contact forms at the two ends of the
cobordism to be proportional (which can always be done) and then
gluing together the boundary components. 
As a particular case of
this we get Banyaga's basic example.
\begin{example} [Banyaga] \label{example:banyaga} Let $(C, \xi)
   $ be a contact manifold with a contact form
   $\lambda$ and take $M=C \times S ^{1}  $ with 2-form $\omega= d
^{\alpha} 
   \lambda : = d \lambda - \alpha \wedge \lambda$, for $\alpha$ the pull-back of the
   volume form on $S ^{1} $ to $C \times S ^{1} $ under the
   projection, and $\lambda$ likewise the pull-back of $\lambda$ by the projection $C \times S ^{1} \to C $.
   %
  % \textcolor{blue}{deformation invariance?} 
   % It is easy to verify that a contact (iso)-morphism $(C
   % _{1} ,
   % \xi _{1} ) \to (C _{2}, \xi _{2}  )$ in the sense of Definition \ref{def:morphismslcs} induces an $\lcs$
   % (iso)-morphism from $(C _{1}  \times S ^{1}, d ^{\alpha}\lambda
   % _{1})   \to (C _{2}  \times S ^{1}, d ^{\alpha}\lambda
   % _{2})   $. 
   % We have an $S ^{1} $ action on the $\lcs$ $C \times S ^{1} $ by
   % rotation in the $S ^{1} $ variable, and the induced
   % $(iso)$-morphism is $S ^{1} $-equivariant.
\end{example}
The operator $d ^{\alpha}: \Omega ^{k} (M) \to \Omega ^{k+1} (M)   $ is called the Lichnerowicz differential with respect to a closed 1-form $\alpha$,
and satisfies $d ^{\alpha} \circ d ^{\alpha} =0  $ so that we have an associated Lichnerowicz complex.

Using above we may then  faithfully embed the category of contact manifolds, and contactomorphism  into the category of $\lcsm$'s, and certain $\lcs$ morphisms as defined below. 
\begin{definition} \label{definition:lcsmap}
   A diffeomorphism $\phi: (M _{0}, \omega _{0} ) \to (M _{1}, \omega _{1}  ) $ is said to be an \textbf{\emph{lcs map}} if $\phi ^{*} \omega _{1}  $ is homotopic through $lcs$ forms $\{\omega _{t} \}$, in the same $d ^{\alpha} $ Lichnerowicz cohomology class, to $\omega _{0} $, where $\alpha $ is the Lee form of $\omega _{0} $ as before. In other words, for each $t _{0}  \in [0,1]$, $$d ^{\alpha} (\frac{d}{dt} \vert _{t=t _{0}} \omega _{t}) = 0. $$
\end{definition}
We also define, following Banyaga,
\textbf{\emph{conformal symplectomorphisms}} $\phi: (M _{1} , \omega _{1} ) \to (M _{2}, \omega _{2}  )$ to be diffeomorphisms satisfying  $\phi ^{*} \omega _{2} = f \omega _{1} $ for a smooth positive function $f$.
% , see \cite{citeBanyagaConformal} for details on how this embedding works.



% We show that it is still possible to extract a variant of Gromov-Witten theory here.
% The story is closely analogous to that of the Fuller index in dynamical
% systems, which is concerned with certain rational counts of periodic orbits. In
% that case sky catastrophes prevent us from obtaining a completely well
% defined invariant, but Fuller constructs certain partial invariants which give
% dynamical information. 
\subsection{Conformal symplectic Weinstein conjecture} \label{sec:highergenus}
We state this conjecture immediately and then motivate it.
An \textbf{\emph{exact lcs structure}} on $M$  is a pair $(\lambda, \alpha)$
with $\alpha$ a closed 1-form, s.t. $\omega=d ^{\alpha} \lambda $ is non-degenerate.
This determines a generalized distribution $\mathcal{V} _{\lambda} $: $$\mathcal{V} _{\lambda} (p) = \{v \in T _{p} {M}| d \lambda (v, \cdot) = 0 \}, %    
$$ which we call the \textbf{\emph{vanishing distribution}}. And 
we have a generalized distribution $\xi _{\lambda} $, which is defined to be the $\omega$-orthogonal complement to $\mathcal{V} _{\lambda}$, which we call \textbf{\emph{co-vanishing distribution}}.
For each $p \in M$, $\mathcal{V} _ {\lambda} (p) $  has dimension at most 2 since $d\lambda - \alpha \wedge \lambda$ is non-degenerate. Moreover $\mathcal{V} _{\lambda} $ cannot identically vanish,  since $M ^{2n} $ is closed and $(d\lambda) ^{n}   $ cannot be non-degenerate by Stokes theorem.
% For $\omega,J$ as above the pair $(\omega,J)$ will be called \textbf{\emph{admissible}}. 
\begin{definition} 
Let $(M,\lambda, \alpha)$ be an exact $\lcs$ structure, $\omega= d ^{\alpha} \lambda. $  Then a smooth map $o: S ^{1} \to M$, s.t. $\forall s \in S ^{1}: o ^{*} \lambda (s) \neq 0  $, is
called a \textbf{\emph{Reeb curve}} for $(M,\lambda,\alpha)$. %    , and if
% $$0 \neq [u ^{*} \alpha] \in H ^{1} _{DR}  (\Sigma). $$ 
% When $\Sigma$ is elliptic the curve $u$ will be called an \textbf{\emph{elliptic Reeb curve}},
% if in addition $\Sigma \simeq T ^{2} $ we will just call such a curve a \textbf{\emph{Reeb torus}}.
\end{definition}
% \begin{definition} 
%    Let $(M,\lambda, \alpha)$ be an exact $\lcs$ structure, $\omega= d ^{\alpha} \lambda, $ and let $\Sigma$ be a closed possibly nodal Riemann surface. A smooth map $u: \Sigma  \to M$ is a \textbf{\emph{Reeb curve}} if $du (T (\Sigma)) \subset \mathcal{V} (\lambda)$ and if there is a smooth submanifold $S _{0} \simeq S ^{1}   \subset \Sigma $ so that $u ^{*}\lambda  $ is non-vanishing on $T {S _{0} } \subset T \Sigma$.
% %    , and if
% % $$0 \neq [u ^{*} \alpha] \in H ^{1} _{DR}  (\Sigma). $$ 
% When $\Sigma$ is elliptic the curve $u$ will be called an \textbf{\emph{elliptic Reeb curve}},
% if in addition $\Sigma \simeq T ^{2} $ we will just call such a curve a \textbf{\emph{Reeb torus}}.
% \end{definition}
% \begin{lemma} \label{lemma:rational} 
%    Every 
% %    Let $(M,\lambda, \alpha)$ be an exact $\lcs$ structure. Suppose that $\alpha$ is rational, then every elliptic Reeb curve is a Reeb torus. 
% % \footnote {It is likely that rationality condition on $\alpha$ is unnecessary.}
% \end{lemma}
We then have the following basic ``conformal symplectic Weinstein conjecture'', %l
later on we state a stronger form of this conjecture.
\begin{conjecture} \label{conjecture:Weinstein} Let $M$ be closed of dimension at least 4, and $\omega$ be an exact $\lcs$ form on $M$, then there is a Reeb curve for $(M, \omega)$. 
\end{conjecture}
The dimension 2 case is special but some version (possibly same version) of the conjecture should hold in this case. As one trivial example, given an lcs 2-manifold $(M, \lambda, \alpha)$, with $d\lambda=0$ and with $\alpha$ rational, the conjecture holds automatically, just take the Reeb curve to parametrize a component of a regular fiber of the map $f: \Sigma \to S ^{1} $ classifying $\alpha$, that is so that $\alpha = q \cdot f ^{*} d\theta  $, for $q \in \mathbb{Q}$.
%
% for such an exact lcs structure , e.g. $M=T ^{2} $, $\lambda = d\theta _{1} $, $\alpha=d \theta _{2} $.
% The following is proved in Section \ref{sectionFuller}.
\begin{lemma} \label{thm:implies}
Conjecture \ref{conjecture:Weinstein} implies the Weinstein conjecture.   
\end{lemma}
\begin{proof} 
   Let $S _{0} \subset  C \times S ^{1} $ be a Reeb curve for the Banyaga $\lcs$ structure $d ^{\alpha} \lambda $.
   Since $T S _{0}  \subset \mathcal{V} _{\lambda} $, $(pr _{C})_* (TS _{0})  \subset \ker d\lambda \subset TC $. 
Since in addition $\lambda$ is non-vanishing on $TS _{0}$, $pr _{C} (S _{0} ) $  is embedded in $C$ and is the image of a Reeb orbit. 
\end{proof}

% Note that the above conjecture is trivially true when $M$ has dimension 2, as elementary Morse theory forces any closed $M$ of dimension 2 with an exact lcs structure to be a torus.
In what follows we use the following $C ^{3}$ metric on the space $\mathcal{L} (M) $ of exact $\lcs$ structures on $M$. For $(\lambda _{1}, \alpha _{1}), (\lambda _{2}, \alpha _{2}) \in \mathcal{L} (M)$ define:
\begin{equation*}
   d _{C ^{3}}  ((\lambda _{1}, \alpha _{1}), (\lambda _{2}, \alpha _{2})) = d _{C ^{3} } (\lambda _{1}, \lambda _{2}  )  + d _{C ^{3}} (\alpha _{1}, \alpha _{2}  ),
\end{equation*}
where $d _{C ^{3}} $ on the right side is the usual $C ^{3} $ metric. We say that an exact $\lcs$ structure $(M ^{2n}, \lambda, \alpha)$ is \textbf{\emph{regular}} if the set:
\begin{equation*}
  {V} (M, \lambda):= \{p \in M| (d\lambda) ^{n} (p)=0 \},
\end{equation*}
is a smooth submanifold of $M$. A \textbf{\emph{regular $C ^{3} $-neighborhood}} of an $\lcs$ structure is then a neighborhood with respect to $d _{C ^{3} } $ intersected with the subset of all regular $\lcs$ structures.

For $\lambda _{H} $ the standard contact structure on $S ^{2k+1} $, so that its Reeb flow is the Hopf flow, we will call $\omega _{H}:= d ^{\alpha} \lambda _{H}$ 
the \textbf{\emph{Hopf $\lcs$ structure}}.
\begin{theorem} \label{thm:C0Weinstein} 
Conjecture \ref{conjecture:Weinstein} holds for a regular $C ^{3} $-neighborhood of the Hopf $\lcs$ structure $(\lambda _{H}, \alpha)$ on $S ^{2k+1} \times S ^{1}  $.
\end{theorem}
This is proved in Section \ref{sectionFuller}. Note that Seifert \cite{citeSeifert} initially found an analogous existence phenomenon of orbits on $S ^{2k+1} $ for a non-singular vector field $C ^{0} $-nearby to the Hopf vector field, \footnote {With more careful analysis we can also likely relax $C ^{3} $ condition to $C ^{0} $ condition.}. And he asked if the nearby condition can be removed, this became known as the Seifert conjecture. This turned out not to be quite true \cite{citeKuperbergvolumepreserving}.
Likewise it is natural for us to conjecture that the nearby condition can be removed and this is the CSW conjecture. In our case this has some additional evidence that we discuss in the next section.

Directly extending Theorem \ref{thm:C0Weinstein} we have the following.
% Call a contact form $\lambda$  on $C$ \textbf{\emph{weakly hyperbolic}} if all its Reeb orbits in a fixed homotopy class are simple. We call $(C,\lambda)$ finite 
% \begin{definition}
% We call an exact $\lcs$ structure $(\alpha, \lambda)$ on $M=C \times S ^{1} $ \textbf{\emph{weakly hyperbolic}}, if for an admissible almost complex structure $J$, in each class $A$, there is exactly one regular, charge (1,0), embedded $J$-holomorphic elliptic curve in $M$.
%  \end{definition}
% \begin{example}
% If $\lambda$ is a hyperbolic contact form on a closed manifold $C$, then by the results in Section \ref{sectionFuller} the associated  Banyaga $\lcs$ structure $(\alpha, \lambda)$ is weakly hyperbolic.
% \end{example}  
\begin{theorem} \label{thm:catastrophyCSW} Let $C$ be a closed contact manifold with contact form $\lambda$. Let  $(\lambda,\alpha)$ be the associated Banyaga lcs structure on $M = C \times S ^{1} $,  with $i (R ^{\lambda}, \beta) \neq 0$, for some $\beta$, where the latter is the extended  Fuller index, as described in Appendix \ref{appendix:Fuller}. 
Then  either the conformal symplectic Weinstein conjecture holds 
% for any regular exact $\lcs$ structure on $M=C \times S ^{1} $, homotopic through (general) $\lcs$ forms to a weakly hyperbolic $\lcs$ structure, and it holds
for any regular exact $\lcs$ structure $(\lambda',\alpha')$ on $M$, so that $\omega _{1} = d ^{\alpha'} \lambda' $ is homotopic through (general) $\lcs$ forms $\{\omega _{t} \}$ to $\omega _{0} = d ^{\alpha} \lambda  $ or holomorphic sky catastrophes exist. 
%    In fact there is a sky catastrophe for a family 
% $\{J _{t} \}$, with $J _{t} $ admissible with respect to an $\lcs$ form $\omega _{t} $ for each $t$.
\end{theorem}
\begin{example} Take $C = S ^{2k+1} $ and $\lambda = \lambda _{H} $, then $i (R ^{\lambda}, 0) = \pm \infty$, (sign depends on $k$), \cite{citeSavelyevFuller}. Or take $C$ to be unit cotangent bundle of a hyperbolic manifold $(X,g)$, $\lambda $ the associated Louiville form, and $(\lambda, \alpha)$ the associated Banyaga lcs structure, in this case $i (R ^{\lambda}, \beta )=\pm 1$ for every $\beta \neq 0$.
\end{example}
To motivate the above conjecture we need pseudo-holomorphic curves in $\lcs$ manifolds.
% We note that the invariant $GW (N, A,  J)$ above is a natural analogue
% of the Fuller index in the theory of dynamical systems and we will discuss this connection further on in more
% detail. The proposition above implies the following.
% \begin{theorem} \label{thm:GWFuller} 
% Suppose that $GW _{g,n}  (N, A, J) \neq 0$ then there
% exists an $\epsilon>0$ so that whenever $(\omega', J')$ is a compatible $\lcs$ tuple, with $J'$ $C ^{0} $ close
% to $J$
% then there is at least  one
% $J'$-holomorphic class $A$ (stable) genus $g$ curve in $M$. \textcolor{blue}{X or M?} 
% \end{theorem}
% Note that the $C ^{\infty} $ version of the above theorem holds for any
% general perturbation of $\omega$ in the space of non-generate 2-forms, (not necessarily
% $\lcs$). This readily follows by Gromov's proof of non-squeezing. The $C ^{0} $
% version for general non-degenerate perturbations of $\omega$ is likely false,
% (there is no obvious obstruction) however constructing an example looks to be very difficult.
% We shall discuss further on some examples where it is possible to work with directly with
% the invariants above. However a different approach to dealing with the inherent
% non-compactness of the moduli spaces $\overline{\mathcal{M}} _{g}   (J,
% A)$ is to use extra geometry on our $\lcs$ $(M, \omega)$ in the form of a
% suitably free $T ^{n} $ action, for which $\energy$ is an analogue of an abstract, proper,
% bounded from below, moment map function, see Karshon
% \cite{citeKarshonMomentmapsandnoncompactcobordisms}. 
% We now describe this and give applications to counting $S ^{1} $ invariant
% orbits of some Reeb vector fields. \textcolor{blue}{give applications here} 
% \subsection{$T ^{2}  $-equivariant Gromov-Witten theory of an $\lcs$}
% % \begin{example} \label{example:globallysymplectic} A trivial example. Suppose that $(M, \omega)$ is a closed $\lcs$ such that
% % $\omega$ is globally conformally symplectic, which is the case for
% % example if $H ^{1} (M)=0 $. Then it immediately
% % follows that $\energy$ is globally bounded for a given class $A$, so
% % that in the theorem above we have compact moduli spaces, and we
% % do Gromov-Witten theory as usual. 
% % , except that the $\energy$ scale is
% % no longer natural, and $\energy$  is not constant on the connected
% % components of the moduli space, let alone moduli spaces in  a given class $A$. 
% % \end{example}
% % \textcolor{blue}{remove this} 
% % We shall show further on that in the setting of Example
% % \ref{example:banyaga} $\energy$ is bounded on the
% % connected components of the moduli space. Strictly speaking we show
% % that it is bounded on
% % components of
% % $\overline{\mathcal{M}} _{1} (J,A)$, but then it is easy to see that
% % every holomorphic curve in $C \times S ^{1} $ is in fact a covering map of genus 1 curve,
% % so that the case of a general curves is implied. 
% % This allows us to obtain interesting invariants even with energy being
% % globally
% % unbounded.       
% % This is closely related to the theory of non-compact cobordisms equipped with abstract, proper,
% % bounded from below, moment map functions, see Karshon
% % \cite{citeKarshonMomentmapsandnoncompactcobordisms}, for a specific
% % circle action. This suggests a path to generalizing some of our
% % constructions 
% % to more general examples of $\lcs$'s, endowed with a circle action,
% % which will be explored elsewhere.  
% % And this is sufficient to have
% % a certain well defined theory of $S ^{1} $-equivariant Gromov-Witten
% % invariants, at least of the type considered here.
% % Although the language of equivariant cobordisms is not absolutely necessary for our
% % specific example  in this note, to discuss invariance properties of
% % our Gromov-Witten invariants it helps in order to have a more
% % conceptual framework, and overall the language is suggestive.
% % \begin{remark} \label{rmk:}
% % Note that the moduli space $\overline{\mathcal{M}}  (\Sigma, J,
% %  A)$ itself has a
% % $\lcs$ structure whenever it forms a smooth manifold.  And the energy
% % function $E$ is then smooth and determines a Hamiltonian flow. When is this a circle action?
% % We do not attempt to answer this, but this indicates that the
% % ``abstract'' moment map property of  $E$ that we consider further on
% % in a special case, is perhaps not
% % that abstract. (It looks rather Hamiltonian.)
% % \end{remark}
% % , for some circle actions
% % coming from a circle action on the $(M, \omega)$.
% Let $(M, \omega)$ be an $\lcs$ equipped with a free $T ^{2} $ action preserving
% $\omega$, and let $J$ be an $T ^{2}  $ invariant, $\omega$-compatible almost complex
% structure  on $M$. We call $(\omega, J)$ a $T ^{2}  $ \emph{invariant pair}.
% \begin{definition}
% Given  $T ^{2}  $ invariant pairs $(\omega _{i}, J _{i}
% )$, $i=0,1$,  we say that they are $T ^{2}  $-equivariantly concordant if
% there exists a smooth family $\{ (\omega _{t}, J _{t}  )\}$ of compatible $T ^{2}  $ invariant pairs interpolating between them.
% \end{definition}
%
%
% \textcolor{blue}{existence of such a J?} 
% Let $$\energy: \overline{\mathcal{M}} (M, J, A) \to \mathbb{R} $$ be the associated
% $L ^{2} $ energy function as before. Then $\energy$ is invariant under the
% induced action of $T ^{2} $ on $\overline{\mathcal{M}} (M, J, A) \to
% \mathbb{R}$. In particular the subspace $\overline{\mathcal{M}} (M, J, A) _{E} $
% defined as before is $T ^{2} $ invariant. Denote by $\overline{\mathcal{M}} (M,
% J, A, E) $ the subspace of $\overline{\mathcal{M}} (M, J, A)  $ consisting of
% elements $u$ with $\energy (u) = E$.
% \begin{definition} \label{def:partiallyfree} We shall say that a continous action of $T
%    ^{2} $ on a topological space $S$ is \textbf{\emph{partially free}} if the
%    isotropy group of each $s \in S$ has codimension at least 1.
% \end{definition}
% \begin{theorem} \label{thm:invariantT2main} Given a $T ^{2} $ invariant pair $(\omega _{i}, J _{i}
% )$, for generic $E>0$ $T ^{2}
%    $ acts acting partially freely on $\overline{\mathcal{M}} (M,
% J, A, E) $, moreover for such an $E$ $GW (\overline{\mathcal{M}} (M, J, A) _{E}, A,J )$ is invariant
% of the $T ^{2} $ equivariant concordance class of $(\omega, J)$.
% \end{theorem}
% as the expected dimension of
% the analytic moduli spaces we consider is 0, this does not give any extra structure on
% Gromov-Witten theory, which is to follow. Nevertheless we may
% understand it as being formally $T$-equivariant. As we have commented
% above, in the future
% we shall consider other $\lcs$ manifolds with $T$ actions, with
% moduli spaces of positive virtual dimension and
% it will be necessary to consider (invariants of) certain $T $-bordism
% classes of these moduli spaces. % section gromov_witten_theory_of_the_lcs_c_times_s_1_ (end)
% In this section we shall prove the following theorem applying the
% theory of equivariant bordisms of the previous section. 
\subsubsection{Pseudo-holomorphic curves in the $\lcsm$ $C \times S ^{1}$} 
\label{sec:gromov_witten_theory_of_the_lcs_c_times_s_1_}
Banyaga type $\lcsm$'s give immediate examples of almost complex manifolds 
where the $L ^{2} $ $\energy$ functional is unbounded on the moduli spaces of fixed class $J$-holomorphic curves, as well as where null-homologous $J$-holomorphic curves can be non-constant. We are going to see this shortly.


Let $(C, \lambda)$ be a closed contact $(2n+1)$-fold with a contact form
 $\lambda$, that is a 1-form satisfying $\lambda \wedge d\lambda ^{n} \neq 0 $. 
The Reeb vector field $R ^{\lambda} $ on $C$ is a vector field satisfying $d\lambda (R ^{\lambda}, \cdot ) = 0$ and $\lambda (R ^{\lambda}) = 1$.
We also denote by $\lambda$ the pull-back of $\lambda$ by the projection $C \times S ^{1} \to C $, and by $\xi \subset T (C \times S ^{1} )$ the distribution $\xi (p) = \ker d\lambda (p) $. 


Identifying $S ^{1} = \mathbb{R}/\mathbb{Z} $, $S ^{1} $ acts on $C \times S ^{1} $ 
by $s \cdot (x,\theta) = (x, \theta + s)$.
We take $J$ to be an almost complex structure on $\xi$, which is $S ^{1} $ invariant, and compatible with $d\lambda$.  The latter means that $g _{J} (\cdot, \cdot):= d \lambda| _{\xi} (\cdot, J \cdot)$ is a $J$ invariant Riemannian metric on the distribution $\xi$. 

There is an induced almost complex structure $J ^{\lambda}  $ on $C \times S ^{1} $, which is
$S ^{1} $-invariant,  coincides with
$J$ on $\xi$ and which satisfies: 
\begin{equation*}
J ^{\lambda} (R ^{\lambda} \oplus \{0\} (p)) = \frac{d}{d
\theta}(p),
\end{equation*}
where $R ^{\lambda} \oplus \{0\}$ is the section of $T (C \times S ^{1} ) \simeq TC \oplus \mathbb{R}$, corresponding to $R ^{\lambda} $, and where $\frac{d}{d
\theta} \subset \{0\} \oplus TS ^{1}  \subset C \times S ^{1} $ denotes the vector field generating the action of $S ^{1} $ on $C \times S ^{1}  $.

% where 
% maps the Reeb vector
% field $$R ^{\lambda} \in TC \oplus 0 \subset T (C \times S ^{1} ) $$
% to $$\frac{d}{d
% \theta} \in \{0\} \oplus TS ^{1} \subset T (C \times S ^{1} ), $$
% for $\theta \in [0, 2 \pi]$ the global angular coordinate on $S ^{1} $.
% This almost complex structure
% is  compatible with $d ^{\alpha} \lambda $.

We now consider a certain moduli space of holomorphic tori in $C \times S ^{1} $, which have a certain charge. Partly the reason for introduction of ``charge'' is that it is now possible for non-constant holomorphic curves to be null-homologous, so we need additional control. Here is a simple example take $S ^{3} \times S ^{1}  $ with $J=J ^{\lambda} $, for the $\lambda$ the standard contact form, then all the Reeb holomorphic tori (as defined further below) are null-homologous. In many cases we can just work with homology classes $A \neq 0$,  but this is inadequate for our setup for conformal symplectic Weinstein conjecture.


Let $\Sigma$ be a complex torus with a chosen marked point $z  \in \Sigma $. These are also known as elliptic curves.  An isomorphism $\phi: (\Sigma _{1}, z _{1} ) \to (\Sigma _{2}, z _{2} )$ is a biholomorphism s.t. $\phi (z _{1} ) = z _{2} $. The set of isomorphism classes forms a smooth orbifold $M _{1,1} $, with a natural compactification, the Deligne-Mumford compactification $\overline{M} _{1,1}  $, by adding a point at infinity corresponding to a nodal curve. 

Suppose then $(M,\omega)$ is an $\lcs$ manifold, $J$ $\omega$-compatible almost complex structure, and $\alpha$ the Lee class corresponding to $\omega$. Assuming for simplicity, at the moment, (otherwise take stable maps) that $(M,J)$ does not admit non-constant $J$-holomorphic maps $(S ^{2},j) \to (M,J)$,  we define: $$\overline{\mathcal{M}} ^{1,0} _{1,1}  (J, A )$$ as a set of equivalence classes of tuples $(u, S)$, for $S= (\Sigma, z) \in \overline {M} _{1,1} $, and $u: \Sigma \to M$ a $J$-holomorphic map satisfying the \textbf{\emph{charge (1,0) condition}}: 
there exists a pair of generators $\rho, \gamma$ for $H _{1} (\Sigma, \mathbb{Z}) $, 
% $\gamma \cdot \rho =1$, or $\gamma=0$ (if $\Sigma$ is nodal), 
such that 
\begin{align*}
& \langle \rho, u _{*} \alpha \rangle =1 \\
& \langle \gamma, u _{*} \alpha \rangle =0,
\end{align*}
and with $[u]=A$. The equivalence relation is $(u _{1}, S _{1}  ) \sim (u _{2}, S _{2}  )$ if there is an isomorphism $\phi: S _{1} \to S _{2}  $ s.t. $u _{2} \circ \phi = u _{1} $.

Note that the charge condition directly makes  sense for nodal curves. 
% $u$ a $J
% ^{\lambda} $-holomorphic map of a stable genus $1$, elliptic curve 
% $\Sigma$ into $C \times S ^{1} $.  So $\Sigma$ is a nodal curve
% with principal component an elliptic curve, and other components
% spherical. So the principal component 
% determines an element of $\overline{M}_{1,1}   $ the compactified moduli space of elliptic curves, which is understood as an orbifold.  When $\Sigma$ is smooth, we may write $[u,j]$ for an equivalence class where
% Given $u \in \overline{\mathcal{M}}
% _{1,1}  (
% {J} ^{\lambda},
% A )$ we may compose $\Sigma \xrightarrow{u} C \times S ^{1} \xrightarrow{pr} S ^{1}  $, for $\Sigma$ the nodal domain of $u$. 
% \begin{definition} \label{definition:class}
%  Let $\alpha \in H ^{1} (C \times S ^{1}) $ be as in Example \ref{example:banyaga}.  We say that $u: \Sigma  \to M$ is in class $A$, if $(pr \circ u) ^{*} \alpha $ can be completed to an integral basis of $H ^{1} (\Sigma, \mathbb{Z}) $, and if the homology class of $u$ is $A$, possibly zero.
% \end{definition}
And it is easy to see that the charge condition is preserved under Gromov convergence, and obviously a charge (1,0) $J$-holomorphic map cannot be constant for any $A$.

By slight abuse we may just denote such an equivalence class above by $u$, so we may write $u \in \overline{\mathcal{M}} ^{1,0} _{1,1}  (J, A ) $, with $S$ implicit.
\subsubsection {Reeb holomorphic tori in $(C \times S ^{1}, J ^{\lambda})  $}
For the almost complex structure $J ^{\lambda} $ as above we have
one natural class of charge (1,0) holomorphic tori in $C \times S ^{1} $.
Let $o$ be a period $c$ Reeb orbit $o$ of $R ^{\lambda} $, that is a map:
\begin{align*}
   & o: S ^{1}  \to C,  \\
   & D _{s}  o (s _{0}) = c \cdot R ^{\lambda} (o (s _{0})),
\end{align*}
for $c>0$, and $\forall s _{0} \in S ^{1}:=\mathbb{R}/\mathbb{Z} $.
A Reeb torus $u _{o} $ for $o$, is the 
map $$u_o (s, t)= (o  (s), t),$$ $s,
t \in S ^{1}$.
A Reeb torus is $J ^{\lambda} $-holomorphic for a uniquely determined holomorphic structure $j$ on $T ^{2} $ defined by:
$$j
(\frac{\partial}{\partial s}) = c \frac{\partial} {\partial t}. $$

Let $\widetilde{S} (\lambda)$ denote the space of general period $\lambda$-Reeb orbits. There is an $S ^{1} $ action on this space by $\theta \cdot o (s)= o (s+\theta) $. Let $ {S} (\lambda):= \widetilde{S} (\lambda)/S ^{1}$ denote the quotient by this action.
We have a map:
\begin{equation*}
   R: S (\lambda) \to \overline{\mathcal{M}} ^{1,0} _{1,1}  (J ^{\lambda}, A), \quad R (o) = u _{o}.
\end{equation*}
\begin{proposition} \label{prop:abstractmomentmap}
The map $R$ is a bijection. \footnote{It is in fact an equivalence of the corresponding topological action groupoids, but we do not need this explicitly.}
%    For a pair
% ${J}^ {\lambda _{0} },
% {J}^ {\lambda _{1} }  $  of almost
% complex structures as above and $\{{J}^ {\lambda _{t} } \} $ a smooth interpolating
% family,  $\overline{\mathcal{M}}_{1}  (
% \{{J}^ {\lambda _{t} } \},
% A),$ has compact connected components.
\end{proposition}
So in the particular case of $J ^{\lambda} $, the domains of elliptic curves in $C \times S ^{1} $ are ``rectangular'', that is are quotients of the complex plane by a rectangular lattice, however for a more general almost complex structure on $C \times S ^{1} $, as we soon consider, the domain almost complex structure on our curves can in principle be arbitrary, in particular we might have nodal degenerations.
% The basis for the second part of the proposition is the following
% crucial lemma.
% \begin{lemma} \label{lemma:compactcomponents}
%  Let  $\{R ^{\lambda _{t} }   \}$, $0 \leq t \leq 1$, be a family of
%  Reeb vector fields corresponding to a smooth family $\{\lambda _{t}
%  \}$ of contact forms on a closed contact manifold $C$.  Then all the connected components
%    of the solution space $$S = \{(o,t)\, | \, o \text{ is a periodic orbit of
%    $R ^{\lambda _{t} }  $}\}$$ are compact.
% \end{lemma}
% We shall say more on this property of Reeb vector fields  in Section \ref{sec:Fuller}.
Also note that the expected dimension of $\overline{\mathcal{M}} _{1,1} ^{1,0}   (
{J} ^{\lambda},
A )$ is 0. It is given by the
Fredholm index of the operator \eqref{eq:fullD} which is 2, minus the dimension of the reparametrization group (for non-nodal curves) which is 2. That is given an elliptic curve $S = (\Sigma,  z) $,  let $\mathcal{G} (\Sigma)$ be the 2-dimensional group of biholomorphisms $\phi$ of $\Sigma$. And given a $J$-holomorphic map $u: \Sigma \to M$, $(\Sigma,z,u)$ is equivalent to 
$(\Sigma, \phi(z), u \circ \phi)$ in $\overline{\mathcal{M}} _{1,1} ^{1,0}  (
{J} ^{\lambda}, A)$, for $\phi \in \mathcal{G} (\Sigma)$. 
% , we get an element $(\Sigma, \gamma, \rho, \phi(z), u \circ \phi)  \in \overline{\mathcal{M}} _{1,1} ^{1,0}   (
% {J} ^{\lambda}, A) $, equivalent to $u$.

In Theorem \ref{thm:GWFullerMain} we relate the (extended) count (Gromov-Witten invariant) of these curves to the (extended) Fuller index, which is reviewed in the Appendix \ref{appendix:Fuller}. This will be one ingredient for the following.
% This will in part motivate the conformal symplectic Weinstein conjecture that
% still requires some preliminaries. 
\begin{definition} \label{def:admissible} Let $(M,\lambda, \alpha)$ be an exact lcs structure, $\omega =  d ^{\alpha} \lambda $. We say that an $\omega$-compatible $J$ is \textbf{\emph{admissible}} if it preserves the vanishing distribution $\mathcal{V} _{\lambda} $,     
and the co-vanishing distribution $\xi _{\lambda} $. We call $(M, \lambda, \alpha, J)$ as above a \textbf{\emph{tamed exact $\lcs$ structure}}. 
  % For $\omega,J$ as above the pair $(\omega,J)$ will be called \textbf{\emph{admissible}}. 
\end{definition}
% \begin{lemma} For each $p$ $\mathcal{V} _{\omega} (p) $ has dimension at most 2.    
% \end{lemma}
% \begin{proof} 
%    Let $M$ have dimension $2n$ and $p \in M$.  $\mathcal{V} _{\omega} (p)$ has dimension at least 2, for if $d\lambda (p)$ is non-degenerate then for some positive bump function $f$, $\int _{M} f (d\lambda) ^{n}  (p) >0 $, while $$0 =\int _{M} \min f (d\lambda) ^{n}  \leq \int _{M} f (d\lambda) ^{n}  \leq \int _{M} \max f (d\lambda) ^{n}  =0 $$ by Stokes theorem as $M$ is closed, a contradiction.
% And 
% \end{proof}
The significance of an admissible almost complex structure is the following.
% \begin{definition} Let $(M,\omega,J)$ be an exact lcs triple, for a smooth $u: \Sigma \to M$ and $\Sigma$ a Riemann surface, define 
%    $$E ^{\pi} (u) = \int _{\Sigma} |\pi _{\xi} \circ du| _{g}  dvol _{\Sigma},    $$ for $\pi _{\xi}: TM \to \xi  $ the projection with respect to the splitting $TM=\xi \oplus \mathcal{V} (d \lambda)$, where $g$ is the metric $\omega (J\cdot,  \cdot)$ and where $|\pi _{\xi} \circ du| _{g}$ denotes the associated operator norm.
% \end{definition}
% The following is immediate.
% \begin{lemma}
%  When $u$ as above is $J$-holomorphic, for $J$ $\omega$-admissible as above, $E ^{\pi} (u)= \int _{\Sigma} u ^{*}d\lambda.  $ 
% \end{lemma}
\begin{lemma} \label{lemma:calibrated} Let $(M,\lambda,\alpha,J)$ be a tamed exact $\lcs$ structure.  Then given a smooth $u: \Sigma \to M$, where $\Sigma$ is a closed (nodal) Riemann surface, $u$ is $J$-holomorphic only if 
$$\image du (z) \subset \mathcal{V} _{\lambda} (u (z)) $$ for all $z \in \Sigma$,
   in particular $u ^{*} d\lambda =0$. 
\end{lemma}
\begin{proof} We have $$I=\int _{\Sigma} u ^{*} d \lambda \geq 0 $$ since $J$ preserves $\mathcal{V} _{\lambda} $. On the other hand $I > 0 $ is impossible by Stokes theorem. So $I=0$. Since $J$ also preserves $\xi _{\lambda} $, this can happen only if $$\image du (z) \subset \mathcal{V} _{\lambda} (u (z)) $$ for all $z \in \Sigma$. 
\end{proof}
\begin{theorem} \label{thm:holomorphicSeifert} Let $M=S ^{2k+1}  \times S ^{1}$, $d ^{\alpha} \lambda _{H} $ the Hopf lcs structure. Then there exists a $\delta>0$ s.t. for any exact $\lcs$ structure $(\lambda', \alpha' )$ on $M$ $C ^{0} $ $\delta$-close to $(\lambda _{H}, \alpha)$, and $J$ compatible with $\omega'=d ^{\alpha'} \lambda' $ and $C ^{0} $ $\delta$-close to $J ^{\lambda _{H} } $, there exists an elliptic, charge (1,0), $J$-holomorphic curve in $S ^{2k+1}  \times S ^{1} $. Moreover, if $k=1$ and $J$ is admissible then this curve may be assumed to be non-nodal and embedded.
\end{theorem}
The following is to be proved in Section \ref{sectionFuller}.
\begin{theorem} \label{lemma:Reeb}
Let $(M,\lambda, \alpha, J)$ be a tamed exact $\lcs$ structure, if $\alpha$ is rational then every non-constant $J$-holomorphic curve $u: \Sigma \to M$ contains Reeb curve in the image $u (\Sigma)$,  
if moreover $\Sigma$ is smooth, connected and immersed then $\Sigma \simeq T ^{2} $.
\end{theorem}
 % This gives further evidence to the CSW conjecture. For if $\omega= d^{\alpha} \lambda _{H}  $ is the Hopf lcs structure on $S ^{2k+1} \times S ^{1} $, or indeed any Banyaga lcs form on $C \times S ^{1} $ on $C$ a three-fold, and if $J=J ^{\lambda} $ then we know there are immersed $J$-holomorphic tori, the Reeb holomorphic tori, since we know there are $\lambda$-Reeb orbits, as the Weinstein conjecture is known to hold in these cases, \cite{citeViterboWeinstein}, \cite{citeTaubesWeinsteinconjecture} and hence there are elliptic Reeb curves, by Lemma \ref{lemma:Reeb}. 
% \begin{remark} Regularity should not be necessary, but then the prove must involve a very difficult Titze extension type theorem, which may not be immediately available.
% \end{remark}
% \begin{conjecture} \label{conjecture:HCSW} Suppose we are given a tamed exact $\lcs$ structure $(M,\lambda, \alpha,J)$,  with $M$ a closed 4-fold, and so that $\alpha$ is rational. Then there is a non-constant $J$-holomorphic curve in $M$.
% \end{conjecture}

% \begin{remark}
% The significance of the rationality condition is that it forces (under our conditions on $J$) any non-constant $J$-holomorphic curves $u$ in $M$ to be smooth, but this will only be shown in a sequel.
% \end{remark}
% Conjecture \ref{conjecture:HCSW} also immediately implies the Weinstein conjecture for a closed contact 3-fold $(C, \lambda)$. For by the proof of Proposition \ref{prop:abstractmomentmap}, any non-constant elliptic curve $u: \Sigma \to M=C \times S ^{1}$, with respect to the Banyaga $\lcs$ structure $d ^{\alpha} \lambda$, $\alpha=d\theta$, must cover a Reeb torus. 

In a sense the above discussion tells us that $J$-holomorphic curves strictify Reeb curves, in the sense that Reeb curves satisfy a partial differential relation while $J$-holomorphic curves satisfy a partial differential equation, but given a solution of the former we also the latter. Strictifying could be helpful because the ``strict'' objects may possibly be counted in some way.  

% pThis immediately implies the CSW conjecture by Theorem \ref{lemma:Reeb} and hence also implies the 3-dimensional Weinstein conjecture.

 It makes sense to try to partially strictify Reeb curves more directly.
 \begin{definition} Let $(M,\lambda, \alpha)$ be an exact $\lcs$ structure,  $\Sigma$ a closed possibly nodal Riemann surface.  A smooth map $u: \Sigma  \to M$ is called a \textbf{\emph{Reeb 2-curve}} if $u_*(T \Sigma) \subset  \mathcal{V} (\lambda)$ and if there is a smooth map $o: S ^{1} \to \Sigma $ s.t. $\forall s \in S ^{1}: o ^{*} u ^{*} \lambda (s) \neq 0  $. 
\end{definition}
 We note that for $u$ satisfying the first condition, the second condition is satisfied for example if $\alpha$ is rational and $u ^{*} \alpha \wedge u ^{*} \lambda  $ is symplectic except at finitely many points.    % %    , and if
% % $$0 \neq [u ^{*} \alpha] \in H ^{1} _{DR}  (\Sigma). $$ 
% % When $\Sigma$ is elliptic the curve $u$ will be called an \textbf{\emph{elliptic Reeb curve}},
% % if in addition $\Sigma \simeq T ^{2} $ we will just call such a curve a \textbf{\emph{Reeb torus}}.
% \end{definition}
% A simple extension of Theorem \ref{lemma:Reeb} tells us that given a Reeb 2-curve in we also have a Reeb curve. Moreover in the proofs of Theorems \ref{thm:C0Weinstein}, \ref{thm:catastrophyCSW} we actually produce Reeb 2-curves.
The proofs of theorems \ref{thm:C0Weinstein}, \ref{thm:catastrophyCSW} actually produce Reeb 2-curves, through which we then deduce existence of Reeb curves. So it makes to further conjecture the following.

\begin{conjecture} \label{conjecture:WeinsteinStrong} Let $M$ be closed, of dimension at least 4,  and $\omega$ an exact $\lcs$ form on $M$ whose Lee form $\alpha$ is rational, then there is a Reeb 2-curve in $M$. 
\end{conjecture}
% In dimension 4 the following conjecture looks similar to the fundamental results of Taubes \cite{citeTaubesCountingPseudoHolomorphic} on Gromov-Witten theory of symplectic 4-folds. 

The above conjectures are not just a curiosity. In contact geometry, rigidity is based on  existence phenomena of Reeb orbits, and $\lcs$ manifolds may be understood as generalized contact manifolds. To attack rigidity questions in $\lcs$ geometry, like Question \ref{q:contactnonsqueezing} further below, we need an analogue of Reeb orbits, we propose that this analogue is Reeb curves as above, from which point of view the above conjectures become more natural.

% On the other hand pseudo-holomorphic curves in $\lcs$ manifolds look to be much more rigid objects than periodic orbits of general smooth dynamical systems. So it is possible that holomorphic sky catastrophes do not exist, at least for families $\{(\omega _{t}, J _{t}  )\}$ with $J _{t} $ admissible as above.
% If sky catastrophes do not exist then the theory of Gromov-Witten invariants of a $\lcsm$ would be very different. In particular we automatically obtain a special case of the the holomorphic Weinstein conjecture below, using Theorem \ref{thm:holomorphicSeifertMain}.
% A nice test case for the above is an exact $\lcs$ $\omega=d ^{\alpha} \lambda $ where $\lambda$ is a non $S ^{1} $ invariant 1-form on $S ^{2k+1} \times S ^{1}  $ which restrict to a contact structure $\lambda _{\theta} $ on each slice $C \times \{\theta\}  $. There is an analogue of Reeb tori in this case coming from $S ^{1} $ families of Reeb orbits $\{o _{\theta} \}$ of $\{\lambda _{\theta} \}$, and which are holomorphic for an almost complex structure $J ^{\lambda} $ on $C \times S ^{1} $ defined analogously to before.
% Note that now basically any non-nodal holomorphic structure on such a ``Reeb torus'' may appear, as opposed to just ``rectangular'' holomorphic structures on the Reeb tori when $\lambda$ was $S ^{1} $ invariant.
\subsubsection {Connection with the extended Fuller index} 
One of the main ingredients for the above is a connection of extended Fuller index with certain extended Gromov-Witten invariants.
If $\beta$ is a free homotopy class of a loop in $C$ set $${A} _{\beta}= [\beta] \times [S ^{1}] \in H _{2} (C \times S ^{1} ).$$
Then we have: 
\begin{theorem} \label{thm:GWFullerMain1} 
Suppose that $\lambda$ is a contact form on a closed manifold $C$, so that its Reeb flow is definite type, see Appendix \ref{appendix:Fuller}, then 
\begin{equation*}
   GW _{1,1} (A _{\beta},J ^{\lambda} ) ([\overline {M} _{1,1} ] \otimes [C \times S ^{1} ]) = i (R ^{\lambda}, \beta),
\end{equation*}
where both sides are certain extended rational numbers $\mathbb{Q} \sqcup \{\pm \infty\}$ valued invariants, so that in particular if either side does not vanish then there are $\lambda$ Reeb orbits in class $\beta$.
\end{theorem} % \begin{remark}$\overline{\mathcal{M}}_{1}  (
% {J} ^{\lambda},
% A )$ and $\overline{\mathcal{M}}_{1}  (
% \{\widetilde{J} (\lambda _{t} ) \},
% A)$ are not smooth manifolds, but they do have virtual dimension 0 and
% 1 respectively. For the above statement to make good sense we should 
% However the language of abstract moment maps is only used here to make
% a connection with future work, technically we need only Lemma
% \ref{lemma:}
% \end{remark}
% For the following proposition we need to assume that there exists a virtual
% $T$-equivariant perturbation theory. In fact since we will be in the
% expected dimension 0 case, and because we shall arrange fixed point
% sets in the moduli space to be composed of regular curves, for the
% proposition below we need only virtual $T$-equivariant perturbation
% theory when $T$ acts freely on the original moduli space. 
% The following proposition may hold in higher dimensions, however our
% approach uses essentially  4 dimensional positivity of intersections
% techniques.
What about higher genus invariants of $C \times S ^{1} $? Following the proof of Proposition \ref{prop:abstractmomentmap}, it is not hard to see that all $J ^{\lambda} $-holomorphic curves must be branched covers of Reeb tori. If one can show that these branched covers are 
regular when the underlying tori are regular, the calculation of invariants would be fairly  automatic from this data, see \cite{citeWendlSuperRigid}, \cite{citeWendlChris} where these kinds of regularity calculation are made.


\subsection {Non-squeezing}
One of the most fascinating early results in symplectic geometry is the so called Gromov non-squeezing theorem appearing in the seminal paper of Gromov~\cite{citeGromovPseudoholomorphiccurvesinsymplecticmanifolds.}.
The most well known formulation of this is that there does not exist a  symplectic embedding $B _{R} \to D ^{2} (r)  \times \mathbb{R} ^{2n-2}   $ for $R>r$, with $ B _{R}  $ the standard closed radius $R$ ball 
in $\mathbb{R} ^{2n} $ centered at $0$.
Gromov's non-squeezing is $C ^{0} $ persistent in the following sense. 

We say that a symplectic form $\omega$ on $M \times N$ is \emph{split} if $\omega= \omega _{1} \oplus \omega _{2} $ for symplectic forms $\omega _{1}, \omega _{2}  $ on $M$ respectively $N$.
\begin{theorem} \label{thm:Gromov} Given $R>r$, there is an $\epsilon>0$ s.t. for any symplectic form $\omega' $ on $S ^{2} \times T ^{2n-2}  $ $C ^{0} $-close to a split symplectic form $\omega $ and satisfying $$ \langle \omega, A  \rangle = \pi r ^{2}, A=[S ^{2} ] \otimes [pt],  $$  there is no symplectic embedding $\phi: B _{R} \hookrightarrow (S ^{2} \times T ^{2n-2}, \omega')   $.
\end{theorem}
 On the other hand it is natural to ask: 
\begin{question} \label{thm:nonrigidity} Given $R>r$ and every $\epsilon > 0 $ is there
a (necessarily non-closed by above) 2-form $\omega'$ on $S ^{2} \times T ^{2n-2}  $  $C ^{0} $ or even $C ^{\infty} $ $\epsilon$-close to a split symplectic form $\omega $, satisfying $ \langle \omega, A  \rangle = \pi r ^{2}  $, and s.t. there is an embedding $\phi: B _{R} \hookrightarrow S ^{2} \times T ^{2n-2}   $, with $\phi ^{*}\omega'=\omega _{st}  $?
\end{question}
The above theorem follows immediately by Gromov's argument in \cite{citeGromovPseudoholomorphiccurvesinsymplecticmanifolds.}, we shall give a certain extension of this theorem for $\lcs$ forms.
One may think that recent work of M\"uller \cite{citeMuller} may be related to the question above and our theorem below. But there seems to be no obvious such relation as pull-backs by diffeomorphisms of nearby forms may not be nearby. Hence there is no way to go from nearby embeddings that we work with to $\epsilon$-symplectic embeddings of M\"uller.



We first give a ridid notion of a morphism of $\lcsm$'s. 
% Gromov's argument trivially generalizes to show the following: 
% \begin{theorem} [Gromov] 
%    Let $(M, \omega)$ be a compact symplectic manifold, with $GW _{0,1}  (\omega,A) ([pt]) \neq 0$ for some class $A$. If $ \langle [\omega], A  \rangle = \pi r ^{2}  $ 
%    then there is no symplectic embedding $$\phi: B _{R}   \hookrightarrow (M, \omega), $$ if $R>r$.
% \end{theorem}
\begin{definition}
   Given a pair of $\lcsm$'s $(M _{i}, \omega _{i} )$, $i=0,1$, we say that $f: M _{1} \to M _{2}  $ is a \textbf{\emph{symplectomorphism}} if 
   $f^{*} \omega _{2} = \omega _{1}$. A \textbf{\emph{symplectic embedding}} then as usual is an embedding by a symplectomorphism.
\end{definition}
% A pair $(\omega,J)$, for $\omega$ $\lcs$ and $J$ compatible, will be called a \textbf{\emph{compatible $\lcs$ pair}}, or just lcs pair.
% \begin{definition} Given $(\omega,J)$ with $\omega$ an $\lcs$ form on $M$ and $J$ an almost complex structure compatible with $\omega$, (as previously defined) we call this a \textbf{\emph{lcs pair}}.
% \end{definition}

% \begin{remark}
%    We say strict here because there are other natural notions of $\lcs$ morphisms. But they will not be considered here.
% \end{remark}
% We say strict here because in some cases it is more natural to consider more
% relaxed notions of $\lcs$ morphisms, for example we may define an $\lcs$
% morphism to be a diffeomorphism $\phi: M _{0} \to M _{1}  $ s.t. $\phi ^{*}
% \omega _{1}  $ is deformation equivalent through $\lcs$ forms to $\omega _{0} $.
% When $\omega _{i} $ are symplectic the latter is just a conformal
% symplectomorphism, but in general it is very different.
% \begin{theorem} \label{thm:nonsqueezing1}
% Let $\omega=\omega _{0} \times \omega _{1}  $ be the product standard
% symplectic form on $M = S ^{2}   \times T ^{2n-2n}  $.
% Set $$R =\min \{\langle \omega, A  \rangle \},
% $$   for $A$ the class $[S ^{2} ] \otimes [pt] \in H ^{2} (M) $.
% Given an $r>0$ there is an $\delta>0$ s.t. if $\omega _{0}, \omega _{1}  $ be
%    $\lcs$ forms on $M$ $C ^{0} $ $\delta$-close to $\omega$ then there is no $\lcs$ diffeomorphism $$\phi: (M, \omega _{0} )   
% \to (M, \omega _{1} )  $$ if $r<R$.
% \end{theorem}
% \begin{theorem} \label{thm:nonsqueezing1}
% Let $\omega=\omega _{0} \times \omega _{1}  $ be the product standard
% symplectic form on $M = \mathbb{R} ^{2}   \times T ^{2n-2n}  $  and let
% $j: T ^{n} \to \mathbb{R} ^{2} \times \mathbb{T} ^{2n-2}   $ a local Lagrangian
%    embedding, meaning that it factors through the quotient map $\mathbb{R} ^{2n}
%    \to \mathbb{R} ^{2} \times T ^{2n-2}  $.
% Set $$R = R (j)=\min \{\langle \omega, A  \rangle \, \vert \,  \langle \omega,  A
%  \rangle \neq 0, $A$ \text{ is a relative class of a disk with boundary on $j
%  (T ^{n} )$}  \}.
%  $$ 
% Given an $r>0$ there is an $\delta>0$ s.t. if $\omega'$ be an $\lcs$ form
% on $M$  homotopic to $\omega$,
% through $\lcs$ forms $C ^{0} $ $\delta$-close to $\omega$ and having the same
% Lagrangian subspaces as $\omega$, 
% then there is no compactly supported $\lcs$ $\omega'$-diffeomorphism $\phi: \mathbb{R} ^{2} \times T ^{2n-2}  
% \to \mathbb{R} ^{2} \times T ^{2n-2}  $  which takes $j (T ^{n} )$ into $D ^{2} (r) \times
% \mathbb{T} ^{2n-2}  $ if $r<R$.
% \end{theorem}
% Note that if in the above theorem $\lcs$ everywhere is replaced by symplectic, then the
% theorem follows by original non-squeezing together with the classical Moser argument.
% At least if want to follow Gromov's original argument we
% need a stronger form of the isoperimetric inequality used by Gromov, which may
% at present be unknown.
% \begin{hypothesis} Let $\mathcal{J}$ be the space of almost complex structures
%    compatible with the standard symplectic form $\omega$ on $\mathbb{R} ^{2n}
%    $. Let $B _{R} $ be the standard closed ball with radius $R$ in $\mathbb{R}
%    ^{2n} $ with center at origin $0$.
% Define $$\mathcal{A}: \mathcal{J} \to \mathbb{R}$$ by $\mathcal{A} (J)$ is the
%    least $\omega$-area of a somewhere injective $J$-holomorphic curve passing through $0$ with
%    boundary on $\partial B$. Then the minimum of $\mathcal{A}$ is  $\pi \cdot R ^{2} $.
% \end{hypothesis} 
% \begin{theorem} \label{thm:nonsqueezing2} 
% Let $(M, \omega)$ be a closed $\lcs$ manifold,   $\omega$ is $C$-comparable with a symplectic form $\omega _{0} $.
% Suppose that $GW _{0,1}  (\omega, A) ([pt]) \neq 0$ for some class $A$. If $$ \langle [\omega _{0} ], A  \rangle = \pi r ^{2}  $$     then there is no $\lcs$ embedding $$\phi: B _{R} \hookrightarrow (M, \omega), $$ if $C< \frac{R}{r}$.
% \end{theorem}
% The following may be more concrete since to obtain one simple application all we have to do is start with the product symplectic form $\omega$ on $M=S ^{2} \times T ^{2n}  $, take $A= [S ^{2} ] \otimes [pt]$ and deform $\omega$ slightly through $\lcs$ forms. The following theorem then tells us that Gromov non-squeezing is ``rigid'' for such a deformation.
% \begin{definition} \label{def:boundeddeformation}
%    We say that a pair of bounded, as in Definition \ref{def:comparable}, $\lcs$ forms $\omega _{0},  \omega _{1} $ on a manifold are
%  \textbf{\emph{$c$-deformation equivalent}} if there is an interpolating continous family $\{\omega _{t} \}$,  $t \in [0,1]$, of $\lcs$ forms,  and a continous family $\widetilde{\omega} _{t}  $, $t \in [0,1]$ of symplectic forms, such that for each $t$ $\omega _{t}, \widetilde{\omega}_{t}
%    $ are $C$-comparable for some $C$, (independent of $t$).
% \end{definition}
% <<<<<<< HEAD
% The following theorem says that it is impossible to even have a ``nearby'' $\lcs$ embedding.  (Note that the $C ^{0} $ norm we use on the space of $\lcs$ structures is (likely strictly) stronger then the obvious norm on the space of $2$-forms).
% \begin{theorem} \label{cor:nonsqueezing} Let $\omega$ be the standard product symplectic form on $M =S ^{2} \times T ^{2}  $,  with $ \langle \omega, A= [\Sigma] \rangle = \pi r ^{2} $. Let $R>r$, and $\Sigma \subset M$ a fixed embedded $\omega$-symplectic, spherical surface in homology class of $A=[S ^{2} \times \{x\}] $, then there is an $\epsilon>0$ s.t. if $\omega _{1} $ is an $\lcs$ on $M$ $C ^{0} $ $\epsilon$-close to $\omega$, then
%    there is no $\lcs$ embedding $$\phi: (B _{R}, \omega _{st})  \hookrightarrow (M, \omega _{1}), $$  s.t $\phi _{*} j  $ preserves $T\Sigma$. 
% \end{theorem}
% For a nearby symplectic manifold the above follows by Gromov's argument, without the condition of $\Sigma$. 
% We note that the image of the embedding $\phi$ would be of course a symplectic submanifold of  $(M, \omega _{1} )$. However it could be highly distorted, so that it might be impossible to complete $\phi _{*} \omega _{st}  $ to a symplectic form on $M$ nearby to  $\omega$, (and perhaps impossible to complete to any symplectic form).
% We also note that it is certainly possibly to have a ``nearby'', in the sense above, volume preserving embedding.
% Take $\omega = \omega _{1}  $, $\Sigma=S ^{2} \times \{x\} $,  then hypothesis of the theorem above are satisfied, while (if the symplectic form on $T ^{2} $ has enough volume) we can find a volume preserving map $\phi: B _{R} \to M $ s.t. $\phi _{*} j$ preserves $\Sigma$. (This is just the squeeze map.)
% =======
% The following tells us that the non-squeezing problem for $\lcsm$ can be is closely tied to pseudo-holomorphic curve theory as in the symplectic case. 
% \begin{theorem} \label{thm:alternative} The following alternative holds.
% Let $\omega$ be the standard product symplectic form on $M =S ^{2} \times T ^{2n}  $,  with $ \langle \omega, A= [S ^{2} \times \{pt\} ] \rangle = \pi r ^{2} $. Let $R>r$, then either there is an $\epsilon>0$ s.t. if $\omega _{1} $ is an $\lcs$ on $M$ $C ^{0} $ $\epsilon$-close to $\omega$, then there is no $\lcs$ embedding $$\phi: (B _{R}, \omega _{st})  \hookrightarrow (M, \omega _{1}), $$  or there is a compatible $\lcs$ family $(\{\omega _{t} \}, \{J _{t} \})$ on $S ^{2} \times T ^{2n}  $ with a sky catastrophe (the definition is given in the section just below).
% \end{theorem}

Let now $M=S ^{2} \times T ^{2n}$, with $\omega$ a split symplectic form on $M$. 
% Note that $\Sigma _{i} $ are naturally foliated by symplectic spheres, we denote by $T ^{fol}  \Sigma _{i} $ the sub-bundle of the tangent bundle consisting of vectors tangent to the foliation. 
The following theorem says that it is impossible to have certain symplectic embeddings into $(M, \omega')$ with $\omega'$ $C ^{0} $ nearby to $\omega$, even in the absence of any volume obstruction. So that we have a first basic rigidity phenomenon for $\lcs$ structures.
Note that in what follows we take a certain natural metric $C ^{0} $ topology $\mathcal{T} ^{0} $ on the space of general $\lcs$ forms, defined in Section \ref{section:basics}, which is finer than the standard $C ^{0} $ metric topology on the space of forms, cf. \cite [Section 6]{citeBanyagaConformal}.
The corresponding metric is denoted $d _{0} $.

We have a real codimension 1 hypersurfaces $$\Sigma _{i} =S ^{2}  \times  (T ^{1} \times \ldots \times T ^{1}  \times   \{pt \} \times T ^{1}   \times \ldots \times T ^{1})   \subset M,   $$ where the singleton $\{pt\} \subset T ^{1}  $ replaces the $i$'th factor of $T ^{2n}= T ^{1} \times \ldots \times T ^{1}  $. These  hypersurfaces are naturally folliated by symplectic submanifolds diffeomorphic to $S ^{2} \times T ^{2n-2}  $. We denote by $T ^{fol} \Sigma _{i} \subset TM  $, the distribution of all tangent vectors tangent to the leaves of the above mentioned folliation.
\begin{theorem} \label{cor:nonsqueezing} Let $\omega$ be a split symplectic form on $M =S ^{2} \times T ^{2n}  $,  and $A$ as above with $ \langle \omega, A\rangle = \pi r ^{2} $. Let $R>r$, then there is an $\epsilon>0$ s.t. if $\{\omega _{t} \} $ is a $\mathcal{T} ^{0} $-continuous family of $\lcs$ forms on $M$, with $d _{0}  (\omega _{t,}, \omega ) < \epsilon$ for all $t$, then
there is no symplectic embedding $$\phi: (B _{R}, \omega _{st})  \hookrightarrow (M, \omega _{1}) - \cup _{i} \Sigma_i. $$ The latter is a full-volume subspace diffeomorphic to $S ^{2} \times \mathbb{R} ^{2n}  $.
More generally there is no symplectic embedding $$\phi: (B _{R}, \omega _{st})  \hookrightarrow (M, \omega _{1}), $$  s.t 
   $\phi _{*} j$   \text{ preserves the bundles } $T ^{fol} \Sigma _{i},$ for $j$ the standard almost complex structure on $B _{R}$. 
\end{theorem}
We note that the image of the embedding $\phi$ would be of course a symplectic submanifold of  $(M, \omega _{1} )$. However it could be highly distorted, so that it might be impossible to complete $\phi _{*} \omega _{st}  $ to a symplectic form on $M$ \emph{nearby} to  $\omega$, so that it is impossible to deduce the above result directly from symplectic Gromov non-squeezing. We also note that it is certainly possible to have a nearby volume preserving as opposed to $\lcs$ embedding which satisfies all other conditions, since as mentioned $(M, \omega _{1}) - \cup _{i} \Sigma_i$ is a full $\omega _{1} $-volume subspace diffeomorphic to $S ^{2} \times \mathbb{R} ^{2n}  $. This extension of Theorem \ref{thm:Gromov} may be optimal, since the $\epsilon$ condition cannot be removed from Theorem \ref{thm:Gromov}.
% Take $\omega = \omega _{1}  $, then if the symplectic form on $T ^{2n} $ has enough volume, we can find a volume preserving map $\phi: B _{R} \to M $ s.t. $\phi _{*} j$ preserves $T ^{fol} \Sigma _{i} $. 
% For example, if $M=S ^{2} \times T ^{2}  $ then we may take  the squeeze map, which as a map $\mathbb{C} ^{2} \to \mathbb{C}^{2}  $ is $(z_1, z _{2} ) \mapsto (\frac{z _{1} }{a}, a \cdot {z _{2}}) $. Or just take any volume preserving map $\phi$ that doesn't hit $\Sigma _{i} $. 
% Note that $\omega _{1} $ must necessarily be globally conformally symplectic on $M - \Sigma_1 \sqcup \Sigma _{2}$, as this is simply connected, however even in this special case the theorem above is not an obvious extension of Gromov non-squeezing since the ambient form is no longer symplectic.
% \subsubsection {Explicit $\lcs$ deformations} 
% \begin{remark}
%  Let us also point out that we may construct explicit non-trivial $\lcs$ deformations of $(S ^{2} \times T ^{2},  \omega _{st} )$. For the pair $(\omega _{st}, j _{st}  )$ is Kahler, then we may use the construction in \cite [Section 6]{citeBanyagaConformal} to obtain such deformations. Specifically $\omega _{st} (X,Y) = g (X, j _{st} Y) $, for $g$ the corresponding Kahler metric. Deforming $g$ through hermitian metrics $\{g _{t} \}$, we get a deformation $\{\omega _{t} \} $ through $\lcs$ structures and the corresponding Lee class (see below), which is an invariant of $\lcs$ structure, will for a general $\{g _{t} \}$ undergo a non-trivial deformation. Hence this will give a non-trivial $\lcs$ deformation.
% \end{remark}
% For instance when $M=S ^{2} \times ^{T ^{2} }  $, $\eta$ integral 1-form dual to a standard generator of $H _{2} (T ^{2} ) $, we may deform with respect to $\eta$ as follows.
% Here is a sketch, take a non-vanishing class $[\eta] \in H ^{1} (M, \mathbb{R}) $, let $\widetilde{M} $  be the associated covering space. We may understand $M$ as being obtained from the fundamental domain in $\widetilde{M}$ via gluing by symplectic maps. We may then scale $\widetilde{\omega} $ on $\widetilde{M} $ by a positive function, 
%
% Let $\eta _{t} $, $t \in [0,1]$ be a continuous family of closed $1$-forms on $M$ s.t. $\{\eta _{t} \} $ does not vanish in cohomology for $t>0$. Given a cover $\{U _{i} \}$ of $M$ by contractible open sets, $x _{i} \in U _{i}  $ fixed points, define smooth functions $f _{i,t}: U _{i} \to \mathbb{R}  $ by $f _{i,t} (x) = e ^ {\int _{0} ^{1} p ^{*} \eta _{t}}       $, for $p$ any smooth path from $x _{i} $ to $x$. Then $d (\ln f _{i,t} )$ glue to $\eta _{t} $. Meanwhile $\{f _{i,t} \omega| _{U _{i} }  \}$ glue to an $\lcs$ form $\omega _{t} $, $\omega _{0}=\omega $, non-symplectic for $t>0$, with $\{\omega _{t} \}$ continuous in our $C ^{0} $-topology on $\lcs$ forms.
% For the proof we need to use geometry to deduce compactness of the moduli space of certain pseudo-holomorphic curves, for what could be fairly general deformations of almost complex structures compatible with $\lcs$ forms, without a priori bounds on $\energy$ and without any $C ^{0} $ bounds on the deformation. 
\subsubsection {Toward direct generalization of contact non-squeezing} What about non-squeezing for $\lcs$ maps as in Definition \ref{definition:lcsmap}? We can try a direct generalization of contact non-squeezing of Eliashberg-Polterovich \cite{citeEKPcontactnonsqueezing}, and Fraser in \cite{citeFraserNonsqueezing}.
Specifically let $R ^{2n}  \times S ^{1}  $ be the prequantization space of $R ^{2n} $, or in other words the contact manifold with the contact form $d\theta - \lambda$, for $\lambda = \frac{1}{2}(ydx - xdy)$. Let $B _{R} $ now denote the open radius $R$ ball in $\mathbb{R} ^{2n} $. 
\begin{question} \label{q:contactnonsqueezing} If $R \geq 1$ is there a compactly supported, $lcs$ embedding map $\phi: \mathbb{R} ^{2n} \times S ^{1} \times S ^{1}  $, so that $\phi (\overline{U} ) \subset U$, for $U := B _{R} \times S ^{1} \times S ^{1}  $ and $\overline{U} $ the topological closure.
\end{question}
We expect the answer is no, but our methods here cannot say anything, as we likely have to extend contact homology rather the Gromov-Witten theory as we do here. 
% Thus we have found that there is ``hidden'' geometric rigidity in the $\lcs$ structure.

% It is also interesting to understand if the condition on $\phi _{*}j $ can be removed. We can say the following. Let $\mathcal{S} _{j} $ be the space of all $j$-holomorphic (stable) curves in $B _{R} $ passing through the origin, with boundary on $\partial B _{R} $, where $j$ is $\omega _{st} $-compatible.
%  Define $$\zeta (R) = \inf _{j, C \in \mathcal{S} _{j} } \area _{g _{j} } (C).
%  $$ 
%  \begin{remark}
%  It seems likely that $\zeta (R) =0$  or  $\zeta (R) =\pi R ^{2} $.
%  \end{remark}
% \begin{theorem} \label{cor:nonsqueezing2} 
% For $(M = S ^{2} \times T ^{2}  ,\omega)$, and $r$ as above, there is an $\epsilon$-neighborhood of $\omega$ in the space of $\lcs$ forms, which do not admit an $\lcs$ embedding of $B _{R} $, for $\zeta (R) > \pi r ^{2} $.
% There is an $\epsilon>0$ with the following property. Given an $\lcs$ $\omega _{1}$ on $M$ which is $C ^{0} $ $\epsilon$-close to $\omega _{0} $, 
% there is no morphism $$\phi: (B _{R}, \omega _{st})  \hookrightarrow (M, \omega _{1}).$$ 
% The proof of this readily follows by the proof of Theorem \ref{cor:nonsqueezing}
% and is omitted.
\subsection {Sky catastrophes} \label{section:skycatastrophy}
This final introductory section will be of a slightly more technical nature. 
The following is well known.
\begin{theorem} \label{thm:quantization} [\cite{citeMcDuffSalamon$J$--holomorphiccurvesandsymplectictopology}, \cite{citeKatrinQuantization}] Let $(M, J)$ be a compact almost complex manifold,  and $u: (S ^{2}, j)   \to M$ a $J$-holomorphic map. Given a Riemannian metric $g$ on $M$, there is an $\hbar = \hbar (g,J) >0$ s.t. if $e _{g} 
 (u) < \hbar $ then $u$ is constant, where $e _{g} $ is the $L ^{2} $-energy functional, 
\begin{equation*}
e _{g}  (u)=\energy _{g} (u) = \int _{S ^{2} } |du| ^{2}  dvol.
\end{equation*}
\end{theorem}
% \begin{remark} This is not a profound remark but may give some perspective to the reader.
%     For the proof we may actually work
%    with a priori more general notion of a $\lcsm$. That is define a
% $\lcsm$ of fancy kind, to be a smooth $2n$-fold with a non-degenerate
% 2-form $\omega$, s.t. for any $p \in M$ there is a chart $\phi: U \subset M  \to \mathbb{R} ^{2n} $,
% $U \ni p$ s.t. for any $\omega$-compatible almost complex structure
% $J$ on $U$, $\phi _{*}J $ is compatible with the standard symplectic
% form on $\mathbb{R} ^{2n} $. It follows that given such a structure
% the tangent transition maps for charts above, will induce linear automorphisms of
% $\mathbb{R} ^{2n} $, which take any $\omega$-compatible linear complex
% structure on the vector space $\mathbb{R} ^{2n} $ to an $\omega$-compatible complex structure, for
% $\omega$ here the standard symplectic form on the vector space
% $\mathbb{R} ^{2n} $. We claim (although this is not obvious, but is
% fairly elementary) that this group is the group of
% linear conformal symplectic automorphisms. It follows immediately from 
% this that the notion of $\lcsm$ of fancy kind is equivalent to the
% classical notion.
% \end{remark}
% Given the theorem above if we had universal $L ^{2} $ energy
% bounds for $J$-holomorphic curves in a fixed class $A$ we would
% immediately obtain Gromov compactness theorem for  class $A$
% $J$-holomorphic curves in an almost complex $(M, J)$. However
% in general this energy
% could be unbounded. We shall see this happen in our examples further
% on. 
% Nevertheless given Theorem \ref{thm:quantization}  the classical ``Gromov compactness'' argument immediately
% gives the following.
Using this we get the following (trivial) extension of Gromov compactness to this setting.
Let $$ \mathcal{M} _{g,n}   (J, A) = \mathcal{M} _{g,n}   (M, J, A)$$ denote the moduli space of isomorphism classes of class $A$, $J$-holomorphic curves in
  $M$, 
 with domain a genus $g$ closed
 Riemann surface, with $n$ marked labeled points. Here
   an isomorphism between $u _{1} : \Sigma _{1}  \to M$, and $u _{2}: \Sigma _{2} \to M  $ is a biholomorphism of marked Riemann surfaces $\phi: \Sigma _{1} \to \Sigma _{2}  $ s.t. $u_2 \circ \phi = u _{1} $.

\begin{theorem} \label{thm:complete}
Let  $ (M,
J)$ be an almost complex manifold.
   Then $\mathcal{M} _{g,n}   (J, A)$  has a pre-compactification  
   \begin{equation*}
\overline{\mathcal{M}} _{g,n}   ( J, A), 
\end{equation*}
by Kontsevich stable maps, with respect to the natural
metrizable Gromov topology see for instance
\cite{citeMcDuffSalamon$J$--holomorphiccurvesandsymplectictopology},
for genus 0 case. Moreover given $E>0$,   the subspace
$\overline{\mathcal{M}} _{g,n}   ( J,
 A) _{E} \subset \overline{\mathcal{M}}_{g,n}   ( J,
 A) $ consisting of elements $u$ with $e (u) \leq E$ is
 compact, where $e$ is the $L ^{2} $ energy with respect to an auxillary metric.   In other words $e$ is a proper function.
\end{theorem}
Thus the most basic situation where we can talk
about Gromov-Witten ``invariants'' of $(M, J)$ is when the $\energy$ function is bounded on 
$\overline{\mathcal{M}} _{g,n}   (J, A)$, and we shall say that $J$ is \textbf{\emph{bounded}} (in class $A$), later on we generalize this in terms of what we call \textbf{\emph{finite type}}.
In this case $ \overline{\mathcal{M}} _{g,n}   (J, A)$ is compact, and has a 
virtual moduli cycle as in the original approach of Fukaya-Ono \cite{citeFukayaOnoArnoldandGW}, or the more algebraic approach \cite{citePardonAlgebraicApproach}.
So we may define functionals:
\begin{equation} \label{eq:functional1}
GW _{g,n}  (\omega, A,J): H_* (\overline{M} _{g,n}) \otimes H _{*} (M ^{n} )  \to
   \mathbb{Q},
\end{equation}
where $\overline {M} _{g,n} $ denotes the compactified moduli space of Riemann surfaces.
Of course symplectic manifolds with any tame almost complex structure is one class of examples, another class of examples comes from some locally conformally symplectic manifolds.


   Given a continuous in the $C ^{\infty} $ topology family $\{J _{t} \}$, $t \in [0,1]$ we denote by $\overline{\mathcal{M}} _{g}
   (\{J _{t} \}, A)$ the space of pairs $(u,t)$, $u \in \overline{\mathcal{M}} _{g}(J _{t}, A)$.
\begin{definition} We say that a continuous family  $\{J _{t}
   \}$ on a compact manifold $M$ has a \textbf{\emph{holomorphic sky catastrophe}} in class
   $A$ if 
there is an element $u \in \overline{\mathcal{M}} _{g} (J _{i}, A)   $, $i=0,1$
which does not belong to any open compact (equivalently energy bounded) subset of $\overline{\mathcal{M}} _{g}
   (\{J _{t} \}, A)$. 
\end{definition} 
Let us slightly expand this definition. If $\overline{\mathcal{M}} _{g}
   (\{J _{t} \}, A)$ is locally connected, so that the connected components 
are open, then we have a sky catastrophe in the sense above if and only if there is a $u \in \overline{\mathcal{M}} _{g} (J _{i}, A)   $ which has a non-compact connected component in $\overline{\mathcal{M}} _{g}
   (\{J _{t} \}, A)$.

   At this point in time there are no known examples of  families $\{ J _{t}  \}$ with sky catastrophes, cf. \cite{citeFullerBlueSky}. 
   \begin{question}
   Do sky catastrophes exist? 
\end{question}
Really what we are interested in is whether they exist generically. 
The author's opinion is that they may appear even generically. However, if we further constrain the geometry to exact $\lcs$ structures as in Section \ref{sec:highergenus}, then the question becomes much more subtle, see also \cite{citeSavelyevFuller} for a related discussion on possible obstructions to sky catastrophes.
   %
   % for  locally conformally symplectic deformations $\{(\omega _{t}, J _{t}  )\}$ as previously defined, it might be possible that holomorphic sky catastrophes cannot exist generically, for example it looks very unlikely that an example can be constructed with Reeb tori (see the following section),  
%    \begin{conjecture} Let $\{(\omega _{t}, J _{t} )\}$ be a family of pairs of locally conformally symplectic forms on a closed manifold and compatible complex structures. Then there exists a $C ^{\infty} $ nearby family $\{J' _{t} \}$ compatible with $\{\omega _{t} \}$ s.t. $\{J _{t} \}$ has no sky catastrophes.
%  \end{conjecture}
% In this direction we have the following. 
% For a $\lcsm$ $(M,\omega)$ we have
% a uniquely associated class $\alpha \in H ^{1} (M, \mathbb{R}) $ s.t. on the associated covering space $\widetilde{M} $, $\widetilde{\omega} $ is globally conformally symplectic. The class $\alpha$ is the Cech 1-cocycle, given as follows. 
% Let $\phi _{\alpha,\beta}$ be the transition map for $\lcs$ charts of $(M, \omega)$. Then $\phi _{\alpha,\beta} ^{*} \omega _{st} = g _{\alpha, \beta} \omega _{st}  $ for a positive real constant $g _{\alpha, \beta} $ and this gives our 1-cocycle. We denote by $PD (\alpha)$ its Poincare dual.
% =======
%    \begin{conjecture} Let $\{(\omega _{t}, J _{t} )\}$, $t \in [0,1]$, be a family of compatible $\lcs$ pairs. Then there exists a $C ^{\infty} $ nearby family $\{J' _{t} \}$ compatible with $\{\omega _{t} \}$ s.t. $\{J _{t} \}$ has no sky catastrophes.
%  \end{conjecture}

Related to this we have the following technical result that will be used in the proof of non-squeezing discussed above.   % For an embedded surface $B \subset M$ let $S (B) \subset M$ denote the boundary of the unit normal disk bundle of $B$, (for any fixed metric).
\begin{theorem} \label{thm:noSkycatastrophe} Let $M$ be closed and $\{\omega _{t} \}$, $t \in [0,1]$, a continuous (with respect to the topology $\mathcal{T} ^{0} $) family of $\lcs$ forms on $M$. Let $\{J _{t} \}$ be a Frechet smooth family of almost complex structures, with $J _{t} $ compatible with $\omega _{t} $ for each $t$.
Let $A \in H _{2} (M) $ be fixed,
   % Let $\Sigma _{i} \subset M$, $i=0, \ldots, m$ be a collection of oriented hypersurfaces.
      % s.t. $PD(\alpha _{t} ) = \sum _{i} a   _{i,t} [\Sigma _{i} ]  $ for each $t$. ($a _{i,t} $ need not be continuous in $t$.) 
and let $D \subset \widetilde{M} $, with $\pi: \widetilde{M} \to  M$ the universal cover of $M$, be a fundamental domain, and $K:= \overline {D}$  its topological closure.
%    , satisfying: $$\pi (\partial K) = \cup _{i} \Sigma _{i},  $$
% $$\forall i: \pi (K ^{\circ} ) \cap \Sigma _{i}= \emptyset, $$
% where $K ^{\circ} $ denotes the interior of $K$.
   Suppose that for each $t$, and for every  $x \in \partial K$ there is a $\widetilde{J} _{t} $-holomorphic hyperplane $H _{x} $ through $x$, with $H _{x} \subset K $,  such that $\pi (H _{x}) \subset M$ \text{ is a closed submanifold and } such that $A \cdot \pi_*([H _{x}]) \leq 0$.
   % such that $$\pi \circ u _{x} \text{ is injective and } \image \pi (u _{x}) \subset \cup _{i} \Sigma _{i},  $$ and such that $A \cdot \pi_*(B) \leq 0$.
Then $\{J _{t} \}$ has no sky catastrophes in class $A$. 
\end{theorem}

If holomorphic sky catastrophes are discovered, this would be a very  interesting  discovery.  The original discovery by Fuller \cite{citeFullerBlueSky} of sky catastrophes in dynamical systems is one of the most important in dynamical systems, see also \cite{citeShilnikovTuraevBlueSky} for an overview.


\section {Elements of Gromov-Witten theory of an $\lcs$ manifold} \label{sec:elements}
Suppose $(M,J)$ is a compact almost complex manifold, where the almost complex structures $J$ are assumed throughout the paper to be $C ^{\infty} $, and let 
   $N \subset
\overline{\mathcal{M}} _{g,k}   (J,
A) $ be an open compact subset with $\energy$ positive on $N$. The latter condition is only relevant when $A =0$.  
 We shall primarily refer in what follows to work of Pardon in \cite{citePardonAlgebraicApproach}, only because this is what is more familiar to the author, due to greater comfort with algebraic topology.
But we should mention that the latter is a follow up to a profound theory that is originally created by Fukaya-Ono \cite{citeFukayaOnoArnoldandGW}, and later expanded with Oh-Ohta \cite{citeFukayaLagrangianIntersectionFloertheoryAnomalyandObstructionIandII}.

 The construction in \cite{citePardonAlgebraicApproach} of implicit atlas,  on the moduli space $\mathcal{M}$  of curves in a symplectic manifold, only needs a neighborhood of $\mathcal{M}$ in the space of all curves. So more generally if we have an almost complex manifold and an \emph{open} compact component $N$ as above,  this will likewise have a natural implicit atlas, or a Kuranishi structure in the setup of \cite{citeFukayaOnoArnoldandGW}.  And so such an $N$ will have
  a virtual fundamental class in the sense of Pardon~\cite{citePardonAlgebraicApproach}, (or in any other approach to virtual fundamental cycle, particularly the original approach of Fukaya-Oh-Ohta-Ono).
  This understanding will be used in other parts of the paper, following  Pardon for the explicit setup. 
We may thus define functionals:
\begin{equation} \label{eq:functionals2}
GW _{g,n}  (N,A,J): H_* (\overline{M} _{g,n}) \otimes H _{*} (M ^{n} )  \to
   \mathbb{Q}.
\end{equation}
How do these functionals depend on $N,J$? 
\begin{lemma} \label{prop:invariance1} Let $\{J _{t} \}$, $t \in [0,1]$ be a Frechet smooth family. Suppose that $\widetilde{N}$ is an open compact subset of the cobordism moduli
space $\overline{\mathcal{M}} _{g,n}   (\{J _{t} \},
A)
$ and that the energy function is positive on $\widetilde{N} $, (the latter only relevant when $A=0$).
   Let $$N _{i} = \widetilde{N} \cap \left( \overline{\mathcal{M}} _{g,n}   (J _{i}, A)\right),  $$  then $$GW _{g,n}  (N _{0}, A, J _{0} ) = GW _{g,n} (N _{1}, A,
 J _{1}).  $$ In particular if $GW _{g,n}  (N _{0}, A,  J _{0} )
   \neq 0$, there is a class $A$ $J _{1} $-holomorphic stable map in $M$.
\end{lemma} 
\begin{proof} [Proof of Lemma \ref{prop:invariance1}]
% Let $\widetilde{N} $ be as in the hypothesis.  
   % Then there is an $\epsilon>0$ s.t. $$\energy ^{-1} ([0,\epsilon)) \subset \overline{\mathcal{M}} _{g,n}  (\{J _{t} \},A)$$ consists of constant curves. 
%  $\widetilde{N}'= \widetilde{N} - \energy ^{-1} ([0,\epsilon))   $ is an open-closed subset of $\overline{\mathcal{M}} _{g,n}  (\{J _{t} \},A)$ and is compact, as $\widetilde{N} $ is compact. 
   % Thus each $N _{i}  $ has a pair of open connected components $N' _{i} $, $N ^{cont} _{i}  $,  $N' _{i}  = \widetilde{N}' \cap \overline{\mathcal{M}} _{g,n}  (J _{i} ,A)$, $N ^{cont} _{i}  $ consisting of constant curves, which must necessarily be all constant curves since this is an open-closed set.
   We may construct exactly as in \cite{citePardonAlgebraicApproach} a natural implicit atlas on $\widetilde{N} $, with boundary $N _{0}  ^{op} \sqcup  N _{1} $, ($op$ denoting opposite orientation). And so 
   \begin{equation*}
   GW _{g,n}   (N  _{0}, A, J _{0} ) = GW _{g,n}    (N _{1}, A, J _{1} ), 
\end{equation*}
as functionals.
\end{proof}



The most basic lemma in this setting is the following, and we shall use it in the following section.
\begin{definition} An \textbf{\emph{almost symplectic pair}} on $M$ is a tuple $(M, \omega, J)$, where $\omega$ is a non-degenerate 2-form on $M$, and $J$ is $\omega$-compatible, meaning that $\omega (\cdot, J \cdot)$ defines $J$-invariant Riemannian metric. When $\omega$ is lcs we call such a pair an \textbf{\emph{lcs pair}}.
\end {definition}
\begin{definition} We say that a pair of almost symplectic pairs $(\omega _{i}, J _{i}  )$ are \textbf{$\delta$-close}, if $\{\omega _{i}\}$ are $C ^{0} $ $\delta$-close, and $\{J _{i} \}$ are $C ^{2} $ $\delta$-close, $i=0,1$. Define this similarly for a pair $(g _{i} ,J _{i} )$ for $g$ a Riemannian metric and $J$ an almost complex structure.
\end{definition}
\begin{definition} For an almost symplectic pair $(\omega,J)$ on $M$, and a smooth map $u: \Sigma \to M$ define:
\begin{equation*}
e_{\omega} (u) = \int _{\Sigma} u ^{*}\omega.
\end{equation*}
\end{definition}
By an elementary calculation this coincides with the $L ^{2} $ $g _{J} $-energy of $u$, for $g _{J} (\cdot, \cdot)= \omega (\cdot, J \cdot) $. That is $e _{\omega} (u)= e _{g _{J} } (u)   $.
In what follows by $f ^{-1} (a,b) $, with $f$ a function,  we mean the preimage by $f$ of the open set $(a,b)$.
\begin{lemma} \label{thm:nearbyGW} 
Given a compact $M$ and an almost symplectic pair $(\omega, J)$ on $M$, suppose that $N \subset \overline{\mathcal{M}} _{g,n}  (J,A)
$ is a compact and open component 
which is energy isolated meaning  that $$N \subset \left( U  =
e _{\omega}  ^{-1} (E ^{0}, E ^{1}) \right)
\subset \left( V  =
e _{\omega}  ^{-1} (E ^{0}  - \epsilon, E ^{1}  + \epsilon  ) \right),
$$ with $\epsilon>0$, $E ^{0} >0$ and with $V  \cap  \overline{\mathcal{M}} _{g,n}  (J,A)
= N$. Suppose also that $GW _{g,n} (N, J,A) \neq 0$.
   Then there is a $\delta>0$ s.t. whenever $(\omega', J')$ is a compatible almost symplectic pair 
   $\delta$-close to $(\omega, J)$, there exists $u \in \overline {\mathcal{M}} _{g,n} (J',A) \neq \emptyset $, with $$E ^{0}  < e _{\omega'} (u) < E ^{1}.
   $$
\end{lemma}
\begin{proof} [Proof of Lemma  \ref{thm:nearbyGW}]
   \begin{lemma} \label{lemma:NearbyEnergy} Given a Riemannian manifold $(M,g)$,
and $J$
an almost complex structure, suppose that $N \subset \overline{\mathcal{M}} _{d,n}  (J,A)
$ is a compact and open component 
which is energy isolated meaning  that $$N \subset \left( U  =
e _{g}  ^{-1} (E ^{0}, E ^{1}) \right)
\subset \left( V  =
e _{g}  ^{-1} (E ^{0}  - \epsilon, E ^{1}  + \epsilon  ) \right),
$$ with $\epsilon>0$, $E _{0}>0 $, and with $V  \cap  \overline{\mathcal{M}} _{g,n}  (J,A)
= N$. Then there is a $\delta>0$ s.t. whenever $(g', J')$ is  
   $\delta$-close to $(g, J)$ if $u \in \overline{\mathcal{M}} _{g,n}  (J',A)
$ and $$E ^{0}  -\epsilon  < e _{g'}  (u) < E ^{1}  +\epsilon  $$
then $$E ^{0}    < e _{g'}  (u) < E ^{1} .
      $$
\end{lemma}
\begin{proof} [Proof of Lemma \ref{lemma:NearbyEnergy}] 
%  The $C ^{0} $ metric on $2$-forms on $M$, is defined with respect to a fixed
%  Riemannian metric $g$ on $M$, and is given by
%  \begin{equation*}
%     d(\omega _{0}, \omega _{1})  = sup _{z; v,w \in T _{z} M} |\omega _{0}  (v,w) - \omega _{1}  (v,w)|,
%  \end{equation*}
%  for $v,w$ a $g$-orthonormal pair in $T _{z} M $.
%     \textcolor{blue}{check $C^0$ convergence vs $C^{\infty}$ convergence} 
    Suppose otherwise then there is a sequence $\{(g _{k}, J _{k}) \}$   converging to $(g,J)$, and a sequence $\{u _{k} \}$ of $J _{k} $-holomorphic stable maps satisfying
    $$E ^{0} - \epsilon < e _{g _{k} }  (u _{k} ) \leq E ^{0}   $$
    or $$E
    ^{1}  \leq  e _{g
    _{k} }  (u _{k} ) < E ^{1}  +\epsilon.
    $$ By Gromov compactness, specifically theorems \cite[B.41, B.42]{citeMcDuffSalamon$J$--holomorphiccurvesandsymplectictopology}, we may find a Gromov convergent subsequence $\{u _{k _{j} } \}$ to a $J$-holomorphic
    stable map $u$, with $$E ^{0}  - \epsilon \leq e _{g}  (u) \leq E ^{0}
      $$ or $$E
    ^{1}  \leq  e _{g}
      (u) \leq E ^{1}  +\epsilon.
      $$ But by our assumptions such a $u$ does not exist.
 \end{proof}
 \begin{lemma} \label{lemma:NearbyEnergyDeformation} Let $M$ be compact, and 
let $(M,\omega,J)$ be an almost symplectic triple, so that $N \subset \overline{\mathcal{M}} _{g,n}  (J,A) $ is exactly as in the lemma above with respect to some $\epsilon>0$.
 Then, there is a $\delta'>0$ s.t. the following is satisfied.
    Let $(\omega',J')$  be $\delta'$-close to
    $(\omega,J)$, then there is a continuous in the $C ^{\infty} $ topology family of almost symplectic pairs $\{(\omega _{t} , J _{t}  )\}$,
    $(\omega _{0}, J_0)= (g, J)$,  $(\omega _{1}, J_1)= (g', J')$
    s.t. there is open compact subset 
 \begin{equation*}
  \widetilde{N}  \subset \overline{\mathcal{M}} _{g,n}  (\{J _{t}\},A),
 \end{equation*}
 and
 with $$\widetilde{N}  \cap \overline{\mathcal{M}} (J,A) = N.
 $$
    Moreover  if $(u,t) \in \widetilde{N} 
       $        then $$E ^{0}    < e _{g _{t} }  (u) < E ^{1}.
      $$
 \end{lemma}
 \begin{proof} For $\epsilon$ as in the hypothesis, let $\delta$ be as in Lemma \ref{lemma:NearbyEnergy}. 
\begin{lemma} \label{lemma:Ret} Given a $\delta>0$  there is a $\delta'>0$ s.t. if $(\omega',J')$ is $\delta'$-near $(\omega, J)$ there is an interpolating, continuous in $C ^{\infty} $ topology family $\{(\omega _{t}, J _{t}  )\}$ with $(\omega _{t},J _{t}  )$ $\delta$-close to $(\omega,J)$ for each $t$.
\end{lemma}
\begin{proof}    Let $\{g _{t} \} $ be the family of metrics on $M$ given by the convex linear combination of $g=g _{\omega _{J} } ,g' = g _{\omega',J'}  $. Clearly $g _{t} $ is $\delta'$-close to $g _{0} $ for each $t$. Likewise the family of 2 forms $\{\omega _{t} \}$ given by the convex linear combination of $\omega $, $\omega'$ is non-degenerate for each $t$ if $\delta'$ was chosen to be sufficiently small and 
is $\delta'$-close to $\omega _{0} = \omega _{g,J}  $ for each moment.

    Let $$ret: Met (M) \times \Omega (M)  \to \mathcal{J} (M)  $$ be the  ``retraction map'' (it can be understood as a retraction followed by projection) as defined in \cite [Prop 2.50]{citeMcDuffSalamonIntroductiontosymplectictopology}, where $Met (M)$ is  space of metrics on $M$, $\Omega (M)$ the space of 2-forms on $M$, and $ \mathcal{J} (M)$ the space of almost complex structures. This map has the property that the almost complex structure $ret (g,\omega)$ is compatible with $\omega$, and that $ret (g _{J}, \omega ) = J$ for $g _{J} = \omega (\cdot, J \cdot) $. Then $ \{(\omega _{t}, ret (g _{t}, \omega _{t})   \} $ is a compatible  family.
   As $ret  $ is continuous in $C ^{2} $-topology, $\delta'$ can be chosen so that $ \{ ret _{t} (g _{t}, \omega _{t}    \} $ are $\delta$-nearby.
\end {proof}
Let $\delta'$ be chosen with respect to $\delta$ as in the above lemma and $\{ (\omega _{t}, J _{t}  )\}$ be the corresponding family.
   % Since $J, J'  $ are also $\delta$-nearby, it is not hard to see that we may find a smooth family $\{J _{t} \}$, $J _{0}=J $, $J _{1}= J {'}  $ so that $\{(\omega _{t}, J _{t}  )\}$ is a compatible $\lcs$ family, and so that $J _{t} $ is $C ^{\infty} $ $\delta$-close to $J$ for each $t$.
     Let $\widetilde{N}  $ consist of all elements $(u,t) \in \overline{\mathcal{M}} (\{J _{t} \},A)$ s.t. $$E ^{0}  -\epsilon  < e _{\omega _{t} }  (u) < E ^{1}  +\epsilon.
     $$

Then by Lemma \ref{lemma:NearbyEnergy} for each $(u,t) \in \widetilde{N}  $, we have:
$$E ^{0} < e _{\omega _{t} }  (u) < E ^{1}.$$

    In particular $\widetilde{N}$ must be closed, it is also clearly open, and is compact as the energy $e$ is a proper function, as discussed.
 \end{proof}
To finish the proof of the main lemma, let $N$ be as in the hypothesis, $\delta'$ as in Lemma \ref{lemma:NearbyEnergyDeformation}, and $\widetilde{N} $ as in the conclusion to Lemma \ref{lemma:NearbyEnergyDeformation}, then by Lemma \ref{prop:invariance1} $$GW _{g,n} (N_1, J', A) =  GW _{g,n} (N, J, A) \neq 0,$$ where $N _{1} = \widetilde{N} \cap \overline{\mathcal{M}} _{g,n}  (J _{1},A)  $. So the conclusion follows.
\end {proof}
% \textcolor{blue}{remove} 
% \begin{proposition} \label{prop:boundeddeformation0} 
%    Let $M$ be a closed manifold, and suppose that  $\{J _{t}\} $, $t \in [0,1]$ has no holomorphic sky catastrophes, then if $J _{i}$, $i =0,1$ are bounded: 
% \begin{equation*}
%    GW _{g,n}   (A, J _{0} ) = GW _{g,n}    (A, J _{1} ), 
% \end{equation*}
% if $A \neq 0$. If only $J _{0} $ is bounded then there is at least one class $A$ $J _{1} $-holomorphic curve in $M$.
% \end {proposition}
% The assumption on $A$ is for simplicity in this case, to avoid energy 0 singularities.
% \begin {proof}[Proof of Proposition \ref{prop:boundeddeformation0}]
% For each  $u \in \overline{\mathcal{M}} _{g,n}   (J _{i} , A)$, $i=0,1$, fix an open-compact subset $$V _{u} \subset \overline{\mathcal{M}} _{g,n}   (\{J _{t} \}, A)$$ containing $u$.
% We can do this by the hypothesis that there are no sky catastrophes.
% Since $\overline{\mathcal{M}} _{g,n}   (J _{i} , A)$ are compact we
% may find a finite subcover $$\{V _{u _{i} } \} \cap (\overline{\mathcal{M}} _{g,n}
% (J _{0} , A) \cup \overline{\mathcal{M}} _{g,n}   (J _{1} , A))$$ of $\overline{\mathcal{M}} _{g,n}
% (J _{0} , A) \cup \overline{\mathcal{M}} _{g,n}   (J _{1} , A)$, considering $\overline{\mathcal{M}} _{g,n}
% (J _{0} , A) \cup \overline{\mathcal{M}} _{g,n}   (J _{1} , A)$ as a subset of $\overline{\mathcal{M}} _{g,n}   (\{J _{t} \}, A)$ naturally.
% Then $V=\bigcup_{i} V _{u _{i} } $ is an open compact subset of
% $\overline{\mathcal{M}} _{g,n}   (\{J _{t} \}, A)$, s.t. $$V \cap \overline{\mathcal{M}} _{g,n}   (J _{i} , A) = \overline{\mathcal{M}} _{g,n}   (J _{i} , A).
% $$ Now apply Lemma \ref{prop:invariance1}.
%
% Likewise if only $J _{0} $ is bounded, for each  $u \in \overline{\mathcal{M}} _{g,n}   (J _{0} , A)$,  fix an open-compact subset $V _{u} $ of $\overline{\mathcal{M}} _{g,n}   (\{J _{t} \}, A)$ containing $u$. Since $\overline{\mathcal{M}} _{g,n}   (J _{0} , A)$ is compact we
% may find a finite subcover $$\{V _{u _{i} } \} \cap (\overline{\mathcal{M}} _{g,n}
% (J _{0} , A) $$ of $\overline{\mathcal{M}} _{g,n}
% (J _{0} , A) $.
% Then $V=\bigcup_{i} V _{u _{i} } $ is an open compact subset of
% $\overline{\mathcal{M}} _{g,n}   (\{J _{t} \}, A)$, s.t. $$V \cap \overline{\mathcal{M}} _{g,n}   (J _{i} , A) = \overline{\mathcal{M}} _{g,n}   (J _{i} , A).
% $$ 
% Again apply Lemma \ref{prop:invariance1}.
% \end{proof}
% The above is proved via the following technical result. \textcolor{blue}{remove} 
 % What follows is one non-classical application of the above theory.
% We will discuss sky catastrophies in more detail in Section \ref{section:skycatastrophy}, for the moment the reader may just think of a sky catastrophy as a having a continuous path $\{J _{t} \}$, $t \in [0,1]$ of almost complex structures on $M$ and a continuous path $t \mapsto u _{t}$ of $J _{t} $-holomorphic maps with $energy (u _{t} ) \to \infty$ as $t \to 1$, although the full definition allows more general phenomena encompassing the above.

While not having sky catastrophes gives us a certain compactness control, the above is not immediate because we can still in principle have total cancellation of the infinitely many components of the moduli space $\overline{\mathcal{M}} _{1,1} (J ^{\lambda}, A ) $.
In other words a virtual 0-dimension Kuranishi space $\overline{\mathcal{M}} ^{1,0} (J ^{\lambda}, A ) $, with an infinite number of compact connected components, can certainly be null-cobordant, by a cobordism all of whose components are compact.  So we need a certain additional algebraic and geometric control to preclude such a total cancellation. 
% The reader may think of the example $M=S ^{2} \times T ^{2}  $ and $\Sigma _{i} $ as in Theorem \ref{cor:nonsqueezing}.
% \begin{definition} We say that an $\lcs$ $\omega$ on $M$ is
%    \textbf{\emph{bounded}} if its lift $\widetilde{\omega} $ to the universal
%    cover $\widetilde{M} $ is
%    diffeomorphic to $f \cdot \alpha  $ for $f>0$ a bounded
%    continuous function and $\alpha$ symplectic.
%    We say that a pair of $\lcs$ forms $\omega _{1}, \omega _{2} $ are \textbf{\emph{bounded deformation equivalent}} if there is a
%    smooth family $\widetilde{\omega _{t} } = f _{t} \cdot \alpha _{t}   $ of
%    $\lcs$ forms on the universal cover $\widetilde{M} $ with each $\alpha _{t} $
%    symplectic, each $f _{t} >0  $ bounded, and with $\widetilde{\omega}_{i}$
%    diffeomorphic to lifts of $\omega _{i} $.
% \end{definition}
% \begin{lemma} \label{prop:bounded} 
% Suppose that $\omega $ is a bounded
% $\lcs$ form on $M$, $J$ is $\omega$-compatible and let $\overline{\mathcal{M}}_{0} ( J,
%  A) $ be the moduli space of genus 0 class $A$ $J$-holomorphic curves in $M$.
%    Then $\energy$ is bounded on $\overline{\mathcal{M}}_{0} ( J,
%  A) $ and so it is compact.
% \end{lemma}
% \begin{proof} This is immediate since a genus 0 $J$-holomorphic curve $u$ lifts to the universal cover and so $$\energy (u) \leq \max f  \langle \alpha, \widetilde{A}
% \rangle,  $$ for $f, \alpha$ as in the definition of bounded, and for
% $\widetilde{A} $ the homology class of the lift.
% \end{proof}
% Thus given a bounded $\lcs$ form $\omega$ on $M$ we may use the virtual moduli
% cycle as constructed for example by Pardon \cite{cite} to define genus 0 Gromov-Witten
% ``invariants'' $$GW _{0}  (\omega, \{a _{k} \}, A)$$, ``counting'' class $A$,
% $J$-holomorphic genus 0 curves
% passing through some homology classes $a _{k} \in H _{i _{k} } (M) $.
% \begin{definition} \label{def:comparable} We say that a pair of 2-forms $\omega, \omega'$ on a manifold are
%    \textbf{\emph{comparable}} or \textbf{\emph{$C$}-comparable}, if there is a constant $C>0$ s.t.
% \begin{equation*}
% \frac{1}{C} \omega (v,w) \leq \omega' (v,w) \leq C \omega (v,w),
% \end{equation*}
%    for all $v,w$ s.t. $\omega' (v,w)>0$. We shall say that an $\lcs$ form $\omega$ is \textbf{\emph{bounded}} if it is $C$-comparable with a symplectic form for some $C$. 
% \end{definition}
% \begin{lemma} \label{lemma:boundedomega} 
%    If $\omega$ is bounded then every compatible pair $(\omega,J)$ is bounded.
% \end{lemma} 
% Moreover the functionals \eqref{eq:functional1} will be shown to be only dependent on $\omega$ for such a pair.
% Thus if $\omega$ is bounded  we have functionals, 
% \begin{equation*}
% GW _{g,n}  (\omega, A): H_* (\overline{M} _{g,n}) \otimes H _{*} (M ^{n} )  \to
%    \mathbb{Q}.
% \end{equation*}
% \begin{definition} \label{def:bounded} We say that an $\lcs$ $\omega$ on $M$ is
%    $A$-\textbf{\emph{bounded}} if 
%     
% \end{definition}
%  
%    
%    
%    $J$ is $\omega$-compatible and let $\overline{\mathcal{M}}_{0} ( J,
%  A) $ be the moduli space of genus 0 class $A$ $J$-holomorphic curves in $M$.
%    Then $\energy$ is bounded on $\overline{\mathcal{M}}_{0} ( J,
%  A) $ and so it is compact.
% We shall use the above invariants in the following.
% Note that the above fails if the $\lcs$ condition is relaxed to just having  non-degenerate 2-forms. Let $M= S ^{2} \times T ^{2n-2} $,  we claim that it is possible to find an approximately $j _{st} $-linear isometric embedding $\phi: B _{R} \to M $. Then $\omega _{1} = - g _{st} (\cdot, \phi_* j _{st} ) $, gives a counterexample.
%
% \begin{corollary} \label{corollary:epsilon}
%    Let $(M=S ^{2} \times V, \omega) $ be the product symplectic manifold with $\pi _{2} V=0 $. Let $A=[S ^{2} ] \times [pt]$. Suppose that $ \langle [\omega], A  \rangle = \pi r ^{2}  $. 
% Suppose that $\omega _{1}$ is an $\lcs$ form  $c$-deformation equivalent to $\omega _{0} $, and $C$-comparable with $\omega $, with $C< \frac{R}{r}$.
%    Then there is no $\lcs$ embedding $$\phi: B _{R} \hookrightarrow (M, \omega _{1}). $$
% \end{corollary}
% \begin{proof}
%    This follows by  the theorem as under the hypothesis of the corollary $GW _{0,1}  (\omega, A) ([pt]) \neq 0$.
% \end{proof}
% \subsection {A possibly non-classical invariance of Gromov-Witten invariants of symplectic manifolds}
% We recall that a pair of symplectic forms are called deformation equivalent if they are joined by a continuous family of symplectic forms. It is well known that Gromov-Witten invariants are independent of choices symplectic forms in the same deformation equivalence class.
% Note that $c$-deformation equivalence also gives a possibly more general notion of equivalence of classical symplectic forms. A key lemma needed for the proof of Theorem \ref{cor:nonsqueezing} is the following.
%   \begin{proposition} \label{prop:boundeddeformation} 
%   Let $M$ be a closed manifold, then if $\omega$ is bounded every pair $(\omega, J)$ is bounded.
%  If $\omega _{0}, \omega _{1} 
%   $ are $c$-deformation equivalent $\lcs$ forms, then:
% \begin{equation*}
%    GW _{g,n}   (\omega _{0}, A ) = GW _{g,n}    (\omega _{1},A). 
% \end{equation*}
% \end{proposition}
% Note that this also tells us that the classical Gromov-Witten invariants for symplectic manifolds are independent of choices symplectic forms in the same $c$-deformation equivalence class. So we should ask:
% \begin{question} Are there $c$-deformation equivalent symplectic forms which are not deformation equivalent?
% \end{question}
% We note that at the moment it is not known whether holomorphic sky catastrophes
% can exist at all. Our opinion is that while generally they may exist, in the
% setup of the theorem above they should not, so that such an $\lcs$ embedding is
% impossible.
% We point out that this is indeed ``rigidity''.  That is suppose we remove the condition that 
% $\phi $ is an $\lcs$ embedding, and put in the only obvious geometric
% invariant
% property of $\lcs$ embeddings that they be volume preserving.
% Then take $\omega' = \omega $, and
% take $\phi$ to be the map induced by:
% \begin{equation*}
%    (z_1, z_2) \mapsto (R/r z_1, z_2 * r/R), \quad $z_1 \in \mathbb{C}$, $z _{2} \in
%    \mathbb{C} ^{n-1} $,
% \end{equation*}
% (enlarge the form on $T ^{2n-2} $ if necessary).
% Then $\phi _{*} j _{st} = j_st  $ which is compatible with $\omega$, and $\phi$
% is volume preserving, giving a volume preserving counter-example to the theorem.
% Thus there is extra rigidity information in the $\lcs$ structure. 

\begin{proof} [Proof of Theorem \ref{thm:complete}] (Outline, as the argument is
   standard.)  Suppose that we have a sequence $u ^{k} $ of
   $J$-holomorphic maps with $L ^{2} $-energy $\leq E$. 
By
\cite
[4.1.1]{citeMcDuffSalamon$J$--holomorphiccurvesandsymplectictopology},
a sequence $u ^{k} $ of $J$-holomorphic curves has a convergent
subsequence if $sup _{k} ||du ^{k} || _{L ^{\infty} } < \infty$.
On the other hand when this condition does not hold rescaling
argument tells us that a holomorphic sphere bubbles off. The
quantization Theorem \ref{thm:quantization}, then tells us that these
bubbles have some minimal energy, so if the total energy is capped by
$E$, only finitely many bubbles may appear, so that a subsequence of
$u ^{k} $ must converge in the Gromov topology to a Kontsevich stable map.
\end{proof}




\section {Rulling out some sky catastrophes and non-squeezing} \label{section:basics} % \begin{proof} [Proof of Theorem \ref{thm:quantization}]
Let $M$ be a smooth manifold of dimension at least 4, which is an assumption as well for the rest of the paper, as dimension 2 case is special.  The $C ^{k} $ metric topology $\mathcal{T} ^{k} $ on the set $LCS (M)$ of smooth $\lcs$ $2$-forms on $M$ is defined with respect to the following metric. 
\begin{definition} \label{def:norm} Fix a Riemannian metric $g$ on $M$. For $\omega _{1}, \omega _{2} \in LCS (M)  $ define
\begin{equation*}
d _{k}  (\omega _{1}, \omega _{2}  )= d _{C ^{k}}  (\omega _{1}, \omega _{2} ) + d _{C ^{k} } (\alpha _{1}, \alpha _{2}),
\end{equation*}
for $\alpha _{i} $ the Lee forms of $\omega _{i} $ and $d _{C ^{k} } $ the 
usual $C ^{k} $ metrics induced by $g$.
% metrics induced by the co-mass norms $|\cdot| _{mass} $  with respect to $g$ on differential $k$-forms. That is $|\eta| _{mass} = \sup _{v} |\eta (v)|   $, where the supremum is over all $g$-unit $k$-vectors $v$. 
% Of course $\mathcal{L} (M)$ is not a convex subspace of the space of all forms, so the above only makes sense to define a metric $d (\omega _{1}, \omega _{2}  ) = ||\omega _{1} ||$
\end{definition}  
The following characterization of convergence will be helpful.
\begin{lemma} \label{lemma:lcsconvergence}
Let $M$ be compact and let $\{\omega _{k}\} \subset LCS (M)  $ be a sequence $\mathcal{T} ^{0} $ converging to a symplectic form $\omega$. Denote by $\{\widetilde{\omega} _{k}  \}$ the lift sequence on the universal cover $\widetilde{M} $.
   Then there is a sequence $\{\omega _{k} ^{symp}\}  $ of symplectic forms on $\widetilde{M} $, and a sequence  $\{f _{k}\}  $ of positive functions pointwise converging to $1$, such that $ \widetilde{\omega} _{k} =  f _{k} \omega _{k} ^{symp}   $.
\end{lemma}
\begin{proof} We may assume that $M$ is connected. Let $\alpha _{k} $ be the Lee form of $\omega _{k} $, and $g _{k}$
functions on $\widetilde{M} $ defined by $g _{k} ([p]) = \int _{[0,1]} p ^{*} \alpha _{k} $, where the universal cover $\widetilde{M} $ is understood as the set equivalence classes of paths $p$ starting at $x _{0} \in M $,
  with a pair $p _{1}, p _{2}  $ equivalent if $p _{1} (1) = p _{2} (1)  $ and $p _{2} ^{-1} \cdot p _{1}$ is null-homotopic, where $\cdot$ is the path concatenation. 



Then we get:
\begin{equation*}
   d \widetilde{\omega} _{k} = dg _{k} \wedge \widetilde{\omega} _{k},  
\end{equation*}
so that if we set $f _{k}:= e ^{g _{k}}  $ then
\begin{equation*}
   d (f ^{-1} _{k} \widetilde{\omega} _{k}) =0.
\end{equation*}
Since by assumption $|\alpha _{k}| _{C ^{0}}  \to 0$, then pointwise $g _{k} \to 0 $ and pointwise $f _{k} \to 1$, so that if we set $\widetilde{\omega} ^{symp}  _{k}:= f ^{-1} _{k} \widetilde{\omega} _{k}$ then we are done.
\end{proof}
 

% Suppose that we have a non-constant $u$ as in the hypothesis, then $du$ is non-singular
% outside a finite collection of points: $\Theta \subset \Sigma  $.
%    \begin{lemma} \label{lemma:boundedmeancurvature}
% For $u$ as above the
% magnitude of the mean curvature $\textbf{H}$  of the image of
% $u$ is bounded from above by a universal constant $C (\omega, J)>0$, outside
% $\Theta$.
%    \end{lemma}
% \begin {proof}
% We may fix a finite cover of $M$ by charts $\phi _{i}: U _{i} \subset
% \mathbb{R} ^{2n} \to M   $, $\phi _{i}^{*} \omega = e^{f _{i}} \omega _{0} 
% $ with $\omega _{0} $ symplectic, with $U _{i} $ contractible.
%  Then 
% $\phi _{i} ^{-1} \circ u  $ is a $\phi _{i} ^{*} J
% $-holomorphic map defined on $u ^{-1} (\phi _{i} (U _{i}) ) $ into $\mathbb{R}^{2n} $, and $\phi _{i} ^{*} J$ is
% compatible with $f _{i} \omega _{0} $ and hence with $\omega _{0} $.
% Fix an (infinite)  cover of $\Sigma - \Theta  $ by disk
%    domains $\{V _{i_j} \}$,
%  with each
% $V _{i _{j} } \subset  u ^{-1} (\phi _{i} (U _{i}) ) $ for some $i$. Then $\phi
% _{i} ^{-1} \circ   u |
% _{V _{i _{j} } } $ is an immersed  $\phi _{i} ^{*} J$-holomorphic
% curve in  $\mathbb{R} ^{2n} $. As this complex structure is compatible
% with the symplectic form $\omega $ it classically follows  that the image $D
% _{i _{j} } $ of  $\phi
% _{i} ^{-1} \circ   u |
% _{V _{i _{j} } } $ is a minimal surface and hence has mean curvature
% 0 with respect to $(\omega _{0} , \phi _{i} ^{*} J )$. Since the cover
% $\{U _{i} \}$ is finite,  the $C ^{\infty} $ norm of the
% functions $f
% _{i} $
% is universally bounded: $A<|f _{i} | <B$, for some some $A,B$ and all $i$.
%    It follows that the
% magnitude of the mean curvature of the surface $D _{i _{j} } $ with respect
% to the metric induced by $(e^{f _{i}}  \omega _{0} , \phi ^{*}_{i} J    )$ is
%    bounded from above by some
% universal constant $C (\omega, J)$. Consequently the same holds for
% $\image u |_{V _{i _{j} } }  $ with respect to  $g _{J} $ from which the result
% follows.
% \end{proof}
%
% Next we shall need the following inequality relating diameter and mean
% curvature deduced from a theorem of Topping
% \cite{citeToppingRelatingDiameter}.
% \begin{theorem} \label{thm:topping}   For a compact Riemannian manifold $(M,
%    g)$, and $f: \Sigma \to M$ a closed surface in $M$, which is
%    non-singular on a set of full-measure (with respect to an
%    auxiliary metric $g'$ on $\Sigma$):
%  \begin{equation*}
%    \diam_{g} (\image f) \leq C (M, g)
%    \int _{\Sigma} |\textbf{H}_{M,g}\circ f|   d vol _{g},
%  \end{equation*}  
% where 
% $$\diam_{g} (\image f) := \max _{x,y \in \Sigma} dist_{\Sigma, g} 
% (f(x), f(y)),$$ $dist_{\Sigma, g}$ is the induced metric from the
% ambient space,  and $dvol _{g} $ is the
% measure on $\Sigma$ induced by $g$. 
% \end{theorem}
% \begin{proof} The theorem of Topping is for closed immersed
%    submanifolds
%    of $R ^{n} $, and for surfaces $\Sigma \subset \mathbb{R} ^{n} $ says that:
%  \begin{equation*}
%    \diam (\Sigma) \leq C (n)
%    \int _{\Sigma} |\textbf{H} _{R ^{n} } | d vol _{\Sigma},
%  \end{equation*}  
% where $\diam$  is  the intrinsic diameter: $\max _{x,y \in \Sigma} dist_{\Sigma, g} 
% (x, y),$ and where $dvol _{\Sigma} $ is the measure induced from the
% standard metric on $\mathbb{R} ^{n} $.
%    To obtain the version stated for a more general but compact $
%    (M,g)$, and non-immersed surfaces, pick an isometric Nash embedding $N$ of $ (M,g)$ into
%    $\mathbb{R} ^{n} $, where $n$ is large enough. Take a small
%    perturbation $(N \circ f)'$ of $N \circ f$ so that $(N \circ f)'$
%    is an immersion (or even embedding) into $\mathbb{R} ^{n}
%    $.  Since $\Sigma$ is
%    closed 
% we get by Topping's theorem:
%  \begin{equation*}
%     \diam (\image (N \circ f)') \leq C (n)
%     \int _{\Sigma} |\textbf{H} _{\mathbb{R} ^{n}} \circ (N \circ f)'  | \, dvol _{\Sigma}.
%  \end{equation*}  
% Note that under hypotheses of the theorem the measures $dvol _{g} $ and
% $dvol _{g'} $ on $\Sigma$ are equivalent, specifically because $f$ is non-singular
% on a set of full $dvol _{g'} $ measure. Since  $M$ is compact the function $z \mapsto |\textbf{H} _{R ^{n}} 
% (N \circ f(z))|  $ on
% $\Sigma$ is bounded on a set of full $dvol _{g'} $ measure, and hence 
% $dvol _{g} $ measure,  from above by the function $z \mapsto C' (N)
% |\textbf{H}_{M, g} (f(z))|  $ on $\Sigma$ for some $C' (N) >>0$, independent
% of $f$. This $C' (N)$ is just an upper bound for the function $\lambda
%  $ on $M$, s.t. $\lambda (m) \cdot |\textbf{H} _{\mathbb{R} ^{n} }  (N
%  (m))| = |\textbf{H}
%  _{g}  (m)|$, $m \in M$. The function $\lambda$ exists and is continuous because  $|\textbf{H}|$ is clearly non-decreasing under isometric embeddings.
%
%
% So we get:
%  \begin{align*}
%     & \diam _{g}  (\image f)= \diam (\image N f) \approx \diam (\image N f') \leq C (n)
%    \int _{\Sigma} |\textbf{H} _{\mathbb{R} ^{n} } ((N \circ f)' ) | \, d vol _{\Sigma}
%    \\ & \approx C (n)
%    \int _{\Sigma} |\textbf{H} _{\mathbb{R} ^{n} } (N \circ f ) | \,
%    d vol _{\Sigma} \leq C' (N) \cdot C (n) \int _{\Sigma} |\textbf{H} _{M,g} \circ f
%     | \, d vol _{g},
%  \end{align*}  
% where the approximate equalities $\approx$ become equalities in the
% limit that  $d _{C
%  ^{\infty} } (N \circ f, (N \circ f)') \mapsto 0$. For the previous assertion to hold
%  for the second approximate equality,  we use that $f$ is non-singular on a
%  set of full measure, and then the assertion is completely elementary.
%  So we get  the required inequality.
% \end{proof}
% Finally let $\epsilon$ be the Lebesgue covering number of $\{U _{i} \}$ with
% respect to the metric $g$.
% Combining Lemma \ref{lemma:boundedmeancurvature} and Theorem
% \ref{thm:topping} we get that for $u$ non-constant as in the hypothesis if $\area
% (u) < \hbar$ than $\diam (u) < \epsilon$, for some $\hbar$ independent
% of $u$. Consequently the image of $u$ is contained in some $U _{i} $,
% and so $\phi _{i} ^{-1} \circ u$ is a $\phi _{i} ^{*} J
% $-holomorphic map of a sphere into the almost Kahler contractible manifold $(U
% _{i} , \omega _{0},  \phi _{i} ^{*} J )$ and so must be constant, which is a contradiction.
% \end{proof}
% =======
% % \begin{proof} [Proof of Theorem \ref{thm:quantization}]
% % For each $x \in M$ fix a diffeomorphism: 
% %  \begin{equation*}
% %  \phi _{x} : B _{r _{x} } (0) \subset \mathbb{R} ^{2n}   \to M,
% %  \end{equation*}  
% %    with $\phi _{x}  (0)=x$, s.t. $\phi _{x}  ^{*} \omega= f \omega _{st}  $, $f>0$, that is $\phi _{x} $ is conformal symplectic, where $B _{r _{x} } (0)$ is the standard metric ball in $\mathbb{R} ^{2n} $, based at 0 with radius $r _{x} $. We may clearly further assume that $\inf _{x} r_x >0 $. 
% %
% %    Now given a non-constant $J$-holomorphic map $u: C \to M$, $u$ goes through some $x \in M$, and $\phi_x ^{-1} \circ u$ determines a $\phi _{x} ^{*}J  $-holomorphic curve in $B _{r _{x} } (0) $ going through $0$ with boundary on $\partial B _{r _{x} } (0) $.
% % Since $\phi _{x} $ is conformal symplectic $\phi _{x} ^{*}J  $ is compatible with $\omega _{st} $, and so our surface 
% % let $r (x)$ be the radius of the metric ball in $\mathbb{R} ^{2n} $ centered at 0, with respect to the standard metric, such that there is a $B _{0} (r) $
% % bbSuppose that we have a non-constant $u$ as in the hypothesis, then $du$ is non-singular
% % outside a finite collection of points: $\Theta \subset \Sigma  $.
% %    \begin{lemma} \label{lemma:boundedmeancurvature}
% % For $u$ as above the
% % magnitude of the mean curvature $\textbf{H}$  of the image of
% % $u$ is bounded from above by a universal constant $C (\omega, J)>0$, outside
% % $\Theta$.
% %    \end{lemma}
% % \begin {proof}
% % We may fix a finite cover of $M$ by charts $\phi _{i}: U _{i} \subset
% % \mathbb{R} ^{2n} \to M   $, $\phi _{i}^{*} \omega = e^{f _{i}} \omega _{0} 
% % $ with $\omega _{0} $ symplectic, with $U _{i} $ contractible.
% %  Then 
% % $\phi _{i} ^{-1} \circ u  $ is a $\phi _{i} ^{*} J
% % $-holomorphic map defined on $u ^{-1} (\phi _{i} (U _{i}) ) $ into $\mathbb{R}^{2n} $, and $\phi _{i} ^{*} J$ is
% % compatible with $f _{i} \omega _{0} $ and hence with $\omega _{0} $.
% % Fix an (infinite) full measure  cover of $\Sigma - \Theta  $ by closed disk
% %    domains $\{V _{i_j} \}$,
% %  with each
% % $V _{i _{j} } \subset  u ^{-1} (\phi _{i} (U _{i}) ) $ for some $i$. Then $\phi
% % _{i} ^{-1} \circ   u |
% % _{V _{i _{j} } } $ is an immersed  $\phi _{i} ^{*} J$-holomorphic
% % curve in  $\mathbb{R} ^{2n} $. As this complex structure is compatible
% % with the symplectic form $\omega $ it classically follows  that the image $D
% % _{i _{j} } $ of  $\phi
% % _{i} ^{-1} \circ   u |
% % _{V _{i _{j} } } $ is a minimal surface and hence has mean curvature
% % 0 with respect to $(\omega _{0} , \phi _{i} ^{*} J )$. Since the cover
% % $\{U _{i} \}$ is finite,  the $C ^{\infty} $ norm of the
% % functions $f
% % _{i} $
% % is universally bounded: $A<|f _{i} | <B$, for some some $A,B$ and all $i$.
% %    It follows that the
% % magnitude of the mean curvature of the surface $D _{i _{j} } $ with respect
% % to the metric induced by $(e^{f _{i}}  \omega _{0} , \phi ^{*}_{i} J    )$ is
% %    bounded from above by some
% % universal constant $C (\omega, J)$. Consequently the same holds for
% % $\image u |_{V _{i _{j} } }  $ with respect to  $g _{J} $ from which the result
% % follows.
% % \end{proof}
% %
% % Next we shall need the following inequality relating diameter and mean
% % curvature deduced from a theorem of Topping
% % \cite{citeToppingRelatingDiameter}.
% % \begin{theorem} \label{thm:topping}   For a compact Riemannian manifold $(M,
% %    g)$, and $f: \Sigma \to M$ a closed surface in $M$, which is
% %    non-singular on a set of full-measure (with respect to an
% %    auxiliary metric $g'$ on $\Sigma$):
% %  \begin{equation*}
% %    \diam_{g} (\image f) \leq C (M, g)
% %    \int _{\Sigma} |\textbf{H}_{M,g}\circ f|   d vol _{g},
% %  \end{equation*}  
% % where 
% % $$\diam_{g} (\image f) := \max _{x,y \in \Sigma} dist_{\Sigma, g} 
% % (f(x), f(y)),$$ $dist_{\Sigma, g}$ is the induced metric from the
% % ambient space,  and $dvol _{g} $ is the
% % measure on $\Sigma$ induced by $g$. 
% % \end{theorem}
% % \begin{proof} The theorem of Topping is for closed immersed
% %    submanifolds
% %    of $R ^{n} $, and for surfaces $\Sigma \subset \mathbb{R} ^{n} $ says that:
% %  \begin{equation*}
% %    \diam (\Sigma) \leq C (n)
% %    \int _{\Sigma} |\textbf{H} _{R ^{n} } | d vol _{\Sigma},
% %  \end{equation*}  
% % where $\diam$  is  the intrinsic diameter: $\max _{x,y \in \Sigma} dist_{\Sigma, g} 
% % (x, y),$ and where $dvol _{\Sigma} $ is the measure induced from the
% % standard metric on $\mathbb{R} ^{n} $.
% %    To obtain the version stated for a more general but compact $
% %    (M,g)$, and non-immersed surfaces, pick an isometric Nash embedding $N$ of $ (M,g)$ into
% %    $\mathbb{R} ^{n} $, where $n$ is large enough. Take a small
% %    perturbation $(N \circ f)'$ of $N \circ f$ so that $(N \circ f)'$
% %    is an immersion (or even embedding) into $\mathbb{R} ^{n}
% %    $.  Since $\Sigma$ is
% %    closed 
% % we get by Topping's theorem:
% %  \begin{equation*}
% %     \diam (\image (N \circ f)') \leq C (n)
% %     \int _{\Sigma} |\textbf{H} _{\mathbb{R} ^{n}} \circ (N \circ f)'  | \, dvol _{\Sigma}.
% %  \end{equation*}  
% % Note that under hypotheses of the theorem the measures $dvol _{g} $ and
% % $dvol _{g'} $ on $\Sigma$ are equivalent, specifically because $f$ is non-singular
% % on a set of full $dvol _{g'} $ measure. Since  $M$ is compact the function $z \mapsto |\textbf{H} _{R ^{n}} 
% % (N \circ f(z))|  $ on
% % $\Sigma$ is bounded on a set of full $dvol _{g'} $ measure, and hence 
% % $dvol _{g} $ measure,  from above by the function $z \mapsto C' (N)
% % |\textbf{H}_{M, g} (f(z))|  $ on $\Sigma$ for some $C' (N) >>0$, independent
% % of $f$. This $C' (N)$ is just an upper bound for the function $\lambda
% %  $ on $M$, s.t. $\lambda (m) \cdot |\textbf{H} _{\mathbb{R} ^{n} }  (N
% %  (m))| = |\textbf{H}
% %  _{g}  (m)|$, $m \in M$. The function $\lambda$ exists and is continuous because  $|\textbf{H}|$ is clearly non-decreasing under isometric embeddings.
% %
% %
% % So we get:
% %  \begin{align*}
% %     & \diam _{g}  (\image f)= \diam (\image N f) \approx \diam (\image N f') \leq C (n)
% %    \int _{\Sigma} |\textbf{H} _{\mathbb{R} ^{n} } ((N \circ f)' ) | \, d vol _{\Sigma}
% %    \\ & \approx C (n)
% %    \int _{\Sigma} |\textbf{H} _{\mathbb{R} ^{n} } (N \circ f ) | \,
% %    d vol _{\Sigma} \leq C' (N) \cdot C (n) \int _{\Sigma} |\textbf{H} _{M,g} \circ f
% %     | \, d vol _{g},
% %  \end{align*}  
% % where the approximate equalities $\approx$ become equalities in the
% % limit that  $d _{C
% %  ^{\infty} } (N \circ f, (N \circ f)') \mapsto 0$. For the previous assertion to hold
% %  for the second approximate equality,  we use that $f$ is non-singular on a
% %  set of full measure, and then the assertion is completely elementary.
% %  So we get  the required inequality.
% % \end{proof}
% % Finally let $\epsilon$ be the Lebesgue covering number of $\{U _{i} \}$ with
% % respect to the metric $g$.
% % Combining Lemma \ref{lemma:boundedmeancurvature} and Theorem
% % \ref{thm:topping} we get that for $u$ non-constant as in the hypothesis if $\area
% % (u) < \hbar$ than $\diam (u) < \epsilon$, for some $\hbar$ independent
% % of $u$. Consequently the image of $u$ is contained in some $U _{i} $,
% % and so $\phi _{i} ^{-1} \circ u$ is a $\phi _{i} ^{*} J
% % $-holomorphic map of a sphere into the almost Kahler contractible manifold $(U
% % _{i} , \omega _{0},  \phi _{i} ^{*} J )$ and so must be constant, which is a contradiction.
% % \end{proof}
% >>>>>>> 828f84d22ae542cd2330702f4fd90fb588e69ecb
% A pair of smooth maps $u _{1}: \Sigma _{1} \to M, u _{2}: \Sigma _{2} \to M   $ are sa
% Let $\hbar (M,J)$ be as in Theorem \ref{thm:quantization}. \textcolor{blue}{something here} 
% Let $\widetilde{S}  _{g,n} (A) $ denote the set of homology class $A$  smooth maps: 
% \begin{equation*}
%    u: \Sigma \to M,
% \end{equation*} for $\Sigma$ a nodal closed genus $g$ Riemann surface, with marked points $z _{1}, \ldots, z _{n} 
%    $. A pair $(u _{1}, \Sigma _{1}  ), (u _{2}, \Sigma _{2}  ) $ are called isomorphic if there is a holomorphism $\phi: \Sigma _{1} \to \Sigma _{2}  $ mapping the marked points to the marked points, preserving order, such that 
% \begin{equation*}
% u _{1} = u _{2} \circ \phi.
% \end{equation*}
% $(u, \Sigma) \in \widetilde{S} _{g,n}  $ is called stable if its automorphism group is finite. \textcolor{blue}{check this} 
% Let $S _{g,n} (A) $ denote the quotient set of $\widetilde{S} _{g,n} (A)  $ identifying isomorphic pairs.

   %    \begin {proof} [Proof of Proposition \ref{prop:boundeddeformation}] 
% Let  $\{\omega _{t} \}$, $\{\widetilde{ \omega} _{t} \}$ be as in the definition of 
% $c$-deformation equivalent, with corresponding constant $C$ and $(\omega _{t}, J _{t}  )$ a compatible $\lcs$ family. 
% Then there is a universal in $t$ bound on the $\energy$ function on $\overline{\mathcal{M}} _{g}   (M, J _{t} , A)$, since for $u \in \overline{\mathcal{M}} _{g}   (M, J _{t} , A) $ we have 
% \begin{equation*}
% \int _{\Sigma} u^{*} \omega _{t} < C  \int _{\Sigma} u^{*} \omega = C
%    \langle [\omega], A \rangle,
% \end{equation*}
% for some $C$, where $\Sigma$ is the domain for a representing map for $u$. The first part of the proposition is then immediate.
% Moreover the above shows that $\{\omega _{t}, J _{t}  \}$ is free of sky catastrophes for every compatible $\{J _{t} \}$, and so the result follows.
% \end {proof}
%  
%    
%    
%    $J$ is $\omega$-compatible and let $\overline{\mathcal{M}}_{0} ( J,
%  A) $ be the moduli space of genus 0 class $A$ $J$-holomorphic curves in $M$.
%    Then $\energy$ is bounded on $\overline{\mathcal{M}}_{0} ( J,
%  A) $ and so it is compact.
\begin{proof} [Proof of Theorem \ref{thm:noSkycatastrophe}] 
We shall actually prove a stronger statement that there is a universal (for all $t$) energy bound from above for class $A$, $J _{t} $-holomorphic curves. 
\begin{lemma} \label{lemma:K}
Let $M,K$ be as in the statement of the theorem, and $A \in H _{2} (M) $ fixed.
Let $(\omega,J)$ be a compatible $\lcs$ pair on $M$ such that for every  $x \in \partial K$ there is a $\widetilde{J}$-holomorphic (real codimension 2) hyperplane $H _{x} \subset K \subset \widetilde{M}  $ through $x$, 
such that $\pi (H _{x}) \subset M$ \text{ is a closed submanifold and } such that $A \cdot [\pi (H _{x} )] \leq 0$.
Then any genus $0$, $J$-holomorphic class $A$ curve $u$ in $M$ has a lift $\widetilde{u} $ with image in $K$.
\end{lemma}
\begin{proof}
For $u$ as in the statement, let $\widetilde{u} $ be a lift intersecting the 
   fundamental domain $D$, (as in the statement of main theorem). Suppose that $\widetilde{u} $ intersects $\partial K$, otherwise we already have $\image \widetilde{u} \subset K ^{\circ} $, for $K ^{\circ} $ the interior, since $\image \widetilde{u}$ is connected (any by elementary topology).  Then $\widetilde{u} $ intersects $u _{x} $ as in the statement, for some $x$. So $u$ is a $J$-holomorphic map intersecting the closed hyperplane $\pi (H _{x} )$ with $A \cdot [\pi (H _{x} )] \leq 0$. By positivity of intersections, \cite{citeMcDuffSalamon$J$--holomorphiccurvesandsymplectictopology}, $\image u \subset \pi (H _{x})$, and so $\image \widetilde{u} \subset H _{x}  $. And so $\image \widetilde{u} \subset \partial K$.
%    Suppose that there is no such $K$. Then there is a sequence $\{(\omega _{k}, J _{k},  u _{k}) \}$, $u _{k} $ $J _{k} $-holomorphic in class $A$, $(\omega _{k},J _{k}  )$ compatible $\lcs$ pair, such that for any compact $K \subset \widetilde{M} $, there is a $k$ such that no lift of $u _{k} $ has image contained $K$. 
%
% By our hypothesis that  $PD(\alpha _{k} ) = \sum _{i} a _{i} [\Sigma _{i} ]  $, $0 =inc _{i} ^{*} \alpha \in H ^{1} _{DR} (\Sigma _{i} )   $ for $inc _{i}:  \Sigma _{i} \to M $ the inclusion maps. 
% Consequently, the maps $inc _{i} $ have a lift to $\widetilde{M} $, let $\widetilde{\Sigma} _{i}  $ denote the images in $\widetilde{M} $ for some fixed choice of lifts. 
%
% We may suppose that $$\forall k,i: \image u _{k} \cap \Sigma _{i} =\emptyset $$  as otherwise such a $u _{k} $ must map into a leaf of the $J _{k} $-holomorphic foliation of $\Sigma _{i} $ by the positivity of intersections.
% And so by the above $u _{k} $ have lifts contained the compact $K = (\cup _{i} \widetilde{\Sigma} _{i} ) \subset \widetilde{M} $.
%
%
%  We have that  $\pi ^{-1} ({M - \bigcup _{i} \Sigma _{i} })$ is a disjoint union $\sqcup _{j \in J} V _{j}  $ of subsets, so that $g _{\alpha} $ is bounded on each $V _{j} $, where $g _{\alpha} $ is as above. 
% For each $k$, let $\widetilde{u} _{k}  $ be any lift of ${u} _{k} $ intersecting $V _{j _{0}} $, for some $j _{0} \in J $ fixed.
% Note that $g _{\alpha} $ is proper, so that if for all $k$ 
% $\widetilde{u} _{k}  $ is contained in $V _{j _{0}} $ this contradicts our hypothesis.
% Then for some $k$, $\widetilde{u} _{k}  $ intersects  some $\Sigma _{i} $, a contradiction.
%
\end{proof}
  
    Suppose otherwise, then there is a sequence $\{u _{k} \} _{k=1} ^{\infty}$, $u _{k}: S ^{2}  \to M$, of $J _{t _{k} } $-holomorphic class $A$ curves, with $$  \int _{S ^{2} } u _{k} ^{*} \omega _{t _{k}} \to \infty, \text{ as $k \to \infty$}.   $$
    We may assume that $t _{k} $ is convergent to $t'  \in [0,1] $, otherwise take a convergent subsequence.  
%    Let $\{ \widetilde{u}_{k}  \}$ be a lift of the curves to the universal covering space $\widetilde{M} \xrightarrow{\pi} M $. 
% 
   % Then the Lee forms $\alpha _{t} $ of $\omega _{t} $ determine a functions $g _{t} $ on $\widetilde{M} $ given by $g _{t} ([p]) = \int _{[0,1]} p ^{*} \alpha _{t} $, for $p$ any representative for $[p]$.


Now, by the lemma above each $u _{t} $ has a lift $\widetilde{u}_{t}$ contained in a compact $K \subset \widetilde{M} $. Then for every $\epsilon>0$ there is a $N$ so that for $k>N$ we have: 
   \begin{equation*}
      \int _{ S ^{2}} \widetilde{u} _{k} ^{*} \omega _{t _{k}}    \leq 
 C _{k} \langle \widetilde{\omega } _{t _{k}}  ^{symp}  , A  \rangle
 \end{equation*}
where $\widetilde{\omega} _{t _{k} } = f _{k}  \widetilde{\omega}  ^{symp} _{k} $, for $\widetilde{\omega} ^{symp} _{k} $ symplectic on $\widetilde{M}$, and  
$f _{k}: \widetilde{M} \to \mathbb{R}$ positive functions constructed as in the proof of Lemma \ref{lemma:lcsconvergence}, and where $C _{k} = \max _{K} f _{k}   $. 
Then 
\begin{equation*}
\lim _{k \to \infty} \int _{ S ^{2}} \widetilde{u} _{k} ^{*} \omega _{t _{k}} \leq C  \langle \widetilde{\omega } _{t'}  ^{symp}, A  \rangle,
\end{equation*}
where $\widetilde{\omega} _{t'}  = f _{t'}  \widetilde{\omega}  ^{symp} _{t'}  $, for  $\widetilde{\omega}  ^{symp} _{t'}  $ symplectic, and $C=\max _{K} f _{t'}  $.
So we have obtained a contradiction.

   % Since $u _{k'} $ cannot intersect $\Sigma$ by positivity of intersections, we may perturb $J _{t _{k'} } $ to $J'$ so that $u _{k}' $ is still $J$'-holomorphic and so that $\Sigma$ is a regular $J'$-holomorphic curve. A bit more detail on this: perturbation of $J _{t _{k'} } $ can be taken to vanish outside a small open ball incident to $\Sigma$ and not incident to $u _{k'} $, and it is just a standard argument cf. \cite{citeMcDuffSalamon$J$--holomorphiccurvesandsymplectictopology}.
   %
   % This perturbation can only in general make $\Sigma$ regular, since as the reader is likely aware we may not be able to regularize the entire moduli space by a perturbation. But then by positivity of intersections class $[\Sigma]$ $J$'-holomorphic curves will foliate a  neighborhood $N$ of $\Sigma$. Taking $r$ so that $D _{r} (\Sigma) $ is contained in $N$, by the above we see that $u' _{k} $ intersects a class $[\Sigma]$ $J$'-holomorphic curve and does not coincide with it by construction. But this contradicts positivity of intersections. 
\end{proof}
\begin{proof} [Proof of Theorem \ref{cor:nonsqueezing}] \label{section:proofnonsqueezing}
% Let us prove a stronger claim for more generality.
% \begin{theorem}  
% Let $(M, \omega)$ be a compact symplectic 4-manifold, with $GW _{0,1}  (\omega, A) ([pt]) \neq 0$ for some class $A$, s.t. $ \langle [\omega], A  \rangle = \pi r ^{2}  $. 
%    Suppose that the Poincare dual of $\alpha$ is represented by a cycle $\sum _{i} a _{i} [\Sigma _{i}]   $ where $\Sigma _{i} $ are hypersurfaces, which admit symplectic folliations by class $B$ curves with $B \cdot A \leq 0$.
%    Let $R>r$, then there is an $\epsilon>0$ with the following property.
%     Given  an $\lcs$ $\omega _{1}$ on $M$ 
% % s.t. there is a class $A$ embedded symplectic surface $\Sigma$ in $(M, \omega _{1} )$, with $\Sigma$ also symplectic with respect to $\omega$ and such that the intersection number $\alpha ^{*} \cdot S (\Sigma) \neq 0$.
% $C ^{0} $ $\epsilon$-close to $\omega _{0} $, 
% there is no $\lcs$ embedding $$\phi: (B _{R}, \omega _{st})  \hookrightarrow (M, \omega _{1}), $$  s.t $\phi _{*} j  $ preserves $T ^{fol} \Sigma _{i} $ for each $i$.
%    Here $j $ is the standard integrable complex structure on $B _{R} $.
% \end{theorem}
% \begin{proof}
 Fix an $\epsilon' >0$ s.t. any 2-form $\omega _{1}  $ on $M$, $C ^{0} $ $\epsilon'$-close to $\omega$ is non-degenerate and is non-degenerate on the leaves of the folliation of each $\Sigma _{i} $, discussed prior to the formulation of the theorem. Suppose by contradiction that for every $\epsilon>0$ there is a homotopy $\{\omega _{t} \}$ of $\lcs$ forms, with $\omega _{0}=\omega$, such that $\forall t: d(\omega_{t}, \omega) < \epsilon$ and such that 
there exists a symplectic embedding $$\phi: B _{R}  \hookrightarrow (M, \omega _{1}), $$  satisfying conditions of the statement of the theorem.
Take $\epsilon<\epsilon'$, and let $\{\omega _{t} \}$ be as in the hypothesis above. In particular $\omega _{t} $ is an $\lcs$ form for each $t$, and is non-degenerate on $\Sigma _{i} $.
    Extend $\phi _{*}j $ to an $\omega _{1} $-compatible almost complex structure $J _{1} $ on $M$, preserving $T ^{fol}  \Sigma _{i} $.
   We may then extend this to a family $\{J _{t} \}  $ of almost complex structures on $M$, s.t. $J _{t} $ is $\omega _{t}  $-compatible for each $t$, with $J _{0} $ is the standard split complex structure on $M$ and such that $J _{t} $ preserves $T\Sigma _{i} $ for each $i$. The latter condition can be satisfied since $\Sigma _{i} $ are $\omega _{t} $-symplectic for each $t$. (For construction of $\{J _{t} \}$ use for example the map $ret$ from Lemma \ref{lemma:Ret}). When the image of $\phi$ does not intersect $\cup _{i} \Sigma _{i} $ these conditions can be trivially satisfied. 

 Then the family $\{(\omega _{t}, J _{t}  )\}$ satisfies the hypothesis of Theorem \ref{thm:noSkycatastrophe}, and so has no sky catastrophes in class $A$. In addition if $N = \overline{\mathcal{M}} _{0,1}   (J _{0}, A)$ (which is compact since $J _{0} $ is tamed by the symplectic form $\omega$)  then
\begin{equation*}
GW _{0,1} (N,A,J _{0} ) ([pt] \otimes [pt]) =1.
\end{equation*}
Consequently by Lemma \ref{prop:invariance1} there is a class $A$ $J _{1}$-holomorphic curve $u$ passing through $\phi ({0}) $.  

   By Lemma \ref{lemma:K} we may choose a lift $\widetilde{u} $ to $\widetilde{M} $, with homology class $[\widetilde{u} ]$ also denoted by $A$, of each $u$ so that the image of $\widetilde{u} $ is contained in a compact set $K \subset \widetilde{M} $, (independent of choice of $\epsilon, \{J _{t} \}$). 
Let $\widetilde{\omega} ^{symp} _{t} $ and $f _{t} $ be as in Lemma \ref{lemma:lcsconvergence}, 
then by this lemma for every $\delta > 0$ we may find an $\epsilon>0$
so that if $d(\omega _{1}, \omega)< \epsilon$ then $d _{C ^{0}} (\widetilde{\omega} ^{symp},  \widetilde{\omega} _{1} ^{symp}) <\delta$ on $K$.


 Since $  \langle \widetilde{\omega} ^{symp}, A  \rangle =\pi r ^{2}  $, if $\delta$ above is chosen to be sufficiently small then $$|\int _{S ^{2}} u ^{*} \omega _{1} - \pi r ^{2}    | \leq  |\max _{K} f _{1} \langle \widetilde{\omega} _{1}  ^{symp}, A  \rangle - \pi \cdot r ^{2}  | < \pi R ^{2} - \pi r^{2},   $$   
   since $$   \lim _{\epsilon \to 0} |\langle \widetilde{\omega} _{1}  ^{symp}, A  \rangle - \pi \cdot r ^{2}| = |\langle \widetilde{\omega}   ^{symp}, A  \rangle - \pi \cdot r ^{2}| = 0,$$ and since 
   $$
d _{0}  (\omega _{1}, \omega ) \to 0 \implies \max _{K} f _{1} \to 1 .$$
In particular we get that $\omega _{1} $-area of $u$ is less then $\pi R ^{2} $.



   % we also have  a universal bound say $C _{1} $ on the area of $\widetilde{u} $ with respect to ${\widetilde{g} } $, since any two metrics on $K$ are comparable. And so we have a universal bound $C _{1} $ on the area of $u$ with respect to $g$.
%
%     
% Let $D$ denote the domain of $u$,  then we have:
% \begin{equation*}
%    |\int _{D} u ^{*} \omega _{0} - {\omega} _{1}|   \leq \int _{D} |\star ( u ^{*} \omega _{0} -u ^{*} {{\omega}} _{1})| dVol _{g} \leq \epsilon \cdot C _{1},
% \end{equation*}
% where $\star$ is the Hodge star with respect to $g$.
% Thus fixing $\epsilon$ so that $\epsilon \cdot C _{1} < \pi  R ^{2} - \pi r ^{2}  $, we get that: 
% \begin{equation*}
% \int _{D} u ^{*} \omega _{1} < \pi r^{2}.
% \end{equation*}
%

  We may then proceed as in the now classical proof of Gromov~\cite{citeGromovPseudoholomorphiccurvesinsymplecticmanifolds.} of the non-squeezing theorem to get a contradiction and finish the proof.  More specifically $\phi ^{-1} ({\image \phi \cap \image u})  $ is a  minimal surface in $B _{R}  $, with boundary on the boundary of $B _{R} $, and passing through $0 \in B _{R} $. By construction it has area strictly less then $\pi R ^{2} $ which is impossible by the classical monotonicity theorem of differential geometry.
\end{proof} 
% \begin{proof} [Proof of Theorem \ref{thm:alternative}]
% Fix an $\epsilon >0$ s.t. for any $\lcs$ $\omega _{1}  $ on $M$, $C ^{0} $ $\epsilon$-close to $\omega$, the convex linear combination $\{\omega _{t} \}$ of $\omega$ and $\omega _{1} $ is non-degenerate and hence conformally symplectic, and s.t. for any  $\omega _{1} $ $\epsilon$-close to $\omega=\omega _{0} $, the foliation of $\Sigma _{i} $ stays symplectic.
%
% Suppose by contradiction that for every $\epsilon>0$ there exists an $\lcs$ embedding
% $$\phi: B _{R}  \hookrightarrow (M, \omega _{1}).
% $$ Extend $\phi _{*}j $ to any  $\omega _{1} $-compatible almost complex structure $J _{1} $ on $M$.
%    We may then extend this to a family $\{{J} _{t} \}  $ of almost complex structures $M$, s.t. $J _{t} $ is $\omega _{t}  $ compatible for each $t$.
%    If $\{(\omega _{t}, J _{t}) \}$ has no sky catastrophes in class $A$, then by Lemma \ref{prop:invariance1} there is a class $A$ $J _{1}$-holomorphic curve $u$ passing through $\phi ({0}) $. 
% Now proceed as in the proof of Theorem \ref{cor:nonsqueezing}.
% %  By the proof of Theorem \ref{thm:noSkycatastrophe} we may choose a lift to $\widetilde{M} $ for each such curve $u$ so that it is contained in a compact set $K \subset \widetilde{M} $, (independent of $\epsilon$ and all other choices). Now by definition of our $C ^{0} $-norm for every $\delta$ we may find an $\epsilon$
% % so that if $\omega _{1} $  is $\epsilon$-close to $\omega$ then $\widetilde{\omega} ^{symp}  $ is $\delta$-close to $\widetilde{\omega} _{1} ^{symp}   $ on $K$.
% %  Since $ \langle \widetilde{\omega} ^{symp}, [\widetilde{u} ]  \rangle =\pi r ^{2}  $, if $\delta$ above is chosen to be sufficiently small then $$| \max _{K} f _{1} \langle \widetilde{\omega} _{1}  ^{symp}, [\widetilde{u} ]  \rangle - \pi \cdot r ^{2}  | < \pi R ^{2} - \pi r^{2},   $$   since $$|\langle \widetilde{\omega} _{1}  ^{symp}, [\widetilde{u} ]  \rangle - \pi \cdot r ^{2}| \simeq |\langle \widetilde{\omega}   ^{symp}, [\widetilde{u} ]  \rangle - \pi \cdot r ^{2}| = 0,$$ for $\delta$ small enough, and $\max _{K} f _{1} \simeq 1$ for $\delta$ small enough, where $\simeq$ denotes approximate equality.
% % In particular we get that $\omega _{1} $-area of $u$ is less then $\pi R ^{2} $.
% % we also have  a universal bound say $C _{1} $ on the area of $\widetilde{u} $ with respect to ${\widetilde{g} } $, since any two metrics on $K$ are comparable. And so we have a universal bound $C _{1} $ on the area of $u$ with respect to $g$.
% %
% %     
% % Let $D$ denote the domain of $u$,  then we have:
% % \begin{equation*}
% %    |\int _{D} u ^{*} \omega _{0} - {\omega} _{1}|   \leq \int _{D} |\star ( u ^{*} \omega _{0} -u ^{*} {{\omega}} _{1})| dVol _{g} \leq \epsilon \cdot C _{1},
% % \end{equation*}
% % where $\star$ is the Hodge star with respect to $g$.
% % Thus fixing $\epsilon$ so that $\epsilon \cdot C _{1} < \pi  R ^{2} - \pi r ^{2}  $, we get that: 
% % \begin{equation*}
% % \int _{D} u ^{*} \omega _{1} < \pi r^{2}.
% % \end{equation*}
% %
% %   We may then proceed as in the classical proof Gromov~\cite{citeGromovPseudoholomorphiccurvesinsymplecticmanifolds.} of the non-squeezing theorem to get a contradiction and finish the proof.  More specifically $\phi ^{-1} ({\image \phi \cap \image u})  $ is a (nodal) minimal surface in $B _{R}  $, with boundary on the boundary of $B _{R} $, and passing through $0 \in B _{R} $. By construction it has area strictly less then $\pi R ^{2} $ which is impossible by a classical result of differential geometry, (the monotonicity theorem.)
% % \
% \end{proof}
% \begin{proof} [Proof of Theorem \ref{cor:nonsqueezing2}]
%    Let $\{\omega _{t}\} $ be as in the proof of Theorem \ref{cor:nonsqueezing}.
%    Let $\{J _{t}\} $  be any $\{\omega _{t} \}$ compatible family, for which $\Sigma$ is $J _{t} $-complex for each $t$. Proceed as before to obtain a $J _{1} $-holomorphic curve $u$ passing through $\phi (0)$, with $\omega _{1} $-area less then $\Zeta (R)$, pulling the curve back to $B _{R} $ we obtain a contradiction. 
% \end{proof}
% \begin{proof} [Proof of Thereom \ref{cor:nonsqueezing}] 
% % The $C ^{0} $ norm on $2$-forms on $M$, is defined with respect to a fixed
% % Riemannian metric $g$ on $M$, and is given by
% % \begin{equation*}
% % ||\omega|| = sup _{z; v,w \in T _{z} M} |\omega (v,w)|,
% % \end{equation*}
% % for $v,w$ a $g$-orthonormal pair in $T _{z} M $.
% % Let $\{\omega _{t} \}$ be as in the statement. 
% % Since $GW _{0,1} (A, \omega _{0}) ([pt]) $ ob
% % Then by Proposition \ref{prop:boundeddeformation}   
% %    
% %    And let $\{J _{t}
% % \}$ be a compatible family of almost complex structures.
% % Suppose by contradiction that 
% % $$\phi: B ^{2n} (R) \hookrightarrow (M, \omega_1), 
% % $$ be an $\lcs$ embedding. 
% %    
% Since $GW _{0,1}  (A, \omega _{0} ) ([pt]) \neq 0$, 
% by Proposition \ref{prop:boundeddeformation}, $GW _{0,1}  (A, \omega _{1} ) ([pt]) \neq 0$.
% Then the result follows by Theorem \ref{thm:nonsqueezing2} applied to $(M, \omega _{1} )$.
% Note that we automatically satisfy the hypothesis of Lemma 
% \ref{lemma:NearbyEnergy} with $$M _{0} =\overline{\mathcal{M}} (J,A, p), $$
% the moduli space of class $A$ $J$-holomorphic curves passing through $p$ and with %    $$E
%    ^{0} _{0} = E ^{1} _{0} =  \pi \cdot r ^{2}.
%    $$ If we define our metric on the space of $2$-forms on $\mathbb{R} ^{2n} $ using the standard metric on
%    $\mathbb{R} ^{2n} $, then fix
%    \begin{equation} \label{eq:epsilon}
%    0 < \epsilon <
%  \frac{1}{\pi \cdot r ^{2}   } | \cdot \pi
%    \cdot R ^{2} - \pi \cdot r^{2}  |, 
%    \end{equation}
%     and let $\delta$ correspond to this
%    $\epsilon$ as given by the 
%    Lemma \ref{lemma:NearbyEnergy} with respect to
%    $(\omega,J)$. Suppose then that we have a homotopy $\{\omega _{t} \} $, with
%    each $\omega _{t} $ $C ^{0} $ $\delta$-close to $\omega$
%    we may choose $J'$ $\omega _{1} $-compatible and $C ^{0} $ $\delta$-close to
%    $J$. Consequently by 
%    Lemma \ref{lemma:NearbyEnergy} the cobordism moduli
%    space $\overline{\mathcal{M}} (\{J _{t} \},A) $ is compact. 
% Moreover, since by assumption $\omega _{1} $ 
% is $C$-comparable with
%  $\omega$, for $C < \frac{R}{r}$ we get that
%    $$ \int _{\Sigma} u ^{*} \omega _{1}  < \frac{R}{r} \int _{\Sigma} u ^{*}
%    \omega _{0} < \pi \cdot R ^{2},
%    $$  as $$\int _{\Sigma} u ^{*} \omega _{0} = \pi \cdot r ^{2}.
%    $$ We may then proceed as in the classical proof Gromov~\cite{cite} of Gromov's non-squeezing to get a contradiction and finish the proof.  \qed
   %
   %
   % and so a $j$-holomorphic surface and hence minimal surface, passing through $0 \in B ^{2n} (0) $ and with boundary on
   % the boundary of the ball. By the 
 


% \begin{proof} [Proof of Corollary \ref{corollary:epsilon}]
%   Under the hypothesis of the corollary $$GW _{0,1}  (\omega, A) ([pt]) \neq 0,
%    $$ c.f. \cite{citeMcDuffSalamon$J$--holomorphiccurvesandsymplectictopology}.
% The $C ^{0} $ metric on the space of $2$-forms on $M$, is defined with respect to an auxilliary metric $g$ on $M$, and is given by:
% \begin{equation*}
% d (\omega _{0}, \omega _{1}  ) = \sup _{m; v,w \in T _{m}M } |\omega _{0} (v,w) - \omega _{1} (v,w)|, 
% \end{equation*}
% where $v,w$ is an orthonormal pair.
% \end{proof}
% Clearly there exists an $\epsilon > 0$ s.t. if $\omega _{1} $ is $\epsilon$-close to $\omega$ then $\omega _{1} $ is $C$-comparable to $\omega$ for some $C$.
\section {Genus 1 curves in the $\lcsm$ $C \times S ^{1} $ and the Fuller index} \label{sectionFuller}  
\begin{proof}[Proof of Proposition \ref{prop:abstractmomentmap}] %  $T$ acts on the moduli space by post-composition of class representative
%  stable maps
%  with the  $J ^{\lambda} $ preserving
%  action of $T$ on $C \times S ^{1} $. 
% Let $u \in [u] \in \overline{\mathcal{M}}_{1}  (
%    {J} ^{\lambda},
% A )$, with domain of $u$ non-nodal, that 
% The projection map $M=C \times S ^{1} \to S ^{1}  $ is $J$
   % convex consequently there are no $J$-holomorphic spheres in $M$ Note first that
% \begin{lemma} \label{lemma:fixed}
% The entire moduli space $\overline{\mathcal{M}}_{1}  (
%     \{{J} ^{\lambda} _{t}\} ,
%    A),$ is fixed by
%    the $T$ action and consists of Reeb tori.
% \end{lemma}
Suppose we a have a curve without spherical nodal components  $u \in \overline{\mathcal{M}}_{1,1} ^{1,0}  ({J} ^{\lambda}, A), $ represented by $u: \Sigma \to M=C \times S ^{1} $. 
Since by Lemma \ref{lemma:calibrated}, $u_*(T\Sigma) \subset \mathcal{V} _{\lambda} $, we get that
$$(pr _{C } \circ u )_* (T\Sigma) \subset \ker d\lambda \subset TC,$$ where $pr _{C}: C \times S ^{1} \to C  $ is the projection. Note that this implies in particular that $\Sigma$ is non-nodal.
     
% Next observe that when the rank of $(pr _{C} \circ u )_{*} $ is $1$, its image is in the Reeb line sub-bundle of $TC$, for otherwise the image
% has a contact component, but this is $J ^{\lambda} $ invariant and so
%    again we get that $\int _{\Sigma } (pr _{C} \circ u) ^{*}
% d\lambda > 0$. We now show that the image of $pr _{C} \circ u $ is in
% fact the image of some Reeb orbit. 
% Let $\rho, \gamma \in H _{1} (\Sigma, \mathbb{Z})  $ be the integral generators satisfying
% \begin{align*}
% & \langle \rho, u _{*} \alpha \rangle =1 \\
% & \langle \gamma, u _{*} \alpha \rangle =0.
% \end{align*}
% Identify the domain $\Sigma$ of $u$, via an isomorphism as defined in Section \ref{sec:gromov_witten_theory_of_the_lcs_c_times_s_1_}, with a marked Riemann surface $(T
%  ^{2}, j) $, $T ^{2} $  the standard torus. \textcolor{blue}{fix} 
% We use coordinates $(s, t)$ on $T ^{2} $
% $s, t \in S ^{1} $. 
   By charge (1,0) condition  $pr _{S ^{1} } \circ u $ is surjective and so by the Sard theorem we have a regular value $\theta _{0} \in S ^{1}   $, so that 
$ u ^{-1} \circ pr _{S ^{1} } ^{-1}  (\theta  _{0}) $ contains  an
embedded circle  $S _{0} \subset \Sigma $, where $pr _{S ^{1} }: C \times S ^{1} \to S ^{1}   $ is the projection.
Now $d (pr _{S ^{1} } \circ u )$ is surjective along 
$T (T ^{2} )| _{S _{0} } $, which means, since $u$ is $J ^{\lambda}
$-holomorphic, that $pr _{C} \circ u| _{S _{0}   } $ has non-vanishing differential.
From this and the discussion above it follows that image of $pr _{C }
\circ u $ is the image of some Reeb orbit. Consequently, by assumption that $u$ has charge $(1,0)$, $u$ is isomorphic to a Reeb torus for a uniquely determined Reeb orbit $o _{u} $.
% ,
% which is the almost complex sub-manifold $R _{{u} } \subset  C \times S ^{1} $ with
% points $(o _{u} (\theta _{1} ), \theta _{2}  )$, $\theta _{1},
% \theta _{2} \in S ^{1}   $. 
%
% By assumptions $u: (T ^{2}, j) \to R _{\gamma
% _{u} }  $ must be degree 1, but a degree 1 holomorphic map between
% elliptic curves is 
%  a bi-holomorphism by classical theory. 
%  It clearly follows that for $\theta \in T$,
% $[\theta \cdot u] = [u]$.
% given by $T ^{2} \to C \times S ^{1}  $, $(\theta _{1}, \theta _{2}  )
% \mapsto (\gamma _{u} (\theta _{1} ),  ) $


% then the curve
% $ pr \circ u \circ i _{\theta ^{2} _{0}  } $ is independent of
% $\theta ^{2} _{0}  $. To see this note that $pr_* \circ u_*$ push-forward of the
% vector field $\frac{\partial}{\partial \theta ^{2} }$ along each curve
% $i _{\theta ^{2} _{0}  }  $, must vanish for each
% $\theta ^{2} _{0}$, by the above discussion. So that there
% can be no change in $pr \circ u \circ i _{\theta _{2} _{0}  } $ in
% $\theta _{0} ^{2}  $.

% Consequently $(pr \circ u \circ i _{\theta ^{2} _{0}
% })_* (\frac{\partial}{\partial \theta}| _{\theta} ) =c (\theta) R$,
% with $c (\theta)$ identically 1, or identically $0$.
% constant which contradicts our assumptions. 
The statement of the lemma
follows when $u$ has no spherical nodal components.   On the
other hand non-constant $J ^{\lambda} $-holomorphic spheres are impossible, which can be seen as follows. Any such a $J ^{\lambda} $-holomorphic sphere $u$ lifts to the covering space $\widetilde{M} = C \times \mathbb{R}$ of $M$, as a $\widetilde{J} $-holomorphic map $\widetilde{u} $, where $\widetilde{J} $ is the lift of $J ^{\lambda} $, and is compatible with the lift $\widetilde{\omega}$ of $\omega = d ^{\alpha} \lambda $. On the other had $\widetilde{\omega} = d\lambda - dt \wedge \lambda $ is conformally symplectomorphic to the exact symplectic form 
$d (e ^{t} \lambda)$, for $t: C \times \mathbb{R} \to \mathbb{R}$ the projection.  So that $\widetilde{u} $ is constant by Stokes theorem.

% So there are no nodal elements in $\overline{\mathcal{M}} ^{0} _{1,1}  (
% {J} ^{\lambda},A) $ which completes the argument.
\end{proof}
\begin{proposition} \label{prop:regular} Let $(C, \xi)$ be a general
    contact manifold. If
   $\lambda$ is a non-degenerate contact 1-form for $\xi$
then all the elements of $\overline{\mathcal{M}}_{1,1} ^{1,0}   ( J ^{\lambda} 
, {A} )$ are regular curves. Moreover, if $\lambda$ is degenerate then
for a period $c$ Reeb orbit $o$ the kernel of the associated real linear Cauchy-Riemann operator for the Reeb torus $u _{o} $ is naturally identified with the 1-eigenspace of $\phi _{c,*} ^{\lambda}  $ - 
the time $c$ linearized return map $\xi (o (0)) \to \xi (o (0)) $
induced by the $R^{\lambda}$ Reeb flow.
% and moreover represents the virtual loose equivariant $T$-bordism
% class of $\overline{\mathcal{M}}_{1}  (T ^{2}, J,
%    A),$.  
   % virtual regularization
   % $\overline{\mathcal{M}}_{1}  (T ^{2}, J ^{virt} ,
   % A),$ of $\overline{\mathcal{M}}_{1}  (T ^{2}, J,
   % A),$ and  $\overline{\mathcal{M}}_{1}  (T ^{2}, \{J ^{virt}  _{t} \},
   % A),$ 
\end{proposition}
\begin{proof}  
% We shall give two proofs as they are of independent interest. The
% first only applies in dimension 3, but has the advantage of being very
%  explicit.
% %  and has the advantage of being sometimes generalizable to the case of
% % $4d$ $\lcs$ with $S ^{1} $ action. (Although we don't consider such
% % generalizations yet.) 
% The second is for general contact manifolds
% ($\lcs$  of the form $C \times S ^{1} $). % Although it is not formally necessary let us proof this first in the case
% % $C$ is dimension 3, as then the argument is very nicely explict.
% We have previously  shown that all  $[u,j] \in \overline{\mathcal{M}}_{1}  (J ^{\lambda} 
% , {A} ),$ are represented by smooth immersed curves, (covering maps of Reeb tori)   % Fixing an appropriate class in $\pi _{1} (C)
%    % $, for the Reeb orbits, the resulting holomorphic tori will be in class
%    % $A$. 
% and these are fixed
%   by the $T$ action. 
% Since each $u$ is immersed we may naturally get a $T$-invariant splitting $u ^{*}T (C \times S ^{1} )
% \simeq N _{u} \times T (T ^{2})   $,
% using $g _{J} $ metric, where $N _{u} $ denotes the pull-back normal
% bundle.
% The full associated real linear Cauchy-Riemann operator takes the
% form:
% \begin{equation} \label{eq:fullD}
%    D ^{J}_{u}: \Omega ^{0} (N _{u} \oplus T (T ^{2})  ) \oplus T _{j} M _{1}   \to \Omega ^{1,0}
%    (T(T ^{2}), N _{u} \oplus T (T ^{2}) ). 
% \end{equation}
% This is an index 2 Fredholm operator (after standard Sobolev
% completions), whose restriction to $\Omega
% ^{0} (N _{u} \oplus T (T ^{2})  )$ preserves the splitting, that is the
% restricted operator splits as 
% \begin{equation*}
% D \oplus D':   \Omega ^{0} (N _{u}) \oplus \Omega ^{0} (T (T ^{2})  )    \to \Omega ^{1,0}
% (T (T ^{2}), N _{u}) \oplus \Omega ^{1,0}(T (T ^{2}), T (T ^{2}) ).
% \end{equation*}
% On the other hand the restricted Fredholm index 2 operator 
% \begin{equation*}
% \Omega ^{0} (T (T ^{2})) \oplus T _{j} M _{1}  \to \Omega ^{1,0}(T (T ^{2}) ),
% \end{equation*}
% is surjective by classical algebraic geometry.
% It follows that $D ^{J}_{u}  $ will be surjective
% if  
% the restricted Fredholm index 0 operator
% \begin{equation*}
% D: \Omega ^{0} (N _{u} ) \to \Omega ^{1,0}
% (N _{u} ),
% \end{equation*}
% has no kernel.
%
%
%
% Note that $N _{u} $ is
% $T$-equivariantly trivial by geometry and moreover the
% $T$-action
% preserves $D$ 
% also by geometry.
% % We need to check that $D$ is surjective.
% For the following we need that dimension of $C$ is 3. A pair of elements $\xi _{1}, \xi _{2} \in \ker D    $
% either coincide or are disjoint. For if they intersect $\xi _{1} - \xi _{2}
% \in \ker D    $ vanishes at some $z \in {T} ^{2} $, and by 
%  Aronsajn's unique continuation theorem  it follows that any such zero
%  must contribute positively to the self intersection number of the
%  zero section, unless $\xi _{1} -\xi _{2}  $ vanishes identically c.f.
%  for instance Taubes \cite[Section
%  5]{citeTaubesCountingPseudoHolomorphic}. 
% In other words this is the "positivity of intersections" argument.
%  It follows from
%  this that the linear span of elements of $\ker D  $, determines a sub-bundle $Ker
%  \subset N _{u} $ of dimension $d:=\dim \ker D  $. Since $Ker$ is
%  spanned by $\ker D   $ which is $T$-invariant $Ker$ is
%  $T$-invariant. Trivializing $Ker$ so that its constant sections
%  correspond to elements of $\ker D $, we see that the induced action of
%  $T$ on $\mathbb{R} ^{d} \times T ^{2}  $ must be  of the form 
% \begin{equation} \label{eq:action}
% \theta \cdot (v,z) = (a (\theta) v, \theta
%  \cdot T),
% \end{equation} 
%  where $a (\theta) \in End (\mathbb{R} ^{d})$, for all $\theta$, as by construction the induced
%  action on $\mathbb{R} ^{d} \times T ^{2}  $ must  take
%  constant sections to constant sections. We claim that $a (\theta)=id$ for all
%  $\theta$. To see this note that by geometry there is clearly a  trivialization of $N _{u} $
%  with respect to which $T$ acts as in \eqref{eq:action} with $a
%  (\theta) =id$ for all $\theta$, and likewise for the sub-bundle $Ker$ as it is $T$-invariant.
%  So  if $a (\theta) \neq id$, for all $ \theta$, there is a
%  $T$-equivariant endomorphism of $\mathbb{R} ^{d} \times T ^{2}  $
%  with respect to a pair of $T$-actions in the form of
%  \eqref{eq:action} one of which is trivial in the $\mathbb{R}
%  ^{d} $ variable, which implies the same for the other.
% So we obtain in the contact 3-fold case:
% % We may reduce the general case to the complex linear case by trick of
% % Hofer-Lizan-Sikorav \cite{cite}, or see for example
% % \cite[Proof of C.1.10
% % (iii)]{citeMcDuffSalamon$J$--holomorphiccurvesandsymplectictopology}.
% % Let us describe this. Denote by $D$ the complex linear part of $D ^{J}
% % $, so that 
% % More
% % specifically one constructs from the real linear CR operator $D ^{J} $
% % a smooth complex linear $CR$ operator ${D'} $ with identical kernel. This
% % construction is natural and $T$ with still act on $(N _{u}, D')$.  
% %  And so we obtain:
% \begin{lemma} \label{lemma:invariantmu}
% Any
% element $\mu$ of $\ker D$, $D: \Omega ^{0} (N _{u} ) \to \Omega ^{1,0}
% (T ^{2}, N _{u} )$ must be $T$-invariant. 
% \end{lemma}
% Suppose that $\ker D \neq 0$,
% we use this to obtain an eigenvector of the time-1 linearized return
% map of the Reeb flow at the orbit $o _{u} $,
% % , $\gamma \mapsto \int _{S ^{1} } \gamma ^{*}
% % \lambda$, 
% which will give a contradiction. As $\mu$ as $T$-invariant and as  the
% $g _{J} $ exponential map from $N _{u} $ into $C \times S ^{1} $ is
% $T$-invariant, we obtain a smooth $T$-invariant embedded submanifold
% $$T ^{2} \times [0,\epsilon) \to C
% \times S ^{1}, $$ containing $\image u$, defined by $ (z,  t)
% \mapsto (exp (t \mu (z )))$.  And we have a vector field $\kappa$ on
% this submanifold $S _{u} $, given by the pushforward of the rotational
% vector field on $T ^{2} \times [0, \epsilon] $ which in coordinates is
% $ (\frac{\partial}{\partial \theta _{1} },0, 0)$, for coordinates $(\theta _{1}, \theta
% _{2}, t)$.
%  The flow $\phi ^{\kappa} (\tau) $ of $\kappa$ induces
% the
% linearized flow $\phi_* ^{\kappa} (\tau)$ on the sub-bundle $\mathbb{R}
% \cdot \mu \subset N _{u} $, which at $\mu (z)$ is just $ \mu _{*}
% (\frac{\partial
% }{\partial \theta _{1} }(z) ) $. On the other hand as $\mu$ is in $\ker D
%   $, it follows by definition of $D $, and
% $T$-invariance of $S _{u} $ that $\kappa$
% $C ^{1} $ converges, as we
% approach $\image u \subset S _{u} $, to $ J ^{\lambda} (\frac{\partial
% }{\partial \theta})  $, which is the Reeb vector field $R ^{\lambda} $  of $\lambda$. Consequently the linearization of
% the Reeb flow preserves $\mathbb{R}
% \cdot \mu \subset N _{u} $
% and coincides with the linearization of the $\kappa$ flow. In
% particular $\mu| _{S ^{1} \times \{\theta _{2} \} } $ must be an
% integral curve
% of the linearized Reeb flow, and since it is closed this is a
% contradiction to $o _{u} $ being non-degenerate.
% higher dimensions we just need to 
% extend  Lemma \ref{lemma:invariantmu}. In fact
% the same statement holds but the argument we give in higher dimensions is less concrete,
% although actually more elementary.
% Note of course that we are giving two independent proofs of the lemma
% in dimension 3 case.
% \begin{proof}[Proof of Lemma \ref{lemma:invariantmu} in general]
We already known that all  $u \in \overline{\mathcal{M}}_{1,1} ^{1,0}   (J ^{\lambda} 
   , {A} ),$ are Reeb tori. In particular have representation by a $J ^{\lambda} $-holomorphic map $$u: (T ^{2},j) \to (Y = C \times S ^{1}, J ^{\lambda}).  $$
% Fixing an appropriate class in $\pi _{1} (C)
   % $, for the Reeb orbits, the resulting holomorphic tori will be in class
   % $A$. 
Since each $u$ is immersed we may naturally get a splitting $u ^{*}T (Y) \simeq N \times T (T ^{2})   $,
using the $g _{J} $ metric, where $N \to T ^{2}  $ denotes the pull-back, of the  $g _{J} $-normal bundle to $\image u$, and which is identified with the pullback of the distribution $\xi _{\lambda} $ on $Y$, (which we also call the co-vanishing distribution).

The full associated real linear Cauchy-Riemann operator takes the
form:
\begin{equation} \label{eq:fullD}
   D ^{J}_{u}: \Omega ^{0} (N  \oplus T (T ^{2})  ) \oplus T _{j} M  _{1,1}   \to \Omega ^{0,1}
   (T(T ^{2}), N \oplus T (T ^{2}) ). 
\end{equation}
This is an index 2 Fredholm operator (after standard Sobolev
completions), whose restriction to $\Omega
^{0} (N \oplus T (T ^{2})  )$ preserves the splitting, that is the
restricted operator splits as 
\begin{equation*}
D \oplus D':   \Omega ^{0} (N) \oplus \Omega ^{0} (T (T ^{2})  )    \to \Omega ^{0,1}
(T (T ^{2}), N ) \oplus \Omega ^{0,1}(T (T ^{2}), T (T ^{2}) ).
\end{equation*}
On the other hand the restricted Fredholm index 2 operator 
\begin{equation*}
\Omega ^{0} (T (T ^{2})) \oplus T _{j} M  _{1,1}  \to \Omega ^{0,1}(T (T ^{2}) ),
\end{equation*}
is surjective by classical Teichmuller theory, see also \cite [Lemma 3.3]{citeWendlAutomatic} for a precise argument in this setting.
It follows that $D ^{J}_{u}  $ will be surjective
if  
the restricted Fredholm index 0 operator
\begin{equation*}
D: \Omega ^{0} (N) \to \Omega ^{0,1}
(N),
\end{equation*}
has no kernel.

The bundle $N$ is symplectic with symplectic form on
the fibers given by restriction of $u ^{*} d \lambda$, and together with $J
^{\lambda} $ this gives a Hermitian structure on $N $. We have a
linear symplectic connection $\mathcal{A}$ on $N$, which over the slices $S ^{1}
\times \{t\} \subset T ^{2} $ is induced by the  pullback
by $u$ of the linearized $R  ^{\lambda} $ Reeb flow. Specifically the $\mathcal{A}$-transport map from the fiber $N  _{(s _{0} , t)}  $ to the fiber $N  _{(s _{1}, t)}  $ over the path $ [s _{0}, s _{1} ]
\times \{t\} \subset T ^{2} $,  is given by $$(u_*| _{N  _{(s _{1}, t)}  }) ^{-1}  \circ \phi ^{\lambda}
_{c(s _{1}  - s _{0})} 
\circ u_*| _{N _{(s _{0} , t  )}  }, $$ 
where $\phi ^{\lambda} 
   _{c(s _{1}  - s _{0})} $ is the time $c \cdot (s _{1}  - s _{0} )$ map for the $R ^{\lambda} $ Reeb flow, where $c$ is the period of the Reeb orbit $o _{u} $,
   and where $u _{*}: N \to TY $ denotes the natural map, (it is the universal map in the pull-back diagram.)

The connection $\mathcal{A}$ is defined to be trivial in the $\theta
_{2} $ direction, where trivial means that the parallel transport  maps are
the
$id$ maps over $\theta _{2} $ rays.  In particular the curvature $R _{\mathcal{A}} $, understood as a lie algebra valued 2-form, of this connection
vanishes. The connection $\mathcal{A}$ determines a real linear CR operator on
$N$ in the standard way (take the complex anti-linear part of
the vertical differential of a section).  It is elementary to verify from the definitions that this
operator is exactly $D$. 

We have a differential 2-form $\Omega$ on the total space of $N$ 
defined as follows. On the fibers $T ^{vert} N$, $\Omega= u _{*}  \omega $, for $\omega= d ^{\alpha} \lambda $, and for $T ^{vert} N \subset TN$ denoting the vertical tangent space, or subspace of vectors $v$ with $\pi _{*} v =0 $, for $\pi: N \to T ^{2} $ the projection. While on the $\mathcal{A}$-horizontal distribution 
$\Omega$ is defined to vanish.
The 2-form $\Omega$ is closed, which we may check explicitly by using that $R _{\mathcal{A}} $ vanishes
to obtain local symplectic trivializations of $N$ in which $\mathcal{A}$ is trivial.
Clearly $\Omega$ must vanish on the
0-section since it is a $\mathcal{A}$-flat section. But any section is homotopic to
the 0-section and so in particular if $\mu \in \ker D$ then $\Omega$
vanishes on $\mu$. But then since $\mu \in \ker D$, and so its
vertical differential is complex linear, it must follow that
the vertical differential  vanishes, since $\Omega (v, J ^{\lambda}v )
>0$, for $0 \neq v \in T ^{vert}N$ and so otherwise we would
have $\int _{\mu} \Omega>0 $. So $\mu$ is
$\mathcal{A}$-flat, in particular the
restriction of $\mu$ over all slices $S ^{1} \times \{t\} $ is
identified with a period $c$ orbit of the linearized at $o$
$R ^{\lambda} $ Reeb flow, and which
does not depend on $t$ as $\mathcal{A}$ is trivial in the $t$ variable. So the kernel of $D$ is identified with the vector
space of period $c$ orbits of the linearized at $o$ $R
^{\lambda} $ Reeb flow, as needed. 
\end{proof}
\begin{proposition} \label{prop:regular2} Let $\lambda$ be a contact form on a  $(2n+1)$-fold $C$, and $o$ a non-degenerate, period $c$,
    $R^{\lambda}$-Reeb orbit, then the orientation of $[u _{o} ]$
    induced by the determinant line bundle orientation of $\overline{\mathcal{M}} ^{1,0} _{1,1}  ( J ^{\lambda} 
       , {A} ),$ is $(-1) ^{CZ (o) -n} $, which is $$\sign \Det (\Id|
       _{\xi (o(0))}  - \phi _{c, *}
    ^{\lambda}| _{\xi (o(0))}   ).$$ 
% and moreover represents the virtual loose equivariant $T$-bordism
% class of $\overline{\mathcal{M}}_{1}  (T ^{2}, J,
%    A),$.  
   % virtual regularization
   % $\overline{\mathcal{M}}_{1}  (T ^{2}, J ^{virt} ,
   % A),$ of $\overline{\mathcal{M}}_{1}  (T ^{2}, J,
   % A),$ and  $\overline{\mathcal{M}}_{1}  (T ^{2}, \{J ^{virt}  _{t} \},
   % A),$ 
\end{proposition}
\begin{proof}[Proof of Proposition \ref{prop:regular2}]
   % \textcolor{blue}{check the notation $u$ vs $u _{o} $} 
Abbreviate $u _{o} $ by $u$. Let $N \to T ^{2} $ be associated to $u$ as in the proof of Proposition \ref{prop:regular}.
% Given the symplectic connection $A$, on
% $N _{u } $ abbreviated as $N$, as in the second
% part of the Proof of Proposition \ref{prop:regular}, we shall
% construct a deformation $\{A _{t} \}$ of $A _{0}=A $, with $A _{1} $
% gauge equivalent to the 
% trivial connection, so that the complex structure  associated to $A
% _{1} $ is integrable. This deformation will have the property that the
% associated family of real linear CR operators,
% $\{D _{t} \}$, $t \in [0,1]$ has one dimensional kernel for a  finite,
% cardinality $N$,
% set of parameters $\{t _{i} \} \in (0,1)$ and is surjective for $t \in
% [0,1) - \{t _{i} \}$. Given this it is clear we may adjust the family $\{D _{t}
% \}$, by adding a small perturbation $$D' _{t} (s) := D _{t} (s) + \tau
% _{t} (s), $$ where $$\tau _{t} \in \Hom (N, \Omega ^{0,1} (TC, 
% N)  ), $$ so that $\tau _{t} $ vanishes outside a neighborhood of $t
% =1$, so that $\{D ' _{t} \}$ is surjective for $t \in [0,1] - \{t _{i}
% \}$, and has one dimensional kernel for the parameter set $\{t _{i}
% \}$. 
Fix a trivialization $\phi$ of $N$ induced by any trivialization of the
contact distribution $\xi$ along $o$ in the obvious sense: $N$
is the pullback of $\xi$ along the composition $$T ^{2} \to S ^{1}
\xrightarrow{o} C.  $$
Let the symplectic connection $\mathcal{A}$ on $N$ be defined as before. Then the pullback connection $\mathcal{A}' := \phi ^{*} \mathcal{A} $ on $T ^{2} \times \mathbb{R} ^{2n}  $ is a connection whose parallel transport 
paths $p _{t}: [0,1] \to \Symp (\mathbb{R} ^{2n} )$, along the closed loops $S ^{1} \times \{t\} $,
are paths starting at $\id$, and are $t$ independent. And so the parallel transport path of $\mathcal{A}'$ along $\{s\} \times S ^{1} $ is constant, that is $\mathcal{A}' 
$ is trivial in the $t$ variable. We shall call such a
connection $\mathcal{A}'$ on $T ^{2} \times
\mathbb{R} ^{2n}  $ \emph{induced by $p$}.  

   By non-degeneracy assumption on $o$, the map $p(1) $
has no 1-eigenvalues. Let $p'': [0,1] \to \Symp (\mathbb{R} ^{2n} )$ be a path from $p (1)$ to a unitary
map $p'' (1)$, with $p'' (1) $ having no $1$-eigenvalues, and s.t. $p''$
has only simple crossings with the Maslov cycle. Let $p'$ be the concatenation of $p$ and $p''$. We then get  $$CZ (p') - \frac{1}{2}\sign \Gamma (p', 0) \equiv
CZ (p') - n \equiv 0 \mod {2}, $$ since
$p'$ is homotopic relative end points to a unitary geodesic path $h$ starting at
$id$, having regular crossings, and since the number of
negative, positive eigenvalues is even at each regular crossing of $h$ by unitarity.  Here $\sign \Gamma (p', 0)$ is the index of the crossing form of the path $p'$ at time $0$, in the notation of \cite{citeRobbinSalamonTheMaslovindexforpaths.}.
Consequently
  \begin{equation} \label{eq:mod2}
  CZ (p'') \equiv CZ (p) -n \mod {2},
  \end{equation} 
    by additivity of
the Conley-Zehnder index. 
   
   Let us then define a free homotopy $\{p _{t} \}$ of $p$ to
$p'$, $p _{t} $ is the concatenation of $p$ with $p''| _{[0,t]} $,
reparametrized to have domain $[0,1]$ at each moment $t$. This
determines a homotopy $\{\mathcal{A}' _{t} \}$ of connections induced by $\{p
_{t} \}$. By the proof of Proposition \ref{prop:regular}, the CR operator $D _{t} $ determined by each  $\mathcal{A}' _{t} $ is surjective except at some finite collection of times $t _{i} \in (0,1) $, $i \in N$ determined by the crossing times of $p''$ with the Maslov cycle, and the dimension of the kernel
of $D _{t _{i} } $ is the 1-eigenspace of $p'' (t _{i} )$, which is 1
by the assumption that the crossings of $p''$ are simple. 
   
 The operator
$D _{1} $ is not complex linear. To fix this we concatenate the homotopy $\{D _{t} \}$ with the homotopy $\{\widetilde{D} _{t}  \}$ defined as follows. Let $\{\widetilde{\mathcal{A}} _{t}  \}$ be a homotopy of $\mathcal{A}' _{1} $ to a
unitary connection $\widetilde{\mathcal{A}} _{1}  $, where the homotopy
$\{\widetilde{\mathcal{A}} _{t}  \}$ is through connections induced by paths
$\{\widetilde{p} _{t} \} $, giving a path homotopy of $p'= \widetilde{p} _{0}  $ to $h$. 
   % unitary path $\widetilde{p} _{1}  $ (for example $h$ above).
Then $\{\widetilde{D} _{t}  \}$ is defined to be induced by $\{\widetilde{\mathcal{A}} _{t}  \}$.

Let us denote by $\{D' _{t} \}$ the
concatenation of $\{D _{t} \}$ with $\{ \widetilde{D} _{t}  \}$. By
construction in the second half of the homotopy $\{ {D}' _{t}
\}$, ${D}' _{t}  $ is surjective. And $D' _{1} $ is induced by a
unitary
connection, since it is induced by unitary path $\widetilde{p}_{1}  $.
Consequently $D' _{1} $ is complex linear. By the above construction,
for the homotopy $\{D' _{t} \}$, $D' _{t} $ is surjective except for
$N$ times in $(0,1)$, where the kernel has dimension one. 
In
particular the sign of $[u]$ by the definition via the determinant
line bundle is exactly $$-1^{N}= -1^{CZ (p) -n},$$
by \eqref{eq:mod2}, which was what to be proved.
\end {proof}
% \subsection {General $T$-equivariant deformations}
% In this section we shall sketch how our Gromov-Witten invariants,
% specifically the quantum Euler number are invariant under more general
% $T$-equivariant deformations. The reading of this may be postponed
% till the end, but it may be helpful for understanding of the main
% mechanism in the paper by putting things into more general framework.
% \subsection{Equivariant cobordism} 
% \label{sec:equivariant_cobordism}
% First let us recast Karshon's definition ~\cite{citeKarshonMomentmapsandnoncompactcobordisms} of abstract moment maps for
% Kuranishi spaces. 
% \begin{definition}
%     \label{def:momentmap}
% Let $T$ act on a space $M$, we say
% that a proper function $H: M \to \mathbb{R},$ is an \textbf{\emph{abstract
% moment map}} for $T$ if:
% \begin{enumerate}
%    \item $H$ is $T$-invariant.
%    \item $H$ is bounded on the fixed point
%       set of $T$.
%    \item Each $E$-sublevel set $M _{E} $ for $E$ has a Kuranishi
%       structure $\mathcal{K} _{E} $, so that 
%       the action of $T$ extends to an action of $T$ on $\mathcal{K}
%       _{E} $.
% \end{enumerate}
% \end{definition}
% A particular case of this when $M$ is a (non-compact) smooth manifold,
% with $M _{E} $ taken to have the trivial Kuranishi structure with
% trivial obstruction bundle is Karshon's definition of an abstract
% momement map, if in addition $H$  is constant on the fixed point set
% of $T$. This condition is not necessary for us.
%
% We shall denote the data of $M$ with $T$ action and an
% abstract moment map by $(M, H)$, keeping track of the  $T$
% action implicitly. Let us call this data a \emph{$T$-space}.
% We then have the following version of equivariant
% cobordism of Karshon.
% \begin{definition} \label{def:equivariantcobordism}
% An \textbf{\emph{equivariant $T$-bordism}} between $T$-spaces $(M_1, H _{1} )$
% and $(M _{2}, H _{2}  )$ is a (possibly non-compact) 
% $T$-space $(W, \widetilde{H}) $, with a $T$-equivariant homeomorphism
% $\phi: \partial
% W \to M _{1} \sqcup M_{2}^{op}   $, which
% pulls back $H_1 \sqcup H
% _{2} $ to $\widetilde{H}| _{\partial W}  $. Finally we ask that  
% the Kuranishi structure $\widetilde{K} _{E}  $ on $W _{E} $, can be
% chosen so that $$\phi| _{(\partial
% W) _{E}}    \to (M _{1}) _{E}  \sqcup (M_{2}^{op}) _{E},  $$
% is a Kuranishi isomorphism with respect to the Kuranishi structures on
% the right determined by their $T$-space structures.
% \end{definition}
% In principal it would be difficult to work purely with manifolds since 
% the spaces $M$  that we shall deal with are (compactified) moduli
% spaces of holomorphic curves. So to make perfect sense Karshon's definition should be
% extended to spaces with Kuranishi atlases or implicit atlases as in
% \cite{citePardonAlgebraicApproach}. Let us call this a $T$-space. The
% interesting thing then is not the (loose) equivariant $T$-bordisms
%  but
% $T$-bordism classes of cycles (maps) of $T$-spaces into a a space. In
% our case these are the moduli cycles. In what follows we supress this
% as the deeper theory of $T$-bordisms does not yet play a technical
% role in our theory here.
%
% Then $T$ acts on $\overline{\mathcal{M}}_{1}  (
% {J} ^{\lambda},
% A )$ and  $\energy$ is an abstract moment
% map for this action.
% Our theory turns out to be equivariant under the natural $T
% $ action on $C \times S ^{1} $, however since we may regularize our
% data by picking a non-degenerate $\lambda$ so that the whole moduli
% space is fixed by the $T$ action, this may appear to give no extra
% structure. This is indeed the case if one is only concerned with
% deformation of $J ^{\lambda} $ corresponding to deformations of
% $\lambda$, which of course is of main geometric interest, as then even
% cobordism moduli spaces are completely fixed by the $T$ action. However we
% claim to define certain $T$-equivariant Gromov-Witten invariants of the $\lcs$ $C \times S
% ^{1} $,  so let us try to make sense of this. 
% \begin{definition} \label{def:invariance}
% Suppose that $\{\omega _{t} \}$ is a family of $T$-invariant $\lcs$
% structures on $C \times S ^{1} $, and suppose that $\{J _{t} \}$ is a
% family $T$-invariant almost
%   complex structures on $C \times S ^{1} $, s.t. $J _{t} $ is
%   compatible with $\omega _{t} $. We say that $\{\omega _{t}, J _{t} \}$ is
%   \emph{admissible} if $\overline{\mathcal{M}} _{1}  (\{J
%      _{t}\}, A ) $ is a loose equivariant $T$-bordism with respect to
%      $\energy$.  
%   \end{definition}
% We may immediately generalize the argument in Proposition
% \ref{prop:abstractmomentmap} to get.
% \begin{lemma} Let $\{J _{t} \}$ be a family of $T$-invariant almost
%    complex structures compatible with a family $\{\omega _{t} \}$ of
%    $\lcs$ structures as in Definition \ref{def:invariance}.
% Suppose that the fixed points of $T$ in $\overline{\mathcal{M}} _{1}  (J
%   _{t}, A ) $ are critical for $\energy$ for each $t$, then 
% $\overline{\mathcal{M}} _{1}  (\{J
%      _{t}\}, A ) $ is a loose equivariant $T$-bordism with respect to
%      $\energy$.
% \end{lemma}
% We claim that invariants that we consider further  on (mainly quantum
% Euler characteristic) are invariant for
% a general admissible isotopy $ \{J _{t} \}  $, as defined above. 
% Note
% that it is in principle essential for this to have Theorem
% \ref{thm:quantization} as we may no longer be able to avoid bubbling
% by purely geometric considerations, as is the case of $J ^{\lambda} $
% almost complex structures.
% % \end{remark}
% Unfortunately the argument for this is very technical since we must
% use virtual $S ^{1} $-localization techniques. Since at the moment we
% have no application for this more general invariance we shall not
% justify our claim here.
% However we will make use of a slightly more
% restrictive 
% notion of admissible family.
% \begin{definition} \label{def:invariance2}
% Suppose that $\{\omega _{t} \}$ is a family of $T$-invariant $\lcs$
% structures on $C \times S ^{1} $, and suppose that $\{J _{t} \}$ is a
% family $T$-invariant almost
%   complex structures on $C \times S ^{1} $, s.t. $J _{t} $ is
%   compatible with $\omega _{t} $. We say that $\{J _{t} \}$ is
%   \emph{strongly admissible} if $\overline{\mathcal{M}} _{1}  (\{J
%      _{t}\}, A ) $ is fixed by the $T$.  
%   \end{definition}
% To apply this lemma we construct a particular perturbation $\lambda'$ 
% of $\lambda$ so that all the Reeb orbits are non-degenerate and hence
% isolated and so that for every $u \in \fix T$, the assumptions of the
% lemma are satisfied. Start with the perturbation $\lamda''$ as before
% so that all the Reeb orbits are non-degenerate. Take a $C ^{\infty}$
% small perturbation of $\lambda''$ to
% a contact form $\lambda'''$  so that the Reeb flow preserves the contact form in the
% neighborhood of the Reeb orbits. Fix a $J$ on the contact distribution
% of $C$, which is invariant under the Reeb flow of $\lambda'''$,  then in particular the Reeb flow for
% $\lambda'''$ preserves $J
% ^{\lambda'''} $ near the Reeb orbits, and has the required properties. 
%
% Denote the
% compliment of $\fix T$ by $\fix T ^{comp}  $. If we assume that 
% there is a well defined theory of virtual $T$-equivariant
% perturbations, in the case where $T$ action is locally free, then the
% result immediately follows since the virtual loose equivariant cobordism type
% of $\fix T ^{comp}$
% would necessarily be $0$. In fact we do not even need the theory of
% $T$-equivariant perturbations we need only argue algebraically as in
% \cite{cite}.
% \begin{proof} [Proof of Theorem \ref{thm:GWFuller}]
% \begin{lemma} \label{lemma:NearbyEnergy} Given a compact $\lcs$ $(M,\omega)$,
%    and $J$
%       an $\omega$-compatible almost complex structure, suppose that $\overline{\mathcal{M}} (J,A)
%       $ is a union of compact and open components $\{M _{i}\} $, 
%       which are energy separated meaning  that each $$M _{i} \subset \left( U _{i} =
%       \energy ^{-1} (E ^{0} _{i}, E ^{1} _{i}) \right)
%       \subset \left( V _{i} =
%       \energy ^{-1} (E ^{0} _{i} - \epsilon, E ^{1} _{i} + \epsilon  ) \right),
%       $$ with $\epsilon>0$ and with $V _{i} \cap V _{j} = \emptyset$ for $i \neq
%       j$. Fix an auxiliary metric $g$ on $M$.
% Then there is a $\delta>0$ s.t. whenever $\omega'$ is an $\lcs$ $C ^{0}  $
%       $\delta$-close to $\omega$ with respect to $g$, and 
%       $J'$ is an $\omega'$ compatible almost complex structure $C ^{\infty} $
%       $\delta$-close to $J$, if $u \in \overline{\mathcal{M}} (J',A)
%       $ and $$E ^{0} _{i} -\epsilon  < \energy _{J'}  (u) < E ^{1} _{i} +\epsilon  $$
%       then $u \in U _{i} $.
% \end{lemma}
% \begin{proof} [Proof of Lemma \ref{lemma:NearbyEnergy}]
% The $C ^{0} $ norm on $2$-forms on $M$, is defined with respect to a fixed
% Riemannian metric $g$ on $M$, and is given by
% \begin{equation*}
% ||\omega|| = sup _{z; v,w \in T _{z} M} |\omega (v,w)|,
% \end{equation*}
% for $v,w$ a $g$-orthonormal pair in $T _{z} M $.
%    \textcolor{blue}{check $C^0$ convergence vs $C^{\infty}$ convergence} 
%    Suppose otherwise then there is a sequence $\{(\omega _{k}, J _{k}) \}$ of
%    compatible pairs with $\{\omega _{k}\}$ a sequence of $\lcs$, $C ^{0} $ converging to $(\omega,
%    J)$, and a sequence $\{u _{k} \}$ of $J _{k} $-holomorphic stable maps with
%    uniformly bounded energy and satisfying
%    $$E ^{0} _{i} - \epsilon < \energy _{J _{k} }  (u _{k} ) \leq E ^{0} _{i}  $$
%    or $$E
%    ^{1} _{i} \leq  \energy _{J
%    _{k} }  (u _{k} ) < E ^{1} _{i} +\epsilon.
%    $$ By Gromov compactness we may find a Gromov convergent subsequence $\{u _{k _{j} } \}$ to a $J$-holomorphic
%    stable map $u$, with $$E ^{0} _{i} - \epsilon < \energy _{J}  (u) \leq E ^{0}
%    _{i}  $$ or $$E
%    ^{1} _{i} \leq  \energy _{J
%      (u) < E ^{1} _{i} +\epsilon.
%      $$ But by our assumptions such a $u$ does not exist.
% \end{proof}
% \begin{lemma} \label{lemma:NearbyEnergyDeformation} Given a compact $\lcs$ $(M,\omega)$,
%    and $J$
%       an $\omega$-compatible almost complex structure, suppose that $\overline{\mathcal{M}} (J,A)
%       $ is a union of compact and open components $\{M _{i}\} $, 
%       which are energy separated meaning  that each $$M _{i} \subset \left( U _{i} =
%       \energy ^{-1} (E ^{0} _{i}, E ^{1} _{i}) \right)
%       \subset \left( V _{i} =
%       \energy ^{-1} (E ^{0} _{i} - \epsilon, E ^{1} _{i} + \epsilon  ) \right),
%       $$ with $\epsilon>0$ and with $V _{i} \cap V _{j} = \emptyset$ for $i \neq
%       j$. Fix an auxiliary metric $g$ on $M$.
% There is a $\delta>0$ s.t. the following is satisfied.
% Let $(\omega', J')$ be a compatible $\lcs$ pair $\delta$-close to
% $(\omega,J)$. Then there is a smooth family $(\omega _{t}, J _{t}  )$,
%    $(\omega _{0}, J_0)= (\omega, J)$,  $(\omega _{1}, J_1)= (\omega', J')$
%    s.t. for every $i$ there is open compact subset 
% \begin{equation*}
%  \widetilde{N} _{i} \subset \overline{\mathcal{M}} (\{J _{t}\},A),
% \end{equation*}
% with $$\widetilde{N} _{i} \cap \overline{\mathcal{M}} (J _{0},A) = M _{i}.
% $$
%       %
%       % if $u \in \overline{\mathcal{M}} (J',A)
%       % $ and $$E ^{0} _{i} -\epsilon  < \energy _{J'}  (u) < E ^{1} _{i} +\epsilon  $$
%       % then $u \in U _{i} $.
% \end{lemma}
% % \begin{proof} 
% %    Observe that the space of $\lcs$ structures on a compact manifold
% %    $M$ is a locally convex subspace of the space of all $2$-forms  with respect to the natural
% %    affine structure. So we may assume that $\omega, \omega'$ are deformation
% %    equivalent through $C ^{0} $ $\delta$-nearby $\lcs$ structures. Clearly if 
% % \end{proof}
% We need a stronger form of Lemma \ref{lemma:NearbyEnergy}.
% \begin{lemma} \label{lemma:Nearby} Given an $\lcs$ $(M,\omega)$, $J$
%       $\omega$-compatible, suppose that $\overline{\mathcal{M}} (J,A)
%       $ is a union of compact and open components $\{M _{i}\} $.
%       Given open neighborhoods $M_i \subset U _{0} \subset U _{1}  $ of $M
%       _{i} $, s.t. $U _{1} \cap M _{j} = \emptyset $ for $i \neq j$, (which can
%       be found by )
%
%       Then there is an $\epsilon>0$ and a
%       $\delta>0$ s.t. whenever $J'$ is $C ^{0} $ $\delta$-close to $J$, if $u
%       \in \overline{\mathcal{M}} (J',A)
%       $ and $$E ^{0} _{i} -\epsilon < \energy (u) < E ^{1} _{i} +\epsilon  $$
%       then $E ^{0} _{i}  < \energy (u) < E ^{1} _{i}    $.
% \end{lemma}
% \end{proof}

\begin{theorem} \label{thm:GWFullerMain} 
\begin{equation*}
   GW _{1,1} (N,A _{\beta},J ^{\lambda} ) ([\overline {M} _{1,1}] \otimes [C \times S ^{1} ]) = i (\widetilde{N}, R ^{\lambda}, \beta),
\end{equation*}
where $N \subset \overline{\mathcal{M}} ^{1,0} 
_{1,1} (
{J} ^{\lambda},
   A _{\beta}  )$ is an open compact set, $\widetilde{N} $ the corresponding subset of periodic orbits of $R ^{\lambda} $, $i (\widetilde{N}, R ^{\lambda}, \beta)$ is the Fuller index as described in the appendix below, and where the left hand side of the equation is a certain Gromov-Witten invariant, that we discuss in Section \ref{sec:elements}.
\end{theorem}
\begin{proof}
 If ${N} \subset \overline{\mathcal{M}} ^{1,0}   _{1,1} (J ^{\lambda}, A _{\beta} ) $ is open-compact and consists of isolated regular Reeb tori $\{u _{i} \}$, corresponding to orbits $\{o _{i} \}$ we have:
\begin{equation*}
   GW _{1,1} (N,  A _{\beta}, J ^{\lambda} ) ([\overline{M} _{1,1}   ] \otimes [C \times S ^{1} ]) = \sum _{i} \frac{(-1) ^{CZ (o _{i} ) - n} }{mult (o _{i} )},
\end{equation*}
where the denominator $mult (o _{i} )$ is there because our moduli space is understood as a non-effective orbifold, see Appendix \ref{sec:GromovWittenprelims}.

The expression on the right is exactly the Fuller index $i (\widetilde{N}, R ^{\lambda},  \beta)$.
Thus the theorem follows for $N$ as above. However in general if $N$ is open and compact then perturbing slightly we obtain a smooth family $\{R ^{\lambda _{t} } \}$, $\lambda _{0} =\lambda $, s.t.
 $\lambda _{1} $ is non-degenerate, that is has non-degenerate orbits. 
And such that there is an open-compact subset $\widetilde{N} $ of $\overline{\mathcal{M}}  _{1,1} ^{1,0}  (\{J ^{\lambda _{t} } \}, A _{\beta} )$ with $(\widetilde{N} \cap \overline{\mathcal{M}}  _{1,1} ^{1,0}  (J ^{\lambda}, A _{\beta} )) = N $, cf. Lemma \ref{lemma:NearbyEnergyDeformation}.
 Then by Lemma \ref{prop:invariance1} if $$N_1=(\widetilde{N} \cap \overline{\mathcal{M}} ^{1,0}   _{1,1} (J ^{\lambda _{1} }, A _{\beta} ))
$$ we get $$GW _{1,1} (N,  A _{\beta}, J ^{\lambda} ) ([\overline{M}  _{1,1} ] \otimes [C \times S ^{1} ]) = GW _{1,1} (N _{1} ,  A _{\beta}, J ^{\lambda _{1} } ) ([\overline{M} _{1,1}  ] \otimes [C \times S ^{1} ]).
$$
By the previous discussion 
\begin{equation*}
   GW _{1,1} (N _{1} ,  A _{\beta}, J ^{\lambda _{1} } ) ([\overline{M} _{1,1}   ] \otimes [C \times S ^{1} ]) = i (N_1, R ^{\lambda_1},  \beta), 
\end{equation*}
but by the invariance of Fuller index (see Appendix \ref{appendix:Fuller}), $$i (N_1, R ^{\lambda_1},  \beta) = i (N, R ^{\lambda},  \beta).
$$ 
\end{proof}
% \subsection* {A few preliminaries on $J$-curves for exact lcs structures} 

% We also have the following holomorphic variant of Lemma \ref{lemma:rational}.
% \begin{lemma} \label{lemma:rational2} Let $(M,\lambda, \alpha)$ be an exact $\lcs$ structure, and $J$ admissible. Suppose that $\alpha$ is rational, then every non-constant $J$-holomorphic curve $u: \Sigma \to M$ is non-nodal, that is $\Sigma \simeq T ^{2} $. \end{lemma}
% \begin{proof} The critical points of $u$ are isolated, we may then find $S _{0} \subset \Sigma $, as in the proof of Lemma \ref{lemma:rational} and so that the restriction of $u$ to $S _{0} $ is an immersion. Then the rest of the proof is identical to the proof of Lemma \ref{lemma:rational}.
%    \end{proof}
\begin{proof} [Proof of Theorem \ref{lemma:Reeb}]
Let $u: \Sigma \to M$ be a non-constant $J$-curve. We first show that $[u ^{*} \alpha] \neq 0$.  Suppose otherwise. Let $\widetilde{M}$ denote the  $\alpha$-covering space of $M$, that is the space of equivalence classes of paths $p$ starting at $x _{0} \in M $, with a pair $p _{1}, p _{2}  $ equivalent if $p _{1} (1) = p _{2} (1)  $ and $ \int _{[0,1]} p _{1} ^{*} \alpha  =  \int _{[0,1]} p _{2} ^{*} \alpha$.
Then the lift of $\omega$ to $\widetilde{M} $ is $\widetilde{\omega}= \frac{1}{f} d (f\lambda) $,
where $f= e ^{g} $ and where $g$ is the primitive for the lift $\widetilde{\alpha} $ of $\alpha$ to $\widetilde{M} $, that is $\widetilde{\alpha} =dg $.
In particular $\widetilde{\omega} $ is conformally exact on $\widetilde{M}$. Now $[u ^{*} \alpha] = 0$,  so $u$ has a lift to a $\widetilde{J} $-holomorphic map $\widetilde{u}: \Sigma \to \widetilde{M}  $,
where $\widetilde{J} $ is the lift of $J$, which is compatible with $\widetilde{\omega} $.
Since $\Sigma$ is closed, it follows that $\widetilde{u} $ is constant, which contradicts the fact that ${u} $ is non-constant. 

Now since $\alpha$ is rational we may construct a smooth $p: M \to S ^{1} $,   so that $\alpha=c \cdot p ^{*} d\theta $ for $c \in \mathbb{Q}$. 
Let $u: \Sigma \to M$ be a non-constant $J$-curve. Let $s _{0}  \in S ^{1}  $ be a regular value of $p \circ u$, and let $S _{0}   \subset \Sigma $, $S _{0}  \simeq S ^{1}  $ be a component of $(p \circ u) ^{-1} (s _{0} ) $. Since the critical points of $u$ are isolated we may suppose that $u$ is non-critical along $S _{0} $. In particular $u ^{*} \omega $ is non-vanishing everywhere on $T\Sigma| _{S _{0}} $, which together with Lemma \ref{lemma:calibrated} implies that $u ^{*} \lambda \wedge u ^{*} \alpha  $ is non-vanishing everywhere on $T\Sigma| _{S _{0}} $, which readily implies that $\image u| _{S _{0}} $ is a Reeb curve.

Now if $u$ is an immersion then $u ^{*}\omega$ is symplectic and by Lemma \ref{lemma:calibrated} $u ^{*}d\lambda = 0 $, so that $\omega _{0}= u ^{*} \alpha \wedge u ^{*} \lambda$ is non-degenerate on $\Sigma$. Let $\widetilde{\Sigma}$ be the $u ^{*} \alpha$-covering space of $\Sigma$ so that $\omega _{0} = dH \wedge u ^{*} \lambda $ for some $H: \widetilde{\Sigma} \to \mathbb{R} $. Since $\omega _{0} $ is non-degenerate, $H$ has no critical points so that $\widetilde{\Sigma} \simeq S ^{1} \times \mathbb{R}  $ by basic Morse theory. It follows that $\Sigma \simeq T ^{2} $.
\end{proof}
\begin{lemma}
    \label{lemma:rational} 
    Let $(M,\lambda, \alpha, J)$ be a tamed exact $\lcs$ structure. Suppose that $\alpha$ is rational, then every non-constant $J$-curve $u: \Sigma \to M$, with $\Sigma$ a closed possibly nodal Riemann surface, is smooth, that is $\Sigma$ is a smooth Riemann surface.
  \end{lemma}
\begin{proof} 
Since $\alpha$ is rational we may construct a smooth $p: M \to S ^{1} $,   so that $\alpha=c \cdot p ^{*} d\theta $ for $c \in \mathbb{Q}$. 
Let $u: \Sigma \to M$ be a non-constant $J$-curve. Let $s _{0}  \in S ^{1}  $ be a regular value of $p \circ u$, and let $S _{0}   \subset \Sigma $, $S _{0}  \simeq S ^{1}  $ be a component of $(p \circ u) ^{-1} (s _{0} ) $. Since the critical points of $u$ are isolated we may suppose that $u$ is non-critical along $S _{0} $.   Suppose by contradiction that $\Sigma$ is nodal. We may then find an embedded disk $i: D ^{2} \to \Sigma $ with $\partial i (D ^{2} )= S$. 
   % Set $D = \image i$.

Since $u ^{*} d\lambda = 0 $ by Lemma \ref{lemma:calibrated}, $\int _{S ^{1}} i ^{*} u ^{*} \lambda =0 $ by Stokes theorem, and so $u ^{*} \lambda (v) = 0$ for some $v \in TS _{0} (z) \subset T _{z} \Sigma  $, $z \in S _{0} $. And let $w \in T _{z} \Sigma $ be such that $v,w$ form a basis for $T _{z} \Sigma $.
Now $u ^{*} \omega$ is symplectic along $S _{0} $ so that $u ^{*} \omega (v,w) \neq 0 $ which implies that $u ^{*} \alpha \wedge u ^{*} \lambda (v,w) \neq 0$ since $u ^{*} d\lambda (v,w) =0 $, but $u ^{*} \alpha (v)=0$ and $u ^{*} \lambda (v) =0 $, so that we have a contradiction.
\end{proof}
\begin{proof} [Proof of Theorem \ref{thm:holomorphicSeifert}]
   Let $N \subset \overline{\mathcal{M}} ^{1,0}  _{1,1} (A, J ^{\lambda} )  $, be the subspace corresponding, (under the Reeb tori, Reeb orbit correspondence $R$) to the subspace $\widetilde{N} $ of all period $2 \pi$ $R ^{\lambda} $-orbits.  It is easy to compute, see for instance \cite{citeFullerIndex}, $$i (\widetilde{N}, R ^{\lambda}) = \pm \chi (\mathbb{CP} ^{k}) \neq 0.
   $$ By Theorem \ref{thm:GWFullerMain} $GW _{1,1} (N, J ^{\lambda}, A)  \neq 0 $. The first part of the theorem then follows by Lemma \ref{thm:nearbyGW}. 

We now verify the second part.
Let $U$ be a $\delta$-neighborhood of $(d ^{\alpha}\lambda _{H}, J ^{\lambda _{H}} )$ guaranteed by the first part of the theorem. Let $(\lambda', \alpha',J) \in U$ and $u \in \overline{\mathcal{M}} ^{1,0}  _{1,1} (A, J) $ guaranteed by the first part of the theorem, with $J$ admissible. 
   Let $\underline{u}$ be a simple $J$-holomorphic curve covered by $u$, which is non-nodal by Lemma \ref{lemma:rational}.
 Let us recall for convenience the adjunction inequality. 
\begin{theorem} [McDuff-Micallef-White] 
 Let $(M, J)$ be an almost complex 4-manifold and $A \in H_2(M)$ be a homology class that is represented by a simple J-holomorphic curve $u$.  Then 
\begin{equation*}
2\delta (u) - \chi (\Sigma) \leq A\cdot A -c _{1} (A),
\end{equation*}
with equality if and only if $u$  is an immersion with only transverse self-intersections. 
\end{theorem}
In our case  $A=0$, $\chi (\Sigma)=0$, so that $\delta (\underline{u})=0$, and so $\underline{u}$ is an embedding.
%
%    To prove the second part let $\delta'$ be as in Lemma \ref{lemma:NearbyEnergyDeformation}, and $\widetilde{N}  \subset \overline{\mathcal{M}} ^{1,0}  _{1,1}  (\{J _{t}\},A),
%    $ as guaranteed by this lemma. From now on if we omit specifying  coefficients for chain groups we mean $\mathbb{Q}$ coefficients.
% Then there is a singular 1-chain (at this point using virtual perturbation techniques) $\sigma \in C _{1} (\widetilde{N}) $, satisfying $$\partial \sigma \in C_1 (\cup _{i} \overline{\mathcal{M}} ^{0}  _{1,1}  (J _{i},A) \cap \widetilde{N}),  $$  more formally $\partial \sigma$ is in the union of images of the inclusion maps $$inc _{i}: C_0(\overline{\mathcal{M}} ^{0}  _{1,1}  (J _{i},A)) \to C _{0}  (\widetilde{N}).$$
%    And $inc _{0} ^{-1}  \partial \sigma \in  C_0 (N),$ representing the the virtual fundamental homology class, which doesn't vanish as calculated above. Then $$inc _{1} ^{-1}  \partial \sigma \in  C_0 (\overline{\mathcal{M}} ^{0}  _{1,1}  (J _{i},A)),$$ likewise does not vanish in homology. So in particular we have a continuous path $\{u _{t} \} \subset \widetilde{N}$, s.t. $u _{0} \in N $, and $u _{1} \in \overline{\mathcal{M}} ^{0}  _{1,1}  (J _{1},A)) $. Since $N$ consists of embedded curves, $u _{0} $ is embedded and it follows by the classical adjunction inequality, see for instance \cite{citeMcDuffSalamon$J$--holomorphiccurvesandsymplectictopology} that $u _{1} $  is embedded.
\end{proof}
\begin{proof} [Proof of Theorem \ref{thm:C0Weinstein}]
   Define a pseudo-metric $d _{0} $ measuring distance between subspaces $W _{1}, W _{2}  $ of an inner product space $(T,g)$ as follows. If $\dim W_1 = \dim W _{2} $ then $$d _{0} (W _{1}, W _{2}  ) := |P _{W _{1} } - P _{W _{2} }  |, $$ for $|\cdot|$ the $g$-operator norm, and $P _{W _{i} } $ $g$-projection operators onto $W _{i} $. If $W _{1} =T $ or $W _{2} =T $ define $d _{0} (W _{1}, W _{2}  ) :=0 $, in all other cases set $d _{0} (W _{1}, W _{2}) := 1 $. We may generalize this to a $C ^{2}$ pseudo-metric $d _{2} $ again in terms of these projection operators.

Let $U$ be a $C ^{2} $  metric $\epsilon$-ball neighborhood of $(\omega _{H}, J _{H}:= J ^{\lambda _{H}})$ as in the first part of Theorem \ref{thm:holomorphicSeifert}. To prove the theorem we need to construct a tamed exact lcs structure $(\lambda, \alpha, J)$, with $(d ^{\alpha} \lambda,J ) \in U$ as Theorem \ref{thm:holomorphicSeifert} then tells us that there is a class $A$, $J$-holomorphic elliptic curve $u$ in $M$, and since $J$ is admissible, by Theorem \ref{lemma:Reeb} there is a Reeb curve for $(\lambda, \alpha)$.

Suppose that $\omega = d ^{\alpha'} \lambda' $ is $\delta$-close to $\omega _{H} $ for the $C ^{3} $ metric $d _{3} $ as in the statement of the theorem.
Then for each $p \in M$, $d _{2} (\mathcal{V} _{\omega} (p), \mathcal{V} _{\omega _{H}} (p)) < \epsilon _{\delta}  $ and  $d _{2} (\xi _{\omega} (p), \xi _{\omega _{H}} (p)) < \epsilon _{\delta}  $  where $\epsilon _{\delta} \to 0 $ as $\delta \to 0$, and where $d _{2} $ is the pseudo-metric as defined above for subspaces of the inner product space $(T _{p} M,g) $.
% , the pairs of distributions $(\mathcal{V} _{\omega}, \xi _{\omega}) $ and $(\mathcal{V} _{\omega _{H} }, \xi _{\omega _{H} }) $ are $C ^{0} $ $\delta'$-close for $\delta'$ arbitrarily small if $\delta$ is taken to be sufficiently small. Here, the $C ^{0} $ metric distance on pairs of distributions is defined naturally by fixing an auxiliary Riemannian metric $g$ on $M$, and then defining one $k$-distribution $\mathcal{V} _{1} $  to be $\delta$-close to another $k$-distribution $\mathcal{V} _{2} $ if for each $p \in M$ the $g$-unit spheres in $\mathcal{V} _{1} (p) $, $\mathcal{V} _{2} (p) $ are Hausdorff $\delta$-close with respect to $g$ on $T _{p} M $. 

   
Then choosing $\delta$ to be suitably be small, for each $p \in V:=\mathcal{V} (M,\lambda')$ we have an isomorphism $\phi (p): T _{p} M \to T _{p} M   $, $\phi _{p}:= P _{1} \oplus P _{2}   $, for $P _{1}: \mathcal{V}_ {\lambda _{H} } (p) \to \mathcal{V} _{\lambda'} (p) $, $P _{2}: \xi _{\omega _{H}} (p) \to \xi _{\omega} (p)   $ the $g$-projection operators. 
Define $J (p):= \phi (p)_*J _{H}  $, and this defines $J$ in the sub-bundle $\pi _{TM} ^{-1} V$, for $\pi _{TM}: TM \to M $ the bundle projection. In addition, if $\delta$ was chosen to be sufficiently small $(\omega,J)$ is a compatible pair in $\pi _{TM} ^{-1} V$, and is $\epsilon$-close to $(\omega _{H}, J _{H}  )$ in $\pi _{TM} ^{-1} V$.

   Now take any extension of $J$ to $TM$ so that $(\omega,J)$ is a compatible pair 
$\epsilon$-close to $(\omega _{H}, J _{H}  )$ in $\pi _{TM} ^{-1} V$.
This can be obtained by using a partition of unity.
Explicitly, $J$ defined in $\pi _{TM} ^{-1} V$, and $\omega$ give a Riemannian metric $g _{J} (\cdot, \cdot) = \omega (\cdot, J \cdot)$ in $\pi _{TM} ^{-1} V$. Use a partition of unity to extend this metric to $TM$, 
and then use the map:
\begin{equation*}
ret: Met (M) \times \Omega (M)  \to \mathcal{J} (M),
\end{equation*}
as in Lemma \ref{lemma:Ret}. 
% but in this case restricting our metrics $g$ so that $\mathcal{V} _{\omega}, \xi _{\omega} 
%  $ are $g$-orthogonal.
\end{proof}
\begin{proof} [Proof of Theorem \ref{thm:catastrophyCSW}] 
Let $\{\omega _{t} \}$, $t \in [0,1]$, be a continuous in usual $C ^{\infty} $ topology homotopy of $\lcs$ forms on $M=C \times S ^{1}$, as in the hypothesis. Fix an almost complex structure $J _{1} $ on $M$ admissible with respect to $(\alpha',\lambda')$.
Extend to a Frechet smooth family $\{J _{t} \}$ of almost complex structures on $M$, so that $J _{t} $ is $\omega _{t} $-compatible for each $t$. Then in the absence of holomorphic sky catastrophes,  by Theorem \ref{thm:holomorphicSeifertMain}, there is a non-constant elliptic $J _{1} $-holomorphic curve in $M$, so that the result follows by Theorem \ref{lemma:Reeb}.

   % The second part follows since $\chi (\Sigma)=0$, $A _{\beta} \cdot A _{\beta} =0$, and $c _{1} (A _{\beta} ) =0 $ (as $A _{\beta} $ is the class of a Reeb holomorphic torus), 
% as in the second part of the proof of Theorem \ref{thm:holomorphicSeifert}, we obtain an embedded $J _{1} $-holomorphic elliptic curve in $M$, 
   % In addition, in the former case, by the adjunction inequality \cite{citeMcDuffSalamon$J$--holomorphiccurvesandsymplectictopology} for self intersections of a $J$-holomorphic curve in a 4-fold, $u _{1} $  must be embedded, since it is homotopic through $J _{t} $-holomorphic curves to an embedded curve. And so by Lemma \ref{lemma:Reeb} we have an elliptic Reeb curve.
\end{proof}

\section {Extended Gromov-Witten invariants and the extended Fuller index} 
In what follows $M$ is a closed oriented $2n$-fold, $n \geq 2$, and $J$ an almost complex structure on $M$. Much of the following discussion extends to general moduli spaces $\mathcal{M} _{g,n} (J,A, a_1, \ldots, a _{n} ) $ with $a _{1}, \ldots, a _{n}  $ homological constraints in $M$. 
We shall however restrict for simplicity to the case $(\omega,J)$ is a compatible lcs pair on $M$, $g=1, n=1$, the homological constraint is $[M]$, as this is the main interest in this paper. Moreover, we restrict our moduli space to consist of non-zero charge pair (for example $(1,0)$) curves, with charge defined with respect to the Lee form $\alpha$ of $\omega$ as in Section \ref{sec:gromov_witten_theory_of_the_lcs_c_times_s_1_}, and this will be implicit, so that we no longer specify this in notation.

In what follows $e (u)$ denotes the energy of a map $u: \Sigma \to M$, with respect to the metric induced by an $\lcs$ pair $(\omega,J)$.
\begin{definition} 
Let $h = \{(\omega _{t}, J _{t}  )\}$ be a homotopy of $\lcs$ pairs on $M$, so that $\{J _{t} \}$ is Frechet smooth, and $\{\omega _{t} \}$ $C ^{0} $ continuous. We say that it is \textbf{\emph{partially admissible for $A$}}
   if every element of $$ \overline{\mathcal{M}} _{1,1} (M, J _{0},A) $$
   is contained in a compact open subset of $\overline{\mathcal{M}} _{1,1} (M, \{J _{t} \},A) $.
 We say that $h$ is \textbf{\emph{admissible for $A$}} 
   if every element of $$ \overline{\mathcal{M}} _{1,1} (M, J _{i},A), $$ $i=0,1$
   is contained in a compact open subset of $ \overline{\mathcal{M}} _{1,1} (M, \{J _{t} \},A) $.
\end{definition}
% We shall say that such an $X$ is 
%  is \emph{compact} if 
% the connected components
%    of the solution spaces $$S  = \{(x,p) \in M \times \mathbb{R} _{+}  \, | \, (x,p) \text{ is a fixed point
%    of $X   $ \textcolor{blue}{what?}  }\}$$ are dcompact. We shall say that a compact $X $ is
%    \emph{weakly generic} if in addition the connected components of $S$ are
%    isolated, that is each component  is contained in an
%    open set  $U _{j} \subset M \times
%     \mathbb{R} _{+} $ with $U _{j} $  non pairwise intersecting. We shall say
%     $X$ is admissible if it is compact and weakly generic.
% [0,1]$ of non-singular $C
% ^{\infty} $ vector fields is \emph{admissible} if $X_0, X_1$ are admissible, $X
% _{t}  $ has isolated
%    periodic orbits for each $t>0$,
%    and such that all the connected components
%    of the solution spaces $$S  = \{(o,t)\, | \, o \text{ is a periodic orbit of
%    $X  _{t} $}\}$$ are compact.
% \begin{definition} \label{def:boundedtype} We say that a non-singular vector field $X$ is of
%    \emph{bounded}
%    type, if the space of its periodic orbits is compact. For such an $X$ we
%    define $$i (X)= \sum _{o} \frac{i (o)}{\mult (o)},$$ where the sum is over all
%    orbits of a nearby vector field $X'$ admissibly homotopic to $X$, having
%    isolated periodic orbits. (The sum is necessarily finite by assumptions.)
%  \end {definition}  
Thus in the above definition, a homotopy is partially admissible if there are sky catastrophes going one way, and admissible if there are no sky catastrophes going either way. 
   
   Partly to simplify notation, we denote by a capital $X$ a compatible general $\lcs$ triple $(M,\omega, J)$, then we introduce the following simplified notation.
\begin{equation}
\begin{aligned}
   S  (X,A)  & = \{u \in \overline{\mathcal{M}} _{1,1} (X,A) \} \\
   S  (X,a,A)  & = \{u \in S  (X,A)  \, \left .  \right | \, e (u)
   \leq a \} \\ 
   S  (h,A)  & = \{u \in \overline{\mathcal{M}} _{1,1} (h,A) \} 
\text{, for $h = \{(\omega _{t}, J _{t}  )\}$ a homotopy as above}
\\
S  (h,a, A)  & = \{u \in S  (h,A)     \, \left .  \right | \,  e (u) \leq a \} \\
\end{aligned} 
\end{equation}
\begin{definition} \label{def:indexmultiplicity} For an isolated element $u$ of $S (X,A)$, which means that $\{u\}$ is open as a subset, we set $gw (u,p) \in \mathbb{Q}$ to be the local Gromov-Witten invariant of $u$. This is defined as:
\begin{equation*}
   gw (u,p)= GW _{1,1}  (\{u\}, A, J) ([ \overline {M} _{1,1}] \otimes [M]),
\end{equation*}
with the right hand side as in \eqref{eq:functionals2}.
\end{definition}
\begin{definition} \label{def:perturbedsystem}
Suppose that $S (X,A)$ has open connected components.
And suppose that we have a collection of $\lcs$ pairs $$\bigcup _{a>0} (X ^{a} =  (M,\omega ^{a}, J ^{a}  )),$$ 
satisfying the following:
\begin{itemize}
   \item 
  $S   (X ^{a},a,A ) $ consists of isolated curves for each $a$.
\item $$S   (X ^{a},a, A ) = S   (X ^{b},a, A ),  $$ (equality of
   subsets of $ \overline{\mathcal{M}} _{1,1} (X,A) \times \mathbb{R} _{+}    $) if $b>a$, and
the local Gromov-Witten invariants corresponding to the identified elements of these sets coincide.
\item
There is a prescribed homotopy $h ^{a}= \{X ^{a}_t \}$ of each $X ^{a} $ 
to $X$, 
called \textbf{\emph{structure
homotopy}}, 
with the property that for every $$y \in S  (X ^{a}_0, A )$$
there is an open
compact subset $\mathcal{C} _{y} \subset S  (h ^{a} ,A)$, $y \in C _{y}  $,
which is 
       \textbf{\emph{non-branching}} which means that 
$$\mathcal{C} _{y}  \cap  S (X ^{a} _{i}   , A),$$ $i=0,1$ are connected.
     %  are contained in $L _{\beta} M 
     %    \times (0, b) \times  [0,1]$, and so are compact. 
     \item $$S   (h ^{a}, a,A) = S (h ^{b},a,A),  $$ (equality of
subsets) if $b>a$ is sufficiently large.
% is contained in $M \times (0,a) \times
      % [0,1]$. 
       % \item Only finitely many components of $S   (F (\{X ^{a}_t \}))$
       % intersect $M \times (0,a) \times [0,1]$ non-trivially.
       % \textcolor{blue}{not needed} 
\end{itemize}
We will then say that $$\mathcal{P} (A) = (\{X ^{a} \} _{a}, h ^{a} )   $$ is a 
\textbf{\emph{perturbation system}} for $X$ in the class $A$.
\end{definition}


% In the context of Reeb vector fields, we shall assume that $X _{E _{n} } $ are
% Reeb and the structure
% homotopies are through Reeb vector fields.

We shall see shortly that, given a contact $(C,\lambda)$, the associated $\lcs$ structure on $C \times S ^{1}$  always admits a perturbation
system for the moduli spaces of charge (1,0) curves in any class, if $\lambda$ is Morse-Bott.
\begin{definition} \label{def:finitetype}  Suppose that  
 $X$ admits a perturbation system $\mathcal{P} (A)$ so that there exists an $E = E (\mathcal{P} (A))$ with the property that $$S  (X
 ^{a}, a, A) = S  (X ^{E}, a, A   )$$ for all $a > E$, where this as
before is equality of subsets, and the
local Gromov-Witten invariants of the identified elements are also identified.
Then we say that $X$ is \textbf{\emph{finite type}} and set:
   \begin{equation*}
      GW (X, A) = \sum _{u \in S   (X ^{E}, A ) }  gw (u).
\end{equation*}
   % where $Per _{\beta}   (X ^{E} ) = S   (X ^{E}, \beta )/S ^{1}  $ denotes the set of unparametrized periodic orbits of $X ^{E} $ in the class $\beta$. 


%    We shall then say that $X$ is
%     and set
% \begin{equation*}
% i (X) = \lim _{a \mapsto \infty} \sum _{(o,p) \in Per _{[o]}  ^{a}  (X ^{a})}  \frac{1}{m
%    (o)} i (o).
% \end{equation*}
\end {definition}
\begin {definition} \label{def:infinitetype}
Suppose that $X$ admits a perturbation system $\mathcal{P} (A)$ and there is an $E = E (\mathcal{P} (A)) >0$ so that $gw (u)>0$ for all $$\{u \in S(X ^{a}, A) \, | \, E \leq e (u) \leq a \} $$ respectively $gw (u)<0$
for all $$\{u \in S(X ^{a}, A) \, | \, E \leq e (u) \leq a \}, $$
and every $a >E$. Suppose in addition that
\begin{equation*}
\lim _{a \mapsto \infty} \sum _{u \in S (X,a,A)}  gw (u) =  \infty, \text{ respectively } \lim _{a \mapsto \infty} \sum _{u \in S (X,a,\beta)}  gw (u) = - \infty.
\end{equation*}
% where $Per _{\beta}   (X ^{a} ) = S (X,a,\beta) /S ^{1} $.
   Then
   we say that $X$ is \textbf{\emph{positive infinite type}}, respectively \textbf{\emph{negative
   infinite type}} and set $$GW (X, A)  = \infty, $$ respectively
   $ GW  (X, A) = - \infty $. These are meant to be interpreted as extended Gromov-Witten invariants, counting elliptic curves in class $A$.
We say that $X$ is
   \textbf{\emph{infinite type}} if it is one or the other.  
\end{definition}   
\begin{definition} We say that $X$ is
   \textbf{\emph{definite}} type if it admits a perturbation system and is infinite type or finite type.
\end{definition}
With the above definitions $$GW (X, A) \in \mathbb{Q} \sqcup {\infty} \sqcup {- \infty}, $$
when it is defined.
\begin{proof} [Proof of Theorem \ref{thm:GWFullerMain1}]
  Given the definitions above, and the definition of the extended Fuller index in \cite{citeSavelyevFuller}, this follows by the same argument as the proof of Theorem \ref{thm:GWFullerMain}.
\end{proof}
% \begin{remark}
% It should be noted that the above admissibility condition on the
% vector field $X$
%  appears to be formally similar to the Wilson renormalization philosophy in
% quantum field theory. At least on a superficial level.
% \end{remark}
\subsubsection {Perturbation systems for Morse-Bott Reeb vector fields}
\begin{definition} \label{def:MorseBott}
A contact form $\lambda$ on $M$, and its associated flow $R ^{\lambda} $ are called \emph{Morse-Bott} if the $\lambda$
action spectrum $\sigma (\lambda)$ - that is the space of
   critical values of $o \mapsto \int _{S ^{1} } o ^{*} \lambda  $,
is discreet and if for every $a \in
\sigma (\lambda)$, the space $$N _{a}: = \{x \in M | \, F _{a} (x)
=x \},   $$ $F _{a} $ the time $a$ flow map for $R ^{\lambda} $ - is a closed smooth
manifold such that rank $d \lambda |  _{N _{a} } $ is locally constant
and $T _{x} N _{a} = \ker (d F _{a} - I ) _{x}   $.
% In this case the (generalized) Conley-Zehnder index,   is locally
% constant and
% we shall denote by $N _{a,l} $ the submanifolds of $N _{a} $ consisting of
% Reeb orbits with  Conley-Zehnder index $l$.
\end{definition}

\begin{proposition} \label{prop:MorseBott1}
  Let  $\lambda$ be a contact form of Morse-Bott type, on a closed contact
  manifold $C$. Then the corresponding $\lcs$ pair $X _{\lambda} =(C \times S ^{1}, d ^{\alpha} \lambda, J ^{\lambda}) $  admits a
  perturbation system $\mathcal{P} (A)$, for moduli spaces of charge (1,0) curves for every class $A$. 
  % We  call
  % this a \textbf{\emph{Reeb perturbation system}}.
\end{proposition}
\begin{proof}
    This follows immediately by \cite[Proposition 2.12 ]{citeSavelyevFuller}, and by Proposition \ref{prop:abstractmomentmap}.
\end{proof}
\begin{lemma} \label{lemma:HopfinfiniteType} The Hopf $\lcs$ pair $(S ^{2k+1}  \times S ^{1}, d ^{\alpha} \lambda _{H} , J ^{\lambda _{H} })$, for $\lambda _{H} $ the standard contact structure on $S ^{2k+1} $ is infinite type.
\end{lemma}
\begin{proof} This follows immediately by \cite [Lemma 2.13]{citeSavelyevFuller}, and by Proposition \ref{prop:abstractmomentmap}.  \end{proof}

The following is the most basic technical result that we need for our applications.  
\begin{theorem} \label{thm:holomorphicSeifertMain} 
Let $(C,\lambda)
    $ be a closed contact manifold so that $R ^{\lambda} $ has definite type.  And suppose that $i (R ^{\lambda}, \beta ) \neq 0$. Let $\omega _{0} = d ^{\alpha} \lambda   $ be the Banyaga structure, and suppose 
we have a partially admissible homotopy $p = \{(\omega _{t,}, J _{t}  )\}$, for class $A _{\beta} $, then there in an element $u \in \overline{\mathcal{M}} _{1,1} ^{1,0} (J _{1}, A _{\beta} ) $.  
\end{theorem}
\subsection {Preliminaries on admissible homotopies}
% Suppose that $\{X _{t} \}$ is partially admissible.
% For each $y \in  S (\{X _{t} \}, \beta) \cap \left( M \times (0, \infty) \times
% \{0\} \right) $ let $\mathcal{C} _{y} $  denote a chosen compact open subset
% of $S= S (\{X _{t} \}, \beta)$ containing it.
% Let $$S _{a,0} = S _{a, \beta, 0} = \bigcup _{y \in S \cap \left( M
% \times (0, a) \times
%  \{0\} \right)} \mathcal{C} _{y}.
%  $$  
% Likewise if $\{X _{t} \}$ is admissible, for each $y \in S \cap \left( M \times (0,
% \infty) \times \partial [0,1] \right) $
% let $\mathcal{C} _{y} $ denote a chosen compact open subset
% of $S$ containing it and
% set $$S _{a} = S _{a, \beta}  = \bigcup _{y \in S \cap \left( M
% \times (0, a) \times
%  \partial [0,1] \right)} \mathcal{C} _{y}.
%  $$ 
% the union of all
% connected components of
% $S$ which non-trivially intersect $M \times (0,a) \times \partial [0,1]$, and so that the class of the associated orbits is
% $\beta$.
%  \begin{notation}
% As the above notation implies we shall often suppress keeping track of  $\beta$ in various notation, keeping implicit
% the fact that we are working relative to  fixed classes of orbits.  We will however make exceptions at key points.
% \end{notation}
\begin{definition} \label{def:admissiblehomotopy}
  Let $h=\{X _{t} \}$ be a smooth homotopy of $\lcs$ pairs. For $b >a >0$ we say that $h$ is \textbf{\emph{partially}} $a,b$-\textbf{\emph{admissible}}, respectively $a,b$-\textbf{\emph{admissible}} (in class $A$)  if 
%    there exists a map   
%  $F ^{pert} $ $C ^{\infty} $ 
%   perturbing $F$,  coinciding with $F$ on $M \times (0,\infty) \times
%  \partial [0,1]$,
  % s.t.:  
   for each $$y \in S (X _{0},a, A) $$ there is a compact open subset $\mathcal{C} _{y} \subset S (h, A)$, $y \in \mathcal{C} _{y}$ with $e (u) <b$, for all $u \in \mathcal{C} _{y}.
   $  %        %        $\mathcal{C} _{y} $ denote the compact clopen subset
% of $S$ containing it.
% Let $$S _{a,0} = S _{a, \beta, 0} (F) = \bigcup _{y \in S _{\beta} (F)   \cap \left( M
% \times (0, a) \times
%  \{0\} \right)} \mathcal{C} _{y}.
%  $$  
%     \item $\{X _{t} \}$ is partially admissible.
% \item  $S _{a, 0} $ is contained in
%     $M \times (0,b) \times
%   [0,1]$.
    % \item components of $S ^{pert} =S (F (\{X ^{pert} _{t}  \})) $ are compact $S (F ^{pert} ) \cap \left(M \times (0,
    % b) \times [0,1] \right)$
%     \item    Only finitely many connected components of the set $S ^{pert}  =
%     S _{ \beta}  (F ^{pert} )$
%        intersect $M \times (0,  b] \times [0,1]$ non-trivially.
Respectively,
if for each $$y \in S (X _{i},a, A), $$ $i=0,1$
there is a compact open subset $\mathcal{C} _{y} \ni y $ of $S (h,A)$ with $e (u) <b$, for all $u \in \mathcal{C} _{y}.$
    % \item components of $S ^{pert} =S (F (\{X ^{pert} _{t}  \})) $ are compact $S (F ^{pert} ) \cap \left(M \times (0,
    % b) \times [0,1] \right)$
%     \item    Only finitely many connected components of the set $S ^{pert}  =
%     S _{ \beta}  (F ^{pert} )$
%        intersect $M \times (0,  b] \times [0,1]$ non-trivially.
 \end {definition}
% Then $S  _{a, 0}, S _{a}$  are compact by compactness of $M$.
\begin{lemma} \label{lemma:partiallyadmissible} Suppose that $X _{0} $ has a perturbation system $\mathcal{P} (A)$, and $\{X _{t} \}$ is
partially admissible,  then for every $a$  
there is a $b>a$ so that $\{\widetilde{X}  ^{b}_t   \} =  \{X _{t}
 \} \cdot \{X ^{b} _ t
 \} $
 is partially $a,b$-admissible, where $\{X _{t}
 \} \cdot \{X ^{b} _ t
 \} $ is the (reparametrized to have $t$ domain $[0,1]$) concatenation of the homotopies $\{X _{t} \}, \{X
 ^{b} _{t}  \}$, and where  $\{X  ^{b} _ t  \}$ is the structure
 homotopy from 
 $X  ^{b  }  $ to $X _{0} $.
\end{lemma}
\begin{proof}
This is a matter of pure topology, and the proof is completely analogous to the proof of \cite [Lemma 3.8]{citeSavelyevFuller}. 
\end{proof}
The analogue of Lemma \ref{lemma:partiallyadmissible} in the admissible case is the following:
\begin{lemma} Suppose that $X _{0}, X _{1}  $ and $\{X _{t} \}$ are admissible,
then for every $a$ there is a $b>a$ so that 
\begin{equation} \label{eq:tilda2}
\{\widetilde{X}  ^{b}_t   \} = \{ {X} ^{b}
_{1,t}   \} ^{-1} \cdot \{X _{t}
 \} \cdot \{X ^{b} _ {0,t}
 \} 
\end{equation} 
   is $a,b$-admissible,  where  $\{X  ^{b} _ {i,t}  \}$ are the structure homotopies from 
 $X _{i}   ^{b}  $ to $X _{i} $.
\end{lemma}

\subsection {Invariance}
% We know that there exists an $C _{n}>0 $, a $j _{n} $ arbitrarily
%  large  and a 
\begin{theorem} \label{thm:welldefined} 
Suppose we have a definite type $\lcs$ pair $X _{0}  $, with $GW (X
_{0}, A) \neq 0$,  which is joined to $X _{1} $  by a partially admissible 
homotopy $\{X _{t} \}$, 
then  $X _{1} $ has non-constant elliptic class $A$ curves. 
\end{theorem}
\begin{proof} [Proof of Theorem \ref{thm:holomorphicSeifertMain}]
This follows by Theorem \ref{thm:welldefined}  
and by Theorem \ref{thm:GWFullerMain1}.
\end{proof}  
We also have a more a more precise result. 
\begin{theorem} \label{thm:welldefinedadmissible} If $X
_{0}, X _{1}  $ are  definite type $\lcs$ pairs and $\{X _{t} \}$ is admissible then $GW (X_0, A) = GW (X _{1},
A)$.
\end{theorem}

\begin{proof} [Proof of Theorem \ref{thm:welldefined}]
Suppose that $X _{0} $ is definite type with $GW (X _{0}, A) \neq 0$, $\{X _{t} \}$ is partially admissible and
   $\overline {\mathcal{M}} _{1,1} (X _{1},A) = \emptyset $.
Let $a$  be given  and $b$ determined so that   $\widetilde{h} ^{b}=  \{\widetilde{X}  ^{b}_t   \} $
is a partially $(a,b)$-admissible homotopy.
We set $$  S _{a}  = \bigcup _{y} \mathcal{C} _{y} \subset S (\widetilde{h} ^{b},A),  $$  for $y \in S (X ^{b} _{0} ,a, A)$. Here we use a natural identification of $S (X ^{b}, a, A ) = S (\widetilde{X} ^{b}  _{0}, a, A)$ as a subset of $S (\widetilde{h} ^{b},A) $ by its construction.
Then $S _{a} $ is an open-compact subset of $S (h,A)$ and so admits an implicit atlas (Kuranishi structure) with boundary, (with virtual dimension 1) s.t:
\begin{equation*} 
\partial S _{a} = S (X ^{b}, a, A ) + Q _{a} ,
\end{equation*}
where $Q _{a} $ as a set is some subset (possibly empty), of elements $u \in S (X ^{b},b, A )$ with $e (u) \geq a$.
So we have for all $a$:
\begin{equation} \label{eq:contradiction}
\sum _{u \in Q _{a}} gw (u)  + \sum _{u \in S
(X ^{b}, a, A)} gw (u) = 0.
\end{equation}
\subsection {Case I, $X _{0} $ is finite type} Let $E=E (\mathcal{P})$ be the
corresponding cutoff value in the definition of finite type, and take any 
$a > E$.
Then $Q _{a} = \emptyset $ and by definition of $E$ we have that the left side is 
\begin{equation*}
\sum _{u \in S
(X ^{b}, E, A)} gw (u)  \neq 0.
\end {equation*}
% as it is a subseries of the remainder series of an absolutely convergent series,
% and 
% \begin {equation*}
% \lim _{a \mapsto \infty}  \sum _{(\textbf{o} ^{0}, p) \in \textbf{S} _{[o]}  ^{0, a} /S ^{1}, } i ({o} ^{0}
%    )\frac{1}{\mult ({o} ^{0} )} = i (X _{0} ) \neq 0.
% \end{equation*}
Clearly this gives a contradiction to \eqref{eq:contradiction}.

\subsection {Case II, $X _{0} $ is infinite type} We may assume that $GW (X _{0},
A) =
\infty$, and take $a > E$, where $E=E (\mathcal{P} (A) )$ is the corresponding cutoff
value in the definition of infinite type.
Then 
\begin{equation*}
\sum _{u  \in Q _{a} } gw (u) \geq 0,
\end {equation*}
as $a > E (\mathcal{P} (A))$.
While 
\begin {equation*}
\lim _{a \mapsto \infty}  \sum _{u \in 
S (X ^{b}, a, A)} gw (u) = \infty,
\end{equation*}
as $GW (X _{0}, A) = \infty$.
This also contradicts \eqref{eq:contradiction}. 
% Suppose that $a > E (X ^{e} )$, and let $a ' > a$.
%  Then 
% \begin{multline}
%    l'(\sum _{(\textbf{o} , p ) \in Q _{a'} } i (o)\frac{1}{\mult (o  )} + \sum _{(\textbf{o} ^{0}, p) \in \textbf{S} ^{0, a'} /S ^{1}, } i ({o} ^{0}
%    )\frac{1}{\mult ({o} ^{0} )}) \\ = l'(\sum _{(\textbf{o} , p ) \in Q _{a} } i (o)\frac{1}{\mult (o  )} + \sum _{(\textbf{o} ^{0}, p) \in \textbf{S} ^{0, a} /S ^{1}, } i ({o} ^{0}
%    )\frac{1}{\mult ({o} ^{0} )} + r _{a'})   \mod k,
% \end{multline}
% % \sum _{(\textbf{o} ^{0}, p) \in \mathcal{S} ^{0} _{ E'}  } i (o ^{0} )\frac{1}{\mult (o ^{0}
% % )}  > 0
% % \sum _{(\textbf{o} ^{1}, p) } i (o ^{1} )\frac{1}{\mult (o ^{1} )} 
% for all $k$, where 
% $$r _{a'} = \sum _{(\textbf{o} , p ) \in R} i (o)\frac{1}{\mult (o  )},  $$ for some $R \subset \textbf{S} ^{0, a,b'}/S ^{1}  $ non-empty
%    (picking $a'$ to be sufficiently large) and by
%    positivity $l'r _{a'} $ is a positive integer.
% Then  we may
% fix a $k$
%  so that $$l'(\sum _{(\textbf{o} , p ) \in Q _{a'} } i (o)\frac{1}{\mult (o  )} + \sum _{(\textbf{o} ^{0}, p) \in \textbf{S} ^{0, a'} /S ^{1}, } i ({o} ^{0}
%    )\frac{1}{\mult ({o} ^{0} )}) \neq 0 \mod k. $$
%
%    % \sum _{(\textbf{o} ^{1}, p) \in S _{1}   } i (o ^{1} )\frac{1}{\mult (o ^{1} )} \mod k $$
% % where $<<$  is in the natural sense (provided $k$ is large enough) - to be
% % explicit if $\mathbb{Z}_{k} = \mathbb{Z} /k \mathbb{Z} $, then the minimal non-negative representative of the left hand side of the
% % equation is $<<k$.
% % Again by positivity the minimal non-negative representative of $\mu(c _{a'})
% % \mod k <<k$
% % provided $k$ is sufficiently large, and it follows that
% % \begin{equation*}
% % \mu(c _{a'}) + \sum _{(\textbf{o} ^{0}, p) \in \textbf{S} ^{0, a} /S ^{1}  } i (o ^{0}
% % )\frac{1}{\mult (o ^{0} )}  \neq 0 \mod k,
% % % \sum _{(\textbf{o} ^{1}, p) } i (o ^{1} )\frac{1}{\mult (o ^{1} )}, 
% % % \mod k,
% % \end{equation*}
%
% But this is a contradiction since we must
% have equality as before. 
\end{proof}
\begin{proof} [Proof of Theorem \ref{thm:welldefinedadmissible}]
This is somewhat analogous to the proof of Theorem \ref{thm:welldefined}. 
Suppose that $X _{i}, \{X _{t} \} $ are  definite type as in the hypothesis.
Let $a$  be given  and $b$ determined so that   $\widetilde{h} ^{b}=  \{\widetilde{X}  ^{b}_t   \} $, see \eqref{eq:tilda2}
is an $(a,b)$-admissible homotopy.
We set $$  S _{a}  = \bigcup _{y} \mathcal{C} _{y} \subset S (\widetilde{h} ^{b},A)  $$ for $y \in S (X ^{b} _{i} ,a, A)$. 
Then $S _{a} $ is an open-compact subset of $S (h,A)$ and so has admits an implicit atlas (Kuranishi structure) with boundary, (with virtual dimension 1) s.t:
\begin{equation*} 
   \partial S _{a} = (S (X ^{b} _{0} , a, A ) + Q _{a,0}) ^{op} +  S (X ^{b} _{1}, a, A ) + Q _{a,1},
\end{equation*}
with $op$ denoting opositite orientation and where $Q _{a,i} $ as sets are some subsets (possibly empty), of elements $u \in S (X ^{b} _{i},b, A)$ with $e (u) \geq a$.
So we have for all $a$:
\begin{equation} \label{eq:contradiction2}
\sum _{u \in Q _{a,0}} gw (u)  + \sum _{u \in S
(X _{0}  ^{b}, a, A)} gw (u) = \sum _{u \in Q _{a,1}} gw (u)  + \sum _{u \in S
(X _{1}  ^{b}, a, A)} gw (u)
\end{equation}
   \subsection {Case I, $X _{0} $ is finite type and $X _{1} $ is infinite type} Suppose in addition $GW (X _{1}, A) = \infty $ and let $E= \max (E (\mathcal{P} _{0} (A)),  E (\mathcal{P} _{1} (A)))$, for $\mathcal{P} _{i} $, the perturbation systems of $X _{i} $. Take any 
$a > E$.
Then $Q _{a,0} = \emptyset $ and the left hand side of \eqref{eq:contradiction2} is
\begin{equation*}
\sum _{u \in S
(X ^{b} _{0} , E, A)} gw (u).
\end {equation*}
While the right hand side tends to $\infty$ as $a$ tends to infinity since,
   \begin{equation*}
\sum _{u  \in Q _{a,1} } gw (u) \geq 0,
\end {equation*}
as $a > E (\mathcal{P} _{1}  (A))$,
and 
\begin {equation*}
\lim _{a \mapsto \infty}  \sum _{u \in 
S (X ^{b} _{1} , a, A)} gw (u) = \infty,
\end{equation*}


% as it is a subseries of the remainder series of an absolutely convergent series,
% and 
% \begin {equation*}
% \lim _{a \mapsto \infty}  \sum _{(\textbf{o} ^{0}, p) \in \textbf{S} _{[o]}  ^{0, a} /S ^{1}, } i ({o} ^{0}
%    )\frac{1}{\mult ({o} ^{0} )} = i (X _{0} ) \neq 0.
% \end{equation*}
Clearly this gives a contradiction to \eqref{eq:contradiction2}.
\subsection {Case II, $X _{i} $ are infinite type} Suppose in addition $GW (X _{0}, A) = -\infty $, $GW (X _{1}, A ) = \infty$ and let $E= \max (E (\mathcal{P} _{0} (A)),  E (\mathcal{P} _{1} (A)))$, for $\mathcal{P} _{i} $, the perturbation systems of $X _{i} $. Take any 
$a > E$.
Then $\sum _{u \in Q _{a,0}} gw (u) \leq 0$, and $\sum _{u \in Q _{a,1}} gw (u) \geq 0$.
   So by definition of $GW (X _{i},A )$ the left hand side of \eqref{eq:contradiction} tends to $-\infty$ as $a$ tends to $\infty$, and the right hand side tends to $\infty$. 
Clearly this gives a contradiction to \eqref{eq:contradiction2}.
\subsection {Case III, $X _{i} $ are finite type}
The argument is analogous.
% \subsection {Case II, $X _{0} $ is infinite type} We may assume that $GW (X _{0},
% A) =
% \infty$, and take $a > E$, where $E=E (\mathcal{P} (A) )$ is the corresponding cutoff
% value in the definition of infinite type.
% Then 
% \begin{equation*}
% \sum _{u  \in Q _{a} } gw (u),
% \end {equation*}
% as $a > E (\mathcal{P} (A))$.
% While 
% \begin {equation*}
% \lim _{a \mapsto \infty}  \sum _{u \in S
% (S (X ^{b} ), a, A)} gw (u) = \infty,
% \end{equation*}
% as $GW (X _{0}, A) = \infty$.
% This also contradicts \eqref{eq:contradiction}. 
%
%
\end{proof}

% \subsection {Quantum Euler number of $(C, \xi)$} 
% % For a general contact
% % manifold we may define its quantum Euler number $\mathcal{E} (C, \xi)$
% % as follows. 
% % as the virtual count of elements 
% % Thus we have verified that $\overline{\mathcal{M}}_{1}  (T
% % ^{2}, J ^{\lambda} 
% % , k)$ consists of regular elements.
% % In
% % particular this set is isolated from the rest of $\overline{\mathcal{M}}_{1}  (T
% % ^{2}, J ^{\lambda} 
% % , A),$ where the action of $T$ must be locally free. This allows us to
% % give the following specialized definition of the Gromov-Witten invariant
% % of the $\lcs$ $C \times S ^{1} $ in the class $A$ as above, for a contact 3-fold
% % $(C, \xi)$.
% % We may think of this invariant as some kind of Euler number of the
% % contact 3-fold.
% % special because the
% % case of a general class will require a more intricate treatment that
% % will (hopefully) appear in the future. 
% Although we have a very strong transversality criterion in the form of
% Proposition \ref{prop:regular}, so that we do not need virtual moduli
% cycle techniques to regularize the moduli space $\overline{\mathcal{M}}_{1}  ( J ^{\lambda}   
% , A)$, 
% it seems that we do need the virtual moduli cycle to regularize
% cobordisms of these moduli spaces. Indeed if one takes the
% ``cobordism''
% of the type $$\overline{\mathcal{M}}_{1}  ( \{J ^{\lambda _{t} }\}   
% , A) :=  \bigsqcup _{t} \overline{\mathcal{M}}_{1}  ( J ^{\lambda _{t} }   
% , A),$$ for a homotopy $\{\lambda _{t} \}$ then there will
%
% vector fields $R ^{\lambda _{t} } $, since these correspond to our
% curves generically $   \overline{\mathcal{M}}_{1}  ( \{J ^{\lambda _{t} }\}   
% , A) $ will not be regular.  Moreover
% one should keep in mind that there are multiply covered
% curves, which must be counted appropriately to get invariants,
%  see Section \ref{sec:prelims}. Formally our moduli spaces $\overline{\mathcal{M}}_{1}  ( J ^{\lambda}   
% , A) $ are 0-dimensional non-effective orbifolds, the ``count'' of 
% elements is the orbifold Euler number of these orbifolds.
% % \begin{definition} \label{def:GromovWitten}
% % Let $(C, \xi)$ be a contact, closed $2n+1$-fold, which admits  a contact
% % form $\lambda$
% % whose Reeb orbits in class $[o]$ have bounded period. Let $A$ be the class of a
% % Reeb torus corresponding to 
% % $[o]$ as above.  We define  $ \mathcal{E}(C \times
% %    S ^{1}, A) = \#  \overline{\mathcal{M}}_{1}  ( J ^{\lambda}   
% % , A) $ 
% % where $\#  \overline{\mathcal{M}}_{1}  (J ^{\lambda}   
% % , A)  $  is the Gromov-Witten invariant:
% % \begin{equation*}
% % \int _{[\overline{\mathcal{M}}_{1}  ( J ^{\lambda}   
% % , A)] ^{vir} } 1,
% % \end{equation*}
% % see Section
% % \ref{sec:GromovWittenprelims}. (We could just ask that $\lambda$ is
% % non-degenerate above but then lose enormous flexibility that is
% % provided by working with abstract perturbations as opposed to $\lambda$
% % perturbations, in particular the Morse-Bott case becomes trivial, from
% % the above view point.)
% Otherwise we set $ \mathcal{E}(C \times
%    S ^{1}, A)$ to be $\pm \infty$, in the case
% when we have a non-degenerate $\lambda$,
% % and
% % and any \emph{regular} energy filtration $\{E _{i} \}$, $E _{i+1} > E _{i}  $, $\lim _{i} E
% %  _{i} = \infty $, where regular means that there are no elements of $ \overline{\mathcal{M}}_{1}  (J ^{\lambda}   
% % , A)  $ with energy exactly $E _{i} $.
% % Consider the spaces $ \overline{\mathcal{M}}_{1}  (J ^{\lambda}   
% % , A) _{E _{i} } $, consisting of points of $ \overline{\mathcal{M}}_{1}  (J ^{\lambda}   
% % , A) $ with $\energy \leq E _{i} $, $E _{i} \mapsto \infty $  as $i
% % \mapsto \infty$.
% and there is an $E>0$, so that all
% the elements of  $ \overline{\mathcal{M}}_{1}  (J ^{\lambda}   
% , A) $, with $\energy > E$ are positively, respectively
% negatively signed. 
% Or if $\lambda$ is degenerate, then the connected components $\mathcal{C}
% _{i} $ of $ \overline{\mathcal{M}}_{1}  (J ^{\lambda}   
% , A) $, are compact, by Proposition \ref{prop:abstractmomentmap}.
% % So each component
% % $\mathcal{C}_{i} $ is completely contained in an $\energy$ sub-level
% % set $\overline{\mathcal{M}}_{1}  (J ^{\lambda}   
% % , A) _{E}$, for $E$ sufficiently large.
% We then ask that there are choices of  abstract perturbations,   
% so that there is an $N>0$, s.t. for every $i > N$ all
% the elements of  $ \mathcal{C} _{i}  ^{vircycle} $,  are positively, respectively
% negatively signed, see Section \ref{sec:vircycle}.
% (As in the bounded case working with abstract perturbations can be much
% simpler than directly perturbing $\lambda$.)
% % Otherwise if the signs of the terms of $\overline{\mathcal{M}}_{1}  ( J ^{\lambda}   
% % , A)$ are alternating, we set:
% % \begin{equation*}
% % \mathcal{E}(C \times
% %    S ^{1}, A) = \lim _{E} \# \overline{\mathcal{M}}_{1}  ( J ^{\lambda}   
% % , A) _{E},
% % \end{equation*}
% % if the limit exists. 
% We shall call the possibilities for $\lambda$: \emph{bounded}, respectively
% \emph{infinite} (positive, negative) type. 
% \end {definition}
% % defined with respect to some $i$
% % dependent perturbations, which are assumed to be coherent, in the
% % sense that $$[\overline{\mathcal{M}}_{1}  (J ^{\lambda}   
% % , A) _{E _{i} }]  ^{vircycle}|_ {\energy \leq E _{k} } = [\overline{\mathcal{M}}_{1}  (J ^{\lambda}   
% % , A) _{E _{k} }]  ^{vircycle},  $$ where $[\overline{\mathcal{M}}_{1}  (J ^{\lambda}   
% % , A) _{E _{i} }]  ^{vircycle}| _{\energy \leq E _{k} }$ just means take all terms
% % of the chain whose underlying points have $\energy \leq E _{k} $.
% % and that $\lim _{i \mapsto \infty} \#  \overline{\mathcal{M}}_{1}  ( J  
% % , A) _{E _{i} } = \infty   $, respectively $- \infty$. (These are
% % chains over $\mathbb{Q}$.)
% The above invariant is in a sense  $T$-equivariant because it is a
% priori
% invariant only under very special $T$-equivariant
% deformations of the pair $(d ^{\alpha} \lambda, J ^{\lambda}  )$ structure, namely the deformations
% corresponding to deformations of $\lambda$. This is the subject of the
% following lemma. We claim that they are in fact invariant under more
% general $T$-equivariant deformations, in a way that is connected to
% the theory of Karshon
% \cite{citeKarshonMomentmapsandnoncompactcobordisms} but we postpone this story.
% \begin{lemma} \label{lemma:welldefined} The invariant $\mathcal{E}(C \times
%    S ^{1}, A)$ is well defined, (in the cases it is defined). That is
%    independent of the choice of $\lambda$.
% \end{lemma}
% \begin{theorem}  \label{thm:invariantcalculation}
% For a contact closed $2n+1$-fold $C$, if $\lambda$ is non-degenerate and bounded 
% type in class $[o]$ then: $$ \mathcal{E}(C \times
% S ^{1}, A) = \sum _{o}  \frac{1}{\mult (o)}(-1) ^{CZ (o)-n}, $$ where the sum is over all
% (unparametrized) $R ^{\lambda} $ Reeb orbits, in class $[o]$, $A$
% determined by $[o]$ as usual.
% \end{theorem}
% \subsection {Connection to the Fuller index} \label{sec:Fuller}
% The Fuller index is an analogue for orbits of the fixed point index.
% But with a couple of new ingredients: we must account for the
% symmetry of the orbits, 
% and since the period is freely varying there is
% an extra compactness issue to deal with. 
% Let us very briefly recall this notion following Fuller's original
% paper \cite{citeFullerIndex} as the connection of this
% index with our quantum Euler characteristic is remarkable.
%
% Let $X$ be a smooth vector field  on $M$, for simplicity without
% zeros, where $M$
% for simplicity is closed. The setting for the Fuller index is the phase space $M \times
% \mathbb{R} _{+} $. Denote by $\phi _{t} $ the time $t$ flow map for
% $X$. We want to consider equivalence classes of points $(p, P) \in M \times \mathbb{R}_{+}
% $ which satisfy $\phi _{P} (p)=p $, with $(p, P) \sim (p', P)$ if
% $p' =\phi _{t} (p) $ for $0 \leq t \leq P$. We call the equivalence classes
% $o = [(p,P)]$: periodic orbits.
% We shall denote by $\underline{o}$ the ($S ^{1} $-reparemetrization equivalence
% class) of a periodic orbit in $M$
% underlying $o$. The multiplicity $\mult (o)$ of a periodic orbit $o= [(p, P)]$ is
% the ratio $P/l$ for $l>0$ the least number with $\phi _{l} (p) =p $.
% We want a kind of fixed point index which counts such
% pairs $(p,P)$ with certain weights, however in general to get
% invariance we must
% have period bounds. This is due to potential existence of so called
% blue sky catastrophe, where one has $\{X _{\tau}\} $ periodic orbits
% $\{o _{\tau}\} $, with this family continuous in the  
% loop space
% $LM $,
% with period of $o (\tau)$ going to infinity as $\tau \mapsto a$,  there are
% then no $X _{a} $ periodic orbits $o _{a} $ for all sufficiently large values of
% the period. So in effect this is a ``bifurcation'' where the orbit
% disappears into the sky.
% Let $N \subset M \times \mathbb{R}_{+} $ be a
% compact isolating neighborhood. That is  there is an open neighborhood of
% $N$ not containing periodic orbits of $X$, which are not in $N$.
% Assume that free homotopy class $c$ periodic orbits of $X$  are
% isolated. Then to such an $X,N$
% Fuller associates an index:
% \begin{equation*}
%    i (X, N) = \sum _{o=[(p,P)],  (p,P) \in N, [\underline{o}] \in c}
%    \frac{1}{\mult (o)}i (o),
% \end{equation*}
%  where $i (o)$, is the fixed point index of the return
% map with respect to
% a local surface of section transverse to $\underline{o}$, and
% $[\underline{o}]$ denotes the free homotopy class. In the case where $X$ is
% the $R ^{\lambda} $-Reeb vector field on $C$, and if $\lambda$ is
% non-degenerate we have: $$i (o) = \sign \Det (\Id|
%    _{\xi (\underline{o}(0))}  - \phi _{P, *}
%     ^{\lambda}| _{\xi (\underline{o}(0))}   ).$$
% Fuller then shows that $i (X,N)$ is invariant in a deformation $\{X _{t}
% \}$ of $X$ if $N$ is isolating for $X _{t} $ for all $t$. (Actually
% Fuller shows that we
% may also suitably vary $N$.)
% Note that if 
% $X$ is a $\lambda$-Reeb vector field on $C$, $\lambda$ non-degenerate and $N= C \times [a,b]$
% is isolating then $i (X,N)$ is just the
% orbifold Euler number of the 0-dimensional orbifold $ \overline{\mathcal{M}}_{1}  (J ^{\lambda}   
% , A) _{a,b} $ consisting of elements of $ \overline{\mathcal{M}}_{1}  (J ^{\lambda}   
% , A)  $ corresponding to Reeb orbits $o$ in class $c$ with $a \leq \period o \leq
% b$.
% The 
% $C ^{\infty} $ condition can probably be relaxed to $C ^{k} $ for
% some small $k>0$, perhaps even $1$.
% We start with some immediate applications, we shall avoid absolute
% generality in places for the sake of simplicity, and state things with
% some restrictions, as our results are 
% pretty strong in any case. it may be apparent to
% the experts how to get around various restrictions.
% \subsection {Relative theory} \textcolor{blue}{rewrite} 
% Most of the content of this paper has a relative analogue. We shall
% only sketch this out as the arguments are mostly identical. 
% First a relative analogue of theorem \ref{thm:complete}, which 
% for simplicity (to avoid extra difficulty with moduli spaces of
% Riemann surfaces with boundary) we state for annuli, which is what is relevant for the
% main example of the paper, and for relative ``Fuller theory''.
% \begin{theorem} \label{thm:completelagrangian}
% Suppose that $X$ is a closed $\lcs$, and $L \subset X$ a closed
% Lagrangian submanifold. Let $\mathcal{M} _{1}   (L, J,
%     A)$ be the space of equivalence classes of $J$-holomorphic maps $(An, \partial
%  An)\to (X, L) $ in
%  relative class class $A$, where $An$ is a Riemann surface with
%  boundary,
%  diffeomorphic to an annulus. Then $\mathcal{M} _{1}   (L, J,
%     A)$ has a metric completion
%  \begin{equation*}
% \overline{\mathcal{M}} _{1}   (L, J, A), 
% \end{equation*}
% by Kontsevich stable maps.  Moreover given $E>0$,   the subspace
% $\overline{\mathcal{M}} _{1}   (L, J,
%  A) _{E} \subset \overline{\mathcal{M}}_{1}   (L, J,
%  A) $ consisting of elements with $L ^{2} $ energy  $ \leq E$ is
%  compact.  
% \end{theorem}
% For the relative analogue of our example Gromov-Witten theory of $C
% \times S ^{1} $, we take a 
%  Lagrangian submanifold $L$ of $C \times S ^{1} $,
%  of the form
% $Leg _1
% \times S ^{1} \sqcup Leg _{2} \times S ^{1}    $ for $Leg _{i} $ closed Legendrian
% submanifolds  of $C$. 
% There are distinguished holomorphic annuli with
% boundary on $L$, which are constructed as follows. Recall that a Reeb
% chord $\gamma$ of $R ^{\lambda} $, 
% going from $Leg _{1}  $  to $Leg _{2} $, is a smooth map $\gamma:
% [0,1] \to C$, $\gamma (0) \in Leg _{1} $, $\gamma (1) \in Leg _{2} $,
% $D _{t} \gamma (t) = c R ^{\lambda} (\gamma (t)) $, $c>0$.
% Given a Reeb chord $\gamma$ from $Leg _{1}  $  to $Leg _{2} $,
% a \emph{Reeb
% annulus} $u _{\gamma} $  is the 
% map $$u_\gamma (r, \theta   )= (\gamma  (r), \theta
%  ),$$ using polar coordinates on the annulus. As with Reeb tori a Reeb
%  annulus is $J ^{\lambda} $-holomorphic for a uniquely determined holomorphic
% structure $j$ on $An $.
% If $$D _{t}  \gamma (t) = c \cdot R ^{\lambda} (\gamma (t)), $$ then $$j
% (\frac{\partial}{\partial r}) = c \frac{\partial} {\partial \theta}. $$
% Let $A$ denote the class of such a curve.
% % We may then
% % construct a moduli space $\mathcal{M} _{1}  (J ^{\lambda}, A)$ of $J ^{\lambda} $-holomorphic annuli with boundary on $L$, 
% % in class $A$.
% We then have an analogue of Proposition \ref{prop:abstractmomentmap}:
% \begin{proposition} \label{prop:abstractmomentmaprelative}
% Let $(C, \lambda)$ and $L \subset C$ be as above. 
% % Let $A $ be the homology class of a
% % map $T ^{2} \to C \times S ^{1}
% % $ with ``degree'' $1$ projection to $S ^{1} $, and ${J} ({\lambda})  $
% % be as above.  Then $T$ acts on 
% The moduli space
% $\overline{\mathcal{M}}_{1}  (L,
% {J} ^{\lambda},
% A )$ consists only of Reeb annuli. 
% % and  $\energy$ is an abstract moment
% % map for this action. In fact the entire moduli space is
% % is fixed by the $T$ action, and contains no nodal curves.
% For a pair $\widetilde{J} ({\lambda _{1} }),
% \widetilde{J} (\lambda _{2} )  $  of almost
% complex structures as above and $\{\widetilde{J} (\lambda _{t} ) \} $ a smooth interpolating
% family, the moduli space $\overline{\mathcal{M}}_{1}  (L,
% \{\widetilde{J} (\lambda _{t} ) \},
% A),$ has compact connected components.
% \end{proposition}
% As the normal bundle $N _{u} $, $[u] \in \overline{\mathcal{M}}
% _{1}  (
% {J} ^{\lambda},
% A ) $, and its real subbundle over the boundary corresponding to $L$
% are both simultaneously trivial, 
% by the Riemann-Roch
% theorem the associated Fredholm operator has index $0+1$, with $+1$
% for the dimension of the  moduli space of complex annuli.
% The formal dimension of $\overline{\mathcal{M}}
% _{1}  (
% {J} ^{\lambda},
% A )$ is then $1$ minus the dimension of
% the automorphism group (for smooth curves) which is 1, so is $0$.
%  It is given by the
% Fredholm index of 
% $$
%    D ^{J ^{\lambda} }_{u}: \Omega ^{0} (N _{u} \oplus T (An)  ) \oplus
%    T _{j} M (An)   \to \Omega ^{0,1}
%    (T(An), N _{u} \oplus T (An) ),
% $$
% which is 1 for the dimension of $T _{j} M ({An})$, minus  , where $M (An)$
% is 
% \begin{remark}
% A small nitpick $\overline{\mathcal{M}}_{1}  (
% {J} ^{\lambda},
% A )$ and $\overline{\mathcal{M}}_{1}  (
% \{\widetilde{J} (\lambda _{t} ) \},
% A)$ are not smooth manifolds, but they do have virtual dimension 0 and
% 1 respectively. For the above statement to make good sense we should 
% However the language of abstract moment maps is only used here to make
% a connection with future work, technically we need only Lemma
% \ref{lemma:}
% \end{remark}
% For the following proposition we need to assume that there exists a virtual
% $T$-equivariant perturbation theory. In fact since we will be in the
% expected dimension 0 case, and because we shall arrange fixed point
% sets in the moduli space to be composed of regular curves, for the
% proposition below we need only virtual $T$-equivariant perturbation
% theory when $T$ acts freely on the original moduli space. 
% The following proposition may hold in higher dimensions, however our
% approach uses essentially  4 dimensional positivity of intersections
% techniques.
% The regularity also works as with Reeb tori. Recall that an action $a$ Reeb cord $\gamma$ between Legendrian submanifolds
% $Leg _{0}, Leg _{1}  
% $ of a contact $2n+1$-fold $C$ is  called non-degenerate if  $\phi ^{\lambda}  _{a,*} (T _{\gamma (0)} Leg _{0})   $ intersects $
% T _{\gamma (a)} Leg _{a}$ transversally,  where $\{\phi ^{\lambda}
% _{t,*}\}$ is the linearized at $\gamma$, $R ^{\lambda} $-Reeb flow.
% We say that $\lambda$ is non-degenerate relative to $Leg _{0}, Leg _{1}  $ if all the Reeb
% cords between $Leg _{i} $ are non-degenerate.
% \begin{proposition} \label{prop:regularan} Let $(C, \xi)$ be a closed 
%     contact manifold. Suppose that
%    $\lambda$ is non-degenerate relative to closed Legendrians $Leg _{0}, Leg _{1}  $ then the moduli space $\overline{\mathcal{M}}_{1}  (L, J ^{\lambda} 
% , {A} )$ is regular. 
% % , that is the
% %    Reeb chords between $Leg _{i} $ are all non-degenerate, which means
% % that if we symplectically 
% %    suppose 
% %    $\phi 
% %    ^{\lambda} _{a} (Leg _{1} )  $ intersects $\phi 
% %    ^{\lambda} _{a} (Leg _{2} )  $ transversally for all $a$, for $\phi
% %    ^{\lambda} _{a}  $ the time $a$ $R ^{\lambda} $ Reeb flow,
% %
% % VMoreover if $\lambda$ is degenerate than for a Reeb orbit $o$ 
% % the kernel of the associated real linear Cauchy-Riemann operator for
% % the Reeb torus of $o$ is identified with the 1-eigenspace of the
% % return map for the linearized at $o$ Reeb flow.
% % and moreover represents the virtual loose equivariant $T$-bordism
% % class of $\overline{\mathcal{M}}_{1}  (T ^{2}, J,
% %    A),$.  
%    % virtual regularization
%    % $\overline{\mathcal{M}}_{1}  (T ^{2}, J ^{virt} ,
%    % A),$ of $\overline{\mathcal{M}}_{1}  (T ^{2}, J,
%    % A),$ and  $\overline{\mathcal{M}}_{1}  (T ^{2}, \{J ^{virt}  _{t} \},
%    % A),$ 
% \end{proposition}
% \begin{remark} To naturally orient $\overline{\mathcal{M}}_{1}  (L, J ^{\lambda}
% , {A} ),$ we do not need any additional assumptions on $L$, this is
% because for a curve $[u] \in \overline{\mathcal{M}}_{1}  (L, J ^{\lambda} 
% , {A} ),$ the normal bundle over the boundary of the curve and its
% real subbundle corresponding to $L$ are simultaneously naturally
% trivial. We claim that the following version of Proposition
% \ref{prop:regular2} holds. 
% Let $(C, \lambda)$ be a closed
%     contact $2n+1$-fold, $\lambda$ non-degenerate with respect to
%     closed Legendrians $Leg
%     _{1}, Leg _{2}  $ and $\gamma$ an action $a$
%     $R^{\lambda}$ Reeb cord from $Leg _{1}   $, to $Leg
%     _{2}  $ then there is natural determinant line
%     bundle theoretic orientation of $[u _{\gamma} ]$, which is
%     identified with the natural orientation of the 
%     intersection $\phi ^{\lambda}  _{a,*} (T _{\gamma (0)}
%     Leg _{0})  \cap T _{\gamma (a)} Leg _{1}        $. 
% We will not verify this here as it is not
% a priority. 
% \end{remark}
% We shall not pursue orientation issues for the annular moduli spaces,
% and content ourself with defining our invariants $\mod 2$ which
% is possible because there are no isotropy groups to deal with in this
% case.
%
%
% \begin{definition} \label{def:GromovWittenrelative}
% Let $(C, \xi)$, be a contact $2n+1$-fold, $\lambda$ non-degenerate with
% respect to $\{Leg _{i} \}$,
% with finitely many Reeb chords in class $[\gamma]$ going from   $Leg
% _{1}
% $, to $Leg _{2} $, and $L = Leg _{1} \times S ^{1} \sqcup Leg _{2}
% \times S ^{1}    $ as before, $A$ the class of Reeb annulus for a Reeb
% chord in class $[\gamma]$.
% We define  $$\mathcal{E}(C \times
%    S ^{1}, L, A) = \#  \overline{\mathcal{M}}_{1}  (L, J ^{\lambda}   
% , A) \mod 2, $$
% where $\#  \overline{\mathcal{M}}_{1}  (L, J ^{\lambda}   
% , A) \mod 2 $  is the $\mod 2$ count of elements of the space. % and
% % and any \emph{regular} energy filtration $\{E _{i} \}$, $E _{i+1} > E _{i}  $, $\lim _{i} E
% %  _{i} = \infty $, where regular means that there are no elements of $ \overline{\mathcal{M}}_{1}  (J ^{\lambda}   
% % , A)  $ with energy exactly $E _{i} $.
% % defined with respect to some $i$
% % dependent perturbations, which are assumed to be coherent, in the
% % sense that $$[\overline{\mathcal{M}}_{1}  (J ^{\lambda}   
% % , A) _{E _{i} }]  ^{vircycle}|_ {\energy \leq E _{k} } = [\overline{\mathcal{M}}_{1}  (J ^{\lambda}   
% % , A) _{E _{k} }]  ^{vircycle},  $$ where $[\overline{\mathcal{M}}_{1}  (J ^{\lambda}   
% % , A) _{E _{i} }]  ^{vircycle}| _{\energy \leq E _{k} }$ just means take all terms
% % of the chain whose underlying points have $\energy \leq E _{k} $.
% % and that $\lim _{i \mapsto \infty} \#  \overline{\mathcal{M}}_{1}  ( J  
% % , A) _{E _{i} } = \infty   $, respectively $- \infty$. (These are
% % chains over $\mathbb{Q}$.)
% \end {definition}   
% Since we have avoided orientation issues we will not discuss the
% infinity type cases, although of course we claim that this can be
% done.
% \begin{lemma} \label{lemma:welldefinedrelative} The  invariant $\mathcal{E}(C \times
%    S ^{1}, L, A)$ is well defined, (when defined).
% \end{lemma}
% The proof is analogous to the proof of Lemma \ref{lemma:welldefined}.
% This together with Proposition \ref{prop:regularan}, this readily implies
% the following.
% \begin{theorem}  \label{thm:invariantcalculationleg} 
% Given a closed contact $2n+1$-fold $C$, with  $Leg _{1}, Leg
% _{2} $, closed Legendrian submanifolds, and $\lambda$ non-degenerate
% with respect to $Leg _{i}$, if there is an odd
% number 
% of class $[\gamma] $, $R ^{\lambda}$-Reeb chords between $Leg _{1}, Leg
% _{2}  $,
% then for any $\lambda'$ homotopic (through contact forms) to $\lambda$ there is at least
% one class $[\gamma] $ $R ^{\lambda} $-Reeb chord between $Leg _{1}, Leg _{2}  $. 
% \end{theorem}
% In this generality,
% this may also be impossible to obtain via contact
% homology.
% \subsection {Relative Fuller index}
% Naturally since our invariants $\mathcal{E} (C \times S ^{1},A )$ are
% closely related to the Fuller index, the invariants $\mathcal{E}(C \times
% S ^{1}, L, A)$ should be related to some kind of relative Fuller
% index, strangely this does not appear anywhere in literature. 
% But there should be no essential difficulty in
%  making such a construction. 
%
%
%
%
% I shall follow the McDuff-Wehrheim approach to the virtual moduli
% cycle,  as there are no effectivity assumptions in their approach, see
% for instance \cite{cite}. 
% I note that Fukaya-Oh-Ohno-Ohta, and the first two authors are of
% course pioneers of virtual techniques, and would be glad to say that
% their approach works just as well, but I don't know how it works for
% non-effective moduli spaces, 
% So let us explain how to count, or how to obtain genus 1 GW
% invariants, when the moduli space of stable maps is given by an
% orbifold $X$, with an orbifold obstruction bundle $E$. 
% \subsection {Virtual fundamental classes and chains} \label{sec:vircycle}
% First we should quickly point out that the construction of Kuranishi
% structures, or any similar construction like Kuranishi atlases of McDuff-Wehrheim obviously extends to moduli spaces of
% $J$-holomorphic curves in a non-symplectic manifold, provided that the
% moduli space, that is the zero set of the non linear Cauchy-Riemann
% section, is compact (modulo reparametrization group), (possibly after adding in
% Kontsevich stable maps). In the main example of this paper compactness holds for connected
% components and we work on the level of these components. 
% % We can also
% % obviously extend the above observation of not needing a symplectic
% % manifold, to the case the moduli space
% % itself is
% % not compact but after adding in  Kontsevich stable maps it is, (one
% % then still needs something like Theorem \ref{thm:quantization}).  We
% % don't need this here because in our moduli spaces nodal degenerations
% % do not happen.
%
% Given the above, we sometimes use notation like $$\mathcal{C}^{vircycle},
% \text{ or } \mathcal{C} ^{virchain}  $$ 
% for $\mathcal{C}$ a connected component (which are compact) of 
% $\overline{\mathcal{M} } _{1}  (J ^{\lambda} , A) $, or respectively for connected
% components of cobordism moduli space $\overline{\mathcal{M} } _{1}
% (\{J ^{\lambda _{t} } \} , A) $, (which are also compact). 
% By this we mean
% the following.
% When $X$ comes with a
% Kuranishi structure $\mathcal{K}$ after some intermediate steps (construction of a good coordinate system, etc.)
% we choose a perturbed (multi)-section $s + \nu$ and a triangulation of its
% zero set. Then appropriately choosing rational weights as in
% \cite{citeFukayaOnoArnoldandGW},
% we get a virtual
% fundamental cycle, $X _{choices, \mathcal{K}} ^{vircycle}  $, whose homology class $[X]
% _{\mathcal{K}}  ^{vir} $ is independent of the
% choices.  Likewise when $X$ is a Kuranishi cobordism, we get a virtual
% fundamental chain $X _{choices, \mathcal{K}} ^{virchain}  $.
%
% When $X$ represents (a component of) the moduli space of holomorphic
% curves $\mathcal{K}$ is natural up to concordance, 
% and we choose to omit $choices, \mathcal{K}$ from the notation  thinking of
% them implicitly. 
% \section {Proofs} \label{section:proofs}
%
%
% \begin{proof} [Proof of \ref{thm:invariantT2main}]
%
% \end{proof}
% % The projection map $M=C \times S ^{1} \to S ^{1}  $ is $J$
%    % convex consequently there are no $J$-holomorphic spheres in $M$ Note first that
%  $T$ acts on the moduli space by post-composition of class representative
%  stable maps
%  with the  $J ^{\lambda} $ preserving
%  action of $T$ on $C \times S ^{1} $. 
% Now $\energy$ is clearly $T$ invariant, thus for the first
% part of the theorem  we just need to check
% that $E$ is constant on $\overline{\mathcal{M}}_{1}  (
%    \widetilde{J},
%    A).$ 
% % Then by assumption on the class $A$ of $u$ $\theta \mapsto pr \circ u(\{\theta ^{1} _{0} \}
% % \times \{\theta\})$, is a degree $1$ curve, where $pr _{S ^{1} } :  C \times S ^{1}
% % \to S ^{1}  $ is the projection. 
% By the previous lemma all elements $u \in \overline{\mathcal{M}}_{1}  (
% {J} ^{\lambda},A)  $, are Reeb tori. If $u$ is a Reeb torus
% corresponding to a Reeb orbit $o$, then
% $$E (u) = \int _{T ^{2} } u ^{*}d ^{\alpha} \lambda = 2\pi \cdot \int _{S ^{1}
% } o ^{*}  \lambda.  $$ 
% And since a  Reeb orbit is a critical point for
% the $\lambda$ functional: $o \mapsto \int _{S ^{1} } o ^{*} \lambda  $, it
% follows that $\energy$ is locally constant on the moduli space
% $\overline{\mathcal{M}}_{1}  (
% {J} ^{\lambda},A)  $.
% An explicit argument is also contained as a special case in the following proof of the second part of the
% theorem concerning cobordism. 
% To prove the claim with regards to cobordisms, it suffices to prove 
% Lemma
% \ref{lemma:compactcomponents}. Since then the connected components of
% the $S ^{1} $-quotient of
% $S$ by the reparametrization $S ^{1} $ action are also compact, and by
% the previous discussion these are identified with connected components of $\overline{\mathcal{M}}_{1}  (
% \{{J} ^{\lambda _{t} } \},A)  $. 
% the
% Conley-Zehnder index of $p''$ 
%
% It is shown in  \cite []{citeRobbinSalamonTheMaslovindexforpaths.}
% that we may smoothly
% homotope $p $ relative end points to a path $p'$ so having simple
% crossings with the Maslov cycle except at time $0$, where $p  (0)
% = \id$ and the crossing is regular. Next define a homotopy $\{p' _{t}
% \}$,
% $t \in [0,1]$ of $p'$ to
% the constant path at $\id$, by setting $p' _{t} $ to be the path $p'| _{[0,t]} $ reparametrized to have domain $[0,1]$.
% Let $\{p'' _{t} \} $ be the concatenation of these two homotopies,
% which goes from $p$ to the constant path at $id$.
%
% Define a smooth family $\{A'
% _{t} \}$, $A' _{0} = A $, so that $A' _{t} $ is induced by the path
% $p'' _{t}  $. By construction $\{A' _{t} \}$ has the property that for
% $t \in [0,1/2]$ the associated operator $D' _{t} $ is surjective since 
% $p'' _{t} (1) $ has no 1-eigenspace, and since we have the Lemma
% \ref{}. Also for $t \in (1,2)$ the kernel of $D' _{t} $, is $0$ except
% for some points $\{t _{i} \} \in (1,2)$, corresponding to the simple crossings
% of $p'$, and so the kernel of $\{D' _{t _{i} } \}$ is dimension 1. On
% the other hand $D' _{1} $ is complex linear since it is induced by the
% trivial connection by construction.
% Pulling back the family $\{D' _{t} \}$ by the
% trivialization map $\phi$ gives the asked for family described at the
% beginning of the proof. Then by construction and by the construction of
% the Conley-Zehnder index in
% \cite{citeRobbinSalamonTheMaslovindexforpaths} if $N$ is the cardinality
% of the set $\{t _{i} \}$, then $$CZ (o) \mod 2 = \frac{1}{2} (2n) 
%  + (-1) ^{N} \mod 2, $$ where the first term on the right is the contribution from
% the regular crossing of $p'$ at the starting end point, the other
% term is the sum of the contributions of the simple crossings. To see
% that the contribution $\mod 2$ for the regular crossing at the
% starting point is $n \mod 2$,
% note that the crossing
% form $\Gamma$ is dimension $2n$, (we are at $\id$) and is non-degenerate by the
% assumption that the crossing is regular so the crossing contributes to
% $CZ (o)$: $1/2 \sign \Gamma$, which is an integer since the number
% $pos$ of
% positive, and $neg$ of negative eigenvalues is even. But then $$1/2 \sign
% \Gamma \mod 2 = \frac{1}{2} \cdot pos - \frac{1}{2} \cdot neg  \mod 2
% = n \mod 2,$$ since $pos + neg =2n$. \textcolor{blue}{check this} 
% %
% % Therefore
% % $[u]$ contributes with sign $(-1)^{CZ (o) -n}$.
% \end{proof}
%
% \begin{proof} [Proof of Theorem \ref{thm:invariantcalculation}] By
%    propositions \ref{prop:regular},
%    \ref{prop:regular2}, if $\lambda$ is non-degenerate of bounded type the
%    moduli space $\overline{\mathcal{M}} (J ^{\lambda} , A) $, is
%    regular, consists only of Reeb tori $[u _{o} ]$, with orientation
%    of $[u _{o} ]$ given by $ {-1} ^{CZ (o) -n}  $. If $o$ has
%    multiplicity $k$, then $[u _{o} ]$ has a symmetry group of order
%    $k$, which is the isotropy group of $[u _{o} ]$ in the orbifold
%    $\overline{\mathcal{M}} (J ^{\lambda} , A) $. Consequently the
%    contribution to the orbifold Euler number of
%    $\overline{\mathcal{M}} (J ^{\lambda} , A) $ from $[u _{o} ]$ is $
%    {-1} ^{CZ (o) -n} /k $.
% \end{proof}
% \begin{proof} [Proof of Lemma \ref{lemma:welldefined}]
% If $\lambda _{0}, \lambda _{1}  $    are both of bounded type then 
% $\overline{\mathcal{M}}_{1}
%    ( J ^{\lambda _{0}} :=J ^{0}      
% , A)$, $\overline{\mathcal{M}}_{1}
% ( J ^{\lambda _{1}}:= J ^{1}   
% , A)$ have Kuranishi structures and are Kuranishi cobordant by $\overline{\mathcal{M}}_{1}
%    ( \{J ^{{t}  }  \}
% , A)$ which in this case is a compact moduli space as readily follows by Proposition
% \ref{prop:abstractmomentmap}. 
%
% % $$\#  \overline{\mathcal{M}}_{1}
% %    ( J ^{\lambda _{0}} :=J ^{0}      
% % , A)= \#  \overline{\mathcal{M}}_{1}
% % ( J ^{\lambda _{1}}:= J ^{1}   
% % , A)$$
% % simply because $[\overline{\mathcal{M}}_{1}
% %    ( \{J ^{\lambda _{t}} := J ^{t}     \}
% % , A)] ^{virtchain}$ will give a homology of the corresponding virtual
% % fundamental 0-chains, as 
%   Let us prove  the infinite type case when $\lambda _{0}, \lambda _{1}  $
%   are regular, as the general case is logically the same. For a given 
% energy level $E >0$, we
% define $\overline{\mathcal{M}}_{1}  (\{J ^{t}\} 
% , A) _{E} ^{vircycle}$ as follows. Take the union
%  $\overline{\mathcal{M}}_{1}  (\{J ^{t}\} 
% , A) _{E}$ of all the
% connected components of the moduli space $\overline{\mathcal{M}}_{1}  (\{J ^{t}\}    
% , A) $, which contain the elements of $\overline{\mathcal{M}}_{1}  (J ^{0}    
%    , A) _{E}$, and of $\overline{\mathcal{M}}_{1}  (J ^{1}    
% , A) _{E} $, with these just being $\energy$ $E$ sublevel sets. These connected components must be in
% the $(E +C _{E} )$ $\energy$ sublevel set of $\overline{\mathcal{M}}_{1}  (\{J ^{t}\}    
% , A) $, for $C _{E}   >0$ depending on $E$, by Proposition \ref{prop:abstractmomentmap}. 
% Clearly $ \overline{\mathcal{M}}_{1}  (\{J ^{t}\}    
% , A) = \bigcup _{E}  \overline{\mathcal{M}}_{1}  (\{J ^{t}\} 
% , A) _{E} $ and each $\overline{\mathcal{M}}_{1}  (\{J ^{t}\} 
% , A) _{E}$ is an open and
% closed subspace of $\overline{\mathcal{M}}_{1}  (\{J ^{t}\}    
% , A) $, moreover each $\overline{\mathcal{M}}_{1}  (\{J ^{t}\} 
% , A) _{E}$ is compact and so has a natural induced $d=1$ Kuranishi
% structure with boundary (that is a Kuranishi cobordism)
%  and so
% with respect to particular abstract perturbation data, there is an induced
% singular 1-chain
% $\overline{\mathcal{M}}_{1}  (\{J ^{t}\} 
%  , A) _{E} ^{virchain}$  whose boundary corresponding to elements 
%  with $\energy \leq E$ is identified with $$- \overline{\mathcal{M}}_{1}  (J ^{0}    
%     , A) _{E}  + \overline{\mathcal{M}}_{1}  (J ^{1}    
%     , A) _{E},$$ where the latter is understood as a (in this case) canonical singular 0-chain, 
% representing the homology Euler class of the non-effective 0-dimensinal orbifold $ \overline{\mathcal{M}}_{1}  (J ^{0}    
%     , A) _{E} ^{op}   \sqcup \overline{\mathcal{M}}_{1}  (J ^{1}    
%  , A) _{E}$, see Section \ref{sec:GromovWittenprelims}.
% Suppose that there is a 
% $\lambda _{0} $ of infinite type (say positive) and $\lambda _{1} $ of
% finite type.
% % then since every term 
% % of $$\overline{\mathcal{M}}_{1}
% % ( J ^{{0}  }   
% % , A)  _{E}   $$ appears as a boundary term of 
% % $[\overline{\mathcal{M}}_{1}
% %    ( \{J ^{{t}  }  \}
% % , A)  _{E}]  ^{virtchain}$ it follows that
% % the number of terms of $$[\overline{\mathcal{M}}_{1}
% %    ( \{J ^{{t}  }  \}
% % , A)] _{E}  ^{virtchain}$$ goes to infinity as $E \mapsto \infty$.
% % Then the number of connected components $\{\mathcal{C} ^{E}  _{i} \} $ of $\overline{\mathcal{M}}_{1}
% %    ( \{J ^{{t}  }  \}
% % , A) _{E} $ goes to infinity as $E \mapsto \infty$, again by
% % Proposition \ref{prop:abstractmomentmap}, since $\energy$ oscillation on
% % a connected component is bounded. 
% Take $E$ sufficiently large so that
% \begin{equation*}
% \overline{\mathcal{M}}_{1}  (J ^{1}    
%  , A) _{E}  = \overline{\mathcal{M}}_{1}  (J ^{1}    
%  , A),
% \end{equation*}
%     and so that there are no negatively signed elements of $\overline{\mathcal{M}}_{1}  (J ^{0}    
%     , A)$, with energy larger than $E$. 
%     Let $\overline{\mathcal{M}}_{1}  (J ^{0}    
%     , A) _{E, E+E _{C} } $ denote the subspace of
%     elements of $[u] \in \overline{\mathcal{M}}_{1}  (J ^{0}    
%     , A)$ with $E \leq \energy ([u]) \leq E + E _{C} $.
% Then $$\partial \overline{\mathcal{M}}_{1}  (\{J ^{t}\} 
%  , A) _{E} ^{virchain} =- c _{E, E_C} -  \overline{\mathcal{M}}_{1}  (J ^{0}    
%     , A) _{E}    + \overline{\mathcal{M}}_{1}  (J ^{1}    
%  , A),$$ where by the positivity assumption $c _{E, E_C}  $ is a 0-chain  with
%  positive coefficients, corresponding to the canonical  representing chain for
%  the orbifold Euler class of some suborbifold of $\overline{\mathcal{M}}_{1}  (J ^{0}, A) _{E, E+C _{E} }  $. 
%     We have
% \begin{equation*}
% \int _{\partial \overline{\mathcal{M}}_{1}  (\{J ^{t}\} 
%    , A) _{E} ^{virchain}}  1 =0,
% \end{equation*}
% but this (together with the positive infinite type assumption) readily implies that we may take an
% $E'>E$ sufficiently large so that in addition to conditions on $E$ above
% we have 
%  $$\int _{\overline{\mathcal{M}}_{1}  (J ^{0}    
%     , A) _{E'}}  1 >  \int _{\overline{\mathcal{M}}_{1}  (J ^{1}    
%     , A) }  1.   $$ 
% Then we still have $$\partial \overline{\mathcal{M}}_{1}  (\{J ^{t}\} 
%  , A) _{E'} ^{virchain} = - c _{E', E'_C} -  \overline{\mathcal{M}}_{1}  (J ^{0}    
%     , A) _{E'}    + \overline{\mathcal{M}}_{1}  (J ^{1}    
%  , A),$$ where $c _{E', E'_C}  $ is as before.
% But then  we get 
% \begin{equation*}
% \int _{\partial \overline{\mathcal{M}}_{1}  (\{J ^{t}\} 
%    , A) _{E'} ^{virchain}}  1 <0,
% \end{equation*}
% which is impossible, and so we obtain a contradiction.
% % Let us abuse notation and write 
% % $\{\mathcal{C}_{i} \}$ for the union $$\bigcup _{E} \mathcal{C} ^{E}
% % _{i}, $$ with each $\mathcal{C} _{i} $ connected.
% % For all but finitely many $i$ the 1-chain $\mathcal{C}_{i} 
% % ^{virtchain} $ has boundary entirely in $\overline{\mathcal{M}}_{1}
% % ( J ^{\lambda _{0}  } , A)$, (in the natural sense) by the assumption that $\lambda _{1}
% % $ is finite type. And this boundary is non empty for infinitely many
% % $i _{j} $, as $\lambda _{0} $ is of infinite type, and as 
% % \begin{equation*}
% %    \bigsqcup _{i}  \partial \mathcal{C} _{i} ^{virtchain}  \simeq \overline{\mathcal{M}}_{1}  (J ^{0}    
% %        , A) ^{op}  \sqcup \overline{\mathcal{M}}_{1}  (J ^{1}    
% %  , A),
% % \end{equation*} where $\simeq$ is understood as natural identification.
% % Consequently there
% % are terms in $\partial \mathcal{C}_{i _{j} } 
% % ^{virtchain} $ that are positively signed and terms that are
% % negatively signed. Since there  are infinitely many $i _{j} $, this
% % contradicts that $\lambda _{0} $ is of infinite positive type. ads
%
% The case of $\lambda _{0} $ of infinite positive type and
% $\lambda _{1} $ of infinite negative type  is similar.
% If $\lambda _{0}, \lambda _{1}  $    are both of bounded type then 
% $\overline{\mathcal{M}}_{1}
%    ( J ^{\lambda _{0}} :=J ^{0}      
% , A)$, $\overline{\mathcal{M}}_{1}
% ( J ^{\lambda _{1}}:= J ^{1}   
% , A)$ have Kuranishi structures and are Kuranishi cobordant by $\overline{\mathcal{M}}_{1}
%    ( \{J ^{{t}  }  \}
% , A)$ which in this case is a compact moduli space as readily follows by Proposition
% \ref{prop:abstractmomentmap}. 
%
% % $$\#  \overline{\mathcal{M}}_{1}
% %    ( J ^{\lambda _{0}} :=J ^{0}      
% % , A)= \#  \overline{\mathcal{M}}_{1}
% % ( J ^{\lambda _{1}}:= J ^{1}   
% % , A)$$
% % simply because $[\overline{\mathcal{M}}_{1}
% %    ( \{J ^{\lambda _{t}} := J ^{t}     \}
% % , A)] ^{virtchain}$ will give a homology of the corresponding virtual
% % fundamental 0-chains, as 
% Let us prove  the unbounded case when $\lambda _{0}, \lambda _{1}  $
% are regular, as the general case is logically the same. For a given 
% energy level $E >0$, we
% define $\overline{\mathcal{M}}_{1}  (\{J ^{t}\} 
% , A) _{E} ^{vircycle}$ as follows. Take the union
%  $\overline{\mathcal{M}}_{1}  (\{J ^{t}\} 
% , A) _{E}$ of all the
% connected components of the moduli space $\overline{\mathcal{M}}_{1}  (\{J ^{t}\}    
% , A) $, which contain the elements of $\overline{\mathcal{M}}_{1}  (J ^{0}    
%    , A) _{E}$, and of $\overline{\mathcal{M}}_{1}  (J ^{1}    
% , A) _{E} $, with these just being $\energy$ $E$ sublevel sets. These connected components must be in
% the $(E +C)$ $\energy$ sublevel set of $\overline{\mathcal{M}}_{1}  (\{J ^{t}\}    
% , A) $, for $C  >0$ depending on $E$, by Proposition \ref{prop:abstractmomentmap}. 
% Clearly $ \overline{\mathcal{M}}_{1}  (\{J ^{t}\}    
% , A) = \bigcup _{E}  \overline{\mathcal{M}}_{1}  (\{J ^{t}\} 
% , A) _{E} $ and each $\overline{\mathcal{M}}_{1}  (\{J ^{t}\} 
% , A) _{E}$ is an open and
% closed subspace of $\overline{\mathcal{M}}_{1}  (\{J ^{t}\}    
% , A) $, moreover each $\overline{\mathcal{M}}_{1}  (\{J ^{t}\} 
% , A) _{E}$ is compact and so has an induced $d=1$ Kuranishi structure and so
% there is an induced weighted branched 1-manifold with boundary 
% $\overline{\mathcal{M}}_{1}  (\{J ^{t}\} 
%  , A) _{E} ^{vircycle}$  whose boundary components (that is elements) with
%  $\energy < E $ are identified with elements of $$- \overline{\mathcal{M}}_{1}  (J ^{0}    
%     , A) _{E}  + \overline{\mathcal{M}}_{1}  (J ^{1}    
%  , A) _{E}.$$ Where the last sum is understood as a sum of weighted branched
%  0-manifolds, i.e. finite sets of points with rational weights.
%
%
% Suppose there is a 
% $\lambda _{0} $ of infinite type (say positive) and $\lambda _{1} $ of
% finite type.
% % then since every term
% % of $$\overline{\mathcal{M}}_{1}
% % ( J ^{{0}  }   
% % , A)  _{E}   $$ appears as a boundary term of 
% % $[\overline{\mathcal{M}}_{1}
% %    ( \{J ^{{t}  }  \}
% % , A)  _{E}]  ^{virtchain}$ it follows that
% % the number of terms of $$[\overline{\mathcal{M}}_{1}
% %    ( \{J ^{{t}  }  \}
% % , A)] _{E}  ^{virtchain}$$ goes to infinity as $E \mapsto \infty$.
% Then the number of connected components $\{\mathcal{C} ^{E}  _{i} \} $ of $\overline{\mathcal{M}}_{1}
%    ( \{J ^{{t}  }  \}
% , A) _{E} $ goes to infinity as $E \mapsto \infty$, again by
% Proposition \ref{prop:abstractmomentmap}, since $\energy$ oscillation on
% a connected component is bounded. Let us abuse notation and write
% $\{\mathcal{C}_{i} \}$ for the union $$\bigcup _{E} \mathcal{C} ^{E}
% _{i}, $$ with each $\mathcal{C} _{i} $ connected.
% For all but finitely many $i$ the 1-chain $\mathcal{C}_{i} 
% ^{virtchain} $ has boundary entirely in $\overline{\mathcal{M}}_{1}
% ( \{J ^{\lambda _{0}  }  \}, A)$, (in the natural sense) by the assumption that $\lambda _{1}
% $ is finite type. And this boundary is non empty for infinitely many
% $i _{j} $, as $\lambda _{0} $ is of infinite type, and as 
% \begin{equation*}
%    \bigsqcup _{i}  \partial \mathcal{C} _{i} ^{virtchain}  = - \overline{\mathcal{M}}_{1}  (J ^{0}    
%        , A) _{E} + \overline{\mathcal{M}}_{1}  (J ^{1}    
%  , A) _{E}.
% \end{equation*}
% Consequently there
% are terms in $\partial \mathcal{C}_{i _{j} } 
% ^{virtchain} $ that are positively signed and terms that are
% negatively signed. Since there  are infinitely many $i _{j} $, this
% contradicts that $\lambda _{0} $ is of infinite positive type. 
%
% % Furthermore
% % all but finitely
% % many of these terms (in the limit as $\mapsto \infty$) must be 1-chains with boundary in $\overline{\mathcal{M}}_{1}
% % ( \{J ^{\lambda _{0}  }  \}, A)$, by the assumption that $\lambda _{1}
% % $ is finite type, unless as $E \mapsto \infty$ arbitrarily many terms of $[\overline{\mathcal{M}}_{1}
% %    ( \{J ^{{t}  }  \}
% % , A)] _{E}  ^{virtchain}$ have one boundary component of the form
% % $q [u]$, $q \in \mathbb{Q}$, $[u] \in \overline{\mathcal{M}}_{1}
% % ( \{J ^{\lambda _{1}  }  \}, A) $, for a fixed $[u]$. 
%
% % But that means that as $E \mapsto
% % \infty$ there are
% % arbitrarily many terms of $\overline{\mathcal{M}}_{1}
% %    ( J ^{\lambda _{0}  }  
% % , A) _{E}  $, with negative signs, unless we again have that 
% % as $E \mapsto \infty$ arbitrarily many terms of $[\overline{\mathcal{M}}_{1}
% %    ( \{J ^{{t}  }  \}
% % , A)] _{E}  ^{virtchain}$ have one boundary component of the form
% % $q [u]$, $q \in \mathbb{Q}$, $[u] \in \overline{\mathcal{M}}_{1}
% % ( \{J ^{\lambda _{0}  }  \}, A) $, for a fixed $[u]$. But the latter is
% % again impossible. So 
% % we get a contradiction.
% The case of $\lambda _{0} $ of infinite positive type and
% $\lambda _{1} $ of infinite negative type  is similar.
% Finally we must deal with the alternating type.  We use the fact that
% there is an upper bound on energy shift for a term of $\overline{\mathcal{M}}_{1}
% ( \{J ^{\lambda _{0}  }  \}, A) _{E} $ under the cobordism $[\overline{\mathcal{M}}_{1}
%    ( \{J ^{{t}  }  \}
% , A)] _{E}  ^{virtchain}$ to prove invariance.asdf
% \end{proof}
% \begin {proof}  [Proof of Theorem \ref{thm:qualitative}]
% \begin{lemma} \label{lemma:growth} Suppose that $\lambda _{t} = (1-t)\lambda + tf \lambda
%    $, $f \geq 1$, and $\{o _{t} \}$, $t \in [0,1]$ a continuous family
%    with $o_t$ a $\lambda _{t} $-Reeb orbit. Then:
%    \begin{equation*}
%    \period (o_0) \leq   \period (o_1)  \leq e^{(\max _{C} f -1) } o_0.
%    \end{equation*}
% \end{lemma}
% \begin{proof}
%  In the argument of the proof of Proposition
%  \ref{prop:abstractmomentmap}, in \eqref{eq:directcalc} we have
%    $$0 \leq D
% _{\tau}| _{\tau _{0} }   \left(\Lambda \circ \widetilde{p} '
% (\tau _{0} )\right) \leq Const \cdot
% \int _{\widetilde{p}' (\tau _{0} )} \lambda \leq (\max _{C} f -1) \cdot
% \Lambda({\widetilde{p}' (\tau _{0} )}), $$ 
% since $f \geq 1$.
% Since $\epsilon$ can be made arbitrarily small, we get
%    \begin{equation*}
%  \Lambda ( \widetilde{p} ' (0)) \leq \Lambda \circ \widetilde{p} ' (1) 
%  \leq  \Lambda ( \widetilde{p} ' (0)) \cdot e ^{ \max _{C} f -1}.
%    \end{equation*}
% \end{proof}
% Given the above lemma it follows that the space
% $\overline{\mathcal{M}} _{1} (\{J
% ^{\lambda _{t} } \}, A) _{2 \pi P e ^{Const}}  $ (defined in the proof
% of Lemma \ref{lemma:welldefined}
% just above) is contained entirely
% in the $2 \pi P e ^{Const} $-sublevel set, for $\energy$, 
% from which the first part of the theorem immediately follows. 
% To prove
% the second part note that the minimal period grows in $t$ by Lemma
% \ref{lemma:growth}. Consequently $ R^{\lambda _{1}}$-Reeb orbits with action bounded from
% above by $P e ^{Const}$ cannot be multiply covered for $Const < \ln 2
% $, from which the conclusion follows.
% \end{proof}
% \begin{proof} [Proof of Theorem \ref{thm:invariantcalculation}]
%  By Proposition \ref{prop:regular}, Proposition
% \ref{prop:regular2} the orbifold Euler number of  $\overline{\mathcal{M}}_{1}
% ( J ^{\lambda}  , A) $ is $\sum _{o}  \frac{1}{\mult (o)}(-1)
% ^{CZ (o)-n}$, which as explained in Section
% \ref{sec:GromovWittenprelims} is the Gromov-Witten invariant.
% \end{proof}
% Clearly we may suppose that all the terms $a ^{0} _{j}  $, and $a ^{1}
% _{j} $ are positively signed, for every $i$,
% since there are finitely many negative terms and we may remove them without affecting the
% limits \eqref{eq:limit1}, \eqref{eq:limit2}.
% Take $i$ to be large enough so that the total negative charge on the
% left with energy between $E _{i}, E _{i+1}  $ is $G << E _{i} $, 
% We compare the sums $\sum _{E ^{0} (j) < E _{i}   } a ^{0}  _{j}  $,
% $\sum _{E ^{1} (j)< E _{i}   } a ^{1}  _{j}  $. 
% Consider the 1-chain 
% $[\overline{\mathcal{M}}_{1}  (\{J ^{t}\} 
% , A) _{E _{i+1}}] ^{vircycle}$, 
% which we suppose for the moment
% has all terms positively signed. 
   % Every term of $\sum _{E ^{0} (j)< E
   % _{i}   } a ^{0}  _{j}  $ corresponds to a unique term of the boundary 
   % of this 1-chain with $\energy < E _{i} $. 
% We argue that 
% \begin{equation*}
% \sum _{E ^{1} (j)< E _{i}   } a ^{1}  _{j} \leq 
% \sum _{E ^{0} (j)< E _{i}   } a ^{0}  _{j} + K _{i} ,
% \end{equation*}
% for some $K _{i}$ with $\lim _{i \mapsto \infty} \frac{K _{i} }{E _{i} } =0 $.
%  \begin {figure}
%  \includegraphics[scale=.5]{Diagrambound.pdf} 
%  \caption {}
%  \label{fig:bound}
%  \end {figure}
% \begin{proof} [Proof of Theorem \ref{cor:MorseBott}]
%  We need to show:
% $$\mathcal{E} (C, A _{o} ) = \sum
% _{a,l} (-1)  \chi (\widetilde{R}  _{a,l} ),$$
%
%
% \begin{lemma}
%    The orbifold obstruction bundle to the moduli
% (orbifold) space $\widetilde{R} _{a,l}  $ is naturally isomorphic to the
% orbifold cotangent bundle of $\widetilde{R}  _{a,l} $.
% \end{lemma}
% \begin{proof} 
%    Given $[u] \in \widetilde{R} _{a,l}  $, by Serre duality 
% the fiber of the obstruction bundle at $u$ is $H ^{1} (N _{u}) \simeq
% H ^{0} (N ^{*}  _{u} \otimes K _{u}  ) $, where $K _{u} $ is the canonical
% line bundle of the domain curve. Now $K _{u} $ is holomorphically
% trivial, and $N
% _{u} $ is holomorphically trivial as it is a line bundle with
% non-trivial $H ^{0} (N _{u} ) = \ker D = T _{[u]} R
% _{a}$ (by the  
% Proposition \ref{prop:regular}) and vanishing degree. It follows that $H ^{0}
% (N ^{*}  _{u} \otimes K _{u}  ) \simeq (H ^{0} (N _{u} )) ^{*} = (\ker
% D) ^{*}  $. 
% (the isomorphism is natural once one chooses an isomorphism $H ^{0} (K
% _{u}) \simeq \mathbb{C}  $, however this isomorphism is also natural
% as domain of $u$ is biholomorphic to a torus $T ^{2}  _{a} $ as described in the proof
% Lemma \ref{lemma:constant}, for which there is a canonical
% identification $H ^{0} (T ^{2} _{a}  ) = \mathbb{C} $. Meanwhile the
% map $H^0 (K _{u} ) \to H ^{0} (T ^{2} _{a}  ) $ is obviously
% independent of the choice of the biholomorphism. (Any two are related
% by an automorphism of $T ^{2} _{a}   $, which act trivially on $H ^{0}
% (T ^{2} _{a}  )$).
%
%
% % Now if for a component $comp$ of  $\overline{\mathcal{M}}_{1}  ( J ^{\lambda}   
% % , A _{o} ) ^{a} $ $fog (comp)$ is not an orbifold point of $M _{1}
% % $, then by Proposition \ref{prop:regular} and the discussion above it is immediate that   
% %    the orbifold obstruction bundle $\mathcal{O}$ to the moduli
% %    orbifold $\overline{\mathcal{M}}_{1}  ( J ^{\lambda}   
% % , A _{o} ) ^{a} $ coincides with
% % the (orbifold) cotangent bundle of $R _{a} $. However if $fog (comp)$
% % is an orbifold point then an essentially identical argument as in the proof of
% % Lemma \ref{lemma:orbifold} gives that this still holds.
%
% % Suppose then that 
% % $fog$ maps to some orbifold point, on a connected component say $comp$ of
% % $\overline{\mathcal{M}}_{1}  ( J ^{\lambda}   
% % , A _{[\gamma]} ) ^{a} $, and let $R _{a, comp} $ be the associated
% % component of $R _{a} $.
% % There is then an orbifold bundle map from
% % $\mathcal{B}: \mathcal{O}| _{comp}   \to T ^{*,orb} R _{a, comp}   $  which is   just the
% % map that ``forgets'' the extra
% % $\mathbb{Z} _{2} $ symmetry group. We have an induced  bundle
% % map $B: O \to T ^{*} R _{a, comp}  $, which is clearly a bundle isomorphism, where $O$ is the underlying
% % vector bundle of $\mathcal{O}| _{comp} $, $T ^{*}R _{a,comp}  $ is the underlying
% % vector bundle of $T ^{*,orb}R _{a,comp}  $.
% % It clearly follows  that the Euler
% % number of $\mathcal{O}| _{comp}$ coincides with the Euler number of
% % $T ^{*,orb} R _{a, comp}  $, from which the conclusion readily follows.
% \end{proof}
% So the contribution to $\mathcal{E} (C,
% A _{o} )$ from $\widetilde{R}_{a,l}  $ is the orbifold Euler number of the orbifold cotangent
% bundle of $\widetilde{R}  _{a} $ or $- \chi (\widetilde{R}  _{a} )$.
% weighted by $(-1) ^{CZ (o) -n} $, with the weight coming from the
% calculation in the proof of Proposition \ref{prop:regular2}.asdf
% A note on orientations.  As usual we orient using Quillen's
% determinant line bundle over the space of real linear Cauchy-Riemann
% operators with fiber $\Det(\ker D)  ^{*} \otimes \Det (\coker D  ) 
% $, with $\Det$ denoting top exterior power. Sometimes one takes the
% dual on the second factor instead, the version above is Quillen's original
% definition which is rather intuitive, although it has a side
% effect of orienting the cotangent bundle when $\coker$ vanishes
% instead of
% the tangent bundle, but this is of no consequence. For a complex
% linear $D$ as in the geometric situation above, an orientation
% preserving (for the canonical complex orientations)
% isomorphism $\ker D  \to \coker D $
% gives a ``positive'' orientation element of the determinant line. 
%  In
% particular for a transverse (multi)-section of the orbifold cotangent bundle of $R _{a}
% $,  an intersection point $p$ contributes positively if the associated
% vertical tangent map $T _{p} R _{a,l} \to (T _{p} R _{a,l}) ^{*}   $
% preserves the complex orientation.
% % \textcolor{blue}{fix notation for elements of the moduli space} 
% % We may sometimes get rid of the
% % modulus $\mod 2$ as follows. 
% %
% % if we knew that the ``natural'' orientation  on the obstruction
% % bundle (coming from determinant line bundle of the CR operator) coincided
% % We now check the assertion on the orbifold Euler number of
% % $\chi (\widetilde{R} _{a,l}  )$.
% \begin{lemma} \label{lemma:constant}
%  On connected components the forgetful map
% $$fog: \widetilde{R}_{a} \to M _{1} $$ is constant. Moreover the
% image is away from orbifold points of $M _{1} $ unless $\frac{a}{2
% \pi} =1$.
% \end{lemma}
% \begin{proof}
% Each Reeb torus corresponding to action $a$ Reeb orbit $o$ is
% clearly (by geometry of the Reeb tori) bi-holomorphic to the quotient
% $T ^{2}  _{a} $,
% identifying opposite sides,
% of the rectangular domain in $\mathbb{C}$ with sides of length
% (for the standard Kahler metric) $1$ and $\frac{a}{2 \pi}$. So the
% complex structure on the domain curve for an element of
% $\widetilde{R}_{a}  $ is completely determined by the action of the
% underlying Reeb orbit. On the other hand the action is clearly
% constant on the connected component $R _{a} $. The last assertion
% clearly follows since $T ^{2} _{a}  $ has no symmetries unless
% $\frac{a}{2 \pi
% } =1$.
% \end{proof}
% It follows immediately from the lemma above that if for a component of $[u] \in \overline{\mathcal{M}}_{1}  ( J ^{\lambda}   
% , A _{[\gamma]} ) $, $fog ([u])$ is an orbifold point of $M _{1} $,
%  the same holds for
% the whole component of $[u]$ and likewise if $fog ([u])$  is not an
% orbifold point, the same holds for the whole component of $[u]$. 
% For a connected component $comp$ of  $\overline{\mathcal{M}}_{1}  ( J ^{\lambda}   
% , A _{o} ) ^{a} $ if its image under $fog$ is not an orbifold
% point of $M _{1} $ then $comp$  is tautologically identified as an orbifold with
% the corresponding connected component of the orbifold $R _{a} $. On
% the other hand if  $fog (comp)$ is an orbifold point of $M _{1}
% $, then $comp$ as an orbifold has a natural orbifold atlas with charts
% (in the notation of \cite{cite})
% $(\widetilde{U}, G \times \mathbb{Z} _{2}, \phi)$, induced by
% charts $(\widetilde{U}, G, \phi)$ for the orbifold $R _{a} $, $\phi$
% denoting $G$ invariant projection, where $G
% \times \mathbb{Z} _{2} $ acts on $\widetilde{U} $ as the induced
% action $G \times \mathbb{Z} _{2} \to G \to Diff (\widetilde{U}) $.
% Such an atlas is obviously equivalent to the original atlas since we
% have an embedding $(\widetilde{U}, G, pr) \to (\widetilde{U}, G \times
% \mathbb{Z} _{2},pr  )$. So $\overline{\mathcal{M}}_{1}  ( J ^{\lambda}   
% , A _{o} ) ^{a} $ is precisely the orbifold $R _{a} $ as
% expected.
%
% additional
% $\mathbb{Z} _{2} $ summand to the isotropy group for each point,
% corresponding to the isotropy group $\mathbb{Z}_{2} $ of orbifold
% points of $M _{1} $, or more precisely the local charts 
%
% however 
% However the underlying space of $comp$ and of the corresponding
% component of $R _{a} $ are obviously identical. 
%  \end{proof}
% \begin{proof} [Proof of Theorem \ref{thm:exampleS2}]
% The standard contact form is Morse-Bott with the corresponding components
% of the moduli space  $\widetilde{R}  _{k
% 2\pi} $, $k \geq 1$ being
% orbifold quotients by the trivial $\mathbb{Z} _{k} $ action of
%  $\mathbb{CP} ^{n}  $. 
%  In this case the associated Cauchy-Riemann operators are complex
%    linear  and the
% obstruction bundle for each component is just the (orbifold) cotangent bundle of
% $\mathbb{CP} ^{n}  $, by the proof of Corollary
% \ref{cor:MorseBott}, so the contribution from each component
% $\widetilde{R} _{k 2 \pi}  $, $k \geq 1$ is
% $ - \frac{1}{k}\chi (\mathbb{CP} ^{n} )$.  
% % We may then take a
% % perturbation of $\lambda$ by a perfect
% % Morse function on $\mathbb{CP} ^{n} $ to obtain a non-degenerate contact form $\lambda'$ on
% % $S ^{2k+1} $ of infinite negative type. The details for this kind of
% % perturbation appear for example in \cite{citeFredericBourgeois}.
% We may then construct abstract perturbations on each component
% $\widetilde{R} _{k 2 \pi}  $ corresponding to a section of the
% cotangent bundle of $\mathbb{CP} ^{n} $ which has isolated
% non-degenerate zeros with negative index, so that 
%  $\widetilde{R} 
% _{2 \pi} ^{vircycle} $ is a 0-dimensional oriented weighted branched
% manifold with negatively signed elements, and so
% $\lambda$  is
% of negative infinite type.
% \end{proof} 
% \begin{proof} [Proof of Theorem \ref{thm:exampleMorseBottextended}]
% % We give two proofs. The first avoids the use of the transcendental
% % Proposition \ref{prop:regular}.
% Given our holomorphic vector bundle $L \to M$, we get a holomorphic torus bundle as follows. 
% % Note that the Chern
% % connection induces a Kahler structure on the total space of $L$.
% Take $L-0$ that is $L$ minus the
% 0-section. This is a holomorphic $\mathbb{C} -0$ bundle with structure
% group $\mathbb{C} ^{\times} $. There is a natural holomorphic
% $\mathbb{Z}$ action on $\mathbb{C}-0$ with quotient the complex torus
% $T ^{2} $, this $\mathbb{Z}$ action clearly commutes with $\mathbb{C}
% ^{\times} $ action, so we get an induced holomorphic  $T ^{2} $ bundle
% $\overline{L}$
% over $M$. The total space of $\overline{L} $ is just $C \times S ^{1}
% $ and the simple degree 1 Reeb tori are the fibers of $\overline{L} $
% over $M$.  
% The Reeb manifolds $R _{a} $ in this case are copies of $M$ consisting of
%    $S ^{1} $-equivalence classes of
%  action $a$ Reeb orbits. If $[o] $ is non-torsion, with $ \langle \lambda, [o] \rangle =a  $,
% then the moduli space $\overline{\mathcal{M}}_{1}  (J ^{\lambda}   , A
% _{o} )$, is clearly identified with $\widetilde{R}  _{a} $; its elements
% degree $\frac{a}{2 \pi}$
% covering maps of the simple degree $1$ Reeb tori corresponding to
% elements $o \in R _{2 \pi} $.
% By the discussion in the  above
% paragraph, for each $[u] \in
% \widetilde{R}_{a}  $, corresponding to $[o] \in R _{a}  $ the
% corresponding Cauchy-Riemann operator $D  $ on $N _{u} $  is (naturally
% identified with) the Dolbeault operator
% for the trivial holomorphic vector bundle.
% More specifically we have by discussion
% above a holomorphic identification of a normal neighborhood of the
% simple Reeb torus with $(U \subset T ^{\mathbb{C}}  _{m _{o} }M) \times T
% ^{2}  $, for some $U \ni 0$, where $m _{o} \in M $ denotes the
% point whose fiber in $\overline{L} $ is the 
% simple degree 1 Reeb torus corresponding to $o$. 
% % \textcolor{blue}{notation check} 
% And so $\ker D $ on $N _{u} $
% is identified with $T ^{\mathbb{C}}  _{m _{o} } M $. 
% Clearly the
% orbifold $\widetilde{R} _{a}  $, $\frac{a}{2 \pi} \geq 1$ is the
% (non-effective) orbifold quotient
% of the smooth manifold $M$ by the trivial action of  $\mathbb{Z}
% _{\mult (o)} $.  In particular $\chi (\widetilde{R}_{a}  ) =
% \frac{1}{ \mult (o)}\chi
% (M)$.
% which is
% identified with $T_{o} R _{a} $. 
% the holomorphic normal bundle
% $N _{u} $ as follows. Let $\underline{\gamma} _u$ denote the simple orbit underlying
% $\gamma _{u} $.  
% The time $2 \pi$ linearized return map $\xi _{\gamma
% (0)}  C \to \xi _{\gamma (0)} $ induced by the $2 \pi $ periodic Reeb
% flow  is the identity map, where $\xi C$ is the contact distribution $\ker
% \alpha$. 
% Since the Reeb flow is actually Hermitian on the contact
% distribution for the metric
% induced by $(\omega, j)$ and is $2 \pi$ periodic, 
% it therefore determines flow invariant Hermitian splitting of the
% contact distribution along $\underline{\gamma} _u$, by
% fixing a Hermitian splitting into line bundles $\{\xi ^{i}
%    (\underline{\gamma} _{u} 
% (0))\} $  of the contant hyperplane at $\underline{\gamma} _{u} 
% (0)$: $\xi (\underline{\gamma} _{u} (0))$. There is then an obviously induced Reeb flow
% invariant Hermitian
% splitting into line bundles $\{\xi ^{i} \}$ of the contact plane distribution along
% $\underline{\gamma} _{u} $, 
% and so a Reeb flow invariant Hermitian splitting 
% into line bundles of degree 0, of the normal bundle $N _{\underline{u}} $ of the Reeb torus in
% $C \times S ^{1} $ associated to
% $\underline{\gamma} _{u} $. There is a canonical Hermitian connection $\nabla _{R} $ on $N _{\underline
% {u}} $
% which in the Reeb direction is induced by the Reeb flow and in the
% $\theta$ direction being trivial. Now a Hermitian connection determines a holomorphic structure on $N
% _{\underline{u}} $, and hence on $N _{u} $ (the pull-back by $u$ of $N 
% _{\underline {u}} $).
%
% In our case the associated $\overline{\partial} $ operator on $N _{u} $ is
% just the $D _{u} $ operator, which readily follows by the fact that
% $(M, L, \nabla, \omega, j)$ is a Kahler prequantization space, and the
% details are left to the reader.
% Since $\nabla _{R} $ preserves the Hermitian splitting,  the
% In particular the holomorphic vector bundle $(N _{u}, D _{u})  $ splits into a
% sum of
% degree 0 holomorphic line
% bundles $(N ^{i} _{u}, D _{u})   $. At this point  the argument of the Proof of
% \ref{prop:regular}, tells us the $\ker D _{u}| _{N ^{i} _{u}  }  $ is
% naturally isomorphic to the space of $2 \pi$ -periodic orbits
% of the linearized Reeb flow on $\xi ^{i} $.
% In particular $\ker D _{u} $ is naturally identified with the tangent space $T _{[u]} R _{a}  $.
% We may then proceed as in  the Proof of \ref{thm:exampleMorseBott}.
% Specifically, we identify the $D$-cokernel bundle over $\widetilde{R}  _{a} $
% with the cotangent bundle of $\widetilde{R}  _{a} $. 
% And so we conclude that
% $$\mathcal{E}(C \times
% S ^{1}, A _{o} )= - \chi (\widetilde{R}  _{a} )= - \frac{1}{ \mult (o)}\chi
% (M),$$ 
% for $\frac{a}{2 \pi } =1$, and 
% $$\mathcal{E}(C \times
% S ^{1}, A _{o} )= - \chi (\widetilde{R}  _{a} )= - \frac{1}{ \mult
% (o)}\chi
% (M),$$
% for 
%  $\frac{a}{2 \pi } >1$. 
%  from which the first part of the theorem
%  follows.  
% Alternatively the first part can also be proved by using Proposition
% \ref{prop:regular} 
% To prove the second part of the theorem, we note that the moduli space $\overline{\mathcal{M}}_{1}  (J ^{\lambda }   
% , A _{o} )  $ is then identified with $\sqcup _{k \geq 0}
% \widetilde{R}  _{(1 + kn) 2 \pi
% }  $, where $n$  is the order  of the image of
% $inc_*: \pi _{1} (S ^{1} ) \to \pi_1(C) $.  We still have that $\widetilde{R}
% _{(1 + kn) 2 \pi}$   is the orbifold quotient
% of the smooth manifold $M$ by the trivial action of $$\mathbb{Z}
% _{1 + kn}. $$ % while 
% % for $\frac{g + kng}{2 \pi} =1$ it is the orbifold quotient
% % of the smooth manifold $M$ by the trivial action of $\mathbb{Z} _{2} \times
% % \mathbb{Z} 
% % _{\mult (o)} $. 
% By the proof of the
% first part of the theorem, the obstruction bundle to each component
% $\widetilde{R} 
% _{(1 + kn) 2 \pi} $, is the (orbifold) cotangent bundle $T ^{*}M/ \mathbb{Z}
% _{1 + kn} $.
% It clearly follows from this that if $M$ admits a vector field with
% isolated non-degenerate zeros with positive index then on each
% component $\widetilde{R} 
% _{(1 + kn) 2 \pi} $ we may construct an abstract perturbation so
% that $\widetilde{R} 
% _{(1 + kn) 2 \pi} ^{vircycle} $ is a 0-dimensional oriented weighted branched
% manifold with negatively signed elements, and so
% $\lambda$  is
% of negative infinite type. 
% \end{proof}
\begin {appendices} 
\section {Fuller index} \label{appendix:Fuller} Let $X$ be a vector field  on $M$. Set \begin{equation*} S (X) = S (X, \beta) = 
   \{(o, p) \in L _{\beta} M \times (0, \infty) \,\left .  \right | \, \text{ $o: \mathbb{R}/\mathbb{Z} \to M $ is a
   periodic orbit of $p X $} \},
\end{equation*}
where $L _{\beta} M  $ denotes
the free homotopy class $\beta$ component of the free loop space.
Elements of $S (X)$ will be called orbits. There
is a natural $S ^{1}$ reparametrization action on $S (X)$, and elements of $S
(X)/S ^{1} $ will be called \emph{unparametrized orbits}, or just orbits. Slightly abusing notation we write $(o,p)$  for
the equivalence class of $(o,p)$. 
%
%
% Let $F _{\tau} $ denote the time $\tau$ flow map for $X$,
% and let $$(x,p) \in M \times
% (0, \infty) = M \times  \mathbb{R} _{+}  $$ be such that $F _{p } (x) =x $, call this pair an ``orbit''.
% and we say that two such orbits are equivalent $(o, p) \sim (o', p)$ if
% $x' =F _{\tau}  (x) $ for $0 \leq \tau \leq p$. We denote the equivalence class
% $[(x,p)]$ of
% $(x,p)$ by $o= (o,p)$ and call it an unparametrized periodic orbit in the Fuller's phase space $M
% \times \mathbb{R} _{+} $, although the word ``unparametrized'' will usually be
% omitted.
% % usual  period $p$
% orbits of $X$. 
% \textcolor{blue}{refactor} 
% For such a class $o$, we will also denote by $o$ the $S ^{1}
% $-reparametrization equivalence class of the corresponding closed integral curve $[0,
% p] \to M$, which we may also understand as an element of
% \begin{equation*}
% S (X) = 
%  \{(o, p) \in LM \times (0, \infty) 
% \,\left .  \right |
% \, \text{ $o: \mathbb{R}/\mathbb{Z} \to M $ is a
% periodic orbit of $p X $} \}.
% \end{equation*}
The multiplicity $m (o,p)$ of a periodic orbit is
the ratio $p/l$ for $l>0$ the least  period of $o$.
We want a kind of fixed point index which counts orbits
$(o,p)$ with certain weights - however in general to get
invariance we must
have period bounds. This is due to potential existence of sky catastrophes as
described in the introduction.

% Let $$emb: S (X) \hookrightarrow M \times (0,\infty)$$ be the embedding $(o,p) \mapsto (o
% (0),p)$ as before.
% Let $N \subset M \times \mathbb{R}_{+} $ be a
% compact \emph{dynamically isolated for $X$} set. This means that there is an open neighborhood $U \supset N$
% not containing image points of $emb$, which are not in $N$.
Let $N \subset S (X) $ be a compact open set.
Assume for simplicity that elements $(o,p) \in N$  are
isolated. (Otherwise we need to perturb.) Then
to such an $(N,X, \beta)$
Fuller associates an index: 
\begin{equation*}
   i (N,X, \beta) = \sum _{(o,p) \in N/ S ^{1}}  \frac{1}{m
   (o,p)} i (o,p),
\end{equation*}
 where $i (o,p)$ is the fixed point index of the time $p$ return
map of the flow of $X$ with respect to
a local surface of section in $M$ transverse to the image of $o$. 
Fuller then shows that $i (N, X, \beta )$ has the following invariance property.
Given a continuous homotopy $\{X _{t}
\}$,  $t \in [0,1]$ let \begin{equation*}  S ( \{X _{t} \}, \beta) = 
   \{(o, p, t) \in L _{\beta} M \times (0, \infty) \times [0,1] \,\left .  \right | \, \text{ $o: \mathbb{R}/\mathbb{Z} \to M $ is a
   periodic orbit of $p X _{t}  $} \}.
\end{equation*}


Given a continuous homotopy $\{X _{t}
\}$, $X _{0} =X $, $t \in [0,1]$, suppose that $\widetilde{N} $ is an open compact subset of $S (\{X _{t} \})$, such that $$\widetilde{N} \cap \left (LM \times   \mathbb{R} _{+} \times \{0\} \right) =N.
$$ Then if $$N_1 = \widetilde{N} \cap \left (LM \times   \mathbb{R} _{+}  \times \{1\} \right) $$ we have \begin{equation*}
i (N, X, \beta ) = i (N_1, X_1, \beta).
\end{equation*} 


In the case where $X$ is the $R ^{\lambda} $-Reeb vector field on a contact manifold $(C ^{2n+1} , \xi)$,
and if $(o,p)$ is
non-degenerate, we have: 
\begin{equation} \label{eq:conleyzenhnder}
i (o,p) = \sign \Det (\Id|
   _{\xi (x)}  - F _{p, *}
^{\lambda}| _{\xi (x)}   ) = (-1)^{CZ (o)-n},
\end{equation}
where $F _{p, *}
    ^{\lambda}$ is the differential at $x$ of the time $p$ flow map of $R ^{\lambda} $,
    and where $CZ ( o)$ is the Conley-Zehnder index, (which is a special
    kind of Maslov index) see
    \cite{citeRobbinSalamonTheMaslovindexforpaths.}.

There is also an extended Fuller index $i (X, \beta) \in \mathbb{Q} \sqcup \{\pm \infty\}$, for certain $X$ having definite type.
This is constructed in \cite{citeSavelyevFuller}, and is conceptually completely analogous to the extended Gromov-Witten invariant constructed in this paper.
\section {Virtual fundamental class} \label{sec:GromovWittenprelims}
This is a small note on how one deals with curves having non-trivial isotropy groups, in the virtual fundamental class technology. We primarily need this for the proof of Theorem \ref{thm:GWFullerMain}.
% In applications here our moduli spaces (for fixed $J$) will be given by orbifolds
% with orbifold obstruction bundles! So we do not have to worry about
% working with Kuranishi atlases. We need the virtual moduli cycle
% theory only to tell us how we should count (especially elements with
% symmetry groups) and to tell us that these counts are invariant.
Given a closed oriented orbifold $X$, with an orbibundle $E$ over $X$
Fukaya-Ono \cite{citeFukayaOnoArnoldandGW} show how to construct
using multi-sections its rational homology Euler
class, which when $X$ represents the moduli space of some stable
curves, is the virtual moduli cycle $[X] ^{vir} $.  (Note that the story of the
Euler class is older than the work of Fukaya-Ono, and there is
possibly prior work in this direction.)
When this is in degree 0, the corresponding Gromov-Witten invariant is $\int _{[X] ^{vir} } 1. $
However they  assume that their orbifolds are
effective. This assumption is not really necessary for the purpose of
construction of the Euler class but  is convenient for other technical reasons. A
different approach to the virtual fundamental class which emphasizes branched manifolds is used by
McDuff-Wehrheim, see for example McDuff
\cite{citeMcDuffNotesOnKuranishi}, which does not have
the effectivity assumption, a similar use of branched manifolds appears in
\cite{citeCieliebakRieraSalamonEquivariantmoduli}. In the case of a
non-effective orbibundle $E \to X$  McDuff \cite{citeMcDuffGroupidsMultisections}, constructs a homological
Euler class $e (E)$ using multi-sections, which  extends the construction
\cite{citeFukayaOnoArnoldandGW}.  McDuff shows that this class $e (E)$ is
Poincare dual to the completely formally natural cohomological Euler class of
$E$, constructed by other authors. In other words there is a natural notion of a
homological Euler class of a possibly non-effective orbibundle.
We shall assume the following black box property of the virtual fundamental
class technology.
\begin{axiom} \label{axiom:GW} Suppose that the moduli space of stable maps is cleanly cut out, 
   which means that it is represented by a (non-effective) orbifold $X$ with an orbifold
   obstruction bundle $E$, that is the bundle over $X$ of cokernel spaces of the linearized CR operators.
   Then the virtual fundamental class $[X]^ {vir} $
   coincides with $e (E)$.
\end{axiom}
Given this axiom it does not matter to us which virtual moduli cycle technique
we use. It is satisfied automatically by the construction of McDuff-Wehrheim,
(at the moment in genus 0, but surely extending).
It can be shown to be satisfied in the approach of John Pardon~\cite{citePardonAlgebraicApproach}.
And it is satisfied by the construction of Fukaya-Oh-Ono-Ohta
\cite{citeFOOOTechnicaldetails}, although not quiet immediately. This is also
communicated to me by Kaoru Ono. 
When $X$ is 0-dimensional this does follow
 immediately by the construction in
\cite{citeFukayaOnoArnoldandGW}, taking any effective Kuranishi neighborhood
at the isolated points of $X$, (this actually suffices for our paper.)  

% While it is always possible to reduce the general case to
% effective case on the level of Kuranishi structures, (when dimension
% of the ambient space is not zero) (c.f.
% \cite[Remark 4.2]{citeFOOOTechnicaldetails}), 
%  it is
% more natural in our context to work directly with non-effective
% orbifolds (since this is what we naturally get in many examples), as McDuff-Wehrheim tells us exactly how to compute in
% this context.  Note that the McDuff-Wehrheim theory at the moment is
% only in genus 0, however it is expected by the authors, and it seems
% rather believable that it extends without problems to higher genus
% case. That said our choice of virtual moduli cycle technology is
% motivated purely by convenience. If the
% reader is more familiar with the technology in
% \cite{citeFOOOTechnicaldetails} we are sure that they may easily adopt
% the arguments to use that. To do this, for the purpose of this paper,
% the reader need only to write out the virtual fundamental class construction in the
% special case when the moduli space is presented by a non-effective
% orbifold, with the obstruction bundle given by its tangent bundle. We
% claim and this is also communicated to me by Kaoru Ono that the resulting virtual fundamental class is just the class
% $[X] ^{vir} $ below, which is just the homology Euler class of the tangent
% bundle. 

As a special case most relevant to us here, suppose we have a moduli space
of elliptic curves in  $X$, which is regular with
expected dimension 0. Then its underlying space is a collection of oriented points.
However as some curves are multiply covered, and so have isotropy
groups,  we must treat this is a non-effective 0 dimensional oriented orbifold.
The contribution of each curve $[u]$ to the Gromov-Witten invariant $\int
_{[X] ^{vir} } 1 $ is $\frac{\pm 1}{[\Gamma ([u])]}$, where $[\Gamma ([u])]$ is
the order of the isotropy group $\Gamma ([u])$ of $[u]$, 
in the McDuff-Wehrheim setup this is explained in \cite[Section
5]{citeMcDuffNotesOnKuranishi}. In the setup of Fukaya-Ono
  \cite{citeFukayaOnoArnoldandGW} we may readily calculate to get the same thing
  taking any effective Kuranishi neighborhood
at the isolated points of $X$. 
\end{appendices}
% \subsection {Conley-Zehnder index}
\section{Acknowledgements} 
I thank Yong-Geun Oh for a number of discussions on related topics, and for an invitation to IBS-CGP, Korea. Thanks also to John Pardon for receiving me during a visit in Princeton. I also thank Dusa McDuff for comments on earlier versions.
\bibliographystyle{siam}  
%  \bibliography{/root/texmf/bibtex/bib/link}  
\bibliography{/home/yashasavelyev/texmf/bibtex/bib/link} 
\end {document}

% to fix, the Lambda functional, discussion on the Fullerindex, path connected %
