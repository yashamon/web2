
\documentclass{amsart}
\usepackage{appendix}
\usepackage{graphicx}
\usepackage[parfill]{parskip} 
\usepackage{amsfonts, amscd, mathrsfs, amsmath, amssymb, amsthm}
\usepackage{url}
\usepackage{hyperref} 
\hypersetup{backref,pdfpagemode=FullScreen,colorlinks=true}
\usepackage{tikz-cd}
\usepackage{}
\usepackage{color, verbatim}
\usepackage{bm}
\usepackage[hmargin=3cm,vmargin=3cm]{geometry}
\numberwithin{equation}{section}
\newtheorem{thm}[equation]{Theorem}
\newtheorem{axiom}[equation]{Axiom}
\newtheorem{theorem}[equation]{Theorem}
\newtheorem{proposition}[equation]{Proposition}
\newtheorem{lma}[equation]{Lemma} 
\newtheorem{lemma}[equation]{Lemma} 
\newtheorem{cpt}[equation]{Computation} 
\newtheorem{corollary}[equation]{Corollary} 
\newtheorem{clm}[equation]{Claim} 
\newtheorem{conjecture}{Conjecture}
\newtheorem{definition}[equation]{Definition}
\theoremstyle{definition}
\newtheorem{ft}{Fact}
\newtheorem{notation}{Notation}
\newtheorem{terminology}{Terminology}
\newtheorem{descr}{Description}[equation]
\theoremstyle{remark}
\newtheorem*{pf}{Proof}
\newtheorem*{pfs}{Proof (sketch)}
\newtheorem{remark}[equation]{Remark}
\newtheorem{example}{Example}
\newtheorem{question}{Question}
\newcommand {\vol}{\operatorname{vol}}
\newcommand{\hatcp}{\widehat{\mathbb {CP}} ^{r-1}}
\DeclareMathOperator {\spann} {span}
\DeclareMathOperator {\period} {period}
\DeclareMathOperator {\sign} {sign}
\DeclareMathOperator {\Id} {Id}
\DeclareMathOperator {\floor} {floor}
\DeclareMathOperator {\ceil} {ceil}
\DeclareMathOperator {\mult} {mult}
\DeclareMathOperator {\Symp} {Symp}
\DeclareMathOperator {\Det} {Det}
\DeclareMathOperator {\comp} {comp}
\DeclareMathOperator {\growth} {growth}
\DeclareMathOperator {\energy} {energy}
\DeclareMathOperator {\Reeb} {Reeb}
\DeclareMathOperator {\Lin} {Lin}
\DeclareMathOperator {\Diff} {Diff}
\DeclareMathOperator {\fix} {fix}
\DeclareMathOperator {\grad} {grad}
\DeclareMathOperator {\area} {area}
\DeclareMathOperator {\diam} {diam}
\DeclareMathOperator {\dvol} {dvol}
\DeclareMathOperator {\quant} {Quant}
\DeclareMathOperator {\ho} {ho}
\DeclareMathOperator {\length} {length}
\DeclareMathOperator {\Proj} {P}
\renewcommand{\i}{\sqrt{-1}}
\DeclareMathOperator{\mVol}{\mathrm{Vol}(M_0,\omega_0)}
\DeclareMathOperator{\Lie}{\mathrm{Lie}}
\DeclareMathOperator{\lie}{\mathrm{lie}}
\DeclareMathOperator{\op}{\mathrm{op}}
\DeclareMathOperator{\rank}{\mathrm{rank}}
\DeclareMathOperator{\ind}{\mathrm{ind}}
\DeclareMathOperator{\trace}{\mathrm{trace}}
\DeclareMathOperator{\image}{\mathrm{image}}
\DeclareMathOperator{\Sym}{\mathrm{Sym}}
\DeclareMathOperator{\Ham}{\mathrm{Ham}}
\DeclareMathOperator{\Aut}{\mathrm{Aut}}
\DeclareMathOperator{\Quant}{\mathrm{Quant}}
\DeclareMathOperator{\Fred}{\mathrm{Fred}}
\DeclareMathOperator{\id}{\mathrm{1}}
\DeclareMathOperator{\lcs}{lcs}
\DeclareMathOperator{\lcsm}{lcsm}
\DeclareMathOperator{\coker}{coker}
\begin{document}
\title{Topology HW}
\maketitle
\subsection*{Hw set 1 for wed}
1.3: 1  
2.1: 2, 3, 4, 5  
\subsection*{Hw set 2 for wed}
2.2: 1,2,3,4,5,6
\subsection*{Hw set 3 for wed}
3.1: 2, 4, 6 (extra credit), 8
\subsection*{Hw set 4 for wed}
3.2: 1,2,3,4,6
\subsection*{Hw set 5 for wed}
3.2: 7, 8,9, 12
\subsection*{Hw set 6 for wed}
3.3: 1,2,3,4,5,6   

% <!-- 3.1: 1, 4 (See chapter 2 for definition of seperable metric space)   -->
% <!---->
% <!---->
% <!-- Hw set 2 for wed   -->
% <!---->
% <!-- . -->
% <!---->
% <!---->
% <!---->
% <!-- Hw set 3 for wed  -->
% <!-- <!--  --> -->
% <!---->
% <!---->
% <!-- Hw set 4 for wed -->
% <!-- <!--  --> -->
% <!-- 3.2: 12   -->
% <!-- 3.3: 1,2,3   -->
% <!---->
% <!-- Hw set 5 for wed  -->
% <!---->
% <!-- 3:3: 4, 5, 6, 7, 8, 9 (extra credit)     -->
% <!---->
% <!---->
% <!-- HW set 6 -->
% <!---->
% <!-- 3.3: 10 -->
% <!-- 3.4: 1, 2, 3    -->
% <!---->
% <!-- HW set 7 -->
% <!---->
% <!-- 3.4: 4, 5, 6,       -->
% <!---->
% <!-- HW set 8 -->
% <!---->
% <!-- 3.4: 5, 6, 8, 9, 10, 11     -->
% <!-- -----  -->
% <!---->
% <!-- Set 9   -->
% <!-- 5.1: 1, 2, 3, 4   -->
% <!-- <!-- , 5, 7    --> -->
% <!---->
% <!-- Set 10: -->
% <!---->
% <!-- 5.1: 5,6,7    -->
% <!---->
% <!-- Set 11:  -->
% <!---->
% <!-- 5.2 1,2   -->
% <!---->
% <!---->
% <!---->
% <!---->
% <!---->
% <!-- <!-- 2, 3, 4, 5, 6 --> -->
% <!-- <!--  --> -->
% <!-- <!-- HW set 7 for wed --> -->
% <!-- <!--  --> -->
% <!-- <!--  --> -->
% <!-- <!-- HW set 8 for wed --> -->
% <!-- <!--  --> -->
% <!-- <!-- 5.2 1,2,3,4   --> -->
% <!-- <!--  --> -->
% <!-- <!--  --> -->
% <!-- <!--  --> -->
% <!-- <!-- <!--  --> --> -->
% <!-- <!-- <!-- Hw set 3 for wed --> --> -->
% <!-- <!-- <!--  --> --> -->
% <!-- <!-- <!-- 3.2: 2, 3, 6, 11, 12   --> --> -->
% <!-- <!-- <!-- 3.3: 1, 2,3   --> --> -->
% <!-- <!-- <!--  --> --> -->
% <!-- <!-- <!-- <!-- Hw set 4 for fri --> --> --> -->
% <!-- <!-- <!-- <!--  --> --> --> -->
% <!-- <!-- <!-- <!-- 3.3: 1,3 --> --> --> -->
% <!-- <!-- <!-- <!--  --> --> --> -->
% <!-- <!-- <!-- HW set 4 for wed --> --> -->
% <!-- <!-- <!-- <!--  --> --> --> -->
% <!-- <!-- <!-- <!-- 1. Show that rationals Q are totally disconnected with its topology inherited from R --> --> --> -->
% <!-- <!-- <!-- <!--  --> --> --> -->
% <!-- <!-- <!-- 3.3: 9,10   --> --> -->
% <!-- <!-- <!-- 3.4: 2, 3, 4 --> --> -->
% <!-- <!-- <!--  --> --> -->
% <!-- <!-- <!-- HW set 5 for wed --> --> -->
% <!-- <!-- <!--  --> --> -->
% <!-- <!-- <!-- 3.4: 5, 6, 8, 9 --> --> -->
% <!-- <!-- <!--  --> --> -->
% <!-- <!-- <!-- HW set 6 for wed --> --> -->
% <!-- <!-- <!--  --> --> -->
% <!-- <!-- <!-- 5.1: 1, 2, 3, 4 --> --> -->
% <!-- <!-- <!--  --> --> -->
% <!-- <!-- <!-- HW set 7 for wed --> --> -->
% <!-- <!-- <!--  --> --> -->
% <!-- <!-- <!-- 5.1: 5, 6, 7   (You already did it) --> --> -->
% <!-- <!-- <!--  --> --> -->
% <!-- <!-- <!-- HW 8 --> --> -->
% <!-- <!-- <!--  --> --> -->
% <!-- <!-- <!-- 5.2: 1,2,3,4   --> --> -->
% <!-- <!-- <!--  --> --> -->
% <!-- <!-- <!-- HW 9 --> --> -->
% <!-- <!-- <!--  --> --> -->
% <!-- <!-- <!-- In the proof of Theorem 17 show that the map is a homeomorphism onto image. --> --> -->
% <!-- <!-- <!--  --> --> -->
% <!-- <!-- <!-- pg 53: 1, 2, 7, 9   --> --> -->
% <!-- <!-- <!-- pg 96 2   --> --> -->
% <!-- <!-- <!--  --> --> -->
% <!-- <!-- <!-- HW 10 mon --> --> -->
% <!-- <!-- <!--  --> --> -->
% <!-- <!-- <!-- 1. Show that the mobius band $M=[0,1] \times R /(0,x) \sim (1,-x)$ is a vector bundle over $S^1$. That is check local triviality.   --> --> -->
% <!-- <!-- <!--  --> --> -->
% <!-- <!-- <!-- Lee: 3-1, 3-2, 3-4, 3-5   --> --> -->
% <!-- <!-- <!--  --> --> -->
% <!-- <!-- <!-- HW 11 fri --> --> -->
% <!-- <!-- <!--  --> --> -->
% <!-- <!-- <!-- Lee: 4-1, 4-2   --> --> -->
% <!-- <!-- <!--  --> --> -->
% <!-- <!-- <!-- HW 12 for fri --> --> -->
% <!-- <!-- <!--  --> --> -->
% <!-- <!-- <!-- Lee, Edition 2: --> --> -->
% <!-- <!-- <!--  --> --> -->
% <!-- <!-- <!-- 11-5,  11-11, 13-5   --> --> -->
% <!-- <!-- <!-- 14-1, 14-5, 14-6   --> --> -->
% <!-- <!-- <!--  --> --> -->
% <!-- <!-- <!-- Lee --> --> -->
% <!-- <!-- <!--  --> --> -->
% <!-- <!-- <!-- 16-2,  --> --> -->
% <!-- <!-- <!--  --> --> -->
% <!-- <!-- <!--  --> --> -->
% <!-- <!-- <!--  --> --> -->
% <!-- <!-- <!--  --> --> -->
\bibliographystyle{siam}
\bibliography{C:/Users/yasha/texmf/bibtex/bib/link}
%  ibliography{/root/texmf/bibtex/bib/link}
% ibliography{/home/yashasavelyev/texmf/bibtex/bib/link}
% ibliography{/home/yasha/texmf/bibtex/bib/link}
\end{document}
