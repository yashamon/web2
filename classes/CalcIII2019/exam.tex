\documentclass{amsart}  
\usepackage{appendix} 
  % \definecolor{gray}{rgb}{.80, .80, .80}
  % \pagecolor {gray}
 %\color {black}
\usepackage{url}        
 \usepackage{color} 
\usepackage[colorlinks]{hyperref}        
\usepackage{amsmath, amssymb, mathrsfs, amsfonts,  amsthm} 
\usepackage[all]{xy}
% \usepackage {mnsymbol}
\newcommand{\ls}{\Omega \Ham (M, \omega)} 
%\newcommand{\freels}{L(\textrm {Ham} (M, \omega))} 
\newcommand{\freels}{L \textrm {Ham}}  
% \newcommand{\Ham}{ \textrm {Ham} (M, \omega)}  
\newcommand{\eqfreels}{L ^{S^1} (\textrm {Ham})}  
\newcommand{\cM}{\mathcal M^*_ {0,2}(P_D, {A}, \{J_n\})}
\newcommand{\cMc}{\mathcal M^*_ {0,1}(P ^{\chi} _{f \ast g}, {A}, \{J_n\})}
\newcommand{\ham}{\Ham (S ^{2}, \omega)}

\newcommand{\hams}{\Ham (\Sigma, \omega)}
\newcommand{\df} {\Omega^t _{r,s}}  
\newcommand{\cMn}{\mathcal M_ {0}(A, \{J_n\})}  
\newcommand{\plus}{(s \times M \times D ^{2})_{+}}
\newcommand{\minus}{(s \times M\times D^2)_\infty}
\newcommand{\delbar}{\bar{\partial}}  
\newcommand {\base} {B}
\newcommand{\ring}{\widehat{QH}( \mathbb{CP} ^{\infty})}
\newcommand{\neigh}{{ \mathcal
{B}} _{U _{\max}}}
\newcommand{\paths}{\mathcal {P} _{g,l,m} (\phi _{ \{l\}} ^{-}, \phi _{ \{m\}} ^{+})}
% \newcommand {\Q}{ \operatorname{QH}_* (M)}
\newcommand{\Om}{\bar{\Omega}}
\newcommand{\G}{\bar {\Gamma}}
\newcommand{\fiber}{X}
%\newcommand{\P}{ P ^{\chi} _{f \ast g}}
\newcommand{\ilim}{\mathop{\varprojlim}\limits}
\newcommand{\dlim}{\mathop{\varinjlim} \limts}
\newcommand {\rect} {[0,4] \times [0,1]}
% \DeclareMathOperator{}{H}
% \DeclareMathOperator{\cohH}{H^*} 
\DeclareMathOperator {\injrad}{injrad}

\DeclareMathOperator {\id}{id}
\DeclareMathOperator {\Loop} {\Omega ^{\hbar} \Ham (S ^{2}, \omega)}
\DeclareMathOperator {\Loops} {\Omega ^{\hbar} \Ham (\Sigma, \omega)}
\DeclareMathOperator {\Ham} {Ham}
\DeclareMathOperator{\C}{\mathbb{R} \times S^{1}}
\DeclareMathOperator{\kareas}{K^s-\text{area}}
\DeclareMathOperator{\rank}{rank}
\DeclareMathOperator{\sys}{sys}
\DeclareMathOperator{\2-sys}{2-sys}
\DeclareMathOperator{\i-sys}{i-sys}
 \DeclareMathOperator{\1,n-sys}{(1,n)-sys}
% \DeclareMathOperator{\1-sys} {1-sys}
\DeclareMathOperator{\ho}{ho}
% \DeclareMathOperator{\index}{index}
\DeclareMathOperator{\karea}{K-\text{area}}
\DeclareMathOperator{\pb}{\mathcal {P} ^{n} _{B}}
\DeclareMathOperator{\area}{area}
\DeclareMathOperator{\ind}{index}
\DeclareMathOperator{\codim}{codim}
\DeclareMathOperator{\su}{\Omega SU (2n)}
\DeclareMathOperator{\im}{im}
\DeclareMathOperator{\coker}{coker}
\DeclareMathOperator{\interior}{int}
\DeclareMathOperator{\I}{\mathcal{I}}
\DeclareMathOperator{\cp}{ \mathbb{CP}^{n-1}}
\DeclareMathOperator{\Aut}{Aut}
\DeclareMathOperator{\Vol}{Vol}
\DeclareMathOperator{\Diff}{Diff}    
\DeclareMathOperator{\sun}{\Omega SU(n)}
\DeclareMathOperator{\gr}{Gr _{n} (2n)}
\DeclareMathOperator{\holG}{G _{\field{C}}}
\DeclareMathOperator{\energy}{c-energy}
\DeclareMathOperator{\qmaps}{\overline{\Omega}}
\DeclareMathOperator{\hocolim}{colim}
\DeclareMathOperator{\image}{image}
% \DeclareMathOperator{\maps}{\Omega^{2} X}
\DeclareMathOperator{\AG}{AG _{0} (X)}
\DeclareMathOperator{\maps}{{\mathcal {P}}}
\DeclareMathOperator {\B} {B _{\hbar/N} }
% \DeclareMathOperator {\P} {\mathcal{P} _{\delta/3}}
\newtheorem {theorem} {Theorem} [section] 
\newtheorem {note} [theorem]{Note} 
\newtheorem {problem} [theorem]{Research Problem} 
\newtheorem{conventions}{Conventions}
\newtheorem {hypothesis} [theorem] {Hypothesis} 
\newtheorem {conjecture} [theorem] {Conjecture}
\newtheorem{lemma}[theorem] {Lemma} 
\newtheorem {claim} [theorem] {Claim}
\newtheorem {question}  [theorem] {Question}
\newtheorem {observation}  [theorem] {Observation}
\newtheorem {definition} [theorem] {Definition} 
\newtheorem {proposition}  [theorem]{Proposition} 
\newtheorem {research} {Research Objective}
\newtheorem {corollary}[theorem]  {Corollary} 
\newtheorem {example} [theorem]  {Example}
\newtheorem {axiom}{Axiom}  
\newtheorem {notation}[theorem] {Notation}
\newtheorem {remark} [theorem] {Remark}
% \newtheorem {axiom2} {Axiom 2}
% \newtheorem {axiom3} {Axiom 3}
\numberwithin {equation} {section}
% \DeclareMathOperator{\image}{image}
\DeclareMathOperator{\grad}{grad}
\DeclareMathOperator{\obj}{obj}
\DeclareMathOperator{\colim}{colim} 
\begin{document} 
\title {Calculus III, Exam 1}
\maketitle
Please justify all answers.
\subsection* {Question 1}
\begin{enumerate}
   \item Find the partial derivatives $a,b,c$: $D_1 f (P) =a, D_2f
      (P)=b, D_3f (P)=b$, $P= (1,0,2)$ of $f(x,y,z)= z\sin (xy)+e
      ^{x}z ^{2}y  $.
   \item Find the gradient of the function $f$ at $P$: $\grad f (P)$ (please express the answer in terms of
      $a,b,c$).
   \item Find the tangent plane to the surface $f(x,y,z)=0$ at the point $P$
      (please express the answer in terms of
      $a,b,c$).
   \item Let $c (t)$ be a curve with $c (0)=P$, and $c' (0)=d$ what is the
      derivative at $0$: $(f \circ
      c)' (0)$ in terms of $a,b,c,d$? (Hint: use Chain rule.)
\end{enumerate}
\newpage
\subsection* {Question 2}
Let $f (x,y,z) = x ^{2} + 3y ^{2}$.
\begin{enumerate}
   \item   
Find critical points in the interior of the closed domain $D$ bounded by the sphere $x
^{2} +y ^{2} +z ^{2}=1   $.
\item Find local $\max$ and $\min$ in the interior of $D$, i.e. in
   $x
^{2} +y ^{2} +z ^{2}<1  $.
\item Find the critical points of $f$ on the boundary of $D$.
    (Hint it may help to use
   Lagrange multipliers.)
\item  Find the absolute $\max$ and $\min$ of $f$ on $D$. Justify your answer.
\end{enumerate} 
\newpage

\subsection* {Question 3}
\begin{enumerate}
\item Find the critical points for $$f (x,y)=x ^{2} e ^{y}   - 6y ^{2}.  $$
   \item Find the Taylor quadratic form $$q=f _{xx} (P)x ^{2} + 2f _{xy} (P) xy + f _{yy} (P) z
^{2},  $$  at each critical point $P$.
\item Determine whether the critical points of $f$ are local maxima, local
   minima, or saddle points. (Justify your answer.)
\end{enumerate}
\newpage

\subsection*{Question 4}
\label{sub:question_4}
\begin{enumerate}
   \item    
Let $F (x,y) = (y e ^{x} + 2x, e ^{y}) $ be a vector field on $\mathbb{R} ^{2}
$. Does it have a potential function? If so find it. Find the curve integral of
$F$:  $\int _{C} F $ for the curve $$c (t) = (sin (t ^{3} )e ^{t}t, t ^{4}+3t +1
),$$ for $0 \leq t
\leq 1$. 
 \item Same question for $F (x,y) = (x ^{2}\sin y, x )$ and the curve $c (t) =
    (t ^{2},t )$.
\end{enumerate}
\newpage

% subsection question_6 (end)
\subsection*{Question 5}
\label{sub:question_5}
Let $f (x,y) = x $. Find the double integral of $f$ over the region $D$
bounded by the curves $x ^{2} + y ^{2} =1  $, and $x=0$. (So $x,y \in D$
satisfies $x \geq 0$ and $x ^{2} + y ^{2} \leq 1  $).
\end{document}






