\documentclass{amsart}
\title{Modern Geometry}
\author{Yasha Savelyev}
\begin{document}
\maketitle
\section{}
my email: yashasavelyev@gmail.com
\\\\
HW Set 1 for wed
\\\\
Kiselev: 
\\
52, 53, 55, 60, 61, 67, 69, 75, 79, 89, 96, 97
\\\\
HW Set 2 for wed
\\
104, 105, 106, 107, 113, 114, 141, 143, 145, 150
\\\\
HW Set 3 for wed
\\
153, 154, 155, 158, 159, 178, 179, 180
\\\\
HW Set 4 for wed
\\
From taste of topology:
\\
2.1: 1,2 
2.3: 2,3 
\\\\
HW Set 5 for wed
\\
1) Sea $(X, d) $ un espacio metrico y sea $(X, L _{d})$ un length space induced by $d$. Demuestra que: $$\forall x,y \, \forall \gamma: d (x,y) \leq L (\gamma)$$ where $\gamma$  is a path from $x$ to $y$. 
\\\\
2) Let $X = \mathbb{R} ^{2}$ and $d _{T}$ the taxicab metric:
$$d _{T} (p _{1}, p _{2}) = |x _{1} - x _{2}| + |y _{1} - y _{2}|$$ for $p _{1} = (x _{1}, y _{1}) $ and $p _{2} = (x _{2}, y _{2})$.
Describe the minimal geodesics from $p _{1}$ to $p _{2}$ for $L _{d _{T}}$. 
\\\\
3) Let $X = \mathbb{R} ^{2}$, $d$  the Euclidean metric and
$d _{P}$ the Paris railroad metric with respect to $P \in
\mathbb{R} ^{2}$. So:
\begin{equation*}
d _{P} (p _{1}, p _{2}) = d (p _{1}, P) + d (p _{2}, P),
\end{equation*}
where $d$ is the Euclidean metric, and $p _{1}, p _{2} \in
\mathbb{R} ^{2} $.

Describe the minimal geodesics from $p _{1}$  to $p _{2}$ for $L _{d _{P}}$. 
\\\\
4) Sea $(\mathbb{R} ^{2}, L) $ un length space. Vamos
a llamar un mapeo inyectivo $$\gamma: \mathbb{R} ^{} \to
\mathbb{R} ^{2} $$ una \textbf{\emph{recta}}, si cado segmento $\gamma | _{[a,b]}$ de $\gamma$ es un geodesico minimal al respecto de $L$. Cuales son rectas en $(\mathbb{R} ^{2}, L _{d _{P}}) $
y cuales son rectas en $(\mathbb{R} ^{2}, L _{d _{T}}) $?
\\\\
5) Con ese definición de las rectas se cumplen los axiomas de geometría de Euclid para $(\mathbb{R} ^{2}, L _{d _{T}}) $ y para $(\mathbb{R} ^{2}, L _{d _{P}}) $? Justify.
\\\\
HW set 6 for wed
1) Let $\phi: (X, d _{X}) \to (Y, d _{Y})$ be a surjective
isometry of metric spaces, so that $\forall x,y \in X:
d _{X} (x, y) = d _{Y} (\phi (x), \phi (y))$. Show that
\begin{itemize}
\item  $\phi $ is continuous.
\item  $\phi $ is injective.
\item $\phi ^{-1}: (Y, d _{Y}) \to (X, d _{X}) $ is also an
isometry.
\end{itemize}
2) For $\phi $ as in problem 1. Show that $\gamma: [a,b] \to
X$ is
a minimal geodesic in $(X, L _{d _{X}})$ if and only if
$\phi \circ \gamma $ is a minimal geodesic in $(Y, L _{d _{Y}})$.
\\\\
3) Extra credit: If $X,Y$ are surfaces in $\mathbb{R} ^{3} $ and $d _{X},
d _{Y}$ the natural metrics as discussed in class. For $\phi $ as above, show that $\phi $ preserves angles between geodesics. 
\\\\
4) Extra credit: Use the above to show that for $\phi $ as
in problem 3), 
the holonomy around a closed piece-wise geodesic curve
$\gamma $ in $X$ coincides with the holonomy around a closed
piece-wise geodesic curve $\phi \circ \gamma $ in $Y$.
Conclude that the zero curvature condition is preserved by
an isometry as in problem 3).
\end{document}

