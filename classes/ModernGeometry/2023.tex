
\documentclass{amsart}
\usepackage{appendix}
\usepackage{graphicx}
\usepackage{amsfonts, amscd, mathrsfs, amsmath, amssymb, amsthm}
\usepackage{url}
\usepackage{hyperref} 
\hypersetup{backref,pdfpagemode=FullScreen,colorlinks=true}
\usepackage{tikz-cd}
\usepackage{}
\usepackage{color, verbatim}
\usepackage{bm}
\usepackage[hmargin=3cm,vmargin=3cm]{geometry}
\numberwithin{equation}{section}
\newtheorem{thm}[equation]{Theorem} 
\newtheorem{axiom}[equation]{Axiom} 
\newtheorem{theorem}[equation]{Theorem} 
\newtheorem{proposition}[equation]{Proposition}
\newtheorem{lma}[equation]{Lemma} 
\newtheorem{lemma}[equation]{Lemma} 
\newtheorem{cpt}[equation]{Computation} 
\newtheorem{corollary}[equation]{Corollary} 
\newtheorem{clm}[equation]{Claim} 
\newtheorem{conjecture}{Conjecture}
\newtheorem{definition}[equation]{Definition}
\theoremstyle{definition}
\newtheorem{ft}{Fact}
\newtheorem{notation}{Notation}
\newtheorem{terminology}{Terminology}
\newtheorem{descr}{Description}[equation]
\theoremstyle{remark}
\newtheorem*{pf}{Proof}
\newtheorem*{pfs}{Proof (sketch)}
\newtheorem{remark}[equation]{Remark}
\newtheorem{example}{Example}
\newtheorem{question}{Question}
\newcommand {\vol}{\operatorname{vol}}
\newcommand{\hatcp}{\widehat{\mathbb {CP}} ^{r-1}}
\DeclareMathOperator {\spann} {span}
\DeclareMathOperator {\period} {period}
\DeclareMathOperator {\sign} {sign}
\DeclareMathOperator {\Id} {Id}
\DeclareMathOperator {\floor} {floor}
\DeclareMathOperator {\ceil} {ceil}
\DeclareMathOperator {\mult} {mult}
\DeclareMathOperator {\Symp} {Symp}
\DeclareMathOperator {\Det} {Det}
\DeclareMathOperator {\comp} {comp}
\DeclareMathOperator {\growth} {growth}
\DeclareMathOperator {\energy} {energy}
\DeclareMathOperator {\Reeb} {Reeb}
\DeclareMathOperator {\Lin} {Lin}
\DeclareMathOperator {\Diff} {Diff}
\DeclareMathOperator {\fix} {fix}
\DeclareMathOperator {\grad} {grad}
\DeclareMathOperator {\area} {area}
\DeclareMathOperator {\diam} {diam}
\DeclareMathOperator {\dvol} {dvol}
\DeclareMathOperator {\quant} {Quant}
\DeclareMathOperator {\ho} {ho}
\DeclareMathOperator {\length} {length}
\DeclareMathOperator {\Proj} {P}
\renewcommand{\i}{\sqrt{-1}}
\DeclareMathOperator{\mVol}{\mathrm{Vol}(M_0,\omega_0)}
\DeclareMathOperator{\Lie}{\mathrm{Lie}}
\DeclareMathOperator{\lie}{\mathrm{lie}}
\DeclareMathOperator{\op}{\mathrm{op}}
\DeclareMathOperator{\rank}{\mathrm{rank}}
\DeclareMathOperator{\ind}{\mathrm{ind}}
\DeclareMathOperator{\trace}{\mathrm{trace}}
\DeclareMathOperator{\image}{\mathrm{image}}
\DeclareMathOperator{\Sym}{\mathrm{Sym}}
\DeclareMathOperator{\Ham}{\mathrm{Ham}}
\DeclareMathOperator{\Aut}{\mathrm{Aut}}
\DeclareMathOperator{\Quant}{\mathrm{Quant}}
\DeclareMathOperator{\Fred}{\mathrm{Fred}}
\DeclareMathOperator{\id}{\mathrm{1}}
\DeclareMathOperator{\lcs}{lcs}
\DeclareMathOperator{\lcsm}{lcsm}
\DeclareMathOperator{\coker}{coker}
\begin{document}
\title{Modern Geometry hw}
\author{Yasha Savelyev}
\thanks {}
\email{yasha.savelyev@gmail.com}
\address{University of Colima}
\maketitle
my email: yashasavelyev@gmail.com
\\

HW Set 1 for wed
\\

Kiselev: 
\\

52, 53, 55, 60, 61, 67, 69, 75, 79, 89, 96, 97
\\

HW Set 2 for wed
\\

104, 105, 106, 107, 113, 114, 141, 143, 145, 150
\\

HW Set 3 for wed
\\

153, 154, 155, 158, 159, 178, 179, 180
\\


HW Set 4 for wed
\\


From taste of topology:
\\

2.1: 1,2 
2.3: 2,3 
\\


HW Set 5 for wed
\\

1) Sea $(X, d) $ un espacio metrico y sea $(X, L _{d})$ un length space induced by $d$. Demuestra que: $$\forall x,y \, \forall \gamma: d (x,y) \leq L (\gamma)$$ where $\gamma$  is a path from $x$ to $y$. 
\\

2) Let $X = \mathbb{R} ^{2}$ and $d _{T}$ the taxicab metric:
$$d _{T} (p _{1}, p _{2}) = |x _{1} - x _{2}| + |y _{1} - y _{2}|$$ for $p _{1} = (x _{1}, y _{1}) $ and $p _{2} = (x _{2}, y _{2})$.
Describe the minimal geodesics from $p _{1}$ to $p _{2}$ for $L _{d _{T}}$. 
\\

3) Let $X = \mathbb{R} ^{2}$, $d$  the Eucleadian metric and $d _{P}$ the Paris railroad metric with respect to $P \in \mathbb{R} ^{2}$. Describe the minimal geodesics from $p _{1}$  to $p _{2}$ for $L _{d _{P}}$. 
\\

4) Vamos a llamar un mapeo injectivo $$\gamma: \mathbb{R} ^{} to \mathbb{R} ^{2} $$ una \textbf{\emph{recta}} si cado segmento $\gamma | _{[a,b]}$ de $\gamma$ es un geodesico minimal. Cuales son rectas en $(\mathbb{R} ^{2}, L _{d _{P}}) $
y cuales son rectas en $(\mathbb{R} ^{2}, L _{d _{T}}) $?
\\

5) Con eso definicion de las rectas se cumplen los axiomas de Euclid para $(\mathbb{R} ^{2}, L _{d _{T}}) $ y para $(\mathbb{R} ^{2}, L _{d _{P}}) $.

%  ibliography{/root/texmf/bibtex/bib/link}
% ibliography{/home/yashasavelyev/texmf/bibtex/bib/link}
% ibliography{/home/yasha/texmf/bibtex/bib/link}
\end{document}

