\documentclass{amsart}  
% \usepackage{appendix} 
  % \definecolor{gray}{rgb}{.80, .80, .80}
  % \pagecolor {gray}
 %\color {black}
\usepackage{url}        
\usepackage{color} 
\usepackage[colorlinks]{hyperref}        
\usepackage{amsmath, amssymb, mathrsfs, amsfonts,  amsthm} 
\usepackage[all]{xy}
% \usepackage {mnsymbol}
\newcommand{\ls}{\Omega \Ham (M, \omega)} 
%\newcommand{\freels}{L(\textrm {Ham} (M, \omega))} 
\newcommand{\freels}{L \textrm {Ham}}  
% \newcommand{\Ham}{ \textrm {Ham} (M, \omega)}  
\newcommand{\eqfreels}{L ^{S^1} (\textrm {Ham})}  
\newcommand{\cM}{\mathcal M^*_ {0,2}(P_D, {A}, \{J_n\})}
\newcommand{\cMc}{\mathcal M^*_ {0,1}(P ^{\chi} _{f \ast g}, {A}, \{J_n\})}
\newcommand{\ham}{\Ham (S ^{2}, \omega)}

\newcommand{\hams}{\Ham (\Sigma, \omega)}
\newcommand{\df} {\Omega^t _{r,s}}  
\newcommand{\cMn}{\mathcal M_ {0}(A, \{J_n\})}  
\newcommand{\plus}{(s \times M \times D ^{2})_{+}}
\newcommand{\minus}{(s \times M\times D^2)_\infty}
\newcommand{\delbar}{\bar{\partial}}  
\newcommand {\base} {B}
\newcommand{\ring}{\widehat{QH}( \mathbb{CP} ^{\infty})}
\newcommand{\neigh}{{ \mathcal
{B}} _{U _{\max}}}
\newcommand{\paths}{\mathcal {P} _{g,l,m} (\phi _{ \{l\}} ^{-}, \phi _{ \{m\}} ^{+})}
% \newcommand {\Q}{ \operatorname{QH}_* (M)}
\newcommand{\Om}{\bar{\Omega}}
\newcommand{\G}{\bar {\Gamma}}
\newcommand{\fiber}{X}
%\newcommand{\P}{ P ^{\chi} _{f \ast g}}
\newcommand{\ilim}{\mathop{\varprojlim}\limits}
\newcommand{\dlim}{\mathop{\varinjlim} \limts}
\newcommand {\rect} {[0,4] \times [0,1]}
% \DeclareMathOperator{}{H}
% \DeclareMathOperator{\cohH}{H^*} 
\DeclareMathOperator {\injrad}{injrad}

\DeclareMathOperator {\id}{id}
\DeclareMathOperator {\Loop} {\Omega ^{\hbar} \Ham (S ^{2}, \omega)}
\DeclareMathOperator {\Loops} {\Omega ^{\hbar} \Ham (\Sigma, \omega)}
\DeclareMathOperator {\Ham} {Ham}
\DeclareMathOperator{\C}{\mathbb{R} \times S^{1}}
\DeclareMathOperator{\kareas}{K^s-\text{area}}
\DeclareMathOperator{\rank}{rank}
\DeclareMathOperator{\sys}{sys}
\DeclareMathOperator{\2-sys}{2-sys}
\DeclareMathOperator{\i-sys}{i-sys}
 \DeclareMathOperator{\1,n-sys}{(1,n)-sys}
% \DeclareMathOperator{\1-sys} {1-sys}
\DeclareMathOperator{\ho}{ho}
% \DeclareMathOperator{\index}{index}
\DeclareMathOperator{\karea}{K-\text{area}}
\DeclareMathOperator{\pb}{\mathcal {P} ^{n} _{B}}
\DeclareMathOperator{\area}{area}
\DeclareMathOperator{\ind}{index}
\DeclareMathOperator{\codim}{codim}
\DeclareMathOperator{\su}{\Omega SU (2n)}
\DeclareMathOperator{\im}{im}
\DeclareMathOperator{\coker}{coker}
\DeclareMathOperator{\interior}{int}
\DeclareMathOperator{\I}{\mathcal{I}}
\DeclareMathOperator{\cp}{ \mathbb{CP}^{n-1}}
\DeclareMathOperator{\Aut}{Aut}
\DeclareMathOperator{\Vol}{Vol}
\DeclareMathOperator{\Diff}{Diff}    
\DeclareMathOperator{\sun}{\Omega SU(n)}
\DeclareMathOperator{\gr}{Gr _{n} (2n)}
\DeclareMathOperator{\holG}{G _{\field{C}}}
\DeclareMathOperator{\energy}{c-energy}
\DeclareMathOperator{\qmaps}{\overline{\Omega}}
\DeclareMathOperator{\hocolim}{colim}
\DeclareMathOperator{\image}{image}
% \DeclareMathOperator{\maps}{\Omega^{2} X}
\DeclareMathOperator{\AG}{AG _{0} (X)}
\DeclareMathOperator{\maps}{{\mathcal {P}}}
\DeclareMathOperator {\B} {B _{\hbar/N} }
% \DeclareMathOperator {\P} {\mathcal{P} _{\delta/3}}
\newtheorem {theorem} {Theorem} [section] 
\newtheorem {note} [theorem]{Note} 
\newtheorem {problem} [theorem]{Research Problem} 
\newtheorem{conventions}{Conventions}
\newtheorem {hypothesis} [theorem] {Hypothesis} 
\newtheorem {conjecture} [theorem] {Conjecture}
\newtheorem{lemma}[theorem] {Lemma} 
\newtheorem {claim} [theorem] {Claim}
\newtheorem {question}  [theorem] {Question}
\newtheorem {observation}  [theorem] {Observation}
\newtheorem {definition} [theorem] {Definition} 
\newtheorem {proposition}  [theorem]{Proposition} 
\newtheorem {research} {Research Objective}
\newtheorem {corollary}[theorem]  {Corollary} 
\newtheorem {example} [theorem]  {Example}
\newtheorem {axiom}{Axiom}  
\newtheorem {notation}[theorem] {Notation}
\newtheorem {remark} [theorem] {Remark}
% \newtheorem {axiom2} {Axiom 2}
% \newtheorem {axiom3} {Axiom 3}
\numberwithin {equation} {section}
% \DeclareMathOperator{\image}{image}
\DeclareMathOperator{\grad}{grad}
\DeclareMathOperator{\obj}{obj}
\DeclareMathOperator{\colim}{colim} 
\begin{document} 
\author {Yasha Savelyev} 
\email {yasha.savelyev@gmail.com}
\title {Notes, and homework problems for Calculus III}
\maketitle
\section* {Basic linear algebra}
\subsection* {Homework set 1}
From 1.5: 1, 4, 5 \\
From 1.6: 1, 2, 3, 8a, 9, 10, 12, 15, 16 \\
\section* {Derivatives}
\subsection* {Homework set 2}
2.1: 1, 2, 3, 4, 5, 7, 8 \\
2.2: 1 \\
3.1 1 \\
3.2 6, 13, 15 \\
3.4 10,11 \\
4.1 2, 3, 4 \\  
\section* {Maxima and minima}
\subsection* {Homework set 3}
5.1: 5, 6, 12 \\
5.2: 1, 2, 4, 5, 8 \\ 
\subsection* {Homework set 4}
5.3: 1, 5, 6, 10, 12, 25  \\
\section* {Potential functions, line integrals}
\subsection* {Homework set 5}
6.2: 3, 4 \\
6.3: 7, 8, 9, 10 \\
7.1: 1, 2 \\
7.2: 5, 6, 8 \\
8.1: 1, 2, 3 \\
\subsection* {Homework set 6}
8.2: 1 \\
8.3: 6,9, 15 \\
9.2: 1:a,b,c,d, 2:a,b, 3 \\

\subsection* {Homework set 7}
9.3: 1,2 \\
10.1: 1:a,b,c, 2 \\ 
\section* {Surface Integrals Divergence and Stokes theorem}
\subsection* {Homework set 8}
11.3: 1,2 \\
12.1: 2 \\
12.2: 1,2,3 \\
\subsection* {Homework set 9}
12.3: 1:a,b, 12 \\
12.5: 2,3,6 \\
12.6: 7, 9 \\
\section* {Change of variables}
\subsection* {Homework set 10}
1) Find the derivative a.k.a Jacobian $L'= DL$ of a linear map $L: \mathbb{R} ^{n} \to R^{m}
   $. \\
17.3: 1, 6, 8, 9 \\
17.5: 1, 2, 3 (just use change of variables method), 4 \\



% \section  {Tangent bundle}
% \subsection {Sheafs, and Tangent bundle as derivations}
%    A few clarifying points on sheafs, and derivations as tangent vectors that we went over
%  in class. We may define something like the tangent bundle
%  $\mathcal{T}(X, \mathcal{F})$ for a
%  somewhat 
%  general sheaf of $\mathbb{R}$-algebras  $\mathcal{F}$ over $X$.
%  We shall make a few assumptions on $\mathcal{F}$ to make this cleaner. First we shall
%  assume that $\mathcal{F} (U)$ is a subalgebra of the algebra of
%  continuous functions on $U$ for each $U$. Second we shall assume the
%  following: for every $U$ and $p \in U$ there is an $f \in \mathcal{F}
%  (U)$, $f (p)=1$, $support f \subset U $.
%  We then define $\mathcal{T} _{p} (X, \mathcal{F}) $ as the space  of
%  $\mathbb{R}$-derivations: linear maps $l: \mathcal{F} (U) \to \mathbb{R}$,  $ U \ni p$, which
%  satisfy the Leibnitz rule: $l (f\cdot g) = l (f) g (p) + f (p) l
%  (g)$. 
%
% Exercise 1: Use the second property of $\mathcal{F}$ to show that if $f$ vanishes
% in a neighborhood of $p$ and $l$ is a derivation then $l (f) =0$. 
%
% Exercise 2: Show that $\mathcal{T} _{p}X $ is well defined i.e. is
% independent of $U$. (We really mean here that it is well defined up to
% a natural isomorphism.)
% (Hint: Use exercise 1.)
%
% Exercise 3. Show that if $X=\mathbb{R} ^{n} $ and the sheaf
% $\mathcal{F}$ is the sheaf of smooth functions then, $\mathcal{T} _{p}
% \mathbb{R} ^{n} $ is naturally isomorphic to $\mathcal{T} _{0}
% \mathbb{R} ^{n}  $ for each $p$. (Hint: use the pushforward map of
% derivations that we discussed in class.)
%
% We have already shown in class in $\mathcal{T} _{0} \mathbb{R} ^{n}  $
% is naturally isomorphic to $\mathbb{R} ^{n} $, in fact the natural
% derivations $D _{i}| _{0}  = \frac{\partial}{\partial x _{i} }| _{0}  $ are a basis.
%
% Consequently we may get from this that for a smooth manifold $M$ defined as a
% sheaf of $\mathbb{R}$-algebras on a topological space $M$, locally isomorphic to the sheaf of
% $\mathbb{R}$-algebras of smooth functions on $\mathbb{R} ^{n} $, there is a well
% defined tangent bundle $\mathcal{T} M$, which is a rank $n$ real
% vector bundle, (fibers are $\mathbb{R}^{n} $)   which is constructed with just
% algebraic data. So there are no $C ^{\infty} $ related charts, atlases or other
% such strange things. This story is equivalent to the other story of
% smooth manifolds defined in terms of smooth atlases, with the
% corresponding tangent bundle, defined in terms of equivalence classes
% of smooth curves for example. And this is what we shall go back to.
% Although it will sometimes still be convinient to speak in terms of
% derivations, which of course still make sense in any case.
%
% This completes our brief excursion into sheafs. 
% \section {Tangent bundle hw problems, due thurs}
% Ex1: Check that the Mobius band is a vector bundle over $S ^{1} $,
% i.e. check local triviality.
% From Spivak Ch3:
%   ex 5 (pick any of the i, ii, iii), 
%  10, 14.  
% \section {Cotangent bundle, due thurs}
% From Spivak: Ch4 1, 2, 3.
% \section {Differential forms}
% Ex0: Let $\{e _{i} \}$ be the standard basis for $\mathbb{R} ^{n} $
% and let $\{e ^{*} _{i} \}  $ be the dual basis. Show that the
% $n$-tensor $e _{1} ^{*} \wedge \ldots e _{n} ^{*}    $, is the
% ``determinant'' tensor. Hint: you only need to know how they evaluate
% on the standard basis, and then use skew-symmetry, multilinearity. \\
% Ex1: Check that the differential satisfies $d ^{2}=0 $. \\
% From Spivak, Ch 7, 27.
% \section {Integration of differential forms}
% From Spivak ch 8: 2, 3, 4, 7.
% \section {Stokes theorem}
% Ex0: Suppose  that  $M, N$ are without boundary. Show that if $f: M ^{k}  \to N$ is homotopic to $g$  then for a
% $k$-form $\omega$ on $N$ $\int _{M} f ^{*} \omega = \int _{M} g
% ^{*} \omega $. Hint: Consider first the case where support of $\omega$
% is in the interiour of the image of a singular cube $f \circ c$, $c$
% is a singular cube into $M$. Then look at 
% the new singular cube $f \circ c \times I \to N$ induced by the
% homotopy what does Stokes theorem tell you? 
% (This allows us to
% define degree without the much more difficult theorem 13, in Spivak,
% which requires construction of something called ``chain homotopy'').
% \\ Spivak ch8: 9,13 \\ 
\end{document}






