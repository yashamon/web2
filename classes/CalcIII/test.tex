\documentclass{amsart}
\usepackage{appendix}
\usepackage{graphicx}
\usepackage{amsfonts, amscd, mathrsfs, amsmath, amssymb, amsthm}
\usepackage{url}
\usepackage{hyperref} 
\hypersetup{backref,pdfpagemode=FullScreen,colorlinks=true}
\usepackage{tikz-cd}
\usepackage{}
\usepackage{color, verbatim}
\usepackage{bm}
\usepackage[hmargin=3cm,vmargin=3cm]{geometry}
\numberwithin{equation}{section}
\newtheorem{thm}[equation]{Theorem} 
\newtheorem{axiom}[equation]{Axiom} 
\newtheorem{theorem}[equation]{Theorem} 
\newtheorem{proposition}[equation]{Proposition}
\newtheorem{lma}[equation]{Lemma} 
\newtheorem{lemma}[equation]{Lemma} 
\newtheorem{cpt}[equation]{Computation} 
\newtheorem{corollary}[equation]{Corollary} 
\newtheorem{clm}[equation]{Claim} 
\newtheorem{conjecture}{Conjecture}
\newtheorem{definition}[equation]{Definition}
\theoremstyle{definition}
\newtheorem{ft}{Fact}
\newtheorem{notation}{Notation}
\newtheorem{terminology}{Terminology}
\newtheorem{descr}{Description}[equation]
\theoremstyle{remark}
\newtheorem*{pf}{Proof}
\newtheorem*{pfs}{Proof (sketch)}
\newtheorem{remark}[equation]{Remark}
\newtheorem{example}{Example}
\newtheorem{question}{Question}
\newcommand {\vol}{\operatorname{vol}}
\newcommand{\hatcp}{\widehat{\mathbb {CP}} ^{r-1}}
\DeclareMathOperator {\spann} {span}
\DeclareMathOperator {\D} {Delta}
\DeclareMathOperator {\obj} {obj}
\DeclareMathOperator {\colim} {colim}
\DeclareMathOperator {\period} {period}
\DeclareMathOperator {\sign} {sign}
\DeclareMathOperator {\Id} {Id}
\DeclareMathOperator {\floor} {floor}
\DeclareMathOperator {\ceil} {ceil}
\DeclareMathOperator {\mult} {mult}
\DeclareMathOperator {\Symp} {Symp}
\DeclareMathOperator {\Det} {Det}
\DeclareMathOperator {\comp} {comp}
\DeclareMathOperator {\growth} {growth}
\DeclareMathOperator {\energy} {energy}
\DeclareMathOperator {\Reeb} {Reeb}
\DeclareMathOperator {\Lin} {Lin}
\DeclareMathOperator {\Diff} {Diff}
\DeclareMathOperator {\fix} {fix}
\DeclareMathOperator {\grad} {grad}
\DeclareMathOperator {\area} {area}
\DeclareMathOperator {\diam} {diam}
\DeclareMathOperator {\dvol} {dvol}
\DeclareMathOperator {\quant} {Quant}
\DeclareMathOperator {\ho} {ho}
\DeclareMathOperator {\length} {length}
\DeclareMathOperator {\Proj} {P}
\renewcommand{\i}{\sqrt{-1}}
\DeclareMathOperator{\mVol}{\mathrm{Vol}(M_0,\omega_0)}
\DeclareMathOperator{\Lie}{\mathrm{Lie}}
\DeclareMathOperator{\lie}{\mathrm{lie}}
\DeclareMathOperator{\op}{\mathrm{op}}
\DeclareMathOperator{\rank}{\mathrm{rank}}
\DeclareMathOperator{\ind}{\mathrm{ind}}
\DeclareMathOperator{\trace}{\mathrm{trace}}
\DeclareMathOperator{\image}{\mathrm{image}}
\DeclareMathOperator{\Sym}{\mathrm{Sym}}
\DeclareMathOperator{\Ham}{\mathrm{Ham}}
\DeclareMathOperator{\Aut}{\mathrm{Aut}}
\DeclareMathOperator{\Quant}{\mathrm{Quant}}
\DeclareMathOperator{\Fred}{\mathrm{Fred}}
\DeclareMathOperator{\id}{\mathrm{1}}
\DeclareMathOperator{\lcs}{lcs}
\DeclareMathOperator{\lcsm}{lcsm}
\DeclareMathOperator{\coker}{coker}
\begin{document}
\title{}
\author{Yasha Savelyev}
\thanks {}
\email{yasha.savelyev@gmail.com}
\address{University of Colima, CUICBAS}
\keywords{}
\begin{abstract}   
\end{abstract}
\maketitle
\section{}
my email: yashasavelyev@gmail.com

HW Set 1 for wed

12.4: 1  \\
12.5: 2 ,3, 12, 13, 23, 25, 59, 65, 71  \\

HW Set 2 for wed

13.3: 3, 4, 5, 8, 14, 18, 28  \\
13.4: 15, 15, 19 \\
14.1: 4, 32, 33, 30

HW Set 3 for wed

14.2: 11, 12, 17, 18, 19, 43, 50 \\
14.3: 6, 33, 34, 37, 41, 77 \\
14.4: 1,2, 23, 25 \\

HW Set 4 for wed

14.5: 1, 2, 11, 12, 15, 17, 18, 51 \\

HW Set 5 for wed \\

1) Using the Chain rule in the book, prove the version of the chain rule:
For $c: [a,b] \to \mathbb{R} ^{n}$ and $f: \mathbb{R} ^{n}
\to \mathbb{R}$, $$\frac{d}{dt} f \circ c (t)  = grad f (c (t)
) \cdot c' (t).$$ 
14.6: 13, 20, 26 \\
14.7: 5, 6, 7, 23, 34 \\

% <!-- 1.5: 1, 4a,b, 5   -->
% <!-- 1.6: 1, 3, 7a,b, 10, 15   -->
% <!-- 1.7: 1, 6 -->
% <!-- 2.1: 1,2, 3,4   -->
% <!-- 2.2: 1,2 -->
% <!---->
% <!-- HW set 2 for wed -->
% <!---->
% <!-- 3.2: 1,2,6   -->
% <!-- 3.3: 1,2,3,4   -->
% <!-- 3.4: 10,11    -->
% <!-- 4.1: 1, 2, 3, 4      -->
% <!-- extra: Show that our definition of differentiability of a map $\mathbb {R} ^n \to \mathbb {R} ^m$, when $n=m=1$ coincides with the usual differentiability of a function. -->
% <!---->
% <!-- HW set 3 for wed -->
% <!-- 4.2: 1abc, 2, 3a, 5a, 12, 13   -->
% <!-- 4.3: 1,3,4   -->
% <!-- 4.4:  1, 3   -->
% <!-- 4.5: 1, 2, 3   -->
% <!-- 5.1: 5, 6, 12   -->
% <!-- 5.2: 1, 2, 4, 5, 8  -->
% <!---->
% <!-- HW set 4 for wed -->
% <!---->
% <!-- 5.3: 1, 5, 6, 10, 12, 25   -->
% <!-- 6.2: 1, 2, 3 a,b,c, 4  -->
% <!-- 6.3: 1, 7, 8, 9, 10, 11   -->
% <!-- 7.1: 1, 2   -->
% <!---->
% <!-- HW set 5 -->
% <!---->
% <!-- 7.2: 5, 6, 8, 11    -->
% <!-- 7.3: 1,2 -->
% <!-- 8.1: 1, 2, 3, 8, 9     -->
% <!-- 8.2: 1   -->
% <!-- 8.3: 6,9, 15      -->
% <!---->
% <!-- HW set 5 -->
% <!---->
% <!-- 9.2: 1:a,b,c,d, 2:a,b, 3, 5, 9   -->
% <!-- 9.3: 1,2, 3,4, 15      -->
% <!-- 10.1: 1:a,b,c, 2, 5, 6      -->
% <!---->
% <!-- HW set 6 -->
% <!---->
% <!-- 10.2: 2   -->
% <!-- 11.1: 1,2,3,4   -->
% <!-- 11.2: 1,2,4,9,13    -->
% <!---->
% <!-- HW set 7 -->
% <!---->
% <!-- 12.1: 1, 2   -->
% <!-- 12.2: 1,2,3    -->
% <!-- 12.3: 1,2, 10   -->
% <!-- 12.4: 6   -->
% <!-- 12.5: 2, 4, 6, 14 -->
% <!-- 12.6: 5 -->
% <!---->
% <!-- HW set 8 -->
% <!---->
% <!-- Q1: What is the Jacobian $L'$ of a linear map $L: \mathbb{R} ^{n}  \to \mathbb{R} ^{m} $? Give a proof.  -->
% <!---->
% <!-- 17.5: 1,2,3 -->
%
% <!-- <!-- 5a,b, 7, 8, 9, 13   --> -->
% <!-- <!--  --> -->
% <!-- HW Set 3, due wed   -->
% <!-- <!--  --> -->
% <!--  -->
% <!-- 3.1: 1,2   -->
% <!--  -->
% <!-- HW set 4, due wed   -->
% <!-- <!-- , 13, 15   --> -->
% <!--  -->
% <!-- HW set 5, due wed    -->
% <!--  -->
% <!--  -->
% <!-- HW set 6, wed -->
% <!--  -->
% <!-- 5.3: 1, 5, 6, 10, 12, 25   -->
% <!-- 6.2: 2, 3, 4   -->
% <!--  -->
% <!--  -->
% <!-- HW set 7, wed    -->
% <!--  -->
% <!-- 6.3: 7, 8, 9, 10, 11   -->
% <!-- 7.1: 1, 2   -->
% <!-- 7.2: 5, 6, 8, 11    -->
% <!--  -->
% <!-- HW set 8 -->
% <!--  -->
% <!-- 8.1: 1, 2, 3, 8, 9     -->
% <!-- <!-- 8.2: 1   --> -->
% <!-- <!-- 8.3: 6,9, 15   -->  -->
% <!--  -->
% <!-- HW set 9  -->
% <!-- 8.2: 1,2    -->
% <!-- 8.3: 1, 6,9, 12, 13, 15   -->
% <!--  -->
% <!-- HW set 10 -->
% <!--  -->
% <!-- 9.2: 1:a,b,c,d, 2:a,b, 3, 5, 9   -->
% <!--  -->
% <!-- Set 11 -->
% <!--  -->
% <!-- 9.3: 1,2, 3,4, 15      -->
% <!-- 10.1: 1:a,b,c, 2, 5, 6       -->
% <!--  -->
% <!-- Set 12 -->
% <!--  -->
% <!-- 11.1: 1,2,3,4   -->
% <!-- 11.2: 1,2,4,9, 13    -->
% <!--  -->
% <!-- Set 13    -->
% <!--  -->
% <!-- 12.1: 1, 2   -->
% <!-- 12.2: 1,2,3    -->
% <!-- 12.3: 1,2, 10   -->
% <!-- 12.4: 6   -->
% <!-- 12.5: 2, 4, 6, 14   -->
% <!--  -->
% <!--  -->
% <!-- Set 14    -->
% <!--  -->
% <!-- Q1: What is the Jacobian $L'$ of a linear map $L: R^n \to R^m $? Give a proof. -->
% <!-- 17.5: 1,2,3   -->
% <!--  -->
% <!-- ======= -->
% <!--  -->
% <!--  -->
% <!-- <!--<h1 id="maxima-and-minima" class="unnumbered">Maxima and minima</h1>--> -->
% <!-- <!--<h2 id="homework-set-3" class="unnumbered">Homework set 3</h2>--> -->
% <!-- <!--<p>5.1: 5, 6, 12<br />--> -->
% <!-- <!--<br />--> -->
% <!-- <!--</p>--> -->
% <!-- <!--<h2 id="homework-set-4" class="unnumbered">Homework set 4</h2>--> -->
% <!-- <!--<p>5.3: 1, 5, 6, 10, 12, 25<br />--> -->
% <!-- <!--</p>--> -->
% <!-- <!--<h1 id="potential-functions-line-integrals" class="unnumbered">Potential functions, line integrals</h1>--> -->
% <!-- <!--<h2 id="homework-set-5" class="unnumbered">Homework set 5</h2>--> -->
% <!-- <!--<p>6.2: 3, 4<br />--> -->
% <!-- <!--6.3: 7, 8, 9, 10<br />--> -->
% <!-- <!--7.1: 1, 2<br />--> -->
% <!-- <!--7.2: 5, 6, 8<br />--> -->
% <!-- <!--8.1: 1, 2, 3<br />--> -->
% <!-- <!--</p>--> -->
% <!-- <!--<h2 id="homework-set-6" class="unnumbered">Homework set 6</h2>--> -->
% <!-- <!--<p>8.2: 1<br />--> -->
% <!-- <!--8.3: 6,9, 15<br />--> -->
% <!-- <!--9.2: 1:a,b,c,d, 2:a,b, 3<br />--> -->
% <!-- <!--</p>--> -->
% <!-- <!--<h2 id="homework-set-7" class="unnumbered">Homework set 7</h2>--> -->
% <!-- <!--<p>9.3: 1,2<br />--> -->
% <!-- <!--10.1: 1:a,b,c, 2<br />--> -->
% <!-- <!--</p>--> -->
% <!-- <!--<h1 id="surface-integrals-divergence-and-stokes-theorem">Surface Integrals Divergence and Stokes theorem</h1>--> -->
% <!-- <!--<h2 id="homework-set-8" class="unnumbered">Homework set 8</h2>--> -->
% <!-- <!--<p>11.3: 1,2<br />--> -->
% <!-- <!--12.1: 2<br />--> -->
% <!-- <!--12.2: 1,2,3<br />--> -->
% <!-- <!--</p>--> -->
% <!-- <!--<h2 id="homework-set-9" class="unnumbered">Homework set 9</h2>--> -->
% <!-- <!--<p>12.3: 1:a,b, 12<br />--> -->
% <!-- <!--12.5: 2,3,6<br />--> -->
% <!-- <!--12.6: 7, 9<br />--> -->
% <!-- <!--</p>--> -->
% <!--  -->
% \bibliographystyle{siam}
% \bibliography{C:/Users/yasha/texmf/bibtex/bib/link}
% %  ibliography{/root/texmf/bibtex/bib/link}
% % ibliography{/home/yashasavelyev/texmf/bibtex/bib/link}
% % ibliography{/home/yasha/texmf/bibtex/bib/link}
\end{document}
