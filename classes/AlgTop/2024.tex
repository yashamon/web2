
\documentclass{amsart}
\usepackage{appendix}
\usepackage{graphicx}
\usepackage[parfill]{parskip} 
\usepackage{amsfonts, amscd, mathrsfs, amsmath, amssymb, amsthm}
\usepackage{url}
\usepackage{hyperref} 
\hypersetup{backref,pdfpagemode=FullScreen,colorlinks=true}
\usepackage{tikz-cd}
\usepackage{}
\usepackage{color, verbatim}
\usepackage{bm}
\usepackage[hmargin=3cm,vmargin=3cm]{geometry}
\numberwithin{equation}{section}
\newtheorem{thm}[equation]{Theorem}
\newtheorem{axiom}[equation]{Axiom}
\newtheorem{theorem}[equation]{Theorem}
\newtheorem{condition}[Condition]{Theorem}
\newtheorem{proposition}[equation]{Proposition}
\newtheorem{lma}[equation]{Lemma} 
\newtheorem{lemma}[equation]{Lemma} 
\newtheorem{cpt}[equation]{Computation} 
\newtheorem{corollary}[equation]{Corollary} 
\newtheorem{clm}[equation]{Claim} 
\newtheorem{conjecture}{Conjecture}
\newtheorem{definition}[equation]{Definition}
\theoremstyle{definition}
\newtheorem{ft}{Fact}
\newtheorem{notation}{Notation}
\newtheorem{terminology}{Terminology}
\newtheorem{descr}{Description}[equation]
\theoremstyle{remark}
\newtheorem*{pf}{Proof}
\newtheorem*{pfs}{Proof (sketch)}
\newtheorem{remark}[equation]{Remark}
\newtheorem{example}{Example}
\newtheorem{question}{Question}
\newcommand {\vol}{\operatorname{vol}}
\newcommand{\hatcp}{\widehat{\mathbb {CP}} ^{r-1}}
\DeclareMathOperator {\spann} {span}
\DeclareMathOperator {\period} {period}
\DeclareMathOperator {\sign} {sign}
\DeclareMathOperator {\Id} {Id}
\DeclareMathOperator {\floor} {floor}
\DeclareMathOperator {\ceil} {ceil}
\DeclareMathOperator {\mult} {mult}
\DeclareMathOperator {\Symp} {Symp}
\DeclareMathOperator {\Det} {Det}
\DeclareMathOperator {\comp} {comp}
\DeclareMathOperator {\growth} {growth}
\DeclareMathOperator {\energy} {energy}
\DeclareMathOperator {\Reeb} {Reeb}
\DeclareMathOperator {\Lin} {Lin}
\DeclareMathOperator {\Diff} {Diff}
\DeclareMathOperator {\fix} {fix}
\DeclareMathOperator {\grad} {grad}
\DeclareMathOperator {\area} {area}
\DeclareMathOperator {\diam} {diam}
\DeclareMathOperator {\dvol} {dvol}
\DeclareMathOperator {\quant} {Quant}
\DeclareMathOperator {\ho} {ho}
\DeclareMathOperator {\length} {length}
\DeclareMathOperator {\Proj} {P}
\renewcommand{\i}{\sqrt{-1}}
\DeclareMathOperator{\mVol}{\mathrm{Vol}(M_0,\omega_0)}
\DeclareMathOperator{\Lie}{\mathrm{Lie}}
\DeclareMathOperator{\lie}{\mathrm{lie}}
\DeclareMathOperator{\op}{\mathrm{op}}
\DeclareMathOperator{\rank}{\mathrm{rank}}
\DeclareMathOperator{\ind}{\mathrm{ind}}
\DeclareMathOperator{\trace}{\mathrm{trace}}
\DeclareMathOperator{\image}{\mathrm{image}}
\DeclareMathOperator{\Sym}{\mathrm{Sym}}
\DeclareMathOperator{\Ham}{\mathrm{Ham}}
\DeclareMathOperator{\Aut}{\mathrm{Aut}}
\DeclareMathOperator{\Quant}{\mathrm{Quant}}
\DeclareMathOperator{\Fred}{\mathrm{Fred}}
\DeclareMathOperator{\id}{\mathrm{1}}
\DeclareMathOperator{\lcs}{lcs}
\DeclareMathOperator{\lcsm}{lcsm}
\DeclareMathOperator{\coker}{coker}
\begin{document}
\title{Algebraic topology}
\author{Yasha Savelyev}
\thanks {}
\email{yasha.savelyev@gmail.com}
\address{University of Colima, CUICBAS}
\keywords{}
\maketitle
\section{HW}
\subsection{Set 1 for wed} \label{sec_Set 1 for wed} 
Work out the three problems mentioned in class. Finish the
calculation of $H _{0} (X) $ for a topological space $X$. 
\subsection{Set 2 for wed}
Chapter 0: 1,4,5, 6, 11, 16 
\subsection{Set 3 for wed}
2.1: 1, 11, 12, 13, 14 
\subsection{Set 4}
2.1: 15,16,17,18,19
2.2: 1, 2, 3, 4
% <span
% class="math inline"><em>X</em></span>.</p>
% <!--  Hw Set 1 for (wed) -->
% <!--  -->
% <!-- # 0: 1,4,5, 6, 11, 16 -->
% <!-- # -->
% <!-- # Hw Set 2 for (wed) -->
% <!-- # -->
% <!-- # 2.1: 1, 11, 12, 13, 14 -->
% <!-- # -->
% <!-- # Hw Set 3 for (wed) -->
% <!-- # -->
% <!-- # 2.1: 15,16,17,18,19 -->
% <!-- # -->
% <!-- # Hw set 4, read section on degrees. Read the definition of CW complexes, and then definition of cellular homology.   -->
% <!-- # -->
% <!-- # HW set 5, Read 3.1, skip axioms. -->
% <!-- # -->
% <!-- # 3.1: 5, 6, 9 -->
% <!-- Hw set 2 for (thurs) -->
% <!--  -->
% <!-- Read ch 2 up to section on exact sequences and excision. -->
% <!--  -->
% <!-- 2.1: 1,4,5,7,11,12 -->
% <!--  -->
% <!-- Set 3 -->
% <!--  -->
% <!-- Read the section on exact sequences and excision. -->
% <!--  -->
% <!-- 15,16,17,18,19 -->
% <!--  -->
% <!-- set 4 -->
% <!--  -->
% <!-- read 2.2 up to cellular homology -->
% <!--  -->
% <!-- 2.2 1, 2, 3, 4, 5, 6 -->
% <!--  -->
% <!-- set 5 -->
% <!--  -->
% <!-- finish reading 2.2 -->
% <!--  -->
% <!-- 9, 10, 27, 28, 29, 31 -->
% <!--  -->
% <!-- set 6  -->
% <!--  -->
% <!-- read additional topics A,B,C: -->
% <!--  -->
% <!-- 1,2,3,7 from (A,B) -->
% <!--  -->
% <!-- 1,2 from C -->
% <!--  -->
% <!--  -->
% \bibliographystyle{siam}
% \bibliography{C:/Users/yasha/texmf/bibtex/bib/link}
% %  ibliography{/root/texmf/bibtex/bib/link}
% % ibliography{/home/yashasavelyev/texmf/bibtex/bib/link}
% % ibliography{/home/yasha/texmf/bibtex/bib/link}
\end{document}

