
\documentclass{amsart}  
\usepackage{graphicx}
\usepackage {appendix}
\usepackage{amsfonts}
\usepackage{url}
\usepackage{hyperref} 
\hypersetup{backref,pdfpagemode=FullScreen,colorlinks=true}
\usepackage{amsmath}
\usepackage{amssymb}
\usepackage{amscd}
\usepackage{color}
\usepackage{amsthm}
\usepackage{bm}
\usepackage{indentfirst}
\usepackage[hmargin=3cm,vmargin=3cm]{geometry}
\usepackage[all, cmtip]{xy}
\numberwithin{equation}{section}
\newtheorem{thm}{Theorem} 
\newtheorem{axiom}[equation]{Axiom} 
\newtheorem{theorem}{Theorem} 
\newtheorem{proposition}[equation]{Proposition}
\newtheorem{lma}[equation]{Lemma} 
\newtheorem{lemma}[equation]{Lemma} 
\newtheorem{cpt}[equation]{Computation} 
\newtheorem{corollary}[equation]{Corollary} 
\newtheorem{clm}[equation]{Claim} 
\newtheorem{conjecture}{Conjecture}
\newtheorem{hypothesis}{Hypothesis}
\newtheorem{definition}[equation]{Definition}
\theoremstyle{definition}
% \newtheorem{definition}[equation]{Definition}
\newtheorem{ft}{Fact}
\newtheorem{notation}{Notation}
\newtheorem{descr}{Description}[equation]

\theoremstyle{remark}
\newtheorem*{pf}{Proof}
\newtheorem*{pfs}{Proof (sketch)}
\newtheorem{remark}{Remark}
\newtheorem{example}{Example}
\newtheorem{question}{Question}

\newcommand{\R}{{\mathbb{R}}}
\newcommand{\Z}{{\mathbb{Z}}}
\newcommand{\C}{{\mathbb{C}}}
\newcommand{\Q}{{\mathbb{Q}}}
\newcommand{\D}{{\mathbb{D}}}
\newcommand{\HH}{{\mathbb{H}}}

\newcommand{\bs}{\bigskip}
\newcommand{\ra}{\rightarrow}
\newcommand{\del}{\partial}
\newcommand{\ddel}[1]{\frac{\partial}{\partial{#1}}}
\newcommand{\sm}[1]{C^\infty(#1)}

\newcommand{\delbar}{\overline{\partial}}
\newcommand{\Sum}{\Sigma}
\newcommand{\G}{\mathcal{G}}
\newcommand{\Pe}{\mathcal{P}}
\newcommand{\X}{\mathfrak{X}}
\newcommand{\J}{\mathcal{J}}
\newcommand{\A}{\mathcal{A}}
\newcommand{\K}{\mathcal{K}}

\newcommand{\ZZ}{\mathcal{Z}}
\newcommand{\eL}{\mathcal{L}}

\newcommand{\mone}{{-1}}
\newcommand{\st}{{^s_t}}
\newcommand{\oi}{_0^1}
\newcommand{\intoi}{\int_0^1}
\newcommand{\til}[1]{\widetilde{#1}}
\newcommand{\wh}[1]{\widehat{#1}}
\newcommand{\arr}[1]{\overrightarrow{#1}}
\newcommand{\paph}[1]{\{ #1 \}_{t=0}^1}
\newcommand{\con}{\#\;}
\newcommand{\codim}{\text{codim}}
\newcommand {\ham} {\text{Ham} (M, \omega)}
\newcommand {\isom} {\text{Isom} ^{h}  (M, \omega, j)}
\newcommand {\lham} {lie \text{Ham} (M, \omega)}
\newcommand {\hamcp} {\text{Ham} (\mathbb{CP} ^{r-1}, \omega )}
\newcommand{\overbar}{\overline}
\newcommand {\vM} {{(T^*)} ^{vert} \cM}
\newcommand{\om}{\omega}
\newcommand{\al}{\alpha}
\newcommand{\la}{\lambda}
\newcommand{\Om}{\Omega}
\newcommand{\ga}{\gamma}
\newcommand{\eps}{\epsilon}
\newcommand{\Cal}{\tex{Cal}}

\newcommand{\cA}{\mathcal{A}}
\newcommand{\cB}{\mathcal{B}}
\newcommand{\cC}{\mathcal{C}}
\newcommand{\cD}{\mathcal{D}}
\newcommand{\cO}{\mathcal{O}}
\newcommand{\cE}{\mathcal{E}}
\newcommand{\cF}{\mathcal{F}}
\newcommand{\cG}{\mathcal{G}}
\newcommand{\cH}{\mathcal{H}}
\newcommand{\cI}{\mathcal{I}}
\newcommand{\cJ}{\mathcal{J}}

%\newcommand{\cO}{\mathcal{O}}
\newcommand{\cS}{\mathcal{S}}

\newcommand{\cU}{\mathcal{U}}

\newcommand{\cQ}{\mathcal{Q}}
\newcommand{\cM}{\bm{M}}
\newcommand{\cP}{\bm{P}}
\newcommand{\cL}{\bm{L}}

\newcommand{\fS}{\mathfrak{S}}
\newcommand{\fk}{\mathfrak{k}}
\newcommand{\fg}{\mathfrak{g}}
% \newcommand{\fz}{\mathfrak{z}}
\newcommand{\fZ}{\mathfrak{Z}}
\newcommand\vol{\operatorname{vol}}
\newcommand {\hatcp}{\widehat{\mathbb {CP}} ^{r-1} }
\newcommand{\rJ}{\mathrm{J}}
\newcommand{\rB}{\mathrm{B}}
\newcommand{\rT}{\mathrm{T}}
\newcommand {\Hpm} {\mathcal{H}^{\pm}}
\newcommand{\bP}{\mathbb{P}}

\DeclareMathOperator {\period} {period}
\DeclareMathOperator {\sign} {sign}
\DeclareMathOperator {\Id} {Id}
\DeclareMathOperator {\floor} {floor}
\DeclareMathOperator {\ceil} {ceil}
\DeclareMathOperator {\mult} {mult}
\DeclareMathOperator {\Symp} {Symp}
\DeclareMathOperator {\Det} {Det}
\DeclareMathOperator {\comp} {comp}
\DeclareMathOperator {\growth} {growth}
\DeclareMathOperator {\energy} {energy}
\DeclareMathOperator {\Reeb} {Reeb}
\DeclareMathOperator {\Lin} {Lin}
\DeclareMathOperator {\Diff} {Diff}
\DeclareMathOperator {\fix} {fix}
% \newcommand{\M}{\mathbb{CP} ^{r-1} }
\DeclareMathOperator {\grad} {grad}
\DeclareMathOperator {\area} {area}
\DeclareMathOperator {\diam} {diam}
% \DeclareMathOperator {\rank} {rank}
\DeclareMathOperator {\dvol} {dvol}
\DeclareMathOperator {\quant} {Quant}
\DeclareMathOperator {\ho} {ho}
\DeclareMathOperator {\length} {length}
\DeclareMathOperator {\Proj} {P}
\renewcommand{\i}{\sqrt{-1}}
\DeclareMathOperator{\mVol}{\mathrm{Vol}(M_0,\omega_0)}
\DeclareMathOperator{\Lie}{\mathrm{Lie}}
\DeclareMathOperator{\lie}{\mathrm{lie}}
\DeclareMathOperator{\op}{\mathrm{op}}
\DeclareMathOperator{\rank}{\mathrm{rank}}
\DeclareMathOperator{\ind}{\mathrm{ind}}
\DeclareMathOperator{\trace}{\mathrm{trace}}
\DeclareMathOperator{\image}{\mathrm{image}}
\DeclareMathOperator{\Sym}{\mathrm{Sym}}
\DeclareMathOperator{\Ham}{\mathrm{Ham}}
\DeclareMathOperator{\Aut}{\mathrm{Aut}}
\DeclareMathOperator{\Quant}{\mathrm{Quant}}
\DeclareMathOperator{\Fred}{\mathrm{Fred}}
\DeclareMathOperator{\id}{\mathrm{1}}
\DeclareMathOperator{\lcs}{l.c.s.}
\DeclareMathOperator{\lcsm}{l.c.s.m.}
% \DeclareMathOperator{\ker}{ker}
\DeclareMathOperator{\coker}{coker}
\begin{document}
\title{Protocol}
\author{Yasha Savelyev}
\thanks {}
\email{yasha.savelyev@gmail.com}
\address{University of Colima, CUICBAS}
\keywords{}
 \maketitle
% Broadly I am working in differential geometry, particularly symplectic, dynamical systems, and algebraic topology. I also recently became interested in computer science. My most recent work branches in a number of research directions. For example, I am researching rigidity phenomena in locally conformally symplectic geometry as partly initiated in \cite{citeSavelyevConformalSymplectic}. Locally conformally symplectic manifolds of $\lcs$ manifolds for short, are an extremely natural generalization of both contact and symplectic manifolds. At the moment this subject is full of fundamental mysteries, and is relatively new.
%
% Particularly, I am interested in extensions of contact and symplectic Gromov non-squeezing theorems to locally conformally symplectic setting.
%    I am also working towards the conjecture on non-existence, or $C ^{0} $-nearby non-existence of sky catastrophes for homotopies of Reeb vector fields.
% The latter conjecture appears in my \cite{citeSavelyevFuller} and its resolution would likely be one of the most exciting possibilities that is presented in this research statement, as the are a number of applications, in particular new existence results for Reeb orbits which traditionally needed techniques of elliptic pde's. 
% I now describe in more detail the pair of research directions mentioned above, and their relationship.
This gives background for the outlined research proposal, outlines methods and ideas, as well as further questions and directions. This is a multi-year proposal, and it is possible that we can do everything that is proposed here in two years. However in the first two years we expect to do at least partially Goals 1, 2a, and 4b, which are described in the following. 
\section {Sky catastrophes, Reeb vector fields, and the no Reeb sky catastrophe conjecture}
The original Seifert conjecture  \cite{citeSeifert} asked if a non-singular vector field on $S
^{3} $ must have a periodic orbit. 
In this formulation the answer was shown to
be no for $C
^{1} $ vector fields by
Schweitzer \cite{citeSchweitzerC1Counterexample}, for $C ^{2} $ vector fields by
Harrison \cite{citeHarrison} and later for $C ^{\infty} $ vector fields by Kuperberg \cite{citeKKuperbergSmoothCounterexample}. A $C^1$ volume preserving counter-example is given by Kuperberg in \cite{citeKuperbergvolumepreserving}.
For a vector field $X$ $C^0$ close to the Hopf vector field it was
shown to hold by Seifert and later by Fuller 
\cite{citeFullerIndex} in his 1967 paper, using his Fuller index, which is a kind fixed point index for orbits.
Part of the importance of the  $C
^{0} $ condition for Fuller is
that it rules out ``sky catastrophes'' for an appropriate
homotopy of non-singular vector
fields connecting $X$ to the Hopf vector field. The latter ``sky catastrophes'' 
are the last discovered kind of bifurcations
originally constructed by Fuller himself \cite{citeFullerBlueSky}. 
He constructs a smooth family $\{X _{t} \}, t \in [0,1] $ of vector fields
on a solid torus, 
for which there is a continuous (and isolated) family of $\{X _{t}\} $ periodic orbits
$\{o _{t}\} $,  with the period of $o _{t} $ going
to infinity as $t \mapsto 1$, and so that for $t =1$ the orbit
disappears. Let us give the following, a bit more general, but still incomplete definition here, a full definition (according to us) appears in \cite{citeSavelyevFuller}.
\begin{definition} \label{def:skyPrelim}[Incomplete]  
  A \textbf{\emph{sky catastrophe}} for a smooth family $\{X _{t} \}$, $t \in [0,1]$,
   of vector
   fields on a manifold $M$ is a continuous family of closed orbits $\tau \mapsto o _{t _{\tau} }$, $o _{t _{\tau} } $ is a non-constant periodic orbit of $X _{t _{\tau} } $, $\tau \in [0, \infty)$, such that the period of $o _{t _{\tau} } $ unbounded from above.
\end{definition}
These sky catastrophes (and their more robust analogues called blue sky catastrophes) turned out to be common in many kinds of systems
appearing in nature and have been studied on their own, see for instance
Shilnikov-Turaev~\cite{citeShilnikovTuraevBlueSky}.

However since the time of Fuller's original papers it has not been
understood if this the only thing that can go wrong. That is if without 
existence of a ``sky catastrophe'' in an appropriate general sense, the time 1
limit of a homotopy of smooth non-singular vector fields on $S ^{2n+1} $
starting at the Hopf vector field must have a periodic orbit.
The difficulty in answering this is that
although our orbits cannot ``disappear into the sky'', as there are infinitely
many of them they may ``cancel each other out'', even if the Fuller index is
``locally positive'' - that is the index of isolated components in the orbit space is positive.
In the $C ^{0} $ nearby case this cancellation is prevented as orbits from isolated components of the orbit space may not
interact.
The reader may think of trying to make sense of the
infinite sum $$ (5-1) + (5-1) + \ldots + (5-1) + \ldots.
$$ 
While generally meaningless it has some meaning if we are not allowed to move
the terms out of the parentheses.
So one has to develop a version
of Fuller's index which precludes such total cancellation in general. We do this in \cite{citeSavelyevFuller} and using this prove as a particular case:
\begin{theorem} \label{conj:preliminary} Let $X = X _{1} $ be a smooth non-singular vector field on $S ^{2k+1} $ homotopic to
the Hopf vector field $H = X _{0} $ through  homotopy  $\{X _{t} \}$ of smooth non-singular vector fields.
Suppose that $\{X _{t} \}$ has no sky catastrophes
then $X$ has periodic orbits.
\end{theorem}
Can we use Theorem \ref{conj:preliminary} and its general analogues
to show existence of orbits? The most promising case where this should be possible is for Reeb vector fields.
\subsection {Reeb vector fields and sky catastrophes}
As a first step we may
ask if a homotopy of Reeb vector fields $\{X _{t}
\}$ on a closed manifold is necessarily free of sky catastrophes.  Our following elementary theorem puts
a very strong restriction on the kinds of sky catastrophes that can happen, for
general contact manifolds. 
It is likely, that if they exist, they must be pathological, and very hard to construct.

We note however that in the proof of \cite[Theorem 1.19]{citeKerman2017} a kind of partial Reeb plug is constructed, which is missing the matching condition, see for instance \cite{citeKKuperbergSmoothCounterexample} for terminology of plugs, also see Kerman \cite{citeKermanHamiltonianSeifert}, Ginzburg \cite{citeGinzburgHamiltonianSeifert}, where plugs are utilized in Hamiltonian context.
If one had a plug with all conditions, then it is simple to construct a sky catastrophe. For we may deform such a plug through partial plugs satisfying all conditions except the trapping condition (condition 3 in \cite{citeKKuperbergSmoothCounterexample}) to a trivial plug. This deformation then readily gives a sky catastrophe corresponding to the trapped orbit.
Without matching, this argument does not obviously work.
%
% We expect that the condition in the following theorem holds for any smooth homotopy $\{X _{t}
% \}$ of Reeb vector fields, for instance it clearly holds if the associated subset $S$ is
% a branched smooth submanifold of $M \times (0, \infty) \times [0,1]$, and there
% are no birth
% bifurcations of orbits for all sufficiently large periods. 
\begin{theorem} \cite{citeSavelyevFuller} \label{prop:AdmissibleReeb} Let $\{X _{t} \}$, $t \in [0,1]$ be a smooth homotopy through Reeb
vector fields on a contact manifold $M$. (Informally) let $S$ denote the space of closed orbits of the family $\{X _{t} \}$, where period is allowed to vary.
Then there is no (period) unbounded from above locally Lipschitz  continuous path $p: [0,\infty)
\to S$ whose composition with the projection $\pi: S
\to [0,1]$ has finite length, with $\pi$ the projection to the time coordinate $t$.
% Suppose that there is an $E$ so that:
% \begin{itemize}
%   \item For all $b>a>E$ there exists an
% arbitrarily $C ^{\infty} $ nearby
%   perturbation 
%  $F ^{pert} $ of $F$ coinciding with $F$ on $M \times (0,\infty) \times
%  \{i\}$, $i=0,1$, such that for any pair $s_1, s_2$ in a component of
%  $S ^{pert} = S (F ^{pert} )   $ with $a \leq \per
% (s_1) \leq \per (s_2) \leq b$ there is a piecewise smooth path $\rho$ in $M
% \times (0, \infty) \times [0,1]$ from $s_1$ to $s_2$, which is contained in $S
% ^{pert}  $,
% such that  $\pi _{3} \circ \rho $ has no local extrema, for $\pi_3: M
% \times (0, \infty) \times [0,1] \to [0,1]$ the projection.
% \item  Only finitely many components of $S ^{pert} $ intersect
% $M \times (0,b) \times
% [0,1]$ non-trivially.
% \end{itemize}
% such that only finitely many connected components of the
% associated set $S$
% intersect $M \times (0,  a) \times [0,1]$ non-trivially for all $a$ sufficiently
% large. Suppose in addition that given any pair $s_1, s_2 \in S  $ with $\per
% (s_1), \per (s_2)$ sufficiently large there is a piecewise smooth $p$ path in $M
% \times (0, \infty) \times [0,1]$ from $s_1$ to $s_2$, which is contained in $S$,
% and such that  $\pi _{3} \circ p $ has no strict local extrema, for $\pi_3: M
% \times (0, \infty) \times [0,1] \to [0,1]$ the projection
%  Then $\{X _{t}\}$ is admissible.
\end{theorem}
The above rules out for example Fuller's sky catastrophe that appears in
\cite{citeFullerBlueSky}, and described in the beginning of our paper. Inspired by this we may further conjecture:
\begin{conjecture} Let $\{X _{t} \}$, $t \in [0,1]$ be a smooth homotopy through Reeb
vector fields on a compact contact manifold $M$. Then there is a $C ^{0} $ nearby smooth family $\{X' _{t} \}$, $t \in [0,1]$, $X' _{i}=X _{i}  $, $i=0,1$, such that $\{X' _{t} \}$ has no sky catastrophes.
\end{conjecture}
Given this conjecture we may readily apply general analogues in \cite{citeSavelyevFuller} of Theorem \ref{conj:preliminary} to get applications to existence of Reeb orbits.
Of course one exciting thing here is that it will be without so called ``hard'' elliptic pde techniques of pseudo-holomorphic curves. Instead the methods would just be conceptual methods of topology. It might be possible however that the proof of the conjecture above, itself may initially involve elliptic techniques. Not surprisingly given how powerful and far reaching these techniques have become.
However whatever the method of proof, if we could verify this conjecture,  we would obtain new qualitative dynamical-topological information about Reeb vector fields.  

This is an ambitious problem, but we could also set some near term goals. For example we can at first restrict to homotopies $\{X _{t} \}$ of vector fields where the space of orbits $S$ is well behaved, for example:
\begin{question} \label{quest:rect} Suppose $S$ that is rectifiable in the sense of measure theory, does the conclusion of the Conjecture \ref{conj:preliminary} hold in this case?
   \end{question}
If $S$ is rectifiable, then there are powerful additional tools at our disposal,  most important of which is Sullivan's theory \cite{citeSullivanCycles} which develops a connection between measure theory, and Hahn-Banach theorem with dynamics. Indeed under the rectifiable hypothesis it should be technically possible to just build on the proof of Theorem \ref{prop:AdmissibleReeb}, and Sullivan's \cite{citeSullivanCycles} to produce this special case of the conjecture, but it should still be interesting and challenging.

\subsection {Goal 1 of the proposal} One of the principal goals of the proposal is to answer Question \ref{quest:rect}. This is possibly work in collaboration with Dennis Sullivan, of Stony Brook University, and City College of New York. We had a number of discussion on related topics.
\section {Locally conformally symplectic geometry and rigidity} A locally conformally symplectic manifold or $\lcsm$ is a smooth $2n$-fold $M$ with an $\lcs$ structure: which is a
non-degenerate 2-form $\omega$, which is locally diffeomorphic to $
{f} \cdot \omega _{st}  $, for some (non-fixed) positive smooth function $f$, with $\omega _{st}  $ the standard symplectic form on
$\mathbb{R} ^{2n} $.
These
% A locally conformally symplectic manifold is a
% smooth $2n$-fold $M$ with a
% non-degenerate 2-form $\omega$ which is locally diffeomorphic to $e
% ^{f} \omega _{0}  $, for some (non-fixed) function $f$, with $\omega _{0}  $ the standard symplectic form on
% $\mathbb{R} ^{2n} $. 
were originally considered by Lee
in \cite{citeLee}, arising naturally as part of an abstract study of
``a kind of even dimensional Riemannian geometry'', and then further studied by
a number of authors see for instance, \cite{citeBanyagaConformal} and
\cite{citeVaismanConformal}.
This is a
fascinating object,  an $\lcsm$ admits all the interesting classical notions of
a symplectic manifold, like Lagrangian submanifolds and Hamiltonian
dynamics, while at the same time forming a much more
flexible class. For example Eliashberg and Murphy show that if a
closed almost complex $2n$-fold $M$ has $H ^{1} (M, \mathbb{R}) \neq 0
$ then it admits a $\lcs$ structure,
\cite{citeEliashbergMurphyMakingcobordisms}, see
also \cite{citeMurphyConformalsymp}.  

$\lcsm$'s can also be understood to generalize contact manifolds. This works as follows.
% However Gromov-Witten type theory of $\lcs$ manifolds, has not been
% considered, (beyond obvious questions on its existence.)
% This is
% tricky, because there often will not be
% any global compactness for spaces of pseudo-holomorphic curves in a $\lcsm$.
First we have a natural class of explicit examples of $\lcsm$'s, obtained
by starting with a symplectic cobordism (see \cite{citeEliashbergMurphyMakingcobordisms}) of a closed contact manifold
$C$ to itself, arranging for the contact forms at the two ends of the
cobordism to be proportional (which can always be done) and then
gluing together the boundary components. 
As a particular case of
this we get Banyaga's basic example.
\begin{example} [Banyaga] \label{example:banyaga} Let $(C, \xi)
   $ be a contact manifold with a contact form
   $\lambda$ and take $M=C \times S ^{1}  $ with 2-form $\omega= d
^{\alpha} 
   \lambda : = d \lambda - \alpha \wedge \lambda$, for $\alpha$ the pull-back of the
   volume form on $S ^{1} $ to $C \times S ^{1} $ under the
   projection.    %
  % \textcolor{blue}{deformation invariance?} 
   % It is easy to verify that a contact (iso)-morphism $(C
   % _{1} ,
   % \xi _{1} ) \to (C _{2}, \xi _{2}  )$ in the sense of Definition \ref{def:morphismslcs} induces an $\lcs$
   % (iso)-morphism from $(C _{1}  \times S ^{1}, d ^{\alpha}\lambda
   % _{1})   \to (C _{2}  \times S ^{1}, d ^{\alpha}\lambda
   % _{2})   $. 
   % We have an $S ^{1} $ action on the $\lcs$ $C \times S ^{1} $ by
   % rotation in the $S ^{1} $ variable, and the induced
   % $(iso)$-morphism is $S ^{1} $-equivariant.
\end{example}
Using above we may then  faithfully embed the category of contact manifolds, and contactomorphism  into the category of $\lcsm$'s, and \textbf{\emph{loose}} $\lcs$ morphisms. These can be defined as diffeomorphisms $\phi: (M _{1} , \omega _{1} ) \to (M _{2}, \omega _{2}  ) $ s.t. $\phi ^{*} \omega _{2}= f \cdot \omega _{1} $, for a positive function $f$.
% Note that when $\omega _{i} $ are symplectic this is just a global conformal symplectomorphism by Moser's trick.  


Banyaga type $\lcsm$'s give immediate examples of almost complex manifolds 
where the $\energy$ function is unbounded on the moduli spaces of fixed class pseudo-holomorphic curves, as well as where null-homologous $J$-holomorphic curves can be non-constant.
We show in \cite{citeSavelyevConformalSymplectic} that it is still possible to extract a variant of Gromov-Witten theory for $\lcsm$'s.
The story is closely analogous to that of the Fuller index in dynamical
systems, which is concerned with certain rational counts of periodic orbits. In
that case sky catastrophes prevent us from obtaining a completely well
defined invariant, but Fuller constructs certain partial invariants which give
dynamical information. In a very particular situation the relationship with the
Fuller index becomes perfect as one of the results 
of \cite{citeSavelyevConformalSymplectic} obtains the classical Fuller index for Reeb vector fields on a
contact manifold $C$ as a
certain genus 1 Gromov-Witten invariant of the $\lcsm$ $C \times S ^{1} $. The
latter also gives a conceptual interpretation for why the Fuller index is
rational, as it is reinterpreted as an (virtual) orbifold Euler number.
\subsubsection {Non-squeezing and rigidity}
Of course what we are really interested in is what kind of rigidity phenomenon can appear in $\lcs$ geometry. As a first attempt what can be said about non-squeezing? Recall that one of the most fascinating early results in symplectic geometry is the so called Gromov non-squeezing theorem appearing in the seminal paper of Gromov~\cite{citeGromovPseudoholomorphiccurvesinsymplecticmanifolds.}.
The most well known formulation of this is that there does not exist a  symplectic embedding $B _{R} \to D ^{2} (r)  \times \mathbb{R} ^{2n-2}   $ for $R>r$, with $ B _{R}  $ the standard closed radius $R$ ball 
in $\mathbb{R} ^{2n} $ centered at $0$.
Gromov's non-squeezing is $C ^{0} $ persistent in the following sense.      
\begin{theorem} \label{thm:Gromov1} Given $R>r$, there is an $\epsilon>0$ s.t. for any symplectic form $\omega' $ on $S ^{2} \times T ^{2n-2}  $ $C ^{0} $ $\epsilon$-close to a split symplectic form $\omega$, which satisfies: $$ \langle \omega, (A=[S ^{2} ] \otimes [pt])  \rangle = \pi r ^{2},  $$  there is no symplectic embedding $\phi: B _{R} \hookrightarrow (S ^{2} \times T ^{2n-2}, \omega')   $.
\end{theorem}
 On the other hand we have: 
\begin{theorem} \label{thm:nonrigidity} Given $R>r$ and every $\epsilon > 0 $  there is
a (necessarily by the theorem above non-closed) 2-form $\omega'$ on $S ^{2} \times T ^{2n-2}  $  $C ^{0} $ close to a split symplectic form $\omega $, satisfying $ \langle \omega, A  \rangle = \pi r ^{2}  $, and s.t. there is an embedding $\phi: B _{R} \hookrightarrow S ^{2} \times T ^{2n-2}   $, with $\phi ^{*}\omega'=\omega _{st}  $.
\end{theorem}
 Theorem \ref{thm:Gromov1} follows immediately by Gromov's argument in \cite{citeGromovPseudoholomorphiccurvesinsymplecticmanifolds.}, while Theorem \ref{thm:nonrigidity} is to be proved in a paper in preparation by the author. In what follows we outline an extension of Theorem \ref{thm:Gromov1} to $\lcs$ manifolds.

One may think that recent work of M\"uller \cite{citeMuller} may be related to the above. But there seems to be no obvious such relation as pull-backs by diffeomorphisms of nearby forms may not be nearby. Hence there is no way to go from nearby embeddings that we work with to $\epsilon$-symplectic embeddings of M\"uller.
% Gromov's argument trivially generalizes to show the following: 
% \begin{theorem} [Gromov] 
%    Let $(M, \omega)$ be a compact symplectic manifold, with $GW _{0,1}  (\omega,A) ([pt]) \neq 0$ for some class $A$. If $ \langle [\omega], A  \rangle = \pi r ^{2}  $ 
%    then there is no symplectic embedding $$\phi: B _{R}   \hookrightarrow (M, \omega), $$ if $R>r$.
% \end{theorem}
\begin{definition}
   Given a pair of $\lcsm$'s $(M _{i}, \omega _{i} )$,
   $i=0,1$, we say that $f: M _{1} \to M _{2}  $ is a \textbf{\emph{morphism}}, if 
   $f^{*} \omega _{2} = \omega _{1}$. A morphism is called an $\lcs$ embedding if it is injective.
\end{definition}
A pair $(\omega,J)$ for $\omega$ $\lcs$ and $J$ compatible will be called a compatible $\lcs$ pair, or just a compatible pair, where there is no confusion.
% \begin{remark}
%    We say strict here because there are other natural notions of $\lcs$ morphisms. But they will not be considered here.
% \end{remark}
% We say strict here because in some cases it is more natural to consider more
% relaxed notions of $\lcs$ morphisms, for example we may define an $\lcs$
% morphism to be a diffeomorphism $\phi: M _{0} \to M _{1}  $ s.t. $\phi ^{*}
% \omega _{1}  $ is deformation equivalent through $\lcs$ forms to $\omega _{0} $.
% When $\omega _{i} $ are symplectic the latter is just a conformal
% symplectomorphism, but in general it is very different.
% \begin{theorem} \label{thm:nonsqueezing1}
% Let $\omega=\omega _{0} \times \omega _{1}  $ be the product standard
% symplectic form on $M = S ^{2}   \times T ^{2n-2n}  $.
% Set $$R =\min \{\langle \omega, A  \rangle \},
% $$   for $A$ the class $[S ^{2} ] \otimes [pt] \in H ^{2} (M) $.
% Given an $r>0$ there is an $\delta>0$ s.t. if $\omega _{0}, \omega _{1}  $ be
%    $\lcs$ forms on $M$ $C ^{0} $ $\delta$-close to $\omega$ then there is no $\lcs$ diffeomorphism $$\phi: (M, \omega _{0} )   
% \to (M, \omega _{1} )  $$ if $r<R$.
% \end{theorem}
% \begin{theorem} \label{thm:nonsqueezing1}
% Let $\omega=\omega _{0} \times \omega _{1}  $ be the product standard
% symplectic form on $M = \mathbb{R} ^{2}   \times T ^{2n-2n}  $  and let
% $j: T ^{n} \to \mathbb{R} ^{2} \times \mathbb{T} ^{2n-2}   $ a local Lagrangian
%    embedding, meaning that it factors through the quotient map $\mathbb{R} ^{2n}
%    \to \mathbb{R} ^{2} \times T ^{2n-2}  $.
% Set $$R = R (j)=\min \{\langle \omega, A  \rangle \, \vert \,  \langle \omega,  A
%  \rangle \neq 0, $A$ \text{ is a relative class of a disk with boundary on $j
%  (T ^{n} )$}  \}.
%  $$ 
% Given an $r>0$ there is an $\delta>0$ s.t. if $\omega'$ be an $\lcs$ form
% on $M$  homotopic to $\omega$,
% through $\lcs$ forms $C ^{0} $ $\delta$-close to $\omega$ and having the same
% Lagrangian subspaces as $\omega$, 
% then there is no compactly supported $\lcs$ $\omega'$-diffeomorphism $\phi: \mathbb{R} ^{2} \times T ^{2n-2}  
% \to \mathbb{R} ^{2} \times T ^{2n-2}  $  which takes $j (T ^{n} )$ into $D ^{2} (r) \times
% \mathbb{T} ^{2n-2}  $ if $r<R$.
% \end{theorem}
% Note that if in the above theorem $\lcs$ everywhere is replaced by symplectic, then the
% theorem follows by original non-squeezing together with the classical Moser argument.
% At least if want to follow Gromov's original argument we
% need a stronger form of the isoperimetric inequality used by Gromov, which may
% at present be unknown.
% \begin{hypothesis} Let $\mathcal{J}$ be the space of almost complex structures
%    compatible with the standard symplectic form $\omega$ on $\mathbb{R} ^{2n}
%    $. Let $B _{R} $ be the standard closed ball with radius $R$ in $\mathbb{R}
%    ^{2n} $ with center at origin $0$.
% Define $$\mathcal{A}: \mathcal{J} \to \mathbb{R}$$ by $\mathcal{A} (J)$ is the
%    least $\omega$-area of a somewhere injective $J$-holomorphic curve passing through $0$ with
%    boundary on $\partial B$. Then the minimum of $\mathcal{A}$ is  $\pi \cdot R ^{2} $.
% \end{hypothesis} 
% \begin{theorem} \label{thm:nonsqueezing2} 
% Let $(M, \omega)$ be a closed $\lcs$ manifold,   $\omega$ is $C$-comparable with a symplectic form $\omega _{0} $.
% Suppose that $GW _{0,1}  (\omega, A) ([pt]) \neq 0$ for some class $A$. If $$ \langle [\omega _{0} ], A  \rangle = \pi r ^{2}  $$     then there is no $\lcs$ embedding $$\phi: B _{R} \hookrightarrow (M, \omega), $$ if $C< \frac{R}{r}$.
% \end{theorem}
% The following may be more concrete since to obtain one simple application all we have to do is start with the product symplectic form $\omega$ on $M=S ^{2} \times T ^{2n}  $, take $A= [S ^{2} ] \otimes [pt]$ and deform $\omega$ slightly through $\lcs$ forms. The following theorem then tells us that Gromov non-squeezing is ``rigid'' for such a deformation.
% \begin{definition} \label{def:boundeddeformation}
%    We say that a pair of bounded, as in Definition \ref{def:comparable}, $\lcs$ forms $\omega _{0},  \omega _{1} $ on a manifold are
%  \textbf{\emph{$c$-deformation equivalent}} if there is an interpolating continous family $\{\omega _{t} \}$,  $t \in [0,1]$, of $\lcs$ forms,  and a continous family $\widetilde{\omega} _{t}  $, $t \in [0,1]$ of symplectic forms, such that for each $t$ $\omega _{t}, \widetilde{\omega}_{t}
%    $ are $C$-comparable for some $C$, (independent of $t$).
% \end{definition}
% <<<<<<< HEAD
% The following theorem says that it is impossible to even have a ``nearby'' $\lcs$ embedding.  (Note that the $C ^{0} $ norm we use on the space of $\lcs$ structures is (likely strictly) stronger then the obvious norm on the space of $2$-forms).
% \begin{theorem} \label{cor:nonsqueezing} Let $\omega$ be the standard product symplectic form on $M =S ^{2} \times T ^{2}  $,  with $ \langle \omega, A= [\Sigma] \rangle = \pi r ^{2} $. Let $R>r$, and $\Sigma \subset M$ a fixed embedded $\omega$-symplectic, spherical surface in homology class of $A=[S ^{2} \times \{x\}] $, then there is an $\epsilon>0$ s.t. if $\omega _{1} $ is an $\lcs$ on $M$ $C ^{0} $ $\epsilon$-close to $\omega$, then
%    there is no $\lcs$ embedding $$\phi: (B _{R}, \omega _{st})  \hookrightarrow (M, \omega _{1}), $$  s.t $\phi _{*} j  $ preserves $T\Sigma$. 
% \end{theorem}
% For a nearby symplectic manifold the above follows by Gromov's argument, without the condition of $\Sigma$. 
% We note that the image of the embedding $\phi$ would be of course a symplectic submanifold of  $(M, \omega _{1} )$. However it could be highly distorted, so that it might be impossible to complete $\phi _{*} \omega _{st}  $ to a symplectic form on $M$ nearby to  $\omega$, (and perhaps impossible to complete to any symplectic form).
% We also note that it is certainly possibly to have a ``nearby'', in the sense above, volume preserving embedding.
% Take $\omega = \omega _{1}  $, $\Sigma=S ^{2} \times \{x\} $,  then hypothesis of the theorem above are satisfied, while (if the symplectic form on $T ^{2} $ has enough volume) we can find a volume preserving map $\phi: B _{R} \to M $ s.t. $\phi _{*} j$ preserves $\Sigma$. (This is just the squeeze map.)
% =======
% The following tells us that the non-squeezing problem for $\lcsm$ can be is closely tied to pseudo-holomorphic curve theory as in the symplectic case. 
% \begin{theorem} \label{thm:alternative} The following alternative holds.
% Let $\omega$ be the standard product symplectic form on $M =S ^{2} \times T ^{2n}  $,  with $ \langle \omega, A= [S ^{2} \times \{pt\} ] \rangle = \pi r ^{2} $. Let $R>r$, then either there is an $\epsilon>0$ s.t. if $\omega _{1} $ is an $\lcs$ on $M$ $C ^{0} $ $\epsilon$-close to $\omega$, then there is no $\lcs$ embedding $$\phi: (B _{R}, \omega _{st})  \hookrightarrow (M, \omega _{1}), $$  or there is a compatible $\lcs$ family $(\{\omega _{t} \}, \{J _{t} \})$ on $S ^{2} \times T ^{2n}  $ with a sky catastrophe (the definition is given in the section just below).
% \end{theorem}
Note that the pair of hypersurfaces $\Sigma _{1} =S ^{2} \times S ^{1} \times \{pt \} \subset S ^{2} \times T ^{2}   $, $\Sigma _{2} =S ^{2} \times  \{pt \} \times S ^{1} \subset S ^{2} \times T ^{2}   $ are naturally foliated by symplectic spheres, we denote by $T ^{fol}  \Sigma _{i} $ the sub-bundle of the tangent bundle consisting of vectors tangent to the foliation. The following theorem proved in my \cite{citeSavelyevConformalSymplectic} says that it is impossible to have  certain ``nearby'' $\lcs$ embeddings, which means that we have a first rigidity phenomenon for $\lcs$ structures.
There is a small caveat here that in what follows we take the $C ^{0} $ norm on the space of $\lcs$ structures that is (likely) stronger then the natural $C ^{0} $ norm (with respect to a metric) on the space of forms.
\begin{theorem} \label{cor:nonsqueezing} Let $\omega$ be a split symplectic form on $M =S ^{2} \times T ^{2}  $,  and $A$ as above with $ \langle \omega, A\rangle = \pi r ^{2} $. Let $R>r$, then there is an $\epsilon>0$ s.t. if $\omega _{1} $ is an $\lcs$ on $M$ $C ^{0} $ $\epsilon$-close to $\omega$, then
   there is no $\lcs$ embedding $$\phi: (B _{R}, \omega _{st})  \hookrightarrow (M, \omega _{1}), $$  s.t 
   $\phi _{*} j$   \text{ preserves the bundles } $T ^{fol} \Sigma _{i},$ for $j$ the standard almost complex structure.
   \end{theorem}
We note that the image of the embedding $\phi$ would be of course a symplectic submanifold of  $(M, \omega _{1} )$. However it could be highly distorted, so that it might be impossible to complete $\phi _{*} \omega _{st}  $ to a symplectic form on $M$ nearby to  $\omega$.
We also note that it is certainly possible to have a nearby volume preserving as opposed to $\lcs$ embedding which satisfies all other conditions.
Take $\omega = \omega _{1}  $, then if the symplectic form on $T ^{2} $ has enough volume, we can find a volume preserving map $\phi: B _{R} \to M $ s.t. $\phi _{*} j$ preserves $T ^{fol} \Sigma _{i} $. 
This is just the squeeze map, which as a map $\mathbb{C} ^{2} \to \mathbb{C}^{2}  $ is $(z_1, z _{2} ) \mapsto (\frac{z _{1} }{a}, a \cdot {z _{2}}) $.
In fact we can just take any volume preserving map $\phi$, which doesn't hit $\Sigma _{i} $.
\subsection {Goal 2.a
}  Remove the condition in the theorem above that $\phi _{*}j $ preserves $T ^{fol} \Sigma _{i} $. With this condition removed we obtain a simple and direct extension of the original Gromov non-squeezing to $\lcs$ manifolds.
% For the proof we need to use geometry to deduce compactness of the moduli space of certain pseudo-holomorphic curves, for what could be fairly general deformations of almost complex structures compatible with $\lcs$ forms, without a priori bounds on $\energy$ and without any $C ^{0} $ bounds on the deformation. 
\subsubsection {Toward non-squeezing for loose morphisms} In some ways loose morphisms of $\lcsm$'s are more natural, particularly when we think about $\lcsm$'s from the contact angle.
So what about non-squeezing for loose morphisms as defined above? We can try a direct generalization of contact non-squeezing of Eliashberg-Polterovich \cite{citeEKPcontactnonsqueezing}, and Fraser in \cite{citeFraserNonsqueezing}.
Specifically let $R ^{2n}   \times S ^{1}  $ be the prequantization space of $R ^{2n} $, or in other words the contact manifold with the contact form $d\theta - \lambda$, for $\lambda = \frac{1}{2}(ydx - xdy)$. Let $B _{R} $ now denote the open radius $R$ ball in $\mathbb{R} ^{2n} $. 
\begin{question} \label{question:loose} If $R\geq 1$ is there a compactly supported, loose endomorphism of the $\lcsm$ $\mathbb{R} ^{2n} \times S ^{1} \times S ^{1}  $ which takes the closure of $U := B _{R} \times S ^{1} \times S ^{1}  $ into $U$? 
\end{question}
\subsection {Goal 2.b} Show that the answer to the above question is no. In this case we obtain an analogue of contact non-squeezing in $\lcs$ geometry.
We can prove the no answer, assuming the following $\lcs$ analogue of the Weinstein conjecture.
\subsection {Holomorphic Weinstein conjecture} In contact geometry rigidity phenomena circulate around existence phenomena of Reeb orbits. The most important conjecture concerning Reeb orbits is the Weinstein conjecture, which says that a closed contact manifold always has a Reeb orbit. For contact 3-folds this is now a very deep theorem of Taubes \cite{citeTaubesWeinsteinconjecture}.
If $\lcsm$'s are generalizations of contact manifolds, what is the analogue of this conjecture in $\lcs$ geometry?
To start we propose the following, which we we show in \cite{citeSavelyevConformalSymplectic} to directly extend the Weinstein conjecture.
% We shall say that an $\lcs$ form $\omega$ on $C \times S ^{1} $ is \emph{simple} if its lift to the cover $C \times \mathbb{R}$ is (globally) conformally symplectic.
\begin{conjecture} \label{conj:Hol} For any compatible pair  $(\omega, J)$, for $\omega$ $\lcs$ form on $C \times S ^{1}  $ for $C$ a threefold or $S ^{2k+1} $, there is an elliptic, non-constant $J$-holomorphic curve in $C \times S ^{1}  $.
\end{conjecture}
In \cite{citeSavelyevConformalSymplectic} we prove this conjecture for $\omega$ $C ^{\infty} $ nearby to the Hopf $\lcs$ structure on $S ^{2k+1} \times S ^{1}  $, by exploiting the connection with Fuller index, to which we already alluded in our discussion of sky catastrophes. Moreover we show that either this conjecture holds for any $\lcs$ structure $\omega$ homotopic to the Hopf $\lcs$ structure, or there exist holomorphic sky catastrophes, which are analogues in world of pseudo-holomorphic curves of sky catastrophes of Definition \ref{def:skyPrelim}. 
In particular this partially elaborates the connection of the rigidity story for $\lcs$ manifolds and dynamical systems.
% We also show there that this conjecture implies the Weinstein conjecture for $C$ a general contact manifold.


Of course it is natural to ask, other then curiosity what is the significance of the above conjecture? Well as we mentioned contact geometry rigidity is linked to existence of Reeb orbits, for example there are certain capacities called ECH capacities, and related constructions \cite{citeHutchingsBeyond} that come from the machine of embedded contact homology of Huchings-Taubes, (and built using Reeb orbits). Recall that the embedded contact homology 
constructed by Hutchings is used by Taubes for the proof of the three dimensional Weinstein.
There is an analogous story in the $\lcs$ setting, and using it we will get new rigidity phenomena in $\lcs$ geometry that we are looking for, in particular Question \ref{question:loose} on loose $\lcs$ non-squeezing will be answered, given the conjecture above.
\subsection {Outline of the proof of Conjecture \ref{conj:Hol} when $C$ is a 3-fold}
In this section we shall assume that the reader knows some basic language of symplectic field theory, Seiberg-Witten theory and embedded contact homology. When $C$ is a 3-fold, and the Seiberg-Witten invariant of $M = C \times S ^{1} $ is non-vanishing the path to the proof of Conjecture \ref{conj:Hol} is almost obvious. First we take a separating contact hypersurface $\Sigma$ in $M$, and perform neck-stretching on the pair $(\omega, J)$ at $\Sigma$. In the end we brake up $M$ into a pair of pieces $M _{i} $ with cylindrical ends, with $\lcs$ forms $\omega _{i} $ which are in fact globally conformally symplectic,  at least when $M _{i} $ are simply connected - for a general $C$ we have to make additional assumptions on the Lee class of $\omega$, and or the hypersurface $\Sigma$. (Lee class is a certain invariant of an $\lcs$ form which vanishes when it is symplectic, see my \cite{citeSavelyevConformalSymplectic} for instance.)


Globally conformally symplectic is as good as symplectic for the purpose of Gromov-Witten or SFT analysis, more specifically compactness works the same way.
We may then consider the count (in the total homology class $A$ determined by the spin-c structure induced by $\omega$) of holomorphic buildings consisting of a pair of holomorphic curves with ECH index 0 in each part $M _{i} $ 
with the same asymptotic constraints at the cylindrical end. This count  is an invariant of the $\lcs$ $(M,\omega)$, even though the count of $J$-holomorphic curves in $M$ by itself is not a priori an invariant, and we claim that this invariant is the Seiberg-Witten invariant of $M$, for the spin-c structure determined by $\omega$.
This of course builds on the foundational work of Hutchings and Taubes, there are series of groundbreaking papers here, we give a very short list, \cite{citeHutchingsECHindex}, \cite{citeHutchingsSullivanRoundingCorners}, \cite{citeTaubesCountingPseudoHolomorphic}, \cite{citeTaubesWeinsteinconjecture}.
Luckily the technical work for this correspondence in a setup closely analogous to the above is worked out in the thesis of Chris Gerig \cite{citeGerigChrisTaming}, \cite{citeGerigChrisSW}, which generalizes work of Taubes on GW-SW correspondence to general smooth 4-folds.
 
So if the Seiberg-Witten invariant of $M$ is non-vanishing then a gluing argument gives an existence of a non-constant, class $A$, $J$-holomorphic curve in $M$. This is enough for the applications that we have in mind (e.g.
Question \ref{question:loose}), however a more careful analysis should yield that this curve is actually an elliptic curve.

% The above assumes non-vanishing of the SW invariant of $M$, it is tempting to guess that this always happens, analogously with the symplectic case. This would then be the goal of further research, towards complete resolution of the conjecture above. 
\subsection {Goal 3} Prove holomorphic Weinstein conjecture for $M = C \times S ^{1} $ with $C$ a 3-fold, following the above outline.


\section {An $\lcs$-homology theory} The above approach to existence of elliptic curves in $C \times S ^{1} $ via Seiberg-Witten theory is very partial, we need $C$ to be dimension 3, and we need topological restrictions on $C$, otherwise the Seiberg-Witten invariant may not even be defined.
For general $\lcs$ manifolds $M$ we need to develop an analogue of contact homology, denoted by $CSH (M)$ for example. This is possibly work in collaboration with Yong-Geun Oh, of IBS, Pohang, Korea, as we held a number of discussions on this problem, during my visit in Spring 2019.

 Indeed for the Banyaga $\lcs$ structure on $M=C\times S ^{1} $ with $(C, \lambda)$ contact, as in Example 1, for an appropriate almost complex structure $J _{\lambda} $ all $J _{\lambda} $-holomorphic tori, are in one to one correspondence with Reeb orbits of $(C, \lambda)$. They are just products of Reeb orbits by the $S ^{1} $ factor of $M \times S ^{1} $.  But these Reeb tori as we call them have an additional structure: the form $d\lambda$ vanishes on them identically, we say that they are \textbf{\emph{calibrated}} by $d\lambda$.
%  There is natural symplectic connection $A _{\lambda} $ on  $T(M \times S ^{1}) $, determined by $\lambda$, such that all Reeb tori are flat for $A _{\lambda} $. 

 For a more general $\lcs$ structure $\omega$ on $M ^{2n-1}  \times S ^{1} $, if there is a fibration $M \times S ^{1} \to S ^{1} $ with all fibers contact type,  there is an analogue $\eta$ (also exact) of the form $d \lambda$, and an analogue $J _{\eta} $ of the almost complex structure $J _{\lambda} $.
We propose then that a generator of $CSH (M)$ is a $J _{\eta} $-holomorphic elliptic curve in $M$, in this case we again have that $\eta$ identically vanishes on such a curve, and we call them \textbf{\emph{calibrated elliptic curves}}.  Instantons (without compactifications) in this homology theory $CSH (M)$ are $J _{\eta} $-holomorphic cylinders in $M$, which are asymptotic in appropriate sense to calibrated elliptic curves. This is equivalent to a condition that these holomorphic cylinders $u$ have finite energy $e (u)$, where $$e (u) = \int _{D} |pr _{\xi} \circ du|,$$ for $pr _{\xi} $ the projection to $\xi$ - a $2n-2$ dimensional distribution of $TM$ on which $\eta$ is non-degenerate, and $D = S ^{1} \times \mathbb{R} $ domain of the curve.

When $M=C \times S ^{1} $ has the Banyaga $\lcs$ structure we naturally have $$CSH (M) = CH (M).
 $$
\subsection {Goal 4.a}  Develop the homology theory $CSH (M)$.
\subsection {Goal 4.b} Use $CSH (M)$ to prove $\lcs$ non-squeezing, as in Goal 2.b.  
\bibliographystyle{siam}  \bibliography{/home/yasha/texmf/bibtex/bib/link} % \bibliography{/root/texmf/bibtex/bib/link} 
\end{document}
