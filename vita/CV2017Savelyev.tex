
\documentclass[overlapped,line,letterpaper]{res} 
\usepackage{amsmath, amssymb, mathrsfs, amsfonts,  amsthm}
% \usepackage{ifpdf}
\usepackage {color}
 \usepackage {hyperref}
% \ifpdf
%   \usepackage[pdftex]{hyperref}
% \else 
%   \usepackage[hypertex]{hyperref}
% \fi
 
\hypersetup{
colorlinks = true,
 urlcolor= blue 
%   pdfpagemode=none,
%   pdftitle={Curriculum Vitae},
%   pdfauthor={Yakov Savelyev},
%   pdfcreator={$ $Id: cv-us.tex,v 1.28 2006/12/12 22:53:52 jrblevin Exp $ $},
%   pdfsubject={Curriculum Vitae},
%   pdfkeywords={}
}

%%===========================================================================%%

\begin{document}

%---------------------------------------------------------------------------
% Document Specific Customizations

% Make lists without bullets and with no indentation
\setlength{\leftmargini}{0em}
\renewcommand{\labelitemi}{}

% Use large bold font for printed name at top of pages
\renewcommand{\namefont}{\large\textbf}
 
%---------------------------------------------------------------------------

\name{Yakov (Yasha) Savelyev}
\begin{resume}
\begin{ncolumn}{2}
Profesor e investigador Titular A \\
CUICBAS \\
University of Colima, Mexico \\
Bernal Díaz del Castillo 340 \\
Col. Villas San Sebastian \\
28045 Colima Colima \\
Mexico \\

yasha.savelyev@gmail.com \\
\textbf{website}: \href{https://sites.google.com/site/yashasavelyev/home} 
{http://yasha.savelyev.googlepages.com/home}\\ \textbf {phone}:
(413) 570-0163\\
\end{ncolumn}
  
%---------------------------------------------------------------------------
 \section {\sc Personal}
  Born: January 17, 1980, Moscow, Russia. \\
 {Citizenship: USA, Russia}
\section{\sc Education}
Ph.D. SUNY Stony Brook, August 2008  \\
Ph.D Advisor: Dusa McDuff\\
B.S. Mathematics, SUNY Stony Brook, 2002\\ 
%---------------------------------------------------------------------------
\section {\sc Appointments}
Visiting Assistant Professor, University of Massachusets, Amherst 2008-
2011 \\ Postdoctoral Fellow, MSRI, spring 2010\\
Postdoctoral Fellow CRM-Montreal, August 2011- August 2013\\
Postdoctoral Fellow, ICMAT, Madrid,  \\
Profesor e investigador Titular A, CUICBAS, University of Colima, current \\
\section {\sc Longer term professional visits}
Tel Aviv University, winter 2009  (Invited by Leonid Polterovich) \\
RIMS Kyoto University, spring 2014 (Invited by Kaoru Ono) \\
\section{\sc Research interests} 
Symplectic and differential geometry, especially Gromov-Witten and
Floer theory. More recently also geometric quantization and Yang-Mills
theory. Connections of the above with algebraic topology, mathematical physics and dynamical
systems. 
% My main research theme concerns the study of an interplay of Gromov-Witten and
% Floer theory, with Hofer geometry and algebraic topology.
% Floer-Gromov-Witten theory is concerned  with probing  symplectic
% manifolds by studying various spaces of pseudo-holomorphic maps of a
% Riemann surface into the manifold. This has close connections to and
% sometimes origin in some structures in modern theoretical physics,
% particularly in string theory. Hofer geometry, which is a certain
% natural Finsler geometry on the group of Hamiltonian
% symplectomorphisms, comes into the picture
% because it often governs perturbation data needed in construction of
% the underlying analytic moduli spaces. On the other hand the algebraic
% topology connection in my work is roughly analogous to the way connections and
% curvature enter into construction of Chern classes, and so into algebraic
% topology. This is partially illustrated via the theory of quantum
% characteristic classes introduced in my thesis.
% I have also done some work involving Yang-Mills theory, and
% geometric quantization. Some of my work also concerns aspects of Hamiltonian
% dynamics.  

% My main research theme concerns the study of an interplay of Gromov-Witten and
% Floer theory,  with Hofer geometry and algebraic topology.
% Floer-Gromov-Witten theory is concerned  with probing  symplectic
% manifolds by studying various spaces of pseudo-holomorphic maps of a
% Riemann surface into the manifold. This has close connections to and
% sometimes origin in some structures in modern theoretical physics,
% particularly in string theory. Hofer geometry comes into the picture
% because it often governs perturbation data needed in construction of
% the underlying analytic moduli spaces. On the other hand the algebraic
% topology connection in my work is roughly analogous to the way connections and
% curvature enter into construction of Chern classes, and so into algebraic
% topology.  
% I am also interested and
% have some papers in Mathematical physics (Yang-Mills theory, and
% quantization). Some of my work also concerns aspects of Hamiltonian
% dynamics.  
% My primary areas of research are in symplectic and differential geometry,
% especially Gromov-Witten and Floer theory, more recently with focus on
% connections to algebraic topology. I have also done much work in
% Hofer geometry, which is a Finsler geometry on the group of
% Hamiltonian diffeomorphisms, some of this work is in connection with a
% certain Gromov-Witten theoretic object known as quantum characteristic
% classes, which was introduced in my thesis. I am also interested and
% have some papers in Mathematical physics (Yang-Mills theory, and
% quantization). Some of my work also concerns aspects of Hamiltonian
% dynamics.  
\section {\sc Publications and Preprints} 
The list together with links is also at:
   \href{http://yashamon.github.io/web2/}{Publication list} \\\\
{\em {Quantum characteristic classes
and the Hofer metric}}, Geometry \& Topology, 12 (2008), pp.~2277--2326.\\\\
{\em {Virtual Morse
  theory on $\Omega $Ham$(M, \omega)$.}}, J. Differ. Geom., 84 (2010),
  pp.~409--425.\\\\ 
{\em{Bott periodicity and stable
quantum classes}}, Selecta Math.(2013) 19: 439-460\\\\ 
 {\em{Gromov K-area and jumping
curves in $ \mathbb{CP} ^{n}$}}, 2012, Algebraic and Geometric Topology \\\\
 {\em {Proof
of the index conjecture in Hofer geometry}}, Math. Res. Letters, Volume 20
(2013), 981-984 \\
\emph{Morse theory for the Hofer length functional}, Journal of topology and
analysis, 08/2013; 06(02), \\\\
{\em {On the injectivity radius in Hofer geometry}}, with Francois Lalonde, 
Electronic Research announcements, Vol 21, 177-185, 2014 \\\\
\emph {Yang Mills theory and jumping curves}, Intern. Journ. of Math.,
26,
2015 \\\\ 
\emph {On the Hofer geometry injectivity radius conjecture}, International Mathematics Research Notices 2016; doi: 10.1093/imrn/rnw023 \\\\
\emph{Global Fukaya category and the space of $A _{\infty} $
categories I},  arxiv preprint, submitted, 2013 \\\\
\emph {Global Fukaya category and the space of $A _{\infty} $
categories II},  arxiv preprint, submitted, 2014 \\\\
\emph {Floer theory and topology of $Diff (S ^{2} )$},  Journal of symplectic
geometry, to appear, 2017 \\\\
\emph {Twisted and untwisted K-theory quantization and symplectic
topology}, (with Egor Shelukhin) arxiv preprint, submitted,  2015 \\\\
\emph {Gromov-Witten theory of a locally conformally symplectic manifold and the
Fuller index}, arxiv preprint, 2016 \\\\
\emph {Extended Fuller index, sky catastrophes and the Seifert conjecture}, arxiv preprint, 2017 \\\\
%  \emph{Hamiltonian string background and elliptic cohomology}, in construction. 
% \emph {On the Morse index in Hofer geometry}, in construction.
%  \section {Collaborators}
%  Leonid Polterovich, Tel Aviv University.
%  \section {Thesis advisor}
%  Dusa McDuff, Barnard College, Columbia University
%  \section {Postdoctoral sponsor}
%  Michael Sullivan, University of Massachusets, Amherst
\emph {Not ready for publication but I am now working on the second one}: \\\\
\emph{Spectral geometry of the group of Hamiltonian diffeomorphisms},  arxiv preprint \\\\
 {\em {On configuration spaces of stable maps}}, arxiv preprint \\\\ 
\section {\sc Service}
Referee for mathematical journals, and reviewer for Zentralblatt. \\
Co-organizer for symplectic geometry seminar
at CRM-Montreal for the 2012-2013 year. \\
\href{https://docs.google.com/spreadsheet/ccc?key=0AlBCuxjt683fdHYzY2VaSWl1TU1lckctNjJWVndyTHc&usp=docslist_api}
{link to list of speakers}
\section{\sc Fellowships and Awards}
Department research award, 2006 Stony Brook\\
Department travel grant, 2007 Stony Brook\\
% Research assistantship, Fall 2007 \\
% NSF sponsored summer support grant, 2006, 2007 \\
% NSF sponsored travel grant, 2005, 2006
NSF travel grant 2006, 2005 \\
Chair's award for outstanding thesis, 2008 Stony Brook\\
MSRI Postdoctoral fellowship, spring 2010,\\
CRM-Montreal postdoctoral fellowship,  2011-2013\\ ICMAT Madrid
postdoctoral fellowship, current 
% % \section { \bf Proposed research description} % % In our previous work we have defined quantum characteristic classes. These are % % certain natural cohomology classes of the free loop space $ L\text % % {Ham}(M, \omega)$ of the group $ \text {Ham}(M, \omega)$ of Hamiltonian % % symplectomorphisms of a symplectic manifold.  % % These classes are defined in terms % % of certain Gromov-Witten invariants and % % % , and % % % greatly generalize the Seidel representation, which is a homomorphism from the % % % fundamental group of $ \text {Ham}(M, \omega)$ to quantum homology $QH _{2n} % % % (M)$.  % % we have shown that they % % are a very powerful tool in the study of higher dimensional topology and Finsler % % geometry of the group $ \text {Ham}(M, \omega)$, with its Hofer metric.  % % % % We are currently exploring  a deep interaction % % between energy flow on the loop spaces of compact Lie groups, holomorphic % % aspects of its Morse theory,  Finsler geometric properties of the % % group $ \text {Ham}(M, \omega)$ and quantum characteristic classes.  % %    % % The immediate importance of the above project is to topology of the group  % % $\text {Ham}(M, \omega)$, about which there is still very little known general % % information.  % % However, a possibly more important application is to the Finsler % % geometry of the group $ \text {Ham}(M, \omega)$, which was in fact the prime % % motivation for construction of quantum characteristic % % classes. Outside of our previous work  % % there is still very little known about higher dimensional Finsler geometrical % % properties of $ \text {Ham}(M, \omega)$. Our project would shed some light % % on this important problem both in immediate applications and perhaps even more % % so by the methods which we intend to develop.   % \section{Research papers} % {\em Quantum characteristic classes and the {H}ofer metric}, Geometry Topology, % (2008).\\ % \emph{Virtual Morse theory on $\Omega \text {Ham}(M, \omega)$}, %   available at arXiv:0804.0059, submitted to Journal of Differential Geometry.  %   \\ %  \emph{Hamiltonian string background}, in %    preparation \\ %  \emph{Hamiltonian cagegory and spectral length}, in preparation \\  \section {\sc Selected invited  talks} % \emph {Morse homology}; Stony Brook, a presentation for graduate students \\ % \emph{On polyfold view point of Morse theory}: Leipzig, Workshop in Symplectic Field Theory, 2005. \\\\ % \emph{Seidel representation}; expository, Stony Brook, symplectic geometry
% seminar, 2005. \\
% \emph{Generalized Seidel representation and the Hofer metric}: 
% Stony Brook,
% symplectic geometry seminar, 2006, 2007 \\\\
% \emph {On the symplectic action functional}; expository, Stony Brook, symplectic
% geometry seminar, 2006. \\
% \emph {Various talks on Quantum characteristic classes}: 
% Stony Brook, symplectic geometry seminar, 2007. \\
\section {\sc Recent invited talks}
 CRM, Montreal,  2007. \\\\
   Courant Institute of 
Mathematics, NYC, 2007.  \\\\
% \emph {Quantum characteristic classes and Morse theory on loop groups}; 
% \emph {Virtual Morse theory for the Hofer length functional}: \\
Tel Aviv University, Topology and dynamics seminar, 2009. \\\\
University of Wisconsin, Madison, geometry-topology seminar, 2009. \\\\
% \emph {Hamiltonian 2d cohomological field theory}: \\ 
 UMASS, Amherst, geometry-topology seminar,  (2 talks), 2008. \\\\
 % \emph{Flow categories and Dirac loop space}: \\
 UMASS, Amherst, geometry-topology seminar, 2009. \\\\
Columbia University, geometry-topology seminar, 2009. \\\\
% \emph{Quantum classes, Mumford conjecture and Hofer geometry}:\\
MSRI, research seminar, 2010. \\\\
% \emph {On Hofer geometry and quantum classes}: \\
CRM-Montreal, symplectic geometry seminar, 2011\\\\
UQAM, Montreal, CIRGET Seminar, 2012 \\\\
IBS, Pohang, Korea, 2013 \\\\
ICMAT, Madrid, May, 2013,   (A talk on "Morse theory for the Hofer length functional") \\\\
CRM Barcelona, Fall 2013  \\\\
ICMAT, Madrid Fall, 2013 (a 5 talk series of lectures on Hofer geometry) \\\\
ICMAT, Madrid Fall, 2013 ("Yang-Mills theory and Jumping curves") \\\\
University of Toronto, topology seminar, 2013 \\\\
QGM, Aarhus, Denmark, 2013 \\\\
Kyoto University, Institute of Mathematical sciences, 2014 \\\\
Jerusalem University, Israel, 2014 \\\\
University of Montpellier, France, 2014 \\\\
Computense University, Madrid, Spain, 2014\\\\
HSE, National Center for Research, Moscow, March 25, 2015 \\\\
symplectix, Inst. H. Poincare, Paris, April 15, 2015 \\\\
University of Colima, Fall 2016 \\\\
Institute for Advance Study, Princeton, 2017
\section {\sc Conference talks}
ICMS, conference on ``Symplectic Geometry and Transformation Groups'', in honor
of H. Hofer, 2010. \\\\
Georgia topology conference, 2011 \\\\
Lodz, Poland conference: ``Contact and symplectic topology, with a focus on open
problems'' (Part of the joint Israeli-Polish mathematical societies meeting.), 2011 \\\\
Tokyo IMPU, Floer and Novikov homology, Contact topology and related topics, 2014 \\\\
% \section {Selected Attended Conferences}
% Simons Workshop in Mathematics and Physics, Stony Brook,  2003, 2004. \\
% Workshop in Symplectic Field Theory I and II, Leipzig, Germany, 2005, 2006. \\
% Symplectic Field theory and applications (A series of lectures by Yasha
% Eliashberg), University Catholique de Louvaine, Belgium, 2005. \\
% Developments in Understanding Symplectic Geometry And Topology, Stony Brook,
% 2006. \\ Clay Workshop on Symplectic Topology, Clay Mathematics Institute, 2007.
% \\
% Floer Theory and Symplectic Dynamics, CRM, Montreal, 2008. \\
% Illinois-Indiana symplectic geometry conference, 2008. \\
% Workshop in Symplectic Field Theory III, Berlin, Germany, 2008.\\
% Symplectic and Contact Topology and Dynamics: Puzzles and Horizons, MSRI,
% 2010.\\
% Symplectic Geometry, Noncommutative Geometry and Physics, MSRI, 2010. \\
% Research Workshop: Homology Theories of Knots and Links, MSRI, 2010. \\
% Symplectic Geometry and Transformation Groups: in honor of H. Hofer, ICMS,
% Edinburg, 2010.
% 
Colima workshop in Geometry, Jan 2015  \\\\
Edinburg, ICMS, Workshop on symplectic geometry and topology, 2016 \\\\
Aguascalientes, 49 Congreso de la Sociedad Matemática Mexicana, geometry section, 2016 \\\\
% \section {\sc Future invited talks}
 \section {\sc Teaching experience}
3 years of lecturing experience as a graduate student at Stony
Brook university. With 1 course per semester load. These were
very large calculus sections with 100 or more enrolled students
per section.  I also TA'd at Stony Brook for calulus, business
calculus, and graduate algebraic topology (twice). 


Another 2.5 years of lecturing experience as a visiting
assistant professor at University of Massachusets, Amherst.  (Away for half a
semester at MSRI).
I was doing 2 courses/sections per semester. Each section was
small about 20-30 enrolled students. 
These were calculus and multivariable calculus sections.


\emph{Teaching at current home institution University of Colima, Mexico.}

Spring 2016, Geometry-Topology: An advanced course for undergraduates following
Spivak's Differential Geometry, and the classical papers of Shing-Shen Chern on
the Gauss-Bonnet theorem.

Fall 2016, Calculus 3: Standard course in vector calculus following Lang.

Spring 2017, Topics in dynamical systems: A somewhat advanced course for
undergraduates including: smooth dynamical systems, elements of
symbolic dynamical systems, Markov chains, structural stability, Smale Horse
shoe, Hamiltonian dynamics, Morse theory
and Morse homology.
  \section {\sc Reference letters}
%  Octav Cornea, CRM Montreal \\
Michael Usher, University of Georgia \\
Kevin Costello, Northwestern and the Perimeter Institute \\ 
% %   Francois Lalonde, CRM Montreal \\
%   Paul Seidel, MIT \\
% %  Helmut Hofer, IAS, Princeton \\
% %  Dennis Sullivan, SUNY Stony Brook and CUNY graduate center, NY, NY \\
Leonid Polterovich, Tel Aviv University and University of Chicago \\
%  Michael Sullivan, University of Massachusets, Amherst \\
Yael Karshon, University of Toronto \\
Kaoru Ono, RIMS Kyoto \\
% %     AlZinger, SUNY Stony Brook\\
Paul Gunnels (teaching), University of Massachusets, Amherst \\
 \end{resume}
 \end{document}


%%===========================================================================%%

