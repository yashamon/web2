%
\documentclass[12pt]{letter} 
\usepackage{amsmath, amssymb, mathrsfs, amsfonts,  amsthm}
\signature{Yasha Savelyev}
%\location{1075 Evans}
%\telephone{(510)643-0893 (office), \,\,\,(510)883-9938 (home)}
%\email{yasha@math.sunysb.edu} 
\name{Yasha (Yakov) Savelyev}
\address{Bernal Díaz del Castillo 340, \\
Col. Villas San Sebastian,\\
28045 Colima, Colima, \\
Mexico \\
 } 
%\letterheadanyway
 \signature{Yasha Savelyev} 
 %\adjustmargins{.25in}
\date{Fall 2018}
\begin{document}

\begin{letter}{
}
\opening{Dear Application Committee:}   
I am applying for the assistant professor position. I am an independent researcher and enjoy teaching and dealing with students.
I also have much experience, having had a teaching postdoc at Umass Amherst, 
and non-teaching fellowships at CRM Montreal, and ICMAT Madrid. I also spent a semester as a postdoctoral fellow at MSRI, and a
month at Tel Aviv university and at the Research Institute for mathematical
sciences at Kyoto university as invited visitor. Currently I have a permanent contract at University of Colima,  Mexico, where I am teaching 1-2 courses a semester.
I received my PhD under the 
guidance of Dusa McDuff
in August 2008 from
the Department of Mathematics at SUNY Stony Brook.  While I am fairly happy at my current university,  
I have some misgivings, which I suppose I should share.
First isolation, I would like to work with more students (to give an example there are only two currently in their junior year, that said they are rather good),
I would also like to advise some PhD students, I have many ideas to share. And of course I would like more fellow mathematicians to talk to, and in more diverse areas. For example I am learning a bit about distributed computing systems, and I have noone to talk to about that here.


Another issue is difficulty in Mexico in obtaining a good grant for research, there are simply too many good applications for the funding available.
Finally, in the last few years there was a very disturbing spike in cartel violence in the state of Colima, and it is now the most dangerous state in Mexico per capita. 


I think I would be very happy at the University of Illinois at Chicago.  It is a very diverse department, 
which suits my rather diverse interests,  and the location of the university
is highly appealing. 



% Michigan state looks like a super exciting place to do mathematics and
% in particular geometry-topology. I would be happy to interact with
% Casim Abbas and Matthew Hedden with whom I share my immediate interests in symplectic
% geometry, as well as Xiaodong Wang and other members of the
% department. The location of the university is appealing to me and I
% really look forward to returning to the USA.
% % The tenure track position at Queens college is highly interesting to me
% % for a number of reasons. Some of the recent  developments in my research
% % intersect aspects of algebra and algebraic geometry, and so could lead
% % to some discussions  with Scott Wilson and John Terilla among others. 
% % I would also have access to all the other great
% % universities in NYC which needless to say would be fantastic for
% % developing my ideas.

% Besides that NYC is a
% fantastic place to teach mathematics and do research. I went to
% the nearby John Bowne high school, straight after emigrating from
% Moscow, Russia and grew to love the student
% diversity of New York. No culture shock for me, it was just a fantastic
% experience, and frankly a refreshing break from the rather elitist, at
% the time, atmosphere in education in Moscow.
% My AP calculus teacher Arthur Samel, with whom I am still in touch, certainly
% played a big role in my choice to pursue study of higher mathematics,
% and I would like to think I could similarly help inspire some students.
%
% On a (even) more personal note my immediate family is currently based in NYC,
% with my sister just starting college,
% and this makes this position even more attractive.

My research interests are in the areas of symplectic and differential geometry, 
with deep interest in exploring connections with algebraic topology,  with a
good percentage of my recent work along these lines.  In particular I recently
establish some  relationships between the theory of Fukaya $A _{\infty} $
categories, Hamiltonian fibrations in symplectic geometry and abstract
algebraic topology, particularly the theory of quasi-categories of Joyal,
especially after work of Lurie. This project may be summarized as trying to categorify Novikov-Pontryagin theory using Floer homology techniques.
In a different and much more basic direction, I am researching new rigidity phenomena in locally conformally symplectic geometry, 
and the dynamical phenomenon of sky catastrophes for Reeb vector fields. I have also been interested in computer science.
% But most exciting is the construction of
% ``Euler'' classes, associated to Hamiltonian fibrations, in a certain generalized cohomology theory coming from the
% space of $A _{\infty} $ categories (constructed recently in my work). In the
% future this
% would allow application of powerful methods of abstract algebraic topology
% (like the Mayer-Vietoris sequence) to
% computation of certain Gromov-Witten invariants, or more specifically quantum
% characteristic classes originally defined in my thesis.
% I also have done some work in Hofer geometry  of the group of Hamiltonian diffeomorphisms of a 
% symplectic manifold, which is a highly interesting and natural
% bi-invariant Finsler geometry on this group. This geometry gives rise to a vast
% number of extremely difficult questions, with very elementary formulation,
% which makes it a very attractive object of study. One of my principal
% contributions in this field
% is the development of ``Morse theory''
% for the Hofer length functional. This had a number of applications, and is also
% recently used for certain computations with the Fukaya
% category. 
% I have extensive teaching experience, I referee for mathematical journals, and for Zentralblatt.  
% I have given many seminar talks and a
% number of conference talks. 

 If
you have any further questions, please send email to 
{\sf yasha.savelyev@gmail.com} or call me at 413-570-0163. 
Thank you for your time and consideration.


\closing{Sincerely,} 


\end{letter}

\end{document} 
