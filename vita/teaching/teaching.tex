% ----------------------------------------------------------------
% eTeX Paper in English *****************************************
% **** -----------------------------------------------------------
\documentclass{amsart}   
% \usepackage{amssymb} 
% \pagestyle{myheadings}   
\title {Teaching Statement, Fall 2018}
\author{Yakov Savelyev}
\begin{document}  
% \makeatletter
% \renewcommand{\@oddhead}%
% {\raisebox{-0.5cm}[\headheight][0pt]
% {\vbox{\hbox  to\textwidth{\strut \rightmark \hfill -- \thepage -- \hfill\leftmark }\hrule}}}
% \makeatother 
% \markboth{ Teaching statement}{Yakov Savelyev}

% \centerline{\bf Teaching Statement} 
% \centerline{Yakov Savelyev} 
% \bigskip
\maketitle
\thispagestyle{plain} 
I have extensive teaching experience.
I have been lecturing for roughly nine years, three years as a graduate student, three as a
postdoc at University of Massachusets, Amherst, where I was
teaching two courses per semester, and three as an associate professor at the University of Colima, Mexico.
In the latter I have been mostly teaching advanced undergraduate courses, one to two course per semester. I am also advising an undergraduate thesis 
at the University of Colima and am responsible for advising a number of students on academic life. In Colima I have also given lectures to high schoolers to attract them to mathematics.
As an undergraduate at Stony Brook University
I was a teaching assistant at
``mathematics learning center''.  

At UMASS, Amherst I have mainly done beginning undergraduate calculus courses including
multivariable calculus, and I was
relatively happy with them. For one thing because I really like
calculus, as it is such a beautiful (relatively) abstract idea, with 
incredible applicational power, which at the same time is within reach of the
mind of all undergraduate student, if they work hard enough
to try to grasp it.  In Colima I have been doing mostly upper division courses, which have often followed graduate level textbooks. For example for differential geometry, I ended up covering Chern's Annals of mathematics paper on the intrinsic proof of the Gauss Bonnet theorem.  This was possible partly because I only had three students who were rather motivated. (All three are now in graduate programs across the world.)
Overall the students here have been motivated, and surprisingly to me well adjusted to university life, in short happy. To give one example our physics students, whom I know also from my courses, have obtained third place in the world wide university physics competition.
That said I am missing out here on the opportunity to advise PhD students, I am a very independent researcher with a lot of ideas to explore, and simply don't have time for everything.

%  Although I have not been teaching at CRM-Montreal, as a postdoctoral
% fellow, I still sometimes reflected back on my teaching, thinking of some
% new and improved way of saying something.  
My main basic philosophy with regard
to teaching mathematics is that ``understanding'' is in a sense
a poor concept for the process of learning mathematics. It is more about growing comfortable with ideas and techniques through trial, repetition  and retrospection based on various points of view,
and making connections as one goes along. 
Of course sometimes the insight
from these mental connections is so deep, that we feel we ``understand''. When I am giving a proof on the blackboard I try to do so extemporaneously, letting the argument flow naturally, pausing on tricky points, exclaiming on key deductions, etc., there is a bit of a play there pretending you are rediscovering the argument (even though I do prepare beforehand).
I find that the students enjoy that and get more into it. 
% I am very flexible and tolerant of personal styles and variability of the students. For example, if they want to study on their own and just come for exams, I strongly advise them against it, but in the end let them.
% I feel that ultimately the art of mathematics is in doing
% hard looking things in interesting, relatively concise, conceptual ways. This is what awes most students, if anything. So this
% is what I try to convey. The other side of mathematics, arriving at interesting concepts and strategies through hard, gritty work is very interesting creatively, but is not
% something to be conveyed in lectures. At least not as a main thread and not in introductory courses. 

I try to invite as many questions as possible. Questions of
students are one of the driving forces behind a lecture in my opinion. They
immediately bring out the curiosity of other students, helping
them stay attentive, which in turn invites more questions. They also help me
feel out the audience for particular points of difficulties in comprehension, and
help me stay motivated. On evaluations I consistently get high marks for personal involvement, attentiveness to students, amiability.
% Of course students, especially at introductory courses,
% won't ask questions if they are not involved in the lecture, which bring us back
% to the first point.

% To keep them interested, when asking questions I try to encourage their own
% ideas first, fixing them if necessary, before giving them a more optimal answer
% or solution. 
%  I like to do lots  of examples, as aside from attracting their
%  attention they also help my own focus, and of course solving problems is what I
%  enjoy.  
%  
Overall I think it's fair to say that my students usually become comfortable
with me very fast. With some becoming very attached. Although I am also
slightly polarising. Hopefully, I can improve on that.
\end{document}
% **** -----------------------------------------------------------
% \documentclass{amsart% \usepackage{amssymb} 
% % \pagestyle{myheadings}   
% \title {Teaching Statement}
% \author{Yakov Savelyev}
% \begin{document}  
% % \makeatletter
% % \renewcommand{\@oddhead}%
% % {\raisebox{-0.5cm}[\headheight][0pt]
% % {\vbox{\hbox  to\textwidth{\strut \rightmark \hfill -- \thepage -- \hfill\leftmark }\hrule}}}
% % \makeatother 
% % \markboth{ Teaching statement}{Yakov Savelyev}
%
% % \centerline{\bf Teaching Statement} 
% % \centerline{Yakov Savelyev} 
% % \bigskip
% \maketitle
% \thispagestyle{plain} 
% I have extensive teaching experience, I have been lecturing for roughly 9 years, 3 as a graduate student,  3 as a
% postdoc at University of Massachusets, Amherst, where I was
% teaching 2 courses per semester.  I have also done TA work as a
% graduate student, and worked as teaching assistant at
% ``mathematics learning center'' at Stony Brook as an
% undergraduate student. For the last few years I have been without teaching  as
% I have been at CRM and then ICMAT as a research fellow, and I
% definitely miss it. Interacting with students seeing their joy at
% learning, their growth, brought me back to my own happy beginnings and I feel
% that this experience made me psychologically stronger.
%
% I have mainly done beginning undergraduate calculus courses including
% multivariable calculus, and I was
% relatively happy with them. For one thing because I really like
% calculus, as it is such a beautiful (relatively to where the
% students are) abstract idea, with 
% incredible applicational power, which at the same time is within reach of the
% mind of an average undergraduate student if they work hard enough
% to try to grasp it.  That said I would also really like to teach more advanced
% material, including graduate courses, where my unique perspective and
% experience may have more of an expression. Some courses I would be
% particularly keen to teach include: quantum mechanics (from a
% more mathematical perspective), functional analysis, (though this is
% not my expertise I have always loved the subject), abstract algebra
% (another subject I just like) symplectic and
% differential geometry, algebraic topology possibly touching on more
% advanced topics like
% topological field theories, string topology, higher categories, and a course on Atiyah-Singer index
% theorem.  
%
%  
% %  Although I have not been teaching at CRM-Montreal, as a postdoctoral
% % fellow, I still sometimes reflected back on my teaching, thinking of some
% % new and improved way of saying something.  
%
%  My main basic philosophy with regard
% to teaching mathematics is that ``understanding'' is in a sense
% a poor concept for the process of learning mathematics. It is more about growing comfortable and at home with
% with various concepts and techniques, from numerous angles,
% and making connections as one goes along. 
% Of course sometimes the insight
% from these mental connections is so deep, that we feel we ``understand'', but that
% certainly does not come for free. 
% For example, one thing in particular that
% I liked to challenge the students with at the very beggining of the
% integral calculus course is to ask them to informally explain to me what is ``area'', or
% what is the significance of this  concept.   I find that
% very often students cannot answer this question, even they have some
% vague intuition for this, I then give them a couple of ways to  think
% about this concept more precisely, (for example via probability, or
% recutting) even if still at an informal level.
% This nicely leads in to calculation of areas via integrals later on.
% I also really like working with examples and doing  problems in my
% lectures which bring out some concepts in a particularly clear way.
%
% I feel that ultimately the art of mathematics is in doing
% hard looking things in interesting, hopefully concise, conceptual ways. This is what awes most students, if anything. So this
% is what I try to convey. The other side of mathematics, arriving at interesting concepts and strategies through hard, gritty work is very interesting creatively, but is not
% something to be conveyed in lectures. At least not as a main thread and not in introductory courses. 
%  
% I try to invite as many questions as possible. Questions of
% students are one of the driving forces behind a lecture in my opinion. They
% immediately bring out the curiosity of other students, helping
% them stay attentive, which in turn invites more questions. They also help me
% feel out the audience for particular points of difficulties in comprehension, and
% help me stay motivated. 
% % Of course students, especially at introductory courses,
% % won't ask questions if they are not involved in the lecture, which bring us back
% % to the first point.
%
% % To keep them interested, when asking questions I try to encourage their own
% % ideas first, fixing them if necessary, before giving them a more optimal answer
% % or solution. 
% %  I like to do lots  of examples, as aside from attracting their
% %  attention they also help my own focus, and of course solving problems is what I
% %  enjoy.  
% %  
% Overal I think it's fair to say that my students usually become comfortable
% with me very fast. With some becoming very attached. Although I am also
% slightly polarising. Hopefully, I can improve on that.
% \end{document}
